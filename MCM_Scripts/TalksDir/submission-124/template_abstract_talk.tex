\documentclass[12pt,a4paper,figuresright]{book}

\usepackage{amsmath,amssymb}
\usepackage{tabularx,graphicx,url,xcolor,rotating,multicol,epsfig,colortbl}

\setlength{\textheight}{25.2cm}
\setlength{\textwidth}{16.5cm} %\setlength{\textwidth}{18.2cm}
\setlength{\voffset}{-1.6cm}
\setlength{\hoffset}{-0.3cm} %\setlength{\hoffset}{-1.2cm}
\setlength{\evensidemargin}{-0.3cm} 
\setlength{\oddsidemargin}{0.3cm}
\setlength{\parindent}{0cm} 
\setlength{\parskip}{0.3cm}

% -- adding a talk
\newenvironment{talk}[6]% [1] talk title
                         % [2] speaker name, [3] affiliations, [4] email,
                         % [5] coauthors, [6] special session
                         % [7] time slot
                         % [8] talk id, [9] session id or photo
 {%\needspace{6\baselineskip}%
  \vskip 0pt\nopagebreak%
%   \colorbox{gray!20!white}{\makebox[0.99\textwidth][r]{}}\nopagebreak%
%   \ifthenelse{\equal{#9}{photo}}{%
%                     \\\\\colorbox{gray!20!white}{\makebox{\includegraphics[width=3cm]{#8}}}\nopagebreak}{}%
 \vskip 0pt\nopagebreak%
%  \label{#8}%
  \textbf{#1}\vspace{3mm}\\\nopagebreak%
  \textit{#2}\\\nopagebreak%
  #3\\\nopagebreak%
  \url{#4}\vspace{3mm}\\\nopagebreak%
  \ifthenelse{\equal{#5}{}}{}{Coauthor(s): #5\vspace{3mm}\\\nopagebreak}%
  \ifthenelse{\equal{#6}{}}{}{Special session: #6\quad \vspace{3mm}\\\nopagebreak}%
 }
 {\vspace{1cm}\nopagebreak}%

\pagestyle{empty}

% ------------------------------------------------------------------------
% Document begins here
% ------------------------------------------------------------------------
\begin{document}
	
\begin{talk}
  {Stochastic collocation for nonlinear and non stationary equations}% [1] talk title
  {Michael Feischl}% [2] speaker name
  {TU Wien}% [3] affiliations
  {michael.feischl@tuwien.ac.at}% [4] email
  {Andrea Scaglioni}% [5] coauthors
  {Efficient methods for uncertainty quantification in differential equations}% [6] special session. Leave this field empty for contributed talks. 
				% Insert the title of the special session if you were invited to give a talk in a special session.
			
We show convergence rates for a sparse grid approximation of the distribution of solutions of the stochastic Landau-Lifshitz-Gilbert equation. Beyond being a frequently studied equation in engineering and physics, the stochastic Landau-Lifshitz-Gilbert equation poses many interesting challenges that do not appear simultaneously in previous works on uncertainty quantification: The equation is strongly non-linear, time-dependent, and has a non-convex side constraint. Moreover, the parametrization of the stochastic noise features countably many unbounded parameters and low regularity compared to other elliptic and parabolic problems studied in uncertainty quantification. We use a novel technique to establish uniform holomorphic regularity of the parameter-to-solution map based on a Gronwall-type estimate and the implicit function theorem. This method is very general and based on a set of abstract assumptions. Thus, it can be applied beyond the Landau-Lifshitz-Gilbert equation as well. We demonstrate numerically the feasibility of approximating with sparse grid and show a clear advantage of a multi-level sparse grid scheme.

\medskip


\begin{enumerate}
	\item[{[1]}] An, X., Dick, J., Feischl, M., Scaglioni, A. and Tran, T. {\it Sparse grid approximation of the stochastic Landau-Lifshitz-Gilbert equation.} arXiv 2310.11225, 2024.
\end{enumerate}

\end{talk}

\end{document}

