
\documentclass[12pt,a4paper,figuresright]{book}





\usepackage{amsmath,amssymb}
\usepackage{tabularx,graphicx,url,xcolor,rotating,multicol,epsfig,colortbl}

\setlength{\textheight}{25.2cm}
\setlength{\textwidth}{16.5cm} %\setlength{\textwidth}{18.2cm}
\setlength{\voffset}{-1.6cm}
\setlength{\hoffset}{-0.3cm} %\setlength{\hoffset}{-1.2cm}
\setlength{\evensidemargin}{-0.3cm} 
\setlength{\oddsidemargin}{0.3cm}
\setlength{\parindent}{0cm} 
\setlength{\parskip}{0.3cm}


\setlength{\floatsep}{12pt plus 2pt minus 2pt}








% -- adding a talk
\newenvironment{talk}[6]% [1] talk title
                         % [2] speaker name, [3] affiliations, [4] email,
                         % [5] coauthors, [6] special session
                         % [7] time slot
                         % [8] talk id, [9] session id or photo
 {%\needspace{6\baselineskip}%
  \vskip 0pt\nopagebreak%
%   \colorbox{gray!20!white}{\makebox[0.99\textwidth][r]{}}\nopagebreak%
%   \ifthenelse{\equal{#9}{photo}}{%
%                     \\\\\colorbox{gray!20!white}{\makebox{\includegraphics[width=3cm]{#8}}}\nopagebreak}{}%
 \vskip 0pt\nopagebreak%
%  \label{#8}%
  \textbf{#1}\vspace{3mm}\\\nopagebreak%
  \textit{#2}\\\nopagebreak%
  #3\\\nopagebreak%
  \url{#4}\vspace{3mm}\\\nopagebreak%
  \ifthenelse{\equal{#5}{}}{}{Coauthor(s): #5\vspace{3mm}\\\nopagebreak}%
  \ifthenelse{\equal{#6}{}}{}{Special session: #6\quad \vspace{3mm}\\\nopagebreak}%
 }
 {\vspace{1cm}\\\nopagebreak}%



\pagestyle{empty}

% ------------------------------------------------------------------------
% Document begins here
% ------------------------------------------------------------------------
\begin{document}



\begin{talk}
  {How to choose an annealing algorithm}% [1] talk title
  {Alexandre Bouchard-C\^ot\'e}% [2] speaker name
  {University of British Columbia}% [3] affiliations
  {bouchard@stat.ubc.ca}% [4] email
  {Saifuddin Syed, Kevin Chern, Arnaud Doucet}% [5] coauthors
  {Continuous-time dynamics in Monte Carlo and beyond}% [6] special session. Leave this field empty for contributed talks. 
				% Insert the title of the special session if you were invited to give a talk in a special session.

				
				

Over the years, several algorithms have been developed to tackle normalization constant estimation. A handful of those have passed the test of time thanks to their capacity to beat the curse of dimensionality in many realistic scenarios: on one hand, Annealed Importance Sampling (AIS) and Sequential Monte Carlo (SMC) methods, and on the other, Parallel Tempering (PT) and Simulated Tempering (ST) algorithms. Indeed many recent developments can be contextualized as members of one of these two families of meta-algorithms.

A priori, these two families of algorithms, AIS/SMC versus PT/ST, appear quite distinct and indeed these communities are largely silos. This leads to an important practical question: for a given problem, which annealing algorithm should be recommended? I will present our work toward tackling this question, in which a key object needed for theoretical and methodological purposes is the rate function of various limiting PDMPs. 
\end{talk}


\end{document}

