\documentclass[12pt,a4paper,figuresright]{book}

\usepackage{amsmath,amssymb}
\usepackage{tabularx,graphicx,url,xcolor,rotating,multicol,epsfig,colortbl}

\setlength{\textheight}{25.2cm}
\setlength{\textwidth}{16.5cm} %\setlength{\textwidth}{18.2cm}
\setlength{\voffset}{-1.6cm}
\setlength{\hoffset}{-0.3cm} %\setlength{\hoffset}{-1.2cm}
\setlength{\evensidemargin}{-0.3cm} 
\setlength{\oddsidemargin}{0.3cm}
\setlength{\parindent}{0cm} 
\setlength{\parskip}{0.3cm}

% -- adding a talk
\newenvironment{talk}[6]% [1] talk title
                         % [2] speaker name, [3] affiliations, [4] email,
                         % [5] coauthors, [6] special session
                         % [7] time slot
                         % [8] talk id, [9] session id or photo
 {%\needspace{6\baselineskip}%
  \vskip 0pt\nopagebreak%
%   \colorbox{gray!20!white}{\makebox[0.99\textwidth][r]{}}\nopagebreak%
%   \ifthenelse{\equal{#9}{photo}}{%
%                     \\\\\colorbox{gray!20!white}{\makebox{\includegraphics[width=3cm]{#8}}}\nopagebreak}{}%
 \vskip 0pt\nopagebreak%
%  \label{#8}%
  \textbf{#1}\vspace{3mm}\\\nopagebreak%
  \textit{#2}\\\nopagebreak%
  #3\\\nopagebreak%
  \url{#4}\vspace{3mm}\\\nopagebreak%
  \ifthenelse{\equal{#5}{}}{}{Coauthor(s): #5\vspace{3mm}\\\nopagebreak}%
  \ifthenelse{\equal{#6}{}}{}{Special session: #6\quad \vspace{3mm}\\\nopagebreak}%
 }
 {\vspace{1cm}\nopagebreak}%

\pagestyle{empty}

% ------------------------------------------------------------------------
% Document begins here
% ------------------------------------------------------------------------
\begin{document}
	
\begin{talk}
  {Conditional Quasi-Monte Carlo with Active Subspaces}% [1] talk title
  {Sifan Liu}% [2] speaker name
  {Department of Statistics, Stanford University}% [3] affiliations
  {sfliu@stanford.edu}% [4] email
  {Art B. Owen}% [5] coauthors
  {Recent advances in QMC methods for computational finance and financial risk management}% [6] special session. Leave this field empty for contributed talks. 
				% Insert the title of the special session if you were invited to give a talk in a special session.
			



Conditional Monte Carlo is a powerful strategy for enhancing the efficiency and accuracy of quasi-Monte Carlo and randomized quasi-Monte Carlo methods. By pre-integrating over a variable in a closed form, it reduces the variance and can potentially improve the smoothness of the integrand. Selecting the appropriate variable for pre-integration requires careful consideration of both the variance reduction factor and the feasibility of the pre-integration step. For integrals with respect to a Gaussian distribution, there is the flexibility to pre-integrate over any linear combination of variables. We introduce a systematic method for choosing the pre-integration direction, leveraging the active subspace decomposition to identify the most important direction. In some scenarios, pre-integration is feasible only along directions that satisfy certain constraints. Therefore, we propose to compute the active subspace decomposition subject to these constraints so that pre-integration can be easily carried out. The effectiveness of the proposed method is demonstrated through its application to derivative pricing problems under the Black-Scholes model and various stochastic volatility models.














\end{talk}

\end{document}

