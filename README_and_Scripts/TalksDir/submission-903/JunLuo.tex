
\documentclass[12pt,a4paper,figuresright]{book}





\usepackage{amsmath,amssymb}
\usepackage{tabularx,graphicx,url,xcolor,rotating,multicol,epsfig,colortbl}

\setlength{\textheight}{25.2cm}
\setlength{\textwidth}{16.5cm} %\setlength{\textwidth}{18.2cm}
\setlength{\voffset}{-1.6cm}
\setlength{\hoffset}{-0.3cm} %\setlength{\hoffset}{-1.2cm}
\setlength{\evensidemargin}{-0.3cm}
\setlength{\oddsidemargin}{0.3cm}
\setlength{\parindent}{0cm}
\setlength{\parskip}{0.3cm}


\setlength{\floatsep}{12pt plus 2pt minus 2pt}








% -- adding a talk
\newenvironment{talk}[6]% [1] talk title
                         % [2] speaker name, [3] affiliations, [4] email,
                         % [5] coauthors, [6] special session
                         % [7] time slot
                         % [8] talk id, [9] session id or photo
 {%\needspace{6\baselineskip}%
  \vskip 0pt\nopagebreak%
%   \colorbox{gray!20!white}{\makebox[0.99\textwidth][r]{}}\nopagebreak%
%   \ifthenelse{\equal{#9}{photo}}{%
%                     \\\\\colorbox{gray!20!white}{\makebox{\includegraphics[width=3cm]{#8}}}\nopagebreak}{}%
 \vskip 0pt\nopagebreak%
%  \label{#8}%
  \textbf{#1}\vspace{3mm}\\\nopagebreak%
  \textit{#2}\\\nopagebreak%
  #3\\\nopagebreak%
  \url{#4}\vspace{3mm}\\\nopagebreak%
  \ifthenelse{\equal{#5}{}}{}{Coauthor(s): #5\vspace{3mm}\\\nopagebreak}%
  \ifthenelse{\equal{#6}{}}{}{Special session: #6\quad \vspace{3mm}\\\nopagebreak}%
 }
 {\vspace{1cm}\\\nopagebreak}%



\pagestyle{empty}

% ------------------------------------------------------------------------
% Document begins here
% ------------------------------------------------------------------------
\begin{document}


\begin{talk}
  {Efficient Risk Quantification via Nested Simulation}% [1] talk title
  {Jun Luo}% [2] speaker name
  {Shanghai Jiao Tong University}% [3] affiliations
  {jluo_ms@sjtu.edu.cn}% [4] email
  {Guo Liang and Kun Zhang}% [5] coauthors
  {}% [6] special session. Leave this field empty for contributed talks.
				% Insert the title of the special session if you were invited to give a talk in a special session.
Risk quantification is pivotal in both portfolio risk measurement and input model uncertainty. This paper aims to quantify the risk by studying widely used risk measures, Value-at-Risk (VaR) and Conditional Value-at-Risk (CVaR). We introduce a jackknife-based nested simulation method to estimate these measures, providing point estimators, confidence intervals (CIs), and deriving their asymptotic properties. Furthermore, we propose an efficient algorithm that ensures the mean squared errors of the estimators and the widths of the CIs decay at their optimal rates in practice. Numerical results are consistent with the theory presented.
\end{talk}


\end{document}
