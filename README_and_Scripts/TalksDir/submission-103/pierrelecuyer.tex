
\documentclass[12pt,a4paper,figuresright]{book}

\usepackage{amsmath,amssymb}
\usepackage{tabularx,graphicx,url,xcolor,rotating,multicol,epsfig,colortbl}

\setlength{\textheight}{25.2cm}
\setlength{\textwidth}{16.5cm} %\setlength{\textwidth}{18.2cm}
\setlength{\voffset}{-1.6cm}
\setlength{\hoffset}{-0.3cm} %\setlength{\hoffset}{-1.2cm}
\setlength{\evensidemargin}{-0.3cm} 
\setlength{\oddsidemargin}{0.3cm}
\setlength{\parindent}{0cm} 
\setlength{\parskip}{0.3cm}
\setlength{\floatsep}{12pt plus 2pt minus 2pt}

% -- adding a talk
\newenvironment{talk}[6]% [1] talk title
                         % [2] speaker name, [3] affiliations, [4] email,
                         % [5] coauthors, [6] special session
                         % [7] time slot
                         % [8] talk id, [9] session id or photo
 {%\needspace{6\baselineskip}%
  \vskip 0pt\nopagebreak%
%   \colorbox{gray!20!white}{\makebox[0.99\textwidth][r]{}}\nopagebreak%
%   \ifthenelse{\equal{#9}{photo}}{%
%                     \\\\\colorbox{gray!20!white}{\makebox{\includegraphics[width=3cm]{#8}}}\nopagebreak}{}%
 \vskip 0pt\nopagebreak%
%  \label{#8}%
  \textbf{#1}\vspace{3mm}\\\nopagebreak%
  \textit{#2}\\\nopagebreak%
  #3\\\nopagebreak%
  \url{#4}\vspace{3mm}\\\nopagebreak%
  \ifthenelse{\equal{#5}{}}{}{Coauthor(s): #5\vspace{3mm}\\\nopagebreak}%
  \ifthenelse{\equal{#6}{}}{}{Special session: #6\quad \vspace{3mm}\\\nopagebreak}%
 }
 {\vspace{1cm}\\\nopagebreak}%

\pagestyle{empty}

% ------------------------------------------------------------------------
% Document begins here
% ------------------------------------------------------------------------
\begin{document}

\begin{talk}
  {A Redesigned C++ Library to Test the Lattice Structure of Linear Generators
	 and Search for Good Ones}% [1] talk title
  {Pierre L'Ecuyer}% [2] speaker name
  {DIRO, Universit\'e de Montr\'eal, Canada}% [3] affiliations
  {lecuyer@iro.umontreal.ca}% [4] email
  {Christian F. Weiss}% [5] coauthors
  {Testing and analysis of pseudo-random number generators}% [6] title of special session.

% Your abstract goes here. 
The spectral test, introduced in [1] and popularized by [2], 
remains the gold standard to measure the uniformity of point sets 
produced by linear random number generators by assessing the quality of their lattice structure.
Multiple recursive generators, multiply-with-carry, matrix linear congruential generators, 
and combined generators of these types, for example, can be constructed and analyzed 
by this type of test [3, 4, 5].  
A software tool named LatMRG was written about 30 years ago in the Modula-2 language 
to perform the spectral test and search for generators with a good lattice structure [4],
but this tool can no longer be used because Modula-2 is no longer supported. 
We are aware of no other similar tool currently available.

In this talk, we present a completely redesigned version of LatMRG, written in C++,
and using NTL to handle computations with large numbers. 
Some of the underlying algorithms have been improved compared with the Modula-2 version.
We illustrate what the software can do and its performance via several examples.

\medskip

%\bibitem{rCOV67a}
[1] R.~R. Coveyou and R.~D. MacPherson.
Fourier analysis of uniform random number generators.
{\em Journal of the ACM}, 14:100--119, 1967.

%\bibitem{rKNU81a}
[2] D.~E. Knuth.
{\em The Art of Computer Programming, Volume 2: Seminumerical Algorithms}.
Addison-Wesley, Reading, MA, second edition, 1981.

%\bibitem{rLEC99b}
[3] P.~L'Ecuyer.
Good parameters and implementations for combined multiple recursive
  random number generators.
{\em Operations Research}, 47(1):159--164, 1999.

%\bibitem{rLEC97c}
[4] P.~L'Ecuyer and R.~Couture.
An implementation of the lattice and spectral tests for multiple
  recursive linear random number generators.
{\em INFORMS Journal on Computing}, 9(2):206--217, 1997.

%\bibitem{rLEC20m}
[5] P.~L'Ecuyer, P.~Wambergue, and E.~Bourceret.
Spectral analysis of the {MIXMAX} random number generators.
{\em INFORMS Journal on Computing}, 32(1):135--144, 2020.
%
\end{talk}
\end{document}

