\documentclass[12pt,a4paper,figuresright]{book}

\usepackage{amsmath,amssymb}
\usepackage{tabularx,graphicx,url,xcolor,rotating,multicol,epsfig,colortbl}

\setlength{\textheight}{25.2cm}
\setlength{\textwidth}{16.5cm} %\setlength{\textwidth}{18.2cm}
\setlength{\voffset}{-1.6cm}
\setlength{\hoffset}{-0.3cm} %\setlength{\hoffset}{-1.2cm}
\setlength{\evensidemargin}{-0.3cm} 
\setlength{\oddsidemargin}{0.3cm}
\setlength{\parindent}{0cm} 
\setlength{\parskip}{0.3cm}

% -- adding a talk
\newenvironment{talk}[6]% [1] talk title
                         % [2] speaker name, [3] affiliations, [4] email,
                         % [5] coauthors, [6] special session
                         % [7] time slot
                         % [8] talk id, [9] session id or photo
 {%\needspace{6\baselineskip}%
  \vskip 0pt\nopagebreak%
%   \colorbox{gray!20!white}{\makebox[0.99\textwidth][r]{}}\nopagebreak%
%   \ifthenelse{\equal{#9}{photo}}{%
%                     \\\\\colorbox{gray!20!white}{\makebox{\includegraphics[width=3cm]{#8}}}\nopagebreak}{}%
 \vskip 0pt\nopagebreak%
%  \label{#8}%
  \textbf{#1}\vspace{3mm}\\\nopagebreak%
  \textit{#2}\\\nopagebreak%
  #3\\\nopagebreak%
  \url{#4}\vspace{3mm}\\\nopagebreak%
  \ifthenelse{\equal{#5}{}}{}{Coauthor(s): #5\vspace{3mm}\\\nopagebreak}%
  \ifthenelse{\equal{#6}{}}{}{Special session: #6\quad \vspace{3mm}\\\nopagebreak}%
 }
 {\vspace{1cm}\nopagebreak}%

\pagestyle{empty}

% ------------------------------------------------------------------------
% Document begins here
% ------------------------------------------------------------------------
\begin{document}
	
\begin{talk}
  {QMC and nonnegative local discrepancy}% [1] talk title
  {Peter Kritzer}% [2] speaker name
  {RICAM, Austrian Academy of Sciences}% [3] affiliations
  {peter.kritzer@oeaw.ac.at}% [4] email
  {Michael Gnewuch, Art B.~Owen, Zexin Pan}% [5] coauthors
  {}% [6] special session. Leave this field empty for contributed talks. 
				% Insert the title of the special session if you were invited to give a talk in a special session.
			
In our talk, we present a way of finding a  non-asymptotic and 
computable upper bound for the integral of a function $f$ over $[0,1]^d$. 		Indeed, let $f:[0,1]^d\to\mathbb{R}$ be a completely monotone integrand
and let points $\boldsymbol{x}_0,\dots,\boldsymbol{x}_{n-1}\in[0,1]^d$
have a non-negative local discrepancy (NNLD) everywhere
in $[0,1]^d$. In such a situation, we can use the points 
$\boldsymbol{x}_0,\dots,\boldsymbol{x}_{n-1}$ in a quasi-Monte Carlo (QMC) rule 
and obtain the desired bound for the integral of $f$. 
An analogous non-positive local
discrepancy (NPLD) property provides a computable lower
bound.

We will also discuss which point sets are candidates 
for having the NNLD or NPLD property.


\end{talk}

\end{document}

