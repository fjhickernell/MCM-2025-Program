
\documentclass[12pt,a4paper,figuresright]{book}





\usepackage{amsmath,amssymb}
\usepackage{tabularx,graphicx,url,xcolor,rotating,multicol,epsfig,colortbl}

\setlength{\textheight}{25.2cm}
\setlength{\textwidth}{16.5cm} %\setlength{\textwidth}{18.2cm}
\setlength{\voffset}{-1.6cm}
\setlength{\hoffset}{-0.3cm} %\setlength{\hoffset}{-1.2cm}
\setlength{\evensidemargin}{-0.3cm} 
\setlength{\oddsidemargin}{0.3cm}
\setlength{\parindent}{0cm} 
\setlength{\parskip}{0.3cm}


\setlength{\floatsep}{12pt plus 2pt minus 2pt}








% -- adding a talk
\newenvironment{talk}[6]% [1] talk title
                         % [2] speaker name, [3] affiliations, [4] email,
                         % [5] coauthors, [6] special session
                         % [7] time slot
                         % [8] talk id, [9] session id or photo
 {%\needspace{6\baselineskip}%
  \vskip 0pt\nopagebreak%
%   \colorbox{gray!20!white}{\makebox[0.99\textwidth][r]{}}\nopagebreak%
%   \ifthenelse{\equal{#9}{photo}}{%
%                     \\\\\colorbox{gray!20!white}{\makebox{\includegraphics[width=3cm]{#8}}}\nopagebreak}{}%
 \vskip 0pt\nopagebreak%
%  \label{#8}%
  \textbf{#1}\vspace{3mm}\\\nopagebreak%
  \textit{#2}\\\nopagebreak%
  #3\\\nopagebreak%
  \url{#4}\vspace{3mm}\\\nopagebreak%
  \ifthenelse{\equal{#5}{}}{}{Coauthor(s): #5\vspace{3mm}\\\nopagebreak}%
  \ifthenelse{\equal{#6}{}}{}{Special session: #6\quad \vspace{3mm}\\\nopagebreak}%
 }
 {\vspace{1cm}\\\nopagebreak}%



\pagestyle{empty}

% ------------------------------------------------------------------------
% Document begins here
% ------------------------------------------------------------------------
\begin{document}



\begin{talk}
  {Metropolis-adjusted interacting particle sampling}% [1] talk title
  {Bj\"orn Sprungk}% [2] speaker name
  {TU Bergakademie Freiberg}% [3] affiliations
  {bjoern.sprungk@math.tu-freiberg.de}% [4] email
  {Simon Weissmann, Jakob Zech}% [5] coauthors
  {}% [6] special session. Leave this field empty for contributed talks. 
				% Insert the title of the special session if you were invited to give a talk in a special session.

In recent years, various interacting particle samplers have been developed to sample from complex target distributions, such as those found in Bayesian inverse problems. These samplers are motivated by the mean-field limit perspective and implemented as ensembles of particles that move in the product state space according to coupled stochastic differential equations. The ensemble approximation and numerical time stepping used to simulate these systems can introduce bias and affect the invariance of the particle system with respect to the target distribution. To correct for this, we investigate the use of a Metropolization step, similar to the Metropolis-adjusted Langevin algorithm. We examine both ensemble- and particle-wise Metropolization and prove basic convergence of the resulting ensemble Markov chain to the target distribution. Our results demonstrate the benefits of this correction in numerical examples for popular interacting particle samplers such as ALDI, CBS, and stochastic SVGD.				
				
\medskip

[1] B. Sprungk, S. Weissmann, J. Zech. Metropolis-adjusted interacting particle sampling. arXiv:2312.13889, 40 pages, 2023.
\end{talk}


\end{document}

