
\documentclass[12pt,a4paper,figuresright]{book}





\usepackage{amsmath,amssymb}
\usepackage{tabularx,graphicx,url,xcolor,rotating,multicol,epsfig,colortbl}

\setlength{\textheight}{25.2cm}
\setlength{\textwidth}{16.5cm} %\setlength{\textwidth}{18.2cm}
\setlength{\voffset}{-1.6cm}
\setlength{\hoffset}{-0.3cm} %\setlength{\hoffset}{-1.2cm}
\setlength{\evensidemargin}{-0.3cm} 
\setlength{\oddsidemargin}{0.3cm}
\setlength{\parindent}{0cm} 
\setlength{\parskip}{0.3cm}


\setlength{\floatsep}{12pt plus 2pt minus 2pt}








% -- adding a talk
\newenvironment{talk}[6]% [1] talk title
                         % [2] speaker name, [3] affiliations, [4] email,
                         % [5] coauthors, [6] special session
                         % [7] time slot
                         % [8] talk id, [9] session id or photo
 {%\needspace{6\baselineskip}%
  \vskip 0pt\nopagebreak%
%   \colorbox{gray!20!white}{\makebox[0.99\textwidth][r]{}}\nopagebreak%
%   \ifthenelse{\equal{#9}{photo}}{%
%                     \\\\\colorbox{gray!20!white}{\makebox{\includegraphics[width=3cm]{#8}}}\nopagebreak}{}%
 \vskip 0pt\nopagebreak%
%  \label{#8}%
  \textbf{#1}\vspace{3mm}\\\nopagebreak%
  \textit{#2}\\\nopagebreak%
  #3\\\nopagebreak%
  \url{#4}\vspace{3mm}\\\nopagebreak%
  \ifthenelse{\equal{#5}{}}{}{Coauthor(s): #5\vspace{3mm}\\\nopagebreak}%
  \ifthenelse{\equal{#6}{}}{}{Special session: #6\quad \vspace{3mm}\\\nopagebreak}%
 }
 {\vspace{1cm}\\\nopagebreak}%



\pagestyle{empty}

% ------------------------------------------------------------------------
% Document begins here
% ------------------------------------------------------------------------
\begin{document}



\begin{talk}
  {Acceleration of true orbit pseudorandom number generators using Newton's method}% [1] talk title
  {Asaki Saito}% [2] speaker name
  {Future University Hakodate}% [3] affiliations
  {saito@fun.ac.jp}% [4] email
  {Akihiro Yamaguchi}% [5] coauthors
  {}% [6] special session. Leave this field empty for contributed talks. 
				% Insert the title of the special session if you were invited to give a talk in a special session.

We developed pseudorandom number generators, termed true orbit
generators, utilizing true orbits of the Bernoulli map on irrational
algebraic integers [1,2].
These generators yield binary sequences that appear in the binary
expansions of irrational algebraic integers, offering nonperiodic
sequences, unlike existing generators.
Supported by ergodic theory [3] and Borel's conjecture [4], these
generators are expected to have high statistical quality, and
extensive computer experiments have confirmed this [1,2].
However, their computational cost is significantly high, with a
worst-case time complexity of $O(N^2)$ for generating a sequence of
length $N$.

To address the issue of the high computational cost, we employ Newton's method, a technique for
producing successively better approximations to the roots of a
function, to accelerate the true orbit generators.
This involves obtaining the exact binary expansion of a true root
(i.e., an irrational algebraic integer) $\alpha$ from its
approximation $x$, which includes an error.
We establish a sufficient condition ensuring that the first $N$ bits
of the binary expansions of $\alpha$ and $x$ match, thereby ensuring
the generation of the same pseudorandom sequence as the true orbit
generators.
Furthermore, we demonstrate that the worst-case time complexity for
generating a sequence of length $N$ using the method proposed in this
study is equivalent to that of multiplying two $N$-bit integers,
showing its efficiency compared to the original generators with
$O(N^2)$ time complexity.

\medskip

[1] A. Saito and A. Yamaguchi,
``Pseudorandom number generation using chaotic true orbits of the Bernoulli map,''
Chaos {\bf 26}, 063122 (2016).

[2] A. Saito and A. Yamaguchi,
``Pseudorandom number generator based on the Bernoulli map on cubic algebraic integers,''
Chaos {\bf 28}, 103122 (2018).

[3] P. Billingsley,
{\it Ergodic Theory and Information}
(Wiley, New York, 1965).

[4] \'{E}. Borel,
``Sur les chiffres d\'{e}cimaux de $\sqrt{2}$ et divers probl\`{e}mes de
probabilit\'{e}s en cha\^{i}ne,''
C. R. Acad. Sci. Paris {\bf 230}, 591--593 (1950).
\end{talk}

\end{document}
