
\documentclass[12pt,a4paper,figuresright]{book}





\usepackage{amsmath,amssymb}
\usepackage{tabularx,graphicx,url,xcolor,rotating,multicol,epsfig,colortbl}

\setlength{\textheight}{25.2cm}
\setlength{\textwidth}{16.5cm} %\setlength{\textwidth}{18.2cm}
\setlength{\voffset}{-1.6cm}
\setlength{\hoffset}{-0.3cm} %\setlength{\hoffset}{-1.2cm}
\setlength{\evensidemargin}{-0.3cm}
\setlength{\oddsidemargin}{0.3cm}
\setlength{\parindent}{0cm}
\setlength{\parskip}{0.3cm}


\setlength{\floatsep}{12pt plus 2pt minus 2pt}








% -- adding a talk
\newenvironment{talk}[6]% [1] talk title
                         % [2] speaker name, [3] affiliations, [4] email,
                         % [5] coauthors, [6] special session
                         % [7] time slot
                         % [8] talk id, [9] session id or photo
 {%\needspace{6\baselineskip}%
  \vskip 0pt\nopagebreak%
%   \colorbox{gray!20!white}{\makebox[0.99\textwidth][r]{}}\nopagebreak%
%   \ifthenelse{\equal{#9}{photo}}{%
%                     \\\\\colorbox{gray!20!white}{\makebox{\includegraphics[width=3cm]{#8}}}\nopagebreak}{}%
 \vskip 0pt\nopagebreak%
%  \label{#8}%
  \textbf{#1}\vspace{3mm}\\\nopagebreak%
  \textit{#2}\\\nopagebreak%
  #3\\\nopagebreak%
  \url{#4}\vspace{3mm}\\\nopagebreak%
  \ifthenelse{\equal{#5}{}}{}{Coauthor(s): #5\vspace{3mm}\\\nopagebreak}%
  \ifthenelse{\equal{#6}{}}{}{Special session: #6\quad \vspace{3mm}\\\nopagebreak}%
 }
 {\vspace{1cm}\\\nopagebreak}%



\pagestyle{empty}

% ------------------------------------------------------------------------
% Document begins here
% ------------------------------------------------------------------------
\begin{document}



\begin{talk}
  {Using Kronecker point sets for function approximation in the Korobov space}% [1] talk title
  {Laurence Wilkes}% [2] speaker name
  {Department of Computer Science, KU Leuven, Belgium}% [3] affiliations
  {laurence.wilkes@kuleuven.be}% [4] email
  {Ronald Cools, Dirk Nuyens}% [5] coauthors
  {Universality in QMC and related algorithms}% [6] Special session title


In this talk, we will introduce the Kronecker point sets and demonstrate that they produce almost the optimal rate of worst case $L_2$ error convergence for function approximation in the Korobov space.
For $n$ sample points, and a generator $\boldsymbol{\zeta} \in [0, 1]^d$, we define the \emph{Kronecker point set} as the set of points $\{ \{ k \boldsymbol{\zeta} \} \: : \: k = 1, \ldots, n \}$.
Unlike rank-1 lattices, which achieve a worst case error convergence of order $\alpha/2$ in the Korobov space of smoothness $\alpha$, the Kronecker point sets are able to achieve the optimal order of~$\alpha$.
This is made possible by a least squares algorithm similar to one presented in [1]. %Krieg and Ullrich (2021).

 \medskip

 \begin{enumerate}%[label={[\arabic*]}]
     \item[{[1]}] Krieg, D., \& Ullrich, M. (2021). Function values are enough for \( L_2 \)-approximation. \textit{Foundations of Computational Mathematics, 21}, 1141-1151.
 \end{enumerate}
\quad
\end{talk}


\end{document}
