\chapter{Special Sessions}\newpage

\begin{session}
 {Stochastic Computation and Complexity, Part I: SDEs, Stochastic optimization and neural networks}% [1] session title
 {Thomas M\"uller-Gronbach}% [2] organizer name
 {University of Passau}% [3] affiliations
 {Thomas.Mueller-Gronbach@uni-passau.de}% [4] email
 {}% [5] organizer name. Leave unchanged if there is no second organizer, otherwise fill in accordingly.
 {}% [6] affiliations. Leave unchanged if there is no second organizer, otherwise fill in accordingly.
 {}% [7] email. Leave unchanged if there is no second organizer, otherwise fill in accordingly.
 {SS3}
 %%{SP03}% [8] session id
 {}% [9] third organizer, if any
The session is devoted to algorithms and complexity for
\begin{itemize}[itemsep=0pt,topsep=0pt]
 \item quadrature and strong approximation of SDEs and SPDEs, in particular under nonstandard assumptions,

 \item high and infinite dimensional integration and approximation, and

 \item stochastic optimization and neural networks,
\end{itemize}
including connections to functional analysis and stochastic analysis.
\end{session}

\sessionPart{}% [1] part
{\hfill\timeslot{Monday, August 19, 2024 -- Morning}
{10:30}{12:30}
{ STC 0020 }}
\sessionTalk{ Almost sure convergence rates of adaptive increasingly rare Markov chain Monte Carlo }
{Daniel Rudolf}
{SS3-1}
\sessionTalk{ Approximation of vectors using adaptive randomized information }
{Marcin Wnuk}
{SS3-2}
\sessionTalk{ Upper and Lower Bounds for Pathwise Approximation of Scalar SDEs with Reflection }
{Klaus Ritter}
{SS3-3}



\clearpage

\begin{session}
    {Optimization under uncertainty}% [1] session title
    {Philipp A. Guth}% [2] organizer name
    {RICAM, Austrian Academy of Sciences}% [3] affiliations
    {philipp.guth@ricam.oeaw.ac.at}% [4] email
    {Vesa Kaarnioja}% [5] organizer name. Leave unchanged if there is no second organizer, otherwise fill in accordingly.
    {University of Potsdam}% [6] affiliations. Leave unchanged if there is no second organizer, otherwise fill in accordingly.
    {vesa.kaarnioja@iki.fi}% [7] email. Leave unchanged if there is no second organizer, otherwise fill in accordingly.
    {SS13}% [8] session id
    {\thirdorganizer{Claudia Schillings}{Free University of Berlin}{c.schillings@fu-berlin.de}}% [9] third organizer, if any
  Large-scale optimization problems based on partial differential equation models typically involve a number of uncertainties: for example, the material parameters, domain shape or sensor locations used to collect the measurements may not be perfectly known. The quantification of these uncertainties leads to challenging high-dimensional integration problems, which can be tackled efficiently using, e.g., multilevel Monte Carlo or quasi-Monte Carlo methods. The intersection of optimization and uncertainty quantification is an actively developing field of research, and this session aims to cover some recent advances in the computational and theoretical treatment of these topics.
 \end{session}

\sessionPart{}% [1] part
{\hfill\timeslot{Monday, August 19, 2024 -- Morning}
{10:30}{12:30}
{ STC 0040 }}
\sessionTalk{ Quasi-Monte Carlo methods for optimal feedback control problems under uncertainty }
{Philipp Guth}
{SS13-1}
\sessionTalk{ Shape optimization under constraints on the probability of a quadratic functional to exceed a given threshold }
{Helmut Harbrecht}
{SS13-2}
\sessionTalk{ Deep learning methods for stochastic Galerkin approximations of random elliptic PDEs }
{Fabio Musco}
{SS13-3}
\sessionTalk{ Randomized quasi-Monte Carlo for nested integration }
{Arved Bartuska}
{SS13-4}



\clearpage

\begin{session}
 {Efficient Bayesian Surrogate Modeling, Part I}% [1] session title
 {Aleksei Sorokin}% [2] organizer name
 {Illinois Institute of Technology}% [3] affiliations
 {asorokin@hawk.iit.edu}% [4] email
 {Pieterjan Robbe}% [5] organizer name. Leave unchanged if there is no second organizer, otherwise fill in accordingly.
 {Sandia National Laboratories}% [6] affiliations. Leave unchanged if there is no second organizer, otherwise fill in accordingly.
 {pmrobbe@sandia.gov}% [7] email. Leave unchanged if there is no second organizer, otherwise fill in accordingly.
 {SS9}% [8] session id
 {}% [9] third organizer, if any
 Common tasks in stochastic modeling include model calibration and sensitivity analysis. These tasks typically require many model evaluations, which can be prohibitively expensive in case model evaluations are costly. This has motivated the development of surrogate models, which are fit offline on a limited budget and then enable rapid online evaluations for predictive purposes. An important decision is where to evaluate the model in order to maximize information captured by the surrogate. While Monte Carlo points are a conventional choice, their independent nature often leads to sampling in locations of little value to the surrogate. In contrast, dependent structures, such as quasi-random (low discrepancy) points or Bayesian optimal experimental designs, have proven to produce more reliable surrogate models. This session will discuss some of the recent developments in these sampling techniques, and will bring together researchers from both communities to explore collaborations.
\end{session}

\sessionTalk{ Efficient surrogate construction for response surfaces with steep gradients }
{Pieterjan Robbe}
{SS9-1}
\sessionTalk{ Constraint active search as an alternative to multiobjective optimization }
{Michael McCourt}
{SS9-2}
\sessionTalk{ Diverse Expected Improvement (DEI): Diverse Optimization of Expensive Black-box Simulators for Internal Combustion Engine Control }
{John Miller}
{SS9-3}
\sessionTalk{ Efficient surrogate construction for response surfaces with steep gradients }
{Pieterjan Robbe}
{SS9-1}
\sessionTalk{ Constraint active search as an alternative to multiobjective optimization }
{Michael McCourt}
{SS9-2}
\sessionTalk{ Diverse Expected Improvement (DEI): Diverse Optimization of Expensive Black-box Simulators for Internal Combustion Engine Control }
{John Miller}
{SS9-3}
\sessionTalk{ Efficient surrogate construction for response surfaces with steep gradients }
{Pieterjan Robbe}
{SS9-1}
\sessionTalk{ Constraint active search as an alternative to multiobjective optimization }
{Michael McCourt}
{SS9-2}
\sessionTalk{ Diverse Expected Improvement (DEI): Diverse Optimization of Expensive Black-box Simulators for Internal Combustion Engine Control }
{John Miller}
{SS9-3}
\sessionTalk{ Efficient surrogate construction for response surfaces with steep gradients }
{Pieterjan Robbe}
{SS9-1}
\sessionTalk{ Constraint active search as an alternative to multiobjective optimization }
{Michael McCourt}
{SS9-2}
\sessionTalk{ Diverse Expected Improvement (DEI): Diverse Optimization of Expensive Black-box Simulators for Internal Combustion Engine Control }
{John Miller}
{SS9-3}
\sessionTalk{ Efficient surrogate construction for response surfaces with steep gradients }
{Pieterjan Robbe}
{SS9-1}
\sessionTalk{ Constraint active search as an alternative to multiobjective optimization }
{Michael McCourt}
{SS9-2}
\sessionTalk{ Diverse Expected Improvement (DEI): Diverse Optimization of Expensive Black-box Simulators for Internal Combustion Engine Control }
{John Miller}
{SS9-3}


%\sessionTalk{ Fast Gaussian Process Regression for Smooth Functions using Lattice and Digital Sequences with Matching Kernels }
{Aleksei Sorokin}
{SS10-1}
\sessionTalk{ Bayesian Optimal Experimental Design for Surrogate Model Training }
{Xun Huan}
{SS10-2}
\sessionTalk{ Rare events and their optimization }
{Vishwas Rao}
{SS10-3}



\clearpage


\begin{session}
 {Stochastic Computation and Complexity, Part II: Approximation of SDEs under non-standard assumptions}% [1] session title
 {Stefan Heinrich}% [2] organizer name
 {RPTU Kaiserslautern-Landau}% [3] affiliations
 {heinrich@informatik.uni-kl.de}% [4] email
 {}% [5] organizer name. Leave unchanged if there is no second organizer, otherwise fill in accordingly.
 {}% [6] affiliations. Leave unchanged if there is no second organizer, otherwise fill in accordingly.
 {}% [7] email. Leave unchanged if there is no second organizer, otherwise fill in accordingly.
 {SS4}
 %%{SP04}% [8] session id
 {}% [9] third organizer, if any
The session is devoted to algorithms and complexity for
\begin{itemize}[itemsep=0pt,topsep=0pt]
 \item quadrature and strong approximation of SDEs and SPDEs, in particular under nonstandard assumptions,

 \item high and infinite dimensional integration and approximation, and

 \item stochastic optimization and neural networks,
\end{itemize}
including connections to functional analysis and stochastic analysis.
\end{session}

\sessionPart{}% [1] part
{\hfill\timeslot{Monday, August 19, 2024 -- Afternoon}
{15:30}{17:30}
{ STC 0020 }}
\sessionTalk{ Milstein-type methods for strong approximation of systems of SDEs with a discontinuous drift coefficient }
{Christopher Rauhögger}
{SS4-1}
\sessionTalk{ On efficient approximation of SDEs driven by countably dimensional Wiener process }
{Łukasz Stepien}
{SS4-2}
\sessionTalk{ On optimal error rates for strong approximation of SDEs with a Hölder-continuous drift coefficient }
{Simon Ellinger}
{SS4-3}




\clearpage

\begin{session}
 {Efficient Bayesian Surrogate Modeling, Part II}% [1] session title
 {Aleksei Sorokin}% [2] organizer name
 {Illinois Institute of Technology}% [3] affiliations
 {asorokin@hawk.iit.edu}% [4] email
 {Pieterjan Robbe}% [5] organizer name. Leave unchanged if there is no second organizer, otherwise fill in accordingly.
 {Sandia National Laboratories}% [6] affiliations. Leave unchanged if there is no second organizer, otherwise fill in accordingly.
 {pmrobbe@sandia.gov}% [7] email. Leave unchanged if there is no second organizer, otherwise fill in accordingly.
 {SS10}% [8] session id
 {}% [9] third organizer, if any
 Common tasks in stochastic modeling include model calibration and sensitivity analysis. These tasks typically require many model evaluations, which can be prohibitively expensive in case model evaluations are costly. This has motivated the development of surrogate models, which are fit offline on a limited budget and then enable rapid online evaluations for predictive purposes. An important decision is where to evaluate the model in order to maximize information captured by the surrogate. While Monte Carlo points are a conventional choice, their independent nature often leads to sampling in locations of little value to the surrogate. In contrast, dependent structures, such as quasi-random (low discrepancy) points or Bayesian optimal experimental designs, have proven to produce more reliable surrogate models. This session will discuss some of the recent developments in these sampling techniques, and will bring together researchers from both communities to explore collaborations.
\end{session}

%\sessionTalk{ Efficient surrogate construction for response surfaces with steep gradients }
{Pieterjan Robbe}
{SS9-1}
\sessionTalk{ Constraint active search as an alternative to multiobjective optimization }
{Michael McCourt}
{SS9-2}
\sessionTalk{ Diverse Expected Improvement (DEI): Diverse Optimization of Expensive Black-box Simulators for Internal Combustion Engine Control }
{John Miller}
{SS9-3}
\sessionTalk{ Efficient surrogate construction for response surfaces with steep gradients }
{Pieterjan Robbe}
{SS9-1}
\sessionTalk{ Constraint active search as an alternative to multiobjective optimization }
{Michael McCourt}
{SS9-2}
\sessionTalk{ Diverse Expected Improvement (DEI): Diverse Optimization of Expensive Black-box Simulators for Internal Combustion Engine Control }
{John Miller}
{SS9-3}
\sessionTalk{ Efficient surrogate construction for response surfaces with steep gradients }
{Pieterjan Robbe}
{SS9-1}
\sessionTalk{ Constraint active search as an alternative to multiobjective optimization }
{Michael McCourt}
{SS9-2}
\sessionTalk{ Diverse Expected Improvement (DEI): Diverse Optimization of Expensive Black-box Simulators for Internal Combustion Engine Control }
{John Miller}
{SS9-3}
\sessionTalk{ Efficient surrogate construction for response surfaces with steep gradients }
{Pieterjan Robbe}
{SS9-1}
\sessionTalk{ Constraint active search as an alternative to multiobjective optimization }
{Michael McCourt}
{SS9-2}
\sessionTalk{ Diverse Expected Improvement (DEI): Diverse Optimization of Expensive Black-box Simulators for Internal Combustion Engine Control }
{John Miller}
{SS9-3}
\sessionTalk{ Efficient surrogate construction for response surfaces with steep gradients }
{Pieterjan Robbe}
{SS9-1}
\sessionTalk{ Constraint active search as an alternative to multiobjective optimization }
{Michael McCourt}
{SS9-2}
\sessionTalk{ Diverse Expected Improvement (DEI): Diverse Optimization of Expensive Black-box Simulators for Internal Combustion Engine Control }
{John Miller}
{SS9-3}


\sessionTalk{ Fast Gaussian Process Regression for Smooth Functions using Lattice and Digital Sequences with Matching Kernels }
{Aleksei Sorokin}
{SS10-1}
\sessionTalk{ Bayesian Optimal Experimental Design for Surrogate Model Training }
{Xun Huan}
{SS10-2}
\sessionTalk{ Rare events and their optimization }
{Vishwas Rao}
{SS10-3}


\clearpage
\begin{session}
 {Variance reduction techniques for rare events}% [1] session title
 {Nadhir Ben Rached}% [2] organizer name
 {University of Leeds}% [3] affiliations
 {n.benrached@leeds.ac.uk}% [4] email
 {Ra\'{u}l Tempone}% [5] organizer name. Leave unchanged if there is no second organizer, otherwise fill in accordingly.
 {RWTH Aachen University and KAUST}% [6] affiliations. Leave unchanged if there is no second organizer, otherwise fill in accordingly.
 {tempone@uq.rwth-aachen.de}% [7] email. Leave unchanged if there is no second organizer, otherwise fill in accordingly.
 {SS20}% [8] session id
 {\thirdorganizer{Shyam Mohan Subbiah Pillai}{RWTH Aachen University}{subbiah@uq.rwth-aachen.de}}% [9] third organizer, if any
Rare events are events with small probabilities, but their occurrences are critical in many real-life applications. The problem of estimating rare event probabilities is encountered in various engineering applications (finance, wireless communications, system reliability, Biology, etc.). Naive Monte Carlo simulations are, in this case, substantially expensive. This session focuses on methods belonging to the class of variance reduction techniques. These alternative methods deliver, when appropriately used, accurate estimates with a substantial amount of variance reduction compared to the naive Monte Carlo estimator.
\end{session}

\sessionTalk{ Importance Sampling Methods with Stochastic Differential Equations for the Estimation of the Right Tail of the CCDF of the Fade Duration }
{Eya Ben Amar}
{SS20-1}
\sessionTalk{ Importance sampling via stochastic optimal control for rare events associated with McKean-Vlasov equation }
{Shyam Mohan Subbiah Pillai}
{SS20-2}
\sessionTalk{ Multilevel reliability analysis: application to a flood risk estimation }
{Romain Espoeys}
{SS20-3}
\sessionTalk{ Importance Sampling Methods with Stochastic Differential Equations for the Estimation of the Right Tail of the CCDF of the Fade Duration }
{Eya Ben Amar}
{SS20-1}
\sessionTalk{ Importance sampling via stochastic optimal control for rare events associated with McKean-Vlasov equation }
{Shyam Mohan Subbiah Pillai}
{SS20-2}
\sessionTalk{ Multilevel reliability analysis: application to a flood risk estimation }
{Romain Espoeys}
{SS20-3}
\sessionTalk{ Importance Sampling Methods with Stochastic Differential Equations for the Estimation of the Right Tail of the CCDF of the Fade Duration }
{Eya Ben Amar}
{SS20-1}
\sessionTalk{ Importance sampling via stochastic optimal control for rare events associated with McKean-Vlasov equation }
{Shyam Mohan Subbiah Pillai}
{SS20-2}
\sessionTalk{ Multilevel reliability analysis: application to a flood risk estimation }
{Romain Espoeys}
{SS20-3}
\sessionTalk{ Importance Sampling Methods with Stochastic Differential Equations for the Estimation of the Right Tail of the CCDF of the Fade Duration }
{Eya Ben Amar}
{SS20-1}
\sessionTalk{ Importance sampling via stochastic optimal control for rare events associated with McKean-Vlasov equation }
{Shyam Mohan Subbiah Pillai}
{SS20-2}
\sessionTalk{ Multilevel reliability analysis: application to a flood risk estimation }
{Romain Espoeys}
{SS20-3}
\sessionTalk{ Importance Sampling Methods with Stochastic Differential Equations for the Estimation of the Right Tail of the CCDF of the Fade Duration }
{Eya Ben Amar}
{SS20-1}
\sessionTalk{ Importance sampling via stochastic optimal control for rare events associated with McKean-Vlasov equation }
{Shyam Mohan Subbiah Pillai}
{SS20-2}
\sessionTalk{ Multilevel reliability analysis: application to a flood risk estimation }
{Romain Espoeys}
{SS20-3}



\clearpage

\begin{session}
 {Stochastic Computation and Complexity, Part III: Approximation of SDEs under non-standard assumptions}% [1] session title
 {Thomas M\"uller-Gronbach}% [2] organizer name
 {University of Passau}% [3] affiliations
 {Thomas.Mueller-Gronbach@uni-passau.de}% [4] email
 {}% [5] organizer name. Leave unchanged if there is no second organizer, otherwise fill in accordingly.
 {}% [6] affiliations. Leave unchanged if there is no second organizer, otherwise fill in accordingly.
 {}% [7] email. Leave unchanged if there is no second organizer, otherwise fill in accordingly.
 {SS5}
 %{SP05}% [8] session id
 {}% [9] third organizer, if any
The session is devoted to algorithms and complexity for
\begin{itemize}[itemsep=0pt,topsep=0pt]
 \item quadrature and strong approximation of SDEs and SPDEs, in particular under nonstandard assumptions,

 \item high and infinite dimensional integration and approximation, and

 \item stochastic optimization and neural networks,
\end{itemize}
including connections to functional analysis and stochastic analysis.
\end{session}

\sessionPart{}% [1] part
{\hfill\timeslot{Tuesday, August 20, 2024 -- Morning}
{10:30}{12:30}
{ STC 0020 }}
\sessionTalk{ Integration and approximation of functions by Monte Carlo and quantum methods }
{Stefan Heinrich}
{SS5-1}
\sessionTalk{ Sampling recovery and sharp norm estimates of projection operators }
{Kateryna Pozharska}
{SS5-2}
\sessionTalk{ The L2-discrepancy of latin hypercubes }
{Nicolas Nagel}
{SS5-3}



\clearpage

\begin{session}
 {Efficient methods for uncertainty quantification in differential equations, Part I}% [1] session title
 {Anastasia Istratuca}% [2] organizer name
 {University of Edinburgh, Heriot-Watt University}% [3] affiliations
 {a.istratuca@sms.ed.ac.uk}% [4] email
 {Aretha Teckentrup}% [5] organizer name. Leave unchanged if there is no second organizer, otherwise fill in accordingly.
 {University of Edinburgh}% [6] affiliations. Leave unchanged if there is no second organizer, otherwise fill in accordingly.
 {a.teckentrup@ed.ac.uk}% [7] email. Leave unchanged if there is no second organizer, otherwise fill in accordingly.
 {SS18}% [8] session id
 {}% [9] third organizer, if any
 One of the most common approaches to modelling physical phenomena consists of ordinary and partial differential equations, which allow for computer simulations through the use of modern numerical solvers. These models encompass parameters that often have to be measured or inferred from data. To account for error measurements and scarce availability of the data, we express our uncertainty about the parameters by associating, for example, a probability distribution to them. This mini-symposium focuses on recent advances in algorithms for quantifying the uncertainty in such models.
\end{session}

\sessionPart{}% [1] part
{\hfill\timeslot{Tuesday, August 20, 2024 -- Morning}
{10:30}{12:30}
{ STC 0040 }}
\sessionTalk{ Nonparametric Inference for Diffusion Processes }
{Sebastian Krumscheid}
{SS18-1}
\sessionTalk{ History Matching and Gaussian Process Emulation in High Dimensions }
{Elliot Addy}
{SS18-2}
\sessionTalk{ A Universal Lattice-based Algorithm for Multivariate Function Approximation in Uncertainty Quantification }
{Weiwen Mo}
{SS18-3}
\sessionTalk{ Revisiting high-dimensional kernel approximation of parametric PDEs over lattice point sets }
{Vesa Kaarnioja}
{SS18-4}


%\sessionTalk{ Bayesian shape inversion in acoustic and electromagnetic scattering }
{Laura Scarabosio}
{SS19-1}
\sessionTalk{ The Quasi Continuous-Level Monte Carlo Method and its Applications }
{Andrea Barth}
{SS19-2}
\sessionTalk{ Multilevel Monte Carlo Methods with Smoothing }
{Aretha Teckentrup}
{SS19-3}
\sessionTalk{ Bayesian shape inversion in acoustic and electromagnetic scattering }
{Laura Scarabosio}
{SS19-1}
\sessionTalk{ The Quasi Continuous-Level Monte Carlo Method and its Applications }
{Andrea Barth}
{SS19-2}
\sessionTalk{ Multilevel Monte Carlo Methods with Smoothing }
{Aretha Teckentrup}
{SS19-3}
\sessionTalk{ Bayesian shape inversion in acoustic and electromagnetic scattering }
{Laura Scarabosio}
{SS19-1}
\sessionTalk{ The Quasi Continuous-Level Monte Carlo Method and its Applications }
{Andrea Barth}
{SS19-2}
\sessionTalk{ Multilevel Monte Carlo Methods with Smoothing }
{Aretha Teckentrup}
{SS19-3}
\sessionTalk{ Bayesian shape inversion in acoustic and electromagnetic scattering }
{Laura Scarabosio}
{SS19-1}
\sessionTalk{ The Quasi Continuous-Level Monte Carlo Method and its Applications }
{Andrea Barth}
{SS19-2}
\sessionTalk{ Multilevel Monte Carlo Methods with Smoothing }
{Aretha Teckentrup}
{SS19-3}
\sessionTalk{ Bayesian shape inversion in acoustic and electromagnetic scattering }
{Laura Scarabosio}
{SS19-1}
\sessionTalk{ The Quasi Continuous-Level Monte Carlo Method and its Applications }
{Andrea Barth}
{SS19-2}
\sessionTalk{ Multilevel Monte Carlo Methods with Smoothing }
{Aretha Teckentrup}
{SS19-3}




\clearpage



\begin{session}
 {Recent Advances in QMC Methods for Computational Finance and Financial Risk Management}% [1] session title
 {Chiheb Ben Hammouda}% [2] organizer name
 {Utrecht University}% [3] affiliations
 {c.benhammouda@uu.nl}% [4] email
 {Ra\'{u}l Tempone}% [5] organizer name. Leave unchanged if there is no second organizer, otherwise fill in accordingly.
 {RWTH Aachen University, King Abdullah University of Science and Technology}% [6] affiliations. Leave unchanged if there is no second organizer, otherwise fill in accordingly.
 {rtempone@gmail.com}% [7] email. Leave unchanged if there is no second organizer, otherwise fill in accordingly.
 {SS14}% [8] session id
 {}% [9] third organizer, if any
 The session is about recent numerical and theoretical advances in quasi-Monte Carlo (QMC) methods to address different challenges in computational finance and Risk management. Challenges range from pricing high-dimensional financial derivatives, computing sensitivities, and efficiently estimating nested expectations arising in financial risk estimation.
\end{session}

\sessionTalk{ Quasi-Monte Carlo for Efficient Fourier Pricing of Multi-Asset Options }
{Michael Samet}
{SS14-1}
\sessionTalk{ Conditional Quasi-Monte Carlo with Active Subspaces }
{Sifan Liu}
{SS14-2}
\sessionTalk{ Application of Randomised QMC for Option Pricing and Greeks }
{Sergei Kucherenko}
{SS14-3}
\sessionTalk{ Estimating quantile and expected shortfall via Hilbert space-filling curve sampling with confidence intervals }
{Zhijian He}
{SS14-4}
\sessionTalk{ Quasi-Monte Carlo for Efficient Fourier Pricing of Multi-Asset Options }
{Michael Samet}
{SS14-1}
\sessionTalk{ Conditional Quasi-Monte Carlo with Active Subspaces }
{Sifan Liu}
{SS14-2}
\sessionTalk{ Application of Randomised QMC for Option Pricing and Greeks }
{Sergei Kucherenko}
{SS14-3}
\sessionTalk{ Estimating quantile and expected shortfall via Hilbert space-filling curve sampling with confidence intervals }
{Zhijian He}
{SS14-4}
\sessionTalk{ Quasi-Monte Carlo for Efficient Fourier Pricing of Multi-Asset Options }
{Michael Samet}
{SS14-1}
\sessionTalk{ Conditional Quasi-Monte Carlo with Active Subspaces }
{Sifan Liu}
{SS14-2}
\sessionTalk{ Application of Randomised QMC for Option Pricing and Greeks }
{Sergei Kucherenko}
{SS14-3}
\sessionTalk{ Estimating quantile and expected shortfall via Hilbert space-filling curve sampling with confidence intervals }
{Zhijian He}
{SS14-4}
\sessionTalk{ Quasi-Monte Carlo for Efficient Fourier Pricing of Multi-Asset Options }
{Michael Samet}
{SS14-1}
\sessionTalk{ Conditional Quasi-Monte Carlo with Active Subspaces }
{Sifan Liu}
{SS14-2}
\sessionTalk{ Application of Randomised QMC for Option Pricing and Greeks }
{Sergei Kucherenko}
{SS14-3}
\sessionTalk{ Estimating quantile and expected shortfall via Hilbert space-filling curve sampling with confidence intervals }
{Zhijian He}
{SS14-4}
\sessionTalk{ Quasi-Monte Carlo for Efficient Fourier Pricing of Multi-Asset Options }
{Michael Samet}
{SS14-1}
\sessionTalk{ Conditional Quasi-Monte Carlo with Active Subspaces }
{Sifan Liu}
{SS14-2}
\sessionTalk{ Application of Randomised QMC for Option Pricing and Greeks }
{Sergei Kucherenko}
{SS14-3}
\sessionTalk{ Estimating quantile and expected shortfall via Hilbert space-filling curve sampling with confidence intervals }
{Zhijian He}
{SS14-4}




\clearpage

\begin{session}
 {Efficient methods for uncertainty quantification in differential equations, Part II}% [1] session title
 {Anastasia Istratuca}% [2] organizer name
 {University of Edinburgh, Heriot-Watt University}% [3] affiliations
 {a.istratuca@sms.ed.ac.uk}% [4] email
 {Aretha Teckentrup}% [5] organizer name. Leave unchanged if there is no second organizer, otherwise fill in accordingly.
 {University of Edinburgh}% [6] affiliations. Leave unchanged if there is no second organizer, otherwise fill in accordingly.
 {a.teckentrup@ed.ac.uk}% [7] email. Leave unchanged if there is no second organizer, otherwise fill in accordingly.
 {SS19}% [8] session id
 {}% [9] third organizer, if any
 One of the most common approaches to modelling physical phenomena consists of ordinary and partial differential equations, which allow for computer simulations through the use of modern numerical solvers. These models encompass parameters that often have to be measured or inferred from data. To account for error measurements and scarce availability of the data, we express our uncertainty about the parameters by associating, for example, a probability distribution to them. This mini-symposium focuses on recent advances in algorithms for quantifying the uncertainty in such models.
\end{session}

%\sessionPart{}% [1] part
{\hfill\timeslot{Tuesday, August 20, 2024 -- Morning}
{10:30}{12:30}
{ STC 0040 }}
\sessionTalk{ Nonparametric Inference for Diffusion Processes }
{Sebastian Krumscheid}
{SS18-1}
\sessionTalk{ History Matching and Gaussian Process Emulation in High Dimensions }
{Elliot Addy}
{SS18-2}
\sessionTalk{ A Universal Lattice-based Algorithm for Multivariate Function Approximation in Uncertainty Quantification }
{Weiwen Mo}
{SS18-3}
\sessionTalk{ Revisiting high-dimensional kernel approximation of parametric PDEs over lattice point sets }
{Vesa Kaarnioja}
{SS18-4}


\sessionTalk{ Bayesian shape inversion in acoustic and electromagnetic scattering }
{Laura Scarabosio}
{SS19-1}
\sessionTalk{ The Quasi Continuous-Level Monte Carlo Method and its Applications }
{Andrea Barth}
{SS19-2}
\sessionTalk{ Multilevel Monte Carlo Methods with Smoothing }
{Aretha Teckentrup}
{SS19-3}
\sessionTalk{ Bayesian shape inversion in acoustic and electromagnetic scattering }
{Laura Scarabosio}
{SS19-1}
\sessionTalk{ The Quasi Continuous-Level Monte Carlo Method and its Applications }
{Andrea Barth}
{SS19-2}
\sessionTalk{ Multilevel Monte Carlo Methods with Smoothing }
{Aretha Teckentrup}
{SS19-3}
\sessionTalk{ Bayesian shape inversion in acoustic and electromagnetic scattering }
{Laura Scarabosio}
{SS19-1}
\sessionTalk{ The Quasi Continuous-Level Monte Carlo Method and its Applications }
{Andrea Barth}
{SS19-2}
\sessionTalk{ Multilevel Monte Carlo Methods with Smoothing }
{Aretha Teckentrup}
{SS19-3}
\sessionTalk{ Bayesian shape inversion in acoustic and electromagnetic scattering }
{Laura Scarabosio}
{SS19-1}
\sessionTalk{ The Quasi Continuous-Level Monte Carlo Method and its Applications }
{Andrea Barth}
{SS19-2}
\sessionTalk{ Multilevel Monte Carlo Methods with Smoothing }
{Aretha Teckentrup}
{SS19-3}
\sessionTalk{ Bayesian shape inversion in acoustic and electromagnetic scattering }
{Laura Scarabosio}
{SS19-1}
\sessionTalk{ The Quasi Continuous-Level Monte Carlo Method and its Applications }
{Andrea Barth}
{SS19-2}
\sessionTalk{ Multilevel Monte Carlo Methods with Smoothing }
{Aretha Teckentrup}
{SS19-3}


\clearpage

\begin{session}
 {Learning to Solve Related Integrals}% [1] session title
 {Chris. J. Oates}% [2] organizer name
 {Newcastle University, UK}% [3] affiliations
 {chris.oates@ncl.ac.uk}% [4] email
 {}% [5] organizer name. Leave unchanged if there is no second organizer, otherwise fill in accordingly.
 {}% [6] affiliations. Leave unchanged if there is no second organizer, otherwise fill in accordingly.
 {}% [7] email. Leave unchanged if there is no second organizer, otherwise fill in accordingly.
 {SS2}
 %%NK on May 7 {SP02}% [8] session id
 {}% [9] third organizer, if any
 The standard perspective on numerical analysis deals with solving individual numerical tasks, but in practice the experience gained from using numerical methods to solve related problems provides valuable insight into their performance, which can shape how and when numerical methods are used.
 Developments at the intersection of probability, statistics, and numerical analysis seek to leverage experience to improve performance on subsequent numerical tasks as they are encountered.
 This session will shine a light on emerging methodology for the solution of related integration problems, arising in areas of research that include sensitivity analysis, computational finance, the solution of partial differential equations, decision-making under uncertainty, and diffusion-based generative modelling.
 Fran\c{c}ois-Xavier Briol from University College London will present a probabilistic approach to estimating related conditional expectations, which operates by sharing statistical information regarding the integrand.
 Jon Cockayne from the University of Southampton will present a novel statistical approach to solving related linear systems of equations, such as occur when integrating a partial differential equation that is parameter-dependent.
 Zheyang Shen from Newcastle University will present a novel perspective on diffusion-based generative modelling, which casts the problem of generating realistic image data as the estimation of related kernel mean embeddings in a reproducing kernel Hilbert space framework.
\end{session}

\sessionTalk{ Estimating parametric expectations through Bayesian quadrature }
{Francois-Xavier Briol}
{SS2-1}
\sessionTalk{ Learning to Solve Related Linear Systems }
{Jon Cockayne}
{SS2-2}
\sessionTalk{ Demystifying diffusion models via their Markov semigroups }
{Zheyang Shen}
{SS2-3}
\sessionTalk{ Estimating parametric expectations through Bayesian quadrature }
{Francois-Xavier Briol}
{SS2-1}
\sessionTalk{ Learning to Solve Related Linear Systems }
{Jon Cockayne}
{SS2-2}
\sessionTalk{ Demystifying diffusion models via their Markov semigroups }
{Zheyang Shen}
{SS2-3}
\sessionTalk{Estimating parametric expectations through Bayesian quadrature}
{Francois-Xavier Briol}
{SS2-1}
\sessionTalk{Learning to Solve Related Linear Systems}
{Jon Cockayne}
{SS2-2}
\sessionTalk{Demystifying diffusion models via their Markov semigroups}
{Zheyang Shen}
{SS2-3}
\sessionTalk{ Estimating parametric expectations through Bayesian quadrature }
{Francois-Xavier Briol}
{SS2-1}
\sessionTalk{ Learning to Solve Related Linear Systems }
{Jon Cockayne}
{SS2-2}
\sessionTalk{ Demystifying diffusion models via their Markov semigroups }
{Zheyang Shen}
{SS2-3}



%%% NK commenting out this and all subsequent similar sections on May 7

\clearpage


\begin{session}
 {Function recovery and discretization problems, Part I}% [1] session title
 {David Krieg}% [2] organizer name
 {Institute of Analysis, Johannes Kepler University Linz, Austria}% [3] affiliations
 {david.krieg@jku.at}% [4] email
 {Kateryna Pozharska}% [5] organizer name. Leave unchanged if there is no second organizer, otherwise fill in accordingly.
 {Institute of Mathematics of NAS of Ukraine, Kyiv, Ukraine; \\ Faculty of Mathematics, Chemnitz University of Technology,
Germany}% [6] affiliations. Leave unchanged if there is no second organizer, otherwise fill in accordingly.
 {pozharska.k@gmail.com}% [7] email. Leave unchanged if there is no second organizer, otherwise fill in accordingly.
 {SS6}% [8] session id
 {}% [9] third organizer, if any
In this session, we would like to bring together experts who contributed to the theory of function recovery and related problems.
Recently, there has been much progress in understanding the power of different types of information (function values vs.\ linear measurements, optimal vs.\ random)
as well as different classes of algorithms (linear vs.\ nonlinear, random vs.\ deterministic, adaptive vs.\ nonadaptive), but also with regard to the error analysis for specific recovery schemes.
The session is concerned with these new developments, which also include the impact of a large dimension, discretization in function spaces and modern methods in data science.
\end{session}

\sessionTalk{ Optimal approximation of infinite-dimensional, Banach-valued, holomorphic functions from i.i.d. samples }
{Ben Adcock}
{SS6-1}
\sessionTalk{ Haar decompositions and Besov-type spaces }
{Winfried Sickel}
{SS6-2}
\sessionTalk{ Entropy numbers of finite-dimensional Lorentz space embeddings }
{Mathias Sonnleitner}
{SS6-3}
\sessionTalk{ Learning the solution of differential equations by sparse high-dimensional approximation }
{Fabian Taubert}
{SS6-4}
\sessionTalk{ Optimal approximation of infinite-dimensional, Banach-valued, holomorphic functions from i.i.d. samples }
{Ben Adcock}
{SS6-1}
\sessionTalk{ Haar decompositions and Besov-type spaces }
{Winfried Sickel}
{SS6-2}
\sessionTalk{ Entropy numbers of finite-dimensional Lorentz space embeddings }
{Mathias Sonnleitner}
{SS6-3}
\sessionTalk{ Learning the solution of differential equations by sparse high-dimensional approximation }
{Fabian Taubert}
{SS6-4}
\sessionTalk{Optimal approximation of infinite-dimensional, Banach-valued, holomorphic functions from i.i.d. samples}
{Ben Adcock}
{SS6-1}
\sessionTalk{Haar decompositions and Besov-type spaces}
{Winfried Sickel}
{SS6-2}
\sessionTalk{Entropy numbers of finite-dimensional Lorentz space embeddings}
{Mathias Sonnleitner}
{SS6-3}
\sessionTalk{Learning the solution of differential equations by sparse high-dimensional approximation}
{Fabian Taubert}
{SS6-4}
\sessionTalk{ Optimal approximation of infinite-dimensional, Banach-valued, holomorphic functions from i.i.d. samples }
{Ben Adcock}
{SS6-1}
\sessionTalk{ Haar decompositions and Besov-type spaces }
{Winfried Sickel}
{SS6-2}
\sessionTalk{ Entropy numbers of finite-dimensional Lorentz space embeddings }
{Mathias Sonnleitner}
{SS6-3}
\sessionTalk{ Learning the solution of differential equations by sparse high-dimensional approximation }
{Fabian Taubert}
{SS6-4}


%\sessionTalk{ Function reconstruction using determinantal sampling }
{Ayoub Belhadji}
{SS7-1}
\sessionTalk{ Sampling numbers of smoothness classes via $\ell^1$-minimization }
{Thomas Jahn}
{SS7-2}
\sessionTalk{ Constructive Sparsification of Finite Frames and Applications to Function Recovery }
{Tino Ullrich}
{SS7-3}
\sessionTalk{ Function reconstruction using determinantal sampling }
{Ayoub Belhadji}
{SS7-1}
\sessionTalk{ Sampling numbers of smoothness classes via $\ell^1$-minimization }
{Thomas Jahn}
{SS7-2}
\sessionTalk{ Constructive Sparsification of Finite Frames and Applications to Function Recovery }
{Tino Ullrich}
{SS7-3}
\sessionTalk{ Function reconstruction using determinantal sampling }
{Ayoub Belhadji}
{SS7-1}
\sessionTalk{ Sampling numbers of smoothness classes via $\ell^1$-minimization }
{Thomas Jahn}
{SS7-2}
\sessionTalk{ Constructive Sparsification of Finite Frames and Applications to Function Recovery }
{Tino Ullrich}
{SS7-3}
\sessionTalk{ Function reconstruction using determinantal sampling }
{Ayoub Belhadji}
{SS7-1}
\sessionTalk{ Sampling numbers of smoothness classes via $\ell^1$-minimization }
{Thomas Jahn}
{SS7-2}
\sessionTalk{ Constructive Sparsification of Finite Frames and Applications to Function Recovery }
{Tino Ullrich}
{SS7-3}



\clearpage


\begin{session}
 {Testing and analysis of pseudorandom number generators}% [1] session title
 {Emil Løvbak}% [2] organizer name
 {Karlsruhe Institute of Technology}% [3] affiliations
 {emil.loevbak@kit.edu}% [4] email
 {Michael Mascagni}% [5] organizer name. Leave unchanged if there is no second organizer, otherwise fill in accordingly.
 {Florida State University}% [6] affiliations. Leave unchanged if there is no second organizer, otherwise fill in accordingly.
 {mascagni@fsu.edu}% [7] email. Leave unchanged if there is no second organizer, otherwise fill in accordingly.
 {SS23}% [8] session id
 {}% [9] third organizer, if any
 Pseudorandom number generators are a core part of scientific computing, lying at the foundation of Monte Carlo methods. Over the history of the field, the quality of such generators has consistently been improved to produce streams of numbers that are hard to distinguish from truly random numbers. There are two approaches to quantify the randomness of a given generator. On the one hand, one can use mathematical techniques to determine the theoretical properties of the generator such as period length, uniformity, and sequence correlation. On the other hand, one can apply statistical benchmarks to empirically test the streams produced by a generator. This minisymposium aims to bring together researchers working on the design and testing of practical random number generators to exchange ideas on how to make use of these two complementary approaches in their evaluation.
\end{session}

\sessionTalk{ Machine Learning and Random Number Generation Testing }
{Michael Mascagni}
{SS23-1}
\sessionTalk{ A Redesigned C++ Library to Test the Lattice Structure of Linear Generators and Search for Good Ones }
{Pierre L'Ecuyer}
{SS23-2}
\sessionTalk{ On NIST's Standards on Random Numbers }
{Meltem Sonmez Turan}
{SS23-3}
\sessionTalk{ Acceleration of true orbit pseudorandom number generators using Newton's method }
{Asaki Saito}
{SS23-4}
\sessionTalk{ Machine Learning and Random Number Generation Testing }
{Michael Mascagni}
{SS23-1}
\sessionTalk{ A Redesigned C++ Library to Test the Lattice Structure of Linear Generators and Search for Good Ones }
{Pierre L'Ecuyer}
{SS23-2}
\sessionTalk{ On NIST's Standards on Random Numbers }
{Meltem Sonmez Turan}
{SS23-3}
\sessionTalk{ Acceleration of true orbit pseudorandom number generators using Newton's method }
{Asaki Saito}
{SS23-4}
\sessionTalk{Machine Learning and Random Number Generation Testing}
{Michael Mascagni}
{SS23-1}
\sessionTalk{A Redesigned C++ Library to Test the Lattice Structure of Linear Generators and Search for Good Ones}
{Pierre L'Ecuyer}
{SS23-2}
\sessionTalk{On NIST's Standards on Random Numbers}
{Meltem Sonmez Turan}
{SS23-3}
\sessionTalk{Acceleration of true orbit pseudorandom number generators using Newton's method}
{Asaki Saito}
{SS23-4}
\sessionTalk{ Machine Learning and Random Number Generation Testing }
{Michael Mascagni}
{SS23-1}
\sessionTalk{ A Redesigned C++ Library to Test the Lattice Structure of Linear Generators and Search for Good Ones }
{Pierre L'Ecuyer}
{SS23-2}
\sessionTalk{ On NIST's Standards on Random Numbers }
{Meltem Sonmez Turan}
{SS23-3}
\sessionTalk{ Acceleration of true orbit pseudorandom number generators using Newton's method }
{Asaki Saito}
{SS23-4}



\clearpage

\begin{session}
 {Function recovery and discretization problems, Part II}% [1] session title
 {David Krieg}% [2] organizer name
 {Institute of Analysis, Johannes Kepler University Linz, Austria}% [3] affiliations
 {david.krieg@jku.at}% [4] email
 {Kateryna Pozharska}% [5] organizer name. Leave unchanged if there is no second organizer, otherwise fill in accordingly.
 {Institute of Mathematics of NAS of Ukraine, Kyiv, Ukraine; \\ Faculty of Mathematics, Chemnitz University of Technology,
Germany}% [6] affiliations. Leave unchanged if there is no second organizer, otherwise fill in accordingly.
 {pozharska.k@gmail.com}% [7] email. Leave unchanged if there is no second organizer, otherwise fill in accordingly.
 {SS7}% [8] session id
 {}% [9] third organizer, if any
In this session, we would like to bring together experts who contributed to the theory of function recovery and related problems.
Recently, there has been much progress in understanding the power of different types of information (function values vs.\ linear measurements, optimal vs.\ random)
as well as different classes of algorithms (linear vs.\ nonlinear, random vs.\ deterministic, adaptive vs.\ nonadaptive), but also with regard to the error analysis for specific recovery schemes.
The session is concerned with these new developments, which also include the impact of a large dimension, discretization in function spaces and modern methods in data science.
\end{session}

\sessionTalk{ Function reconstruction using determinantal sampling }
{Ayoub Belhadji}
{SS7-1}
\sessionTalk{ Sampling numbers of smoothness classes via $\ell^1$-minimization }
{Thomas Jahn}
{SS7-2}
\sessionTalk{ Constructive Sparsification of Finite Frames and Applications to Function Recovery }
{Tino Ullrich}
{SS7-3}
\sessionTalk{ Function reconstruction using determinantal sampling }
{Ayoub Belhadji}
{SS7-1}
\sessionTalk{ Sampling numbers of smoothness classes via $\ell^1$-minimization }
{Thomas Jahn}
{SS7-2}
\sessionTalk{ Constructive Sparsification of Finite Frames and Applications to Function Recovery }
{Tino Ullrich}
{SS7-3}
\sessionTalk{ Function reconstruction using determinantal sampling }
{Ayoub Belhadji}
{SS7-1}
\sessionTalk{ Sampling numbers of smoothness classes via $\ell^1$-minimization }
{Thomas Jahn}
{SS7-2}
\sessionTalk{ Constructive Sparsification of Finite Frames and Applications to Function Recovery }
{Tino Ullrich}
{SS7-3}
\sessionTalk{ Function reconstruction using determinantal sampling }
{Ayoub Belhadji}
{SS7-1}
\sessionTalk{ Sampling numbers of smoothness classes via $\ell^1$-minimization }
{Thomas Jahn}
{SS7-2}
\sessionTalk{ Constructive Sparsification of Finite Frames and Applications to Function Recovery }
{Tino Ullrich}
{SS7-3}








\begin{session}
 {Universality in QMC and related algorithms}% [1] session title
 {Peter Kritzer}% [2] organizer name
 {RICAM, Austrian Academy of Sciences}% [3] affiliations
 {peter.kritzer@oeaw.ac.at}% [4] email
 {}% [5] organizer name. Leave unchanged if there is no second organizer, otherwise fill in accordingly.
 {}% [6] affiliations. Leave unchanged if there is no second organizer, otherwise fill in accordingly.
 {}% [7] email. Leave unchanged if there is no second organizer, otherwise fill in accordingly.
 {SS8}% [8] session id
 {}% [9] third organizer, if any
 In the literature on QMC and related methods, it is often the case that one can tailor an algorithm to a specific problem, usually depending on a certain (fixed) choice
 of problem parameters such as smoothness parameters or coordinate weights. This may
 have the advantage that one obtains an excellent algorithm for this particular problem,
 but the obvious downside is that it is not clear whether the same algorithm could be applied in other settings, e.g., when some of the parameters change. There have been recent attempts to make QMC and related algorithms more universal, and a number of interesting open questions remain. This special session brings together four speakers who have recently contributed to this aspect of multivariate algorithms.
\end{session}

\sessionPart{}% [1] part
{\hfill\timeslot{Wednesday, August 21, 2024 -- Afternoon}
{15:30}{17:30}
{ STC 0040 }}
\sessionTalk{ Explicit constructions of point sets whose worst-case error in certain spaces depends polynomially on the dimension }
{Josef Dick}
{SS8-1}
\sessionTalk{ Quasi-Monte Carlo Kernel Density Estimation }
{Fred Hickernell}
{SS8-2}
\sessionTalk{ A universal median quasi-Monte Carlo integration }
{Kosuke Suzuki}
{SS8-3}
\sessionTalk{ Using Kronecker point sets for function approximation in the Korobov space }
{Laurence Wilkes}
{SS8-4}



\clearpage


\begin{session}
 {Multilevel methods for SDEs and SPDEs, Part I}% [1] session title
 {Mike Giles}% [2] organizer name
 {University of Oxford}% [3] affiliations
 {mike.giles@maths.ox.ac.uk}% [4] email
 {}% [5] organizer name. Leave unchanged if there is no second organizer, otherwise fill in accordingly.
 {}% [6] affiliations. Leave unchanged if there is no second organizer, otherwise fill in accordingly.
 {}% [7] email. Leave unchanged if there is no second organizer, otherwise fill in accordingly.
 {SS21}% [8] session id
 {}% [9] third organizer, if any
 Speakers in this session will present and analyse multilevel algorithms for an interesting variety of applications, including chaotic SDEs, stochastic PDEs and kinetic particle models.
\end{session}

\sessionTalk{ Multilevel Monte Carlo Methods for Chaotic Dynamical Systems }
{Anastasia Istratuca}
{SS21-1}
\sessionTalk{ Multiindex Monte Carlo for semilinear stochastic partial differential equations }
{Håkon Hoel}
{SS21-2}
\sessionTalk{ Multilevel Monte Carlo for kinetic particle models }
{Emil Løvbak}
{SS21-3}
\sessionTalk{ Mixed Precision Multilevel Monte Carlo Method }
{Josef Martínek}
{SS21-4}
\sessionTalk{ Multilevel Monte Carlo Methods for Chaotic Dynamical Systems }
{Anastasia Istratuca}
{SS21-1}
\sessionTalk{ Multiindex Monte Carlo for semilinear stochastic partial differential equations }
{Håkon Hoel}
{SS21-2}
\sessionTalk{ Multilevel Monte Carlo for kinetic particle models }
{Emil Løvbak}
{SS21-3}
\sessionTalk{ Mixed Precision Multilevel Monte Carlo Method }
{Josef Martínek}
{SS21-4}
\sessionTalk{ Multilevel Monte Carlo Methods for Chaotic Dynamical Systems }
{Anastasia Istratuca}
{SS21-1}
\sessionTalk{ Multiindex Monte Carlo for semilinear stochastic partial differential equations }
{Håkon Hoel}
{SS21-2}
\sessionTalk{ Multilevel Monte Carlo for kinetic particle models }
{Emil Løvbak}
{SS21-3}
\sessionTalk{ Mixed Precision Multilevel Monte Carlo Method }
{Josef Martínek}
{SS21-4}
\sessionTalk{ Multilevel Monte Carlo Methods for Chaotic Dynamical Systems }
{Anastasia Istratuca}
{SS21-1}
\sessionTalk{ Multiindex Monte Carlo for semilinear stochastic partial differential equations }
{Håkon Hoel}
{SS21-2}
\sessionTalk{ Multilevel Monte Carlo for kinetic particle models }
{Emil Løvbak}
{SS21-3}
\sessionTalk{ Mixed Precision Multilevel Monte Carlo Method }
{Josef Martínek}
{SS21-4}
\sessionTalk{ Multilevel Monte Carlo Methods for Chaotic Dynamical Systems }
{Anastasia Istratuca}
{SS21-1}
\sessionTalk{ Multiindex Monte Carlo for semilinear stochastic partial differential equations }
{Håkon Hoel}
{SS21-2}
\sessionTalk{ Multilevel Monte Carlo for kinetic particle models }
{Emil Løvbak}
{SS21-3}
\sessionTalk{ Mixed Precision Multilevel Monte Carlo Method }
{Josef Martínek}
{SS21-4}


%\sessionTalk{ Multilevel Active Subspaces for High Dimensional Function Approximation }
{Fabio Nobile}
{SS22-1}
\sessionTalk{ Multilevel function approximation I: meta-theorems and PDE analysis }
{Filippo De Angelis}
{SS22-2}
\sessionTalk{ Multilevel function approximation II: SDE analysis }
{Michael Giles}
{SS22-3}
\sessionTalk{ Multilevel Active Subspaces for High Dimensional Function Approximation }
{Fabio Nobile}
{SS22-1}
\sessionTalk{ Multilevel function approximation I: meta-theorems and PDE analysis }
{Filippo De Angelis}
{SS22-2}
\sessionTalk{ Multilevel function approximation II: SDE analysis }
{Michael Giles}
{SS22-3}
\sessionTalk{Multilevel Active Subspaces for High Dimensional Function Approximation}
{Fabio Nobile}
{SS22-1}
\sessionTalk{Multilevel function approximation I: meta-theorems and PDE analysis}
{Filippo De Angelis}
{SS22-2}
\sessionTalk{Multilevel function approximation II: SDE analysis}
{Michael Giles}
{SS22-3}
\sessionTalk{ Multilevel Active Subspaces for High Dimensional Function Approximation }
{Fabio Nobile}
{SS22-1}
\sessionTalk{ Multilevel function approximation I: meta-theorems and PDE analysis }
{Filippo De Angelis}
{SS22-2}
\sessionTalk{ Multilevel function approximation II: SDE analysis }
{Michael Giles}
{SS22-3}



\clearpage

\begin{session}
 {Recent Advances in Monte Carlo Methods for Forward and Inverse Problems for Stochastic Reaction Networks, Part I}% [1] session title
 {Chiheb Ben Hammouda}% [2] organizer name
 {Utrecht University}% [3] affiliations
 {c.benhammouda@uu.nl}% [4] email
 {Sophia Wiechert}% [5] organizer name. Leave unchanged if there is no second organizer, otherwise fill in accordingly.
 {RWTH Aachen University}% [6] affiliations. Leave unchanged if there is no second organizer, otherwise fill in accordingly.
 {wiechert@uq.rwth-aachen.de}% [7] email. Leave unchanged if there is no second organizer, otherwise fill in accordingly.
 {SS15}
 %{SP13}% [8] session id
 {\thirdorganizer{Ra\'{u}l Tempone}{RWTH Aachen University}{tempone@uq.rwth-aachen.de}}% [9] third organizer, if any
The session is about recent advances related to Monte Carlo methods and variance/dimension reduction techniques for forward/inverse problems and sensitivity analysis for pure jump processes and stochastic reaction networks, with a particular focus on stochastic biological and chemical systems.
\end{session}

\sessionPart{}% [1] part
{\hfill\timeslot{Thursday, August 22, 2024 -- Morning}
{10:30}{12:30}
{ STC 0040 }}
\sessionTalk{ Chemical reaction networks with stochastic switching behavior and machine learning applications }
{Hye-Won Kang}
{SS15-1}
\sessionTalk{ Dimension Reduction via Markovian Projection for Stochastic Reaction Networks }
{Sophia Wiechert}
{SS15-2}
\sessionTalk{ Guided simulation of conditioned chemical reaction networks }
{Frank Meulen}
{SS15-3}


%\sessionTalk{ Stochastic Filtering of Partially Observed Reaction Networks }
{Muruhan Rathinam}
{SS16-1}
\sessionTalk{ Dimensionality Reduction via Markovian Projection in Filtering for Stochastic Reaction Networks: Bridging Accuracy and Efficiency }
{Chiheb Ben Hammouda}
{SS16-2}
\sessionTalk{ Spectral Estimation of the Koopman operator for Stochastic Reaction Networks }
{Ankit Gupta}
{SS16-3}
\sessionTalk{ Stochastic Filtering of Partially Observed Reaction Networks }
{Muruhan Rathinam}
{SS16-1}
\sessionTalk{ Dimensionality Reduction via Markovian Projection in Filtering for Stochastic Reaction Networks: Bridging Accuracy and Efficiency }
{Chiheb Ben Hammouda}
{SS16-2}
\sessionTalk{ Spectral Estimation of the Koopman operator for Stochastic Reaction Networks }
{Ankit Gupta}
{SS16-3}
\sessionTalk{ Stochastic Filtering of Partially Observed Reaction Networks }
{Muruhan Rathinam}
{SS16-1}
\sessionTalk{ Dimensionality Reduction via Markovian Projection in Filtering for Stochastic Reaction Networks: Bridging Accuracy and Efficiency }
{Chiheb Ben Hammouda}
{SS16-2}
\sessionTalk{ Spectral Estimation of the Koopman operator for Stochastic Reaction Networks }
{Ankit Gupta}
{SS16-3}
\sessionTalk{ Stochastic Filtering of Partially Observed Reaction Networks }
{Muruhan Rathinam}
{SS16-1}
\sessionTalk{ Dimensionality Reduction via Markovian Projection in Filtering for Stochastic Reaction Networks: Bridging Accuracy and Efficiency }
{Chiheb Ben Hammouda}
{SS16-2}
\sessionTalk{ Spectral Estimation of the Koopman operator for Stochastic Reaction Networks }
{Ankit Gupta}
{SS16-3}
\sessionTalk{ Stochastic Filtering of Partially Observed Reaction Networks }
{Muruhan Rathinam}
{SS16-1}
\sessionTalk{ Dimensionality Reduction via Markovian Projection in Filtering for Stochastic Reaction Networks: Bridging Accuracy and Efficiency }
{Chiheb Ben Hammouda}
{SS16-2}
\sessionTalk{ Spectral Estimation of the Koopman operator for Stochastic Reaction Networks }
{Ankit Gupta}
{SS16-3}



\clearpage

\begin{session}
 {Kernel approximation and cubature, Part I}% [1] session title
 {Vesa Kaarnioja}% [5] organizer name. Leave unchanged if there is no second organizer, otherwise fill in accordingly.
 {University of Potsdam}% [6] affiliations. Leave unchanged if there is no second organizer, otherwise fill in accordingly.
 {vesa.kaarnioja@iki.fi}% [7] email. Leave unchanged if there is no second organizer, otherwise fill in accordingly.
 {Ilja Klebanov}% [2] organizer name
 {Free University of Berlin}% [3] affiliations
 {klebanov@zedat.fu-berlin.de}% [4] email
 {SS11}
 %{SP10}% [8] session id
 {}% [9] third organizer, if any
 Reproducing kernel Hilbert spaces (RKHSs) are very amenable to the development of efficient approximation and cubature methods. To this end, there has been a surge of interest in recent years regarding some of the advantages that kernel-based methods can offer in applications involving collocation over Monte Carlo or quasi-Monte Carlo point sets---some examples include, e.g., Gaussian process regression (kriging), Bayesian neural networks or uncertainty quantification for partial differential equations. This minisymposium showcases some recent theoretical and computational developments in the study of kernel-based approximation and cubature methods.
\end{session}

\sessionPart{}% [1] part
{\hfill\timeslot{Thursday, August 22, 2024 -- Morning}
{10:30}{12:30}
{ STC 0050 }}
\sessionTalk{ High Dimensional Approximation -- Making life easy with kernels }
{Ian Sloan}
{SS11-1}
\sessionTalk{ Quasi-Monte Carlo meets kernel cubature }
{Robert Gruhlke}
{SS11-2}
\sessionTalk{ Sampling with Stein Discrepancies }
{Chris Oates}
{SS11-3}
\sessionTalk{ Enhanced Lattice-Based Kernel Cubature through Weight Optimization }
{Ilja Klebanov}
{SS11-4}


%\sessionTalk{ A comparison of lattice based kernel and truncated least squares approximations }
{Dirk Nuyens}
{SS12-1}
\sessionTalk{ Approximating distribution functions in uncertainty quantification using quasi-Monte Carlo methods }
{Abirami Srikumar}
{SS12-2}
\sessionTalk{ Quasi-Monte Carlo for Electrical Impedance Tomography }
{Laura Bazahica}
{SS12-3}
\sessionTalk{ Quasi-Monte Carlo Methods for PDEs on Randomly Moving Domains }
{André-Alexander Zepernick}
{SS12-4}
\sessionTalk{ A comparison of lattice based kernel and truncated least squares approximations }
{Dirk Nuyens}
{SS12-1}
\sessionTalk{ Approximating distribution functions in uncertainty quantification using quasi-Monte Carlo methods }
{Abirami Srikumar}
{SS12-2}
\sessionTalk{ Quasi-Monte Carlo for Electrical Impedance Tomography }
{Laura Bazahica}
{SS12-3}
\sessionTalk{ Quasi-Monte Carlo Methods for PDEs on Randomly Moving Domains }
{André-Alexander Zepernick}
{SS12-4}
\sessionTalk{ A comparison of lattice based kernel and truncated least squares approximations }
{Dirk Nuyens}
{SS12-1}
\sessionTalk{ Approximating distribution functions in uncertainty quantification using quasi-Monte Carlo methods }
{Abirami Srikumar}
{SS12-2}
\sessionTalk{ Quasi-Monte Carlo for Electrical Impedance Tomography }
{Laura Bazahica}
{SS12-3}
\sessionTalk{ Quasi-Monte Carlo Methods for PDEs on Randomly Moving Domains }
{André-Alexander Zepernick}
{SS12-4}
\sessionTalk{ A comparison of lattice based kernel and truncated least squares approximations }
{Dirk Nuyens}
{SS12-1}
\sessionTalk{ Approximating distribution functions in uncertainty quantification using quasi-Monte Carlo methods }
{Abirami Srikumar}
{SS12-2}
\sessionTalk{ Quasi-Monte Carlo for Electrical Impedance Tomography }
{Laura Bazahica}
{SS12-3}
\sessionTalk{ Quasi-Monte Carlo Methods for PDEs on Randomly Moving Domains }
{André-Alexander Zepernick}
{SS12-4}
\sessionTalk{ A comparison of lattice based kernel and truncated least squares approximations }
{Dirk Nuyens}
{SS12-1}
\sessionTalk{ Approximating distribution functions in uncertainty quantification using quasi-Monte Carlo methods }
{Abirami Srikumar}
{SS12-2}
\sessionTalk{ Quasi-Monte Carlo for Electrical Impedance Tomography }
{Laura Bazahica}
{SS12-3}
\sessionTalk{ Quasi-Monte Carlo Methods for PDEs on Randomly Moving Domains }
{André-Alexander Zepernick}
{SS12-4}



\clearpage
\begin{session}
 {Multilevel methods for SDEs and SPDEs, Part II}% [1] session title
 {Mike Giles}% [2] organizer name
 {University of Oxford}% [3] affiliations
 {mike.giles@maths.ox.ac.uk}% [4] email
 {}% [5] organizer name. Leave unchanged if there is no second organizer, otherwise fill in accordingly.
 {}% [6] affiliations. Leave unchanged if there is no second organizer, otherwise fill in accordingly.
 {}% [7] email. Leave unchanged if there is no second organizer, otherwise fill in accordingly.
 {SS22}% [8] session id
 {}% [9] third organizer, if any
 Speakers in this session will present and analyse multilevel algorithms for an interesting variety of applications, including chaotic SDEs, stochastic PDEs and kinetic particle models.
\end{session}

%\sessionTalk{ Multilevel Monte Carlo Methods for Chaotic Dynamical Systems }
{Anastasia Istratuca}
{SS21-1}
\sessionTalk{ Multiindex Monte Carlo for semilinear stochastic partial differential equations }
{Håkon Hoel}
{SS21-2}
\sessionTalk{ Multilevel Monte Carlo for kinetic particle models }
{Emil Løvbak}
{SS21-3}
\sessionTalk{ Mixed Precision Multilevel Monte Carlo Method }
{Josef Martínek}
{SS21-4}
\sessionTalk{ Multilevel Monte Carlo Methods for Chaotic Dynamical Systems }
{Anastasia Istratuca}
{SS21-1}
\sessionTalk{ Multiindex Monte Carlo for semilinear stochastic partial differential equations }
{Håkon Hoel}
{SS21-2}
\sessionTalk{ Multilevel Monte Carlo for kinetic particle models }
{Emil Løvbak}
{SS21-3}
\sessionTalk{ Mixed Precision Multilevel Monte Carlo Method }
{Josef Martínek}
{SS21-4}
\sessionTalk{ Multilevel Monte Carlo Methods for Chaotic Dynamical Systems }
{Anastasia Istratuca}
{SS21-1}
\sessionTalk{ Multiindex Monte Carlo for semilinear stochastic partial differential equations }
{Håkon Hoel}
{SS21-2}
\sessionTalk{ Multilevel Monte Carlo for kinetic particle models }
{Emil Løvbak}
{SS21-3}
\sessionTalk{ Mixed Precision Multilevel Monte Carlo Method }
{Josef Martínek}
{SS21-4}
\sessionTalk{ Multilevel Monte Carlo Methods for Chaotic Dynamical Systems }
{Anastasia Istratuca}
{SS21-1}
\sessionTalk{ Multiindex Monte Carlo for semilinear stochastic partial differential equations }
{Håkon Hoel}
{SS21-2}
\sessionTalk{ Multilevel Monte Carlo for kinetic particle models }
{Emil Løvbak}
{SS21-3}
\sessionTalk{ Mixed Precision Multilevel Monte Carlo Method }
{Josef Martínek}
{SS21-4}
\sessionTalk{ Multilevel Monte Carlo Methods for Chaotic Dynamical Systems }
{Anastasia Istratuca}
{SS21-1}
\sessionTalk{ Multiindex Monte Carlo for semilinear stochastic partial differential equations }
{Håkon Hoel}
{SS21-2}
\sessionTalk{ Multilevel Monte Carlo for kinetic particle models }
{Emil Løvbak}
{SS21-3}
\sessionTalk{ Mixed Precision Multilevel Monte Carlo Method }
{Josef Martínek}
{SS21-4}


\sessionTalk{ Multilevel Active Subspaces for High Dimensional Function Approximation }
{Fabio Nobile}
{SS22-1}
\sessionTalk{ Multilevel function approximation I: meta-theorems and PDE analysis }
{Filippo De Angelis}
{SS22-2}
\sessionTalk{ Multilevel function approximation II: SDE analysis }
{Michael Giles}
{SS22-3}
\sessionTalk{ Multilevel Active Subspaces for High Dimensional Function Approximation }
{Fabio Nobile}
{SS22-1}
\sessionTalk{ Multilevel function approximation I: meta-theorems and PDE analysis }
{Filippo De Angelis}
{SS22-2}
\sessionTalk{ Multilevel function approximation II: SDE analysis }
{Michael Giles}
{SS22-3}
\sessionTalk{Multilevel Active Subspaces for High Dimensional Function Approximation}
{Fabio Nobile}
{SS22-1}
\sessionTalk{Multilevel function approximation I: meta-theorems and PDE analysis}
{Filippo De Angelis}
{SS22-2}
\sessionTalk{Multilevel function approximation II: SDE analysis}
{Michael Giles}
{SS22-3}
\sessionTalk{ Multilevel Active Subspaces for High Dimensional Function Approximation }
{Fabio Nobile}
{SS22-1}
\sessionTalk{ Multilevel function approximation I: meta-theorems and PDE analysis }
{Filippo De Angelis}
{SS22-2}
\sessionTalk{ Multilevel function approximation II: SDE analysis }
{Michael Giles}
{SS22-3}

\clearpage

\begin{session}
 {Recent Advances in Monte Carlo Methods for Forward and Inverse Problems for Stochastic Reaction Networks, Part II}% [1] session title
 {Chiheb Ben Hammouda}% [2] organizer name
 {Utrecht University}% [3] affiliations
 {c.benhammouda@uu.nl}% [4] email
 {Sophia Wiechert}% [5] organizer name. Leave unchanged if there is no second organizer, otherwise fill in accordingly.
 {RWTH Aachen University}% [6] affiliations. Leave unchanged if there is no second organizer, otherwise fill in accordingly.
 {wiechert@uq.rwth-aachen.de}% [7] email. Leave unchanged if there is no second organizer, otherwise fill in accordingly.
 {SS16}
 %{SP13}% [8] session id
 {\thirdorganizer{Ra\'{u}l Tempone}{RWTH Aachen University}{tempone@uq.rwth-aachen.de}}% [9] third organizer, if any
The session is about recent advances related to Monte Carlo methods and variance/dimension reduction techniques for forward/inverse problems and sensitivity analysis for pure jump processes and stochastic reaction networks, with a particular focus on stochastic biological and chemical systems.
\end{session}

%\sessionPart{}% [1] part
{\hfill\timeslot{Thursday, August 22, 2024 -- Morning}
{10:30}{12:30}
{ STC 0040 }}
\sessionTalk{ Chemical reaction networks with stochastic switching behavior and machine learning applications }
{Hye-Won Kang}
{SS15-1}
\sessionTalk{ Dimension Reduction via Markovian Projection for Stochastic Reaction Networks }
{Sophia Wiechert}
{SS15-2}
\sessionTalk{ Guided simulation of conditioned chemical reaction networks }
{Frank Meulen}
{SS15-3}


\sessionTalk{ Stochastic Filtering of Partially Observed Reaction Networks }
{Muruhan Rathinam}
{SS16-1}
\sessionTalk{ Dimensionality Reduction via Markovian Projection in Filtering for Stochastic Reaction Networks: Bridging Accuracy and Efficiency }
{Chiheb Ben Hammouda}
{SS16-2}
\sessionTalk{ Spectral Estimation of the Koopman operator for Stochastic Reaction Networks }
{Ankit Gupta}
{SS16-3}
\sessionTalk{ Stochastic Filtering of Partially Observed Reaction Networks }
{Muruhan Rathinam}
{SS16-1}
\sessionTalk{ Dimensionality Reduction via Markovian Projection in Filtering for Stochastic Reaction Networks: Bridging Accuracy and Efficiency }
{Chiheb Ben Hammouda}
{SS16-2}
\sessionTalk{ Spectral Estimation of the Koopman operator for Stochastic Reaction Networks }
{Ankit Gupta}
{SS16-3}
\sessionTalk{ Stochastic Filtering of Partially Observed Reaction Networks }
{Muruhan Rathinam}
{SS16-1}
\sessionTalk{ Dimensionality Reduction via Markovian Projection in Filtering for Stochastic Reaction Networks: Bridging Accuracy and Efficiency }
{Chiheb Ben Hammouda}
{SS16-2}
\sessionTalk{ Spectral Estimation of the Koopman operator for Stochastic Reaction Networks }
{Ankit Gupta}
{SS16-3}
\sessionTalk{ Stochastic Filtering of Partially Observed Reaction Networks }
{Muruhan Rathinam}
{SS16-1}
\sessionTalk{ Dimensionality Reduction via Markovian Projection in Filtering for Stochastic Reaction Networks: Bridging Accuracy and Efficiency }
{Chiheb Ben Hammouda}
{SS16-2}
\sessionTalk{ Spectral Estimation of the Koopman operator for Stochastic Reaction Networks }
{Ankit Gupta}
{SS16-3}
\sessionTalk{ Stochastic Filtering of Partially Observed Reaction Networks }
{Muruhan Rathinam}
{SS16-1}
\sessionTalk{ Dimensionality Reduction via Markovian Projection in Filtering for Stochastic Reaction Networks: Bridging Accuracy and Efficiency }
{Chiheb Ben Hammouda}
{SS16-2}
\sessionTalk{ Spectral Estimation of the Koopman operator for Stochastic Reaction Networks }
{Ankit Gupta}
{SS16-3}

\clearpage

\begin{session}
 {Kernel approximation and cubature, Part II}% [1] session title
 {Vesa Kaarnioja}% [5] organizer name. Leave unchanged if there is no second organizer, otherwise fill in accordingly.
 {University of Potsdam}% [6] affiliations. Leave unchanged if there is no second organizer, otherwise fill in accordingly.
 {vesa.kaarnioja@iki.fi}% [7] email. Leave unchanged if there is no second organizer, otherwise fill in accordingly.
 {Ilja Klebanov}% [2] organizer name
 {Free University of Berlin}% [3] affiliations
 {klebanov@zedat.fu-berlin.de}% [4] email
 {SS12}
 %{SP10}% [8] session id
 {}% [9] third organizer, if any
 Reproducing kernel Hilbert spaces (RKHSs) are very amenable to the development of efficient approximation and cubature methods. To this end, there has been a surge of interest in recent years regarding some of the advantages that kernel-based methods can offer in applications involving collocation over Monte Carlo or quasi-Monte Carlo point sets---some examples include, e.g., Gaussian process regression (kriging), Bayesian neural networks or uncertainty quantification for partial differential equations. This minisymposium showcases some recent theoretical and computational developments in the study of kernel-based approximation and cubature methods.
\end{session}

%\sessionPart{}% [1] part
{\hfill\timeslot{Thursday, August 22, 2024 -- Morning}
{10:30}{12:30}
{ STC 0050 }}
\sessionTalk{ High Dimensional Approximation -- Making life easy with kernels }
{Ian Sloan}
{SS11-1}
\sessionTalk{ Quasi-Monte Carlo meets kernel cubature }
{Robert Gruhlke}
{SS11-2}
\sessionTalk{ Sampling with Stein Discrepancies }
{Chris Oates}
{SS11-3}
\sessionTalk{ Enhanced Lattice-Based Kernel Cubature through Weight Optimization }
{Ilja Klebanov}
{SS11-4}


\sessionTalk{ A comparison of lattice based kernel and truncated least squares approximations }
{Dirk Nuyens}
{SS12-1}
\sessionTalk{ Approximating distribution functions in uncertainty quantification using quasi-Monte Carlo methods }
{Abirami Srikumar}
{SS12-2}
\sessionTalk{ Quasi-Monte Carlo for Electrical Impedance Tomography }
{Laura Bazahica}
{SS12-3}
\sessionTalk{ Quasi-Monte Carlo Methods for PDEs on Randomly Moving Domains }
{André-Alexander Zepernick}
{SS12-4}
\sessionTalk{ A comparison of lattice based kernel and truncated least squares approximations }
{Dirk Nuyens}
{SS12-1}
\sessionTalk{ Approximating distribution functions in uncertainty quantification using quasi-Monte Carlo methods }
{Abirami Srikumar}
{SS12-2}
\sessionTalk{ Quasi-Monte Carlo for Electrical Impedance Tomography }
{Laura Bazahica}
{SS12-3}
\sessionTalk{ Quasi-Monte Carlo Methods for PDEs on Randomly Moving Domains }
{André-Alexander Zepernick}
{SS12-4}
\sessionTalk{ A comparison of lattice based kernel and truncated least squares approximations }
{Dirk Nuyens}
{SS12-1}
\sessionTalk{ Approximating distribution functions in uncertainty quantification using quasi-Monte Carlo methods }
{Abirami Srikumar}
{SS12-2}
\sessionTalk{ Quasi-Monte Carlo for Electrical Impedance Tomography }
{Laura Bazahica}
{SS12-3}
\sessionTalk{ Quasi-Monte Carlo Methods for PDEs on Randomly Moving Domains }
{André-Alexander Zepernick}
{SS12-4}
\sessionTalk{ A comparison of lattice based kernel and truncated least squares approximations }
{Dirk Nuyens}
{SS12-1}
\sessionTalk{ Approximating distribution functions in uncertainty quantification using quasi-Monte Carlo methods }
{Abirami Srikumar}
{SS12-2}
\sessionTalk{ Quasi-Monte Carlo for Electrical Impedance Tomography }
{Laura Bazahica}
{SS12-3}
\sessionTalk{ Quasi-Monte Carlo Methods for PDEs on Randomly Moving Domains }
{André-Alexander Zepernick}
{SS12-4}
\sessionTalk{ A comparison of lattice based kernel and truncated least squares approximations }
{Dirk Nuyens}
{SS12-1}
\sessionTalk{ Approximating distribution functions in uncertainty quantification using quasi-Monte Carlo methods }
{Abirami Srikumar}
{SS12-2}
\sessionTalk{ Quasi-Monte Carlo for Electrical Impedance Tomography }
{Laura Bazahica}
{SS12-3}
\sessionTalk{ Quasi-Monte Carlo Methods for PDEs on Randomly Moving Domains }
{André-Alexander Zepernick}
{SS12-4}


\cleardoublepage
\begin{session}
 {MCMC: Convergence and Robustness}% [1] session title
 {Alex Shestopaloff}% [2] organizer name
 {Queen Mary University of London}% [3] affiliations
 {a.shestopaloff@qmul.ac.uk}% [4] email
 {Jun Yang}% [5] organizer name. Leave unchanged if there is no second organizer, otherwise fill in accordingly.
 {University of Copenhagen}% [6] affiliations. Leave unchanged if there is no second organizer, otherwise fill in accordingly.
 {jy@math.ku.dk}% [7] email. Leave unchanged if there is no second organizer, otherwise fill in accordingly.
 %%%%% from CL, May 4: this session is now labeled SS1
 %%%%% see MasterLists and MCQMC2024Data
 %%%%% we need to change all the labels for the sessions
 {SS1}% [8] session id
 %%%%% CL is commenting out old label on May 4
 %%%%%{SP01}% [8] session id
 {}% [9] third organizer, if any
 As Markov Chain Monte Carlo (MCMC) methods become more complex, a deeper understanding of their convergence and performance guarantees in realistic scenarios becomes an important aspect of using these methods in computational Bayesian statistics. This session aims to further this understanding by focusing on the convergence and robustness of complex MCMC samplers, covering recent work on topics such as convergence of hybrid Gibbs sampling [1], novel methods for evaluation of convergence rates using random number simulations [2] the study of convergence with Dirichlet forms [3] as well as techniques for making MCMC samplers more robust and a study of their corresponding convergence properties, such as [4].

\begin{enumerate}
	\item[{[1]}] Qian Qin, Nianqiao Ju, Guanyang Wang (2023). Spectral gap bounds for reversible hybrid Gibbs chains. arXiv:2312.12782.
	\item[{[2]}] Sabrina Sixta and Jeffrey S. Rosenthal (2023). Bounding and estimating MCMC convergence rates using common random number simulations. arXiv:2309.15735.
	\item[{[3]}] Ning Ning (2022). Convergence of Dirichlet Forms for MCMC Optimal Scaling with General Target Distributions on Large Graphs. arXiv:2210:17042.
	\item[{[4]}] Michael C.H. Choi (2020). Improved Metropolis-Hastings algorithms via landscape modifcation with applications to simulated annealing and the Curie-Weiss model. arXiv: 2011:09680.
\end{enumerate}
\end{session}

%%%%%% from CL on May 4
%%%%%% now we just need to input the corresponding
%%%%%% latex file with sessions description
%%%%%% convention is sess[sessionID].tex so here sessSS1.tex

\sessionTalk{ Geometric unification of central MCMC algorithms via rate distortion theory and factorizability of multivariate Markov chains }
{Michael Choi}
{SS1-1}
\sessionTalk{ A large deviation principle for Metropolis-Hastings sampling }
{Federica Milinanni}
{SS1-2}
\sessionTalk{ On the Convergence of MCMCs with Quantum Speedup }
{Ning Ning}
{SS1-3}
\sessionTalk{ Adapting the Stereographic Bouncy Particle Sampler }
{Cameron Bell}
{SS1-4}
\sessionTalk{ Geometric unification of central MCMC algorithms via rate distortion theory and factorizability of multivariate Markov chains }
{Michael Choi}
{SS1-1}
\sessionTalk{ A large deviation principle for Metropolis-Hastings sampling }
{Federica Milinanni}
{SS1-2}
\sessionTalk{ On the Convergence of MCMCs with Quantum Speedup }
{Ning Ning}
{SS1-3}
\sessionTalk{ Adapting the Stereographic Bouncy Particle Sampler }
{Cameron Bell}
{SS1-4}
\sessionTalk{ Geometric unification of central MCMC algorithms via rate distortion theory and factorizability of multivariate Markov chains }
{Michael Choi}
{SS1-1}
\sessionTalk{ A large deviation principle for Metropolis-Hastings sampling }
{Federica Milinanni}
{SS1-2}
\sessionTalk{ On the Convergence of MCMCs with Quantum Speedup }
{Ning Ning}
{SS1-3}
\sessionTalk{ Adapting the Stereographic Bouncy Particle Sampler }
{Cameron Bell}
{SS1-4}
\sessionTalk{ Geometric unification of central MCMC algorithms via rate distortion theory and factorizability of multivariate Markov chains }
{Michael Choi}
{SS1-1}
\sessionTalk{ A large deviation principle for Metropolis-Hastings sampling }
{Federica Milinanni}
{SS1-2}
\sessionTalk{ On the Convergence of MCMCs with Quantum Speedup }
{Ning Ning}
{SS1-3}
\sessionTalk{ Adapting the Stereographic Bouncy Particle Sampler }
{Cameron Bell}
{SS1-4}


\clearpage
\begin{session}
 {Continuous-time dynamics in Monte Carlo and beyond}% [1] session title
 {Neil Chada}% [2] organizer name
 {Heriot-Watt University}% [3] affiliations
 {n.chada@hw.ac.uk}% [4] email
 {Jonas Latz}% [5] organizer name. Leave unchanged if there is no second organizer, otherwise fill in accordingly.
 {University of Manchester } %[6] affiliations. Leave unchanged if there is no second organizer, otherwise fill in accordingly.
 {jonas.latz@manchester.ac.uk }% [7] email. Leave unchanged if there is no second organizer, otherwise fill in accordingly.
 {SS17}
 %{SP14}% [8] session id
 {}
 Langevin Monte Carlo methods — such as MALA and ULA [3] — construct a Monte Carlo Markov chain by appropriately discretising certain stochastic differential equations.
 This has the fortunate effect that certain properties of the resulting MCMC algorithms can be derived by studying these SDEs rather than the arising discrete-time Markov chains.
 The idea of analysing an underlying continuous-time system to understand a discrete-time algorithm is much broader and shall be one focus of this minisymposium – with `algorithm’,
 we foremost want to focus on methods in computational statistics, but also look forward to optimisation methods, such as [2], data assimilation, diffusion models, and partial differential equation
 methods in data science. The second focus are Monte Carlo methods that are both posed and used in continuous time, such as piecewise-deterministic Markov processes (cf. [1]).
 \medskip

 \begin{enumerate}
 \item[{[1]}] Bierkens, Joris, Paul Fearnhead \& Gareth Roberts (2019). {\it The Zig-Zag process and super-efficient sampling for Bayesian analysis of big data}. Ann.\ Statist.\ 47(3): 1288-1320.
 \item[{[2]}] Li, Qianxiao , Cheng Tai \& Weinan E (2019). {\it Stochastic Modified Equations and Dynamics of Stochastic Gradient Algorithms I: Mathematical Foundations}. J.\ Mach.\ Learn.\ Res.\ 20(40):1-47.
 \item[{[3]}] Roberts, Gareth \& Richard Tweedie (2002). {\it Exponential convergence of Langevin distributions and their discrete approximations}. Bernoulli 2(4): 341-363.
 \end{enumerate}
\end{session}

\sessionTalk{ Losing momentum in continuous-time stochastic optimisation }
{Jonas Latz}
{SS17-1}
\sessionTalk{ How to choose an annealing algorithm }
{Alexandre Bouchard-Cote}
{SS17-2}
\sessionTalk{ Projected ensemble data assimilation }
{Svetlana Dubinkina}
{SS17-3}
\sessionTalk{ Approximation of vectors using adaptive randomized information }
{Marcin Wnuk}
{SS17-4}
\sessionTalk{ Losing momentum in continuous-time stochastic optimisation }
{Jonas Latz}
{SS17-1}
\sessionTalk{ How to choose an annealing algorithm }
{Alexandre Bouchard-Cote}
{SS17-2}
\sessionTalk{ Projected ensemble data assimilation }
{Svetlana Dubinkina}
{SS17-3}
\sessionTalk{ Approximation of vectors using adaptive randomized information }
{Marcin Wnuk}
{SS17-4}
\sessionTalk{ Losing momentum in continuous-time stochastic optimisation }
{Jonas Latz}
{SS17-1}
\sessionTalk{ How to choose an annealing algorithm }
{Alexandre Bouchard-Cote}
{SS17-2}
\sessionTalk{ Projected ensemble data assimilation }
{Svetlana Dubinkina}
{SS17-3}
\sessionTalk{ Approximation of vectors using adaptive randomized information }
{Marcin Wnuk}
{SS17-4}
\sessionTalk{ Losing momentum in continuous-time stochastic optimisation }
{Jonas Latz}
{SS17-1}
\sessionTalk{ How to choose an annealing algorithm }
{Alexandre Bouchard-Cote}
{SS17-2}
\sessionTalk{ Projected ensemble data assimilation }
{Svetlana Dubinkina}
{SS17-3}
\sessionTalk{ Approximation of vectors using adaptive randomized information }
{Marcin Wnuk}
{SS17-4}
\sessionTalk{ Losing momentum in continuous-time stochastic optimisation }
{Jonas Latz}
{SS17-1}
\sessionTalk{ How to choose an annealing algorithm }
{Alexandre Bouchard-Cote}
{SS17-2}
\sessionTalk{ Projected ensemble data assimilation }
{Svetlana Dubinkina}
{SS17-3}
\sessionTalk{ Approximation of vectors using adaptive randomized information }
{Marcin Wnuk}
{SS17-4}



\clearpage


%%FGW

\begin{session}
    {Function spaces and algorithms for high-dimensional problems}% [1] session title
    {Michael Gnewuch}% [2] organizer name
    {University of Osnabr\"uck, Germany}% [3] affiliation(s)
    {michael.gnewuch@uni-osnabrueck.de}% [4] email
    {Klaus Ritter }% [5] Second organizer's name. Leave unchanged if there is no second organizer, otherwise fill in accordingly.
    {RPTU Kaiserslautern, Germany}% [6] Second organizer's affiliation(s). Leave unchanged if there is no second organizer, otherwise fill in accordingly.
    {ritter@mathematik.uni-kl.de}% [7] Second organizer's email. Leave unchanged if there is no second organizer, otherwise fill in accordingly.
    {SS24}
    %{SP19}%
    {}% [9] third organizer, if any
  High- and infinite-dimensional problems pose serious challenges in numerical practice. An approach to surpass these obstacles is to identify common structural features of the underlying problems. These features are usually encoded in the specific function spaces that are considered in the analysis.
  In this special session we want to bring together researchers from analysis, approximation theory, and information-based complexity to discuss different types of function spaces and algorithmic approaches for high- and infinite-dimensional integration and approximation problems.
\end{session}

\sessionTalk{ Function space embeddings for non-tensor product spaces and application to high-dimensional approximation }
{Michael Gnewuch}
{SS24-1}
\sessionTalk{ ANOVA-boosting for high-dimensional approximation }
{Laura Weidensager}
{SS24-2}
\sessionTalk{ Tractability results for integration on Gaussian spaces }
{Robin Rüßmann}
{SS24-3}
\sessionTalk{ Function space embeddings for non-tensor product spaces and application to high-dimensional approximation }
{Michael Gnewuch}
{SS24-1}
\sessionTalk{ ANOVA-boosting for high-dimensional approximation }
{Laura Weidensager}
{SS24-2}
\sessionTalk{ Tractability results for integration on Gaussian spaces }
{Robin Rüßmann}
{SS24-3}
\sessionTalk{Function space embeddings for non-tensor product spaces and application to high-dimensional approximation}
{Michael Gnewuch}
{SS24-1}
\sessionTalk{ANOVA-boosting for high-dimensional approximation}
{Laura Weidensager}
{SS24-2}
\sessionTalk{Tractability results for integration on Gaussian spaces}
{Robin Rüßmann}
{SS24-3}
\sessionTalk{ Function space embeddings for non-tensor product spaces and application to high-dimensional approximation }
{Michael Gnewuch}
{SS24-1}
\sessionTalk{ ANOVA-boosting for high-dimensional approximation }
{Laura Weidensager}
{SS24-2}
\sessionTalk{ Tractability results for integration on Gaussian spaces }
{Robin Rüßmann}
{SS24-3}



\clearpage

%% NK: organizers cancelled one of four sessions, commenting out part 4 (May 7)
\iffalse
\begin{session}
 {Stochastic Computation and Complexity, Part IV: High dimensional approximation and integration}% [1] session title
 {Larisa Yaroslavtseva}% [2] organizer name
 {University of Graz}% [3] affiliations
 {larisa.yaroslavtseva@uni-graz.at}% [4] email
 {}% [5] organizer name. Leave unchanged if there is no second organizer, otherwise fill in accordingly.
 {}% [6] affiliations. Leave unchanged if there is no second organizer, otherwise fill in accordingly.
 {}% [7] email. Leave unchanged if there is no second organizer, otherwise fill in accordingly.
 {SS6}
 %{SP06}% [8] session id
 {}% [9] third organizer, if any
The session is devoted to algorithms and complexity for
\begin{itemize}[itemsep=0pt,topsep=0pt]
 \item quadrature and strong approximation of SDEs and SPDEs, in particular under nonstandard assumptions,

 \item high and infinite dimensional integration and approximation, and

 \item stochastic optimization and neural networks,
\end{itemize}
including connections to functional analysis and stochastic analysis.
\end{session}



 \fi
\clearpage
