\documentclass[12pt,a4paper,figuresright]{book}

\usepackage{amsmath,amssymb}
\usepackage{tabularx,graphicx,url,xcolor,rotating,multicol,epsfig,colortbl}

\setlength{\textheight}{25.2cm}
\setlength{\textwidth}{16.5cm} %\setlength{\textwidth}{18.2cm}
\setlength{\voffset}{-1.6cm}
\setlength{\hoffset}{-0.3cm} %\setlength{\hoffset}{-1.2cm}
\setlength{\evensidemargin}{-0.3cm} 
\setlength{\oddsidemargin}{0.3cm}
\setlength{\parindent}{0cm} 
\setlength{\parskip}{0.3cm}

% -- adding a talk
\newenvironment{talk}[6]% [1] talk title
                         % [2] speaker name, [3] affiliations, [4] email,
                         % [5] coauthors, [6] special session
                         % [7] time slot
                         % [8] talk id, [9] session id or photo
 {%\needspace{6\baselineskip}%
  \vskip 0pt\nopagebreak%
%   \colorbox{gray!20!white}{\makebox[0.99\textwidth][r]{}}\nopagebreak%
%   \ifthenelse{\equal{#9}{photo}}{%
%                     \\\\\colorbox{gray!20!white}{\makebox{\includegraphics[width=3cm]{#8}}}\nopagebreak}{}%
 \vskip 0pt\nopagebreak%
%  \label{#8}%
  \textbf{#1}\vspace{3mm}\\\nopagebreak%
  \textit{#2}\\\nopagebreak%
  #3\\\nopagebreak%
  \url{#4}\vspace{3mm}\\\nopagebreak%
  \ifthenelse{\equal{#5}{}}{}{Coauthor(s): #5\vspace{3mm}\\\nopagebreak}%
  \ifthenelse{\equal{#6}{}}{}{Special session: #6\quad \vspace{3mm}\\\nopagebreak}%
 }
 {\vspace{1cm}\nopagebreak}%

\pagestyle{empty}

% ------------------------------------------------------------------------
% Document begins here
% ------------------------------------------------------------------------
\begin{document}
	
\begin{talk}
  {Quasi-uniform quasi-Monte Carlo digital nets}% [1] talk title
  {Takashi Goda}% [2] speaker name
  {Graduate School of Engineering, The University of Tokyo}% [3] affiliations
  {goda@frcer.t.u-tokyo.ac.jp}% [4] email
  {Josef Dick, Kosuke Suzuki}% [5] coauthors
  {Frontiers in (Quasi-)Monte Carlo and Markov Chain Monte Carlo Methods}% [6] special session. Leave this field empty for contributed talks. 
				% Insert the title of the special session if you were invited to give a talk in a special session.
			
We investigate the quasi-uniformity properties of digital nets, a class of quasi-Monte Carlo point sets. Quasi-uniformity is a space-filling property that plays a crucial role in applications such as designs of computer experiments and radial basis function approximation. However, it remains open whether common low-discrepancy digital nets satisfy quasi-uniformity.

In this talk, we introduce the concept of \emph{well-separated} point sets as a tool for constructing quasi-uniform low-discrepancy digital nets. We establish an algebraic criterion to determine whether a given digital net is well-separated and use this criterion to construct an explicit example of a two-dimensional digital net that is both low-discrepancy and quasi-uniform. Furthermore, we present counterexamples of low-discrepancy digital nets that fail to achieve quasi-uniformity, highlighting the limitations of existing constructions.

\begin{enumerate}
	\item[{[1]}] T. Goda, The Sobol’ sequence is not quasi-uniform in dimension 2. \emph{Proc. Amer. Math. Soc.}, 152(8):3209–3213, 2024.
	\item[{[2]}] J. Dick, T. Goda, \& K. Suzuki, On the quasi-uniformity properties of quasi-Monte Carlo digital nets and sequences. \emph{arXiv preprint arXiv:2501.18226}, 2025.
\end{enumerate}
\end{talk}

\end{document}

