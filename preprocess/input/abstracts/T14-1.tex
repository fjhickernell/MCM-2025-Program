\documentclass[12pt,a4paper,figuresright]{book}

\usepackage{amsmath,amssymb}
\usepackage{tabularx,graphicx,url,xcolor,rotating,multicol,epsfig,colortbl}

\setlength{\textheight}{25.2cm}
\setlength{\textwidth}{16.5cm} %\setlength{\textwidth}{18.2cm}
\setlength{\voffset}{-1.6cm}
\setlength{\hoffset}{-0.3cm} %\setlength{\hoffset}{-1.2cm}
\setlength{\evensidemargin}{-0.3cm} 
\setlength{\oddsidemargin}{0.3cm}
\setlength{\parindent}{0cm} 
\setlength{\parskip}{0.3cm}

% -- adding a talk
\newenvironment{talk}[6]% [1] talk title
                         % [2] speaker name, [3] affiliations, [4] email,
                         % [5] coauthors, [6] special session
                         % [7] time slot
                         % [8] talk id, [9] session id or photo
 {%\needspace{6\baselineskip}%
  \vskip 0pt\nopagebreak%
%   \colorbox{gray!20!white}{\makebox[0.99\textwidth][r]{}}\nopagebreak%
%   \ifthenelse{\equal{#9}{photo}}{%
%                     \\\\\colorbox{gray!20!white}{\makebox{\includegraphics[width=3cm]{#8}}}\nopagebreak}{}%
 \vskip 0pt\nopagebreak%
%  \label{#8}%
  \textbf{#1}\vspace{3mm}\\\nopagebreak%
  \textit{#2}\\\nopagebreak%
  #3\\\nopagebreak%
  \url{#4}\vspace{3mm}\\\nopagebreak%
  \ifthenelse{\equal{#5}{}}{}{Coauthor(s): #5\vspace{3mm}\\\nopagebreak}%
  \ifthenelse{\equal{#6}{}}{}{Special session: #6\quad \vspace{3mm}\\\nopagebreak}%
 }
 {\vspace{1cm}\nopagebreak}%

\pagestyle{empty}

% ------------------------------------------------------------------------
% Document begins here
% ------------------------------------------------------------------------
\begin{document}
	
\begin{talk}
  {Delayed Acceptance Slice Sampling: A Two-Level method for Improved Efficiency in High-Dimensional Settings}% [1] talk title
  {Kevin Bitterlich}% [2] speaker name
  {TU Bergakademie Freiberg}% [3] affiliations
  {kevin.bitterlich@math.tu-freiberg.de}% [4] email
  {Bjoern Sprungk, Daniel Rudolf}% [5] coauthors
  {}% [6] special session. Leave this field empty for contributed talks. 
				% Insert the title of the special session if you were invited to give a talk in a special session.
			
Slice sampling is a Markov chain Monte Carlo (MCMC) method for drawing (approximately) random samples from a posterior distribution that 
is typically only known up to a normalizing constant. 
The method is based on sampling a new state on a slice, i.e., a level set of the target density function. 
Slice sampling is especially interesting because it is tuning-free and guarantees a move to a new state, which can 
result in a lower autocorrelation compared to other MCMC methods. 
However, finding such a new state can be computationally expensive due to frequent evaluations of the target density, 
especially in high-dimensional settings. 
To mitigate these costs, we introduce a delayed acceptance mechanism that incorporates an approximate target density for finding potential 
new states. We will demonstrate the effectiveness of our method through various numerical experiments and outline an extension of our two-level method into a multilevel framework.


\end{talk}

\end{document}

