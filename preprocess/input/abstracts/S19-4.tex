\documentclass[12pt,a4paper,figuresright]{book}

\usepackage{amsmath,amssymb}
\usepackage{tabularx,graphicx,url,xcolor,rotating,multicol,epsfig,colortbl}

\setlength{\textheight}{25.2cm}
\setlength{\textwidth}{16.5cm} %\setlength{\textwidth}{18.2cm}
\setlength{\voffset}{-1.6cm}
\setlength{\hoffset}{-0.3cm} %\setlength{\hoffset}{-1.2cm}
\setlength{\evensidemargin}{-0.3cm} 
\setlength{\oddsidemargin}{0.3cm}
\setlength{\parindent}{0cm} 
\setlength{\parskip}{0.3cm}

% -- adding a talk
\newenvironment{talk}[6]% [1] talk title
                         % [2] speaker name, [3] affiliations, [4] email,
                         % [5] coauthors, [6] special session
                         % [7] time slot
                         % [8] talk id, [9] session id or photo
 {%\needspace{6\baselineskip}%
  \vskip 0pt\nopagebreak%
%   \colorbox{gray!20!white}{\makebox[0.99\textwidth][r]{}}\nopagebreak%
%   \ifthenelse{\equal{#9}{photo}}{%
%                     \\\\\colorbox{gray!20!white}{\makebox{\includegraphics[width=3cm]{#8}}}\nopagebreak}{}%
 \vskip 0pt\nopagebreak%
%  \label{#8}%
  \textbf{#1}\vspace{3mm}\\\nopagebreak%
  \textit{#2}\\\nopagebreak%
  #3\\\nopagebreak%
  \url{#4}\vspace{3mm}\\\nopagebreak%
  \ifthenelse{\equal{#5}{}}{}{Coauthor(s): #5\vspace{3mm}\\\nopagebreak}%
  \ifthenelse{\equal{#6}{}}{}{Special session: #6\quad \vspace{3mm}\\\nopagebreak}%
 }
 {\vspace{1cm}\nopagebreak}%

\pagestyle{empty}

% ------------------------------------------------------------------------
% Document begins here
% ------------------------------------------------------------------------
\begin{document}
	
\begin{talk}
  {Stochastic Gradient with Testing Functionals}% [1] talk title
  {Akshita Gupta}% [2] speaker name
  {Purdue University}% [3] affiliations
  {gupta417@purdue.edu}% [4] email
  {R. Bollapragada, R. Pasupathy, A. Yip}% [5] coauthors
  {}% [6] special session. Leave this field empty for contributed talks. 
				% Insert the title of the special session if you were invited to give a talk in a special session.
			
A folklore stochastic gradient (SG) algorithm works as follows. Execute SG with a \emph{fixed step} $\eta >0$ until the iterates approach ``stationarity,'' so that no further gains are possible without a reduction in step size. Now re-execute SG starting from the final iterate of the previous execution, but with the smaller fixed step  $\eta \leftarrow \beta_0\eta, \beta_0 <1$ until the iterates again approach ``stationarity,'' and so on. This simple restart scheme, originally conceived by G. Pflug in 1983, has been rediscovered and commented on by several prominent authors over the decades, with some reporting strikingly favorable numerical results, robustness, and a self-correcting nature. And yet, no complete analysis of such algorithms exists to date, probably due to the need to handle stopping times, and the need to rigorously detect stationarity. In this work, we first unify adaptive fixed-step stochastic gradient methods through \emph{testing functionals} --- essentially stopping time policies that use algorithmic history to decide stopping. We characterize general conditions on testing functionals to guarantee algorithm consistency and optimality, and then identify a specific testing functional that is based on a statistic entering a fixed stopping region, whose optimal size is intimately tied to SG's fixed step size. The proposed testing functional framework also provides a path to analyze the many adaptive SG heuristics that have emerged over the years.

\medskip




\end{talk}

\end{document}
