\documentclass[12pt,a4paper,figuresright]{book}

\usepackage{amsmath,amssymb}
\usepackage{tabularx,graphicx,url,xcolor,rotating,multicol,epsfig,colortbl}

\setlength{\textheight}{25.2cm}
\setlength{\textwidth}{16.5cm} %\setlength{\textwidth}{18.2cm}
\setlength{\voffset}{-1.6cm}
\setlength{\hoffset}{-0.3cm} %\setlength{\hoffset}{-1.2cm}
\setlength{\evensidemargin}{-0.3cm} 
\setlength{\oddsidemargin}{0.3cm}
\setlength{\parindent}{0cm} 
\setlength{\parskip}{0.3cm}

% -- adding a talk
\newenvironment{talk}[6]% [1] talk title
                         % [2] speaker name, [3] affiliations, [4] email,
                         % [5] coauthors, [6] special session
                         % [7] time slot
                         % [8] talk id, [9] session id or photo
 {%\needspace{6\baselineskip}%
  \vskip 0pt\nopagebreak%
%   \colorbox{gray!20!white}{\makebox[0.99\textwidth][r]{}}\nopagebreak%
%   \ifthenelse{\equal{#9}{photo}}{%
%                     \\\\\colorbox{gray!20!white}{\makebox{\includegraphics[width=3cm]{#8}}}\nopagebreak}{}%
 \vskip 0pt\nopagebreak%
%  \label{#8}%
  \textbf{#1}\vspace{3mm}\\\nopagebreak%
  \textit{#2}\\\nopagebreak%
  #3\\\nopagebreak%
  \url{#4}\vspace{3mm}\\\nopagebreak%
  \ifthenelse{\equal{#5}{}}{}{Coauthor(s): #5\vspace{3mm}\\\nopagebreak}%
  \ifthenelse{\equal{#6}{}}{}{Special session: #6\quad \vspace{3mm}\\\nopagebreak}%
 }
 {\vspace{1cm}\nopagebreak}%

\pagestyle{empty}

% ------------------------------------------------------------------------
% Document begins here
% ------------------------------------------------------------------------
\begin{document}
	
\begin{talk}
  {Complexity of approximating piecewise smooth functions in the presence of deterministic or random noise}% [1] talk title
  {Leszek Plaskota}% [2] speaker name
  {University of Warsaw}% [3] affiliations
  {L.Plaskota@mimuw.edu.pl}% [4] email
  {}% [5] coauthors
  {Stochastic Computation and Complexity}% [6] special session. Leave this field empty for contributed talks. 
				% Insert the title of the special session if you were invited to give a talk in a special session.
			
Consider the smoothness class of $1$-periodic functions $f:\mathbb R\to\mathbb R$ for which 
$$|f^{(r)}(x)-f^{(r)}(y)|\le |x-y|^\rho,\quad x,y\in\mathbb R,$$ 
where $r\in\{0,1,2,\ldots\}$ and $0<\rho\le 1.$ It is well known that the optimal worst case error of $L^p$-approximation ($1\le p\le\infty$) of such functions that can be achieved from $n$ exact evaluations of $f$ is proportional to $e_n=n^{-(r+\rho)}.$ Less obvious is what happens when the functions are piecewise smooth only with unknown break points. Even less obvious is the situation when the function values are additionally corrupted by some noise, i.e., when evaluating the value of $f$ at $x$ we obtain $y=f(x)+\xi$ where $|\xi|\le\delta$ (determnistic noise) or $\xi$ is a zero-mean random variable of variance $\sigma^2$ (random noise). In this talk we construct an algorithm which despite the presence of noise and break points achieves the worst case $L^p$-error still proportional to $e_n$ provided the noise level $\delta$ or $\sigma$ is of the same order $e_n$ (exept the case of $p=\infty$ and random noise where we have an additional logarithmic factor in the error). The algorithm uses divided differences and special adaptive extrapolation technique to locate the break points and approximate in their neighborhoods. 

\end{talk}

\end{document}
