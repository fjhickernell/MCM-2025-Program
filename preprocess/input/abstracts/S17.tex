\documentclass[12pt,a4paper,figuresright]{book}

\usepackage{amsmath,amssymb}
\usepackage{tabularx,multirow,graphicx,url,wrapfig,xcolor,rotating,multicol,epsfig,colortbl}

\setlength{\textheight}{25.2cm}
\setlength{\textwidth}{16.5cm} %\setlength{\textwidth}{18.2cm}
\setlength{\voffset}{-1.6cm}
\setlength{\hoffset}{-0.3cm} %\setlength{\hoffset}{-1.2cm}
\setlength{\evensidemargin}{-0.3cm}
\setlength{\oddsidemargin}{0.3cm}
\setlength{\parindent}{0cm}
\setlength{\parskip}{0.3cm}

\renewcommand{\topfraction}{1}
\renewcommand{\textfraction}{0}
\setlength{\floatsep}{12pt plus 2pt minus 2pt}

\newcommand{\organizer}[3]{%
	{\textit{#1}}\\\nopagebreak%
	#2\\\nopagebreak%
	\url{#3}\vspace{3mm}\\\nopagebreak%
	}

\newenvironment{session}[5] % [1] session title
							% [2] number of organizers
                            % [3] organizer 1 info
                            % [4] organizer 2 info
                            % [5] organizer 3 info
                            % [6] session id for later
 {%\needspace{6\baselineskip}
  \vskip 0pt\nopagebreak%
  %\label{#5}%
  \textbf{#1}\vspace{3mm}\\\nopagebreak%
  \ifthenelse{\equal{#2}{1}}{Organizer:}{Organizers:}%
  \vspace{2mm}\\\nopagebreak%
  #3
  \ifthenelse{\equal{#2}{2}}{#4}{}%
  \ifthenelse{\equal{#2}{3}}{#4#5}{}%
  \quad\\\nopagebreak%
  %Session Description:\vspace{3mm}\\\nopagebreak%
 }
 {\nopagebreak}%


\pagestyle{empty}

% ------------------------------------------------------------------------
% Document begins here
% ------------------------------------------------------------------------
\begin{document}

%Input the relevant information below
\begin{session}
  {Statistical Design of Experiments}% [1] session title
  {2} %[2]  number of organizers
  {\organizer{Lulu Kang}% organizer one name
    {University of Massachusetts Amherst}% orgnizer one affiliations
    {lulukang@umass.edu}}% organizer one email
  {\organizer{Chunfang Devon Lin}% organizer two name, if needed
	{Queen's University}% orgnizer two affiliations, if needed
	{devon.lin@queensu.ca}}% organizer two email
  {\organizer{Name three}% organizer one name
	{Affiliation(s) three}% orgnizer one affiliations
	{organizer-three-email-goes@here}}% organizer one email


\medskip

This session explores innovative methodologies for optimizing experimental design and factor analysis in complex, high-dimensional, and resource-constrained settings. The first talk introduces QuIP, a novel framework for designing experiments with qualitative factors using integer programming and Gaussian process models, demonstrating its effectiveness in path planning and rover trajectory optimization. The second talk addresses the challenge of cost-efficient predictive computing by proposing a multi-fidelity emulator design inspired by Multilevel Monte Carlo methods, which ensures predictive accuracy while minimizing computational costs under a tight budget. The third talk shifts focus to experiments involving both quantitative and sequence factors, presenting a new class of optimal quantitative-sequence (QS) designs that are flexible, space-filling, and asymptotically orthogonal, making them ideal for high-dimensional applications in medical science and bio-engineering. Finally, the fourth talk introduces FIRST, a model-free framework for factor importance ranking and selection using total Sobol' indices, offering a computationally efficient and consistent approach to identifying key factors in regression and classification tasks. Together, these talks highlight cutting-edge advancements in experimental design, optimization, and factor analysis, with broad applications across scientific and engineering disciplines.
\end{session}

We list the speakers and title of their talks below. 

\begin{enumerate}
\item Simon Mak, Assistant Professor, Department of Statistical Science at Duke University. \\
Talk title: QuIP: Experimental design for expensive simulators with many Qualitative factors via Integer Programming

% Abstract: The need to explore and/or optimize expensive simulators with many qualitative factors arises in broad scientific and engineering problems. Our motivating application lies in path planning -- the exploration of feasible paths for navigation, which plays an important role in robotics, surgical planning and assembly planning. Here, the feasibility of a path is evaluated via expensive virtual experiments, and its parameter space is typically discrete and high-dimensional. A carefully selected experimental design is thus essential for timely decision-making. We propose here a novel framework, called QuIP, for experimental design of Qualitative factors via integer programming under a Gaussian process surrogate model with exchangeable covariance kernel. For initial design, we show that its asymptotic D-optimal design can be formulated as a variant of the well-known assignment problem in operations research, which can be efficiently solved to global optimality using state-of-the-art integer programming solvers. For sequential design (specifically, for active learning or black-box optimization), we show that its design criterion can similarly be formulated as an assignment problem, thus enabling efficient and reliable optimization with existing solvers. We then demonstrate the effectiveness of QuIP over existing methods in a suite of path planning experiments and an application to rover trajectory optimization.

\item Chih-Li Sung, Assistant Professor in the Department of Statistics and Probability at Michigan State University.\\
Talk title: Stacking designs: designing multi-fidelity computer experiments with target predictive accuracy
% Abstract: In an era where scientific experiments can be very costly, multi-fidelity emulators provide a useful tool for cost-efficient predictive scientific computing. For scientific applications, the experimenter is often limited by a tight computational budget, and thus wishes to (i) maximize predictive power of the multi-fidelity emulator via a careful design of experiments, and (ii) ensure this model achieves a desired error tolerance with some notion of confidence. Existing design methods, however, do not jointly tackle objectives (i) and (ii). Inspired by the Multilevel Monte Carlo (MLMC) methods, we propose a novel stacking design approach that addresses both goals. A multi-level reproducing kernel Hilbert space (RKHS) interpolator is first introduced to build the emulator, under which our stacking design provides a sequential approach for designing multi-fidelity runs such that a desired prediction error is met under regularity assumptions. We then prove a novel cost complexity theorem that, under this multi-level interpolator, establishes a bound on the computation cost (for training data simulation) needed to achieve a prediction bound. This result provides novel insights on conditions under which the proposed multi-fidelity approach improves upon a conventional RKHS interpolator which relies on a single fidelity level.

\item Qian Xiao, Associate Professor in the Department of Statistics, School of Mathematical Sciences at Shanghai Jiao Tong University, Shanghai, China. \\
Talk title: Optimal Design of Experiments With Quantitative-sequence Factors
% Abstract: A new type of experiments with joint considerations of quantitative and sequence factors are recently drawing much attention in medical science, bio-engineering and many other disciplines. The input spaces of such experiments are semi-discrete and often very large. Thus, efficient and economic experimental designs are required. Based on the transformations and aggregations of good lattice point sets, we construct a new class of optimal quantitative-sequence (QS) designs which are marginally coupled, pair-balanced, space-filling and asymptotically orthogonal. The proposed QS designs have certain flexibility in run and factor sizes, and are especially appealing for high-dimensional cases.
\item Chaofan Huang, Ph.D. student in the School of Industrial and Systems Engineering at Georgia Institute of Technology. \\
Title: Factor Importance Ranking and Selection using Total Indices
% Abstract: Factor importance measures the impact of each feature on output prediction accuracy. In this paper, we focus on the intrinsic importance as proposed by Williamson et al. (2023), which defines the importance of a factor as the reduction in predictive potential when that factor is removed. To bypass the modeling step required by the existing estimator, we present the equivalence between predictiveness potential and total Sobol' indices from global sensitivity analysis, and introduce a novel model-free consistent estimator that can be directly computed from noisy data. Integrating with forward selection and backward elimination gives rise to FIRST, Factor Importance Ranking and Selection using Total (Sobol') indices. Extensive simulations are provided to demonstrate the effectiveness of FIRST on regression and binary classification problems, and a clear advantage over the state-of-the-art methods.
\end{enumerate}


\end{document}



The relevant papers/publications/preprints related to the talks are

%If you would like to include references, please do so by creating a simple list numbered by [1], [2], [3], \ldots. See example below.
%Please do not use the \texttt{bibliography} environment or \texttt{bibtex} files.
%APA reference style is recommended.
\begin{enumerate}
\item[{[1]}] Sung, C. L., Ji, Y., Mak, S., Wang, W., \& Tang, T. (2024). Stacking Designs: Designing Multifidelity Computer Experiments with Target Predictive Accuracy. \emph{SIAM/ASA Journal on Uncertainty Quantification}, 12(1), 157-181.
\end{enumerate}

%Equations may be used if they are referenced. Please note that the equation numbers may be different (but will be cross-referenced correctly) in the final program book.

