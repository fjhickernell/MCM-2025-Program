\documentclass[12pt,a4paper,figuresright]{book}

\usepackage{amsmath,amssymb}
\usepackage{tabularx,graphicx,url,xcolor,rotating,multicol,epsfig,colortbl}

\setlength{\textheight}{25.2cm}
\setlength{\textwidth}{16.5cm} %\setlength{\textwidth}{18.2cm}
\setlength{\voffset}{-1.6cm}
\setlength{\hoffset}{-0.3cm} %\setlength{\hoffset}{-1.2cm}
\setlength{\evensidemargin}{-0.3cm} 
\setlength{\oddsidemargin}{0.3cm}
\setlength{\parindent}{0cm} 
\setlength{\parskip}{0.3cm}

% -- adding a talk
\newenvironment{talk}[6]% [1] talk title
                         % [2] speaker name, [3] affiliations, [4] email,
                         % [5] coauthors, [6] special session
                         % [7] time slot
                         % [8] talk id, [9] session id or photo
 {%\needspace{6\baselineskip}%
  \vskip 0pt\nopagebreak%
%   \colorbox{gray!20!white}{\makebox[0.99\textwidth][r]{}}\nopagebreak%
%   \ifthenelse{\equal{#9}{photo}}{%
%                     \\\\\colorbox{gray!20!white}{\makebox{\includegraphics[width=3cm]{#8}}}\nopagebreak}{}%
 \vskip 0pt\nopagebreak%
%  \label{#8}%
  \textbf{#1}\vspace{3mm}\\\nopagebreak%
  \textit{#2}\\\nopagebreak%
  #3\\\nopagebreak%
  \url{#4}\vspace{3mm}\\\nopagebreak%
  \ifthenelse{\equal{#5}{}}{}{Coauthor(s): #5\vspace{3mm}\\\nopagebreak}%
  \ifthenelse{\equal{#6}{}}{}{Special session: #6\quad \vspace{3mm}\\\nopagebreak}%
 }
 {\vspace{1cm}\nopagebreak}%

\pagestyle{empty}

% ------------------------------------------------------------------------
% Document begins here
% ------------------------------------------------------------------------
\begin{document}
	
\begin{talk}
  {Model Problems for PDEs on Uncertain Domains}% [1] talk title
  {Harri Hakula}% [2] speaker name
  {Aalto University}% [3] affiliations
  {Harri.Hakula@aalto.fi}% [4] email
  {}% [5] coauthors
  {Domain Uncertainty Quantification}% [6] special session. Leave this field empty for contributed talks. 
				% Insert the title of the special session if you were invited to give a talk in a special session.
			
%Your abstract goes here. Please do not use your own commands or macros.
Partial differential equation related uncertainty quantification has become one of the
topical research areas in applied mathematics and, in particular, engineering.
Stochastic finite element methods are applied both in source and eigenvalue problems.
Remarkably, computational function theory provides a rich set of invariants 
and identities that
can be applied in designing model problems where the domain is random or uncertain. 
In this talk the focus is on conformal capacity in a simple, 
yet general case where the sides of a quadrilateral are assumed be random 
and parameterised with a suitable Karhunen-Loève expansion [1].
Lattice quasi-Monte Carlo (QMC) cubature rules are used for computing the expected value of the solution to the resulting Poisson problem subject to domain uncertainty. 

High-order finite element methods ($hp$-FEM) are used in the deterministic problems.
The special features related to modelling random domains in $hp$-context are discussed.
Convergence properties of the lattice QMC quadratures are presented. The talk
concentrates on numerical experiments demonstrating the theoretical error estimates.
The new results on the associated Steklov eigenvalue problem are also covered.


\medskip

\begin{enumerate}
	\item[{[1]}] Hakula, H., Harbrecht, H., Kaarnioja, V., Kuo, F. Y., \& Sloan, I. H. (2024). Uncertainty quantification for random domains using periodic random variables. Numerische Mathematik, 156(2), 273–317. https://doi.org/10.1007/s00211-023-01392-6
\end{enumerate}

% Equations may be used if they are referenced. Please note that the equation numbers may be different (but will be cross-referenced correctly) in the final program book.
\end{talk}

\end{document}

