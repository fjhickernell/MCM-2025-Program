\documentclass[12pt,a4paper,figuresright]{book}

\usepackage{amsmath,amssymb}
\usepackage{tabularx,multirow,graphicx,url,wrapfig,xcolor,rotating,multicol,epsfig,colortbl,verbatim}

\setlength{\textheight}{25.2cm}
\setlength{\textwidth}{16.5cm} %\setlength{\textwidth}{18.2cm}
\setlength{\voffset}{-1.6cm}
\setlength{\hoffset}{-0.3cm} %\setlength{\hoffset}{-1.2cm}
\setlength{\evensidemargin}{-0.3cm}
\setlength{\oddsidemargin}{0.3cm}
\setlength{\parindent}{0cm}
\setlength{\parskip}{0.3cm}

\renewcommand{\topfraction}{1}
\renewcommand{\textfraction}{0}
\setlength{\floatsep}{12pt plus 2pt minus 2pt}

\newcommand{\organizer}[3]{%
	{\textit{#1}}\\\nopagebreak%
	#2\\\nopagebreak%
	\url{#3}\vspace{3mm}\\\nopagebreak%
	}

\newenvironment{session}[5] % [1] session title
							% [2] number of organizers
                            % [3] organizer 1 info
                            % [4] organizer 2 info
                            % [5] organizer 3 info
                            % [6] session id for later
 {%\needspace{6\baselineskip}
  \vskip 0pt\nopagebreak%
  %\label{#5}%
  \textbf{#1}\vspace{3mm}\\\nopagebreak%
  \ifthenelse{\equal{#2}{1}}{Organizer:}{Organizers:}%
  \vspace{2mm}\\\nopagebreak%
  #3
  \ifthenelse{\equal{#2}{2}}{#4}{}%
  \ifthenelse{\equal{#2}{3}}{#4#5}{}%
  \quad\\\nopagebreak%
  %Session Description:\vspace{3mm}\\\nopagebreak%
 }
 {\nopagebreak}%


\pagestyle{empty}

% ------------------------------------------------------------------------
% Document begins here
% ------------------------------------------------------------------------
\begin{document}

%Input the relevant information below
\begin{session}
  {Hardware or Software for (Quasi-)Monte Carlo Algorithms}% [1] session title
  {3} %[2]  number of organizers
  {\organizer{Sou-Cheng T.  Choi}% organizer one name
    {Illinois Institute of Technology}% orgnizer one affiliations
    {schoi32@iit.edu}}% organizer one email
  {\organizer{Pieterjan Robbe}% organizer two name, if needed
	{Sandia National Laboratories}% orgnizer two affiliations, if needed
	{pmrobbe@sandia.gov}}% organizer two email
  {\organizer{Mike Giles}% organizer three name
	{University of Oxford}% orgnizer three affiliations
	{mike.giles@maths.ox.ac.uk}}% organizer three email

Monte Carlo (MC) or quasi-Monte Carlo (QMC) algorithms are widely used in various fields such as finance, physics, and engineering for their ability to handle high-dimensional integration problems. The development and maintenance of software for (quasi-)Monte Carlo ((Q)MC) algorithms can significantly enhance the accessibility and usability of these techniques. This special session aims to bring together experts from academia and industry to discuss recent advances in (Q)MC software, share best practices, and explore future directions, fostering collaboration among researchers and practitioners.

Topics of interest for the session include:
\begin{itemize}
    \item Novel hardware or architectural designs for open-source (Q)MC libraries.
    \item Best collaborative practices for developing and maintaining efficient and reliable (Q)MC software.
    \item Challenges and opportunities in integrating (Q)MC methods with machine learning and AI techniques.
    \item High-performance computing solutions for (Q)MC software.
    \item Adaptation of (Q)MC software to application fields such as finance, computer graphics, sensitivity analysis, Bayesian optimization, and uncertainty quantification.
    \item Innovative approaches to enhancing and extending existing (Q)MC tools.
\end{itemize}

\end{session}

\end{document}

