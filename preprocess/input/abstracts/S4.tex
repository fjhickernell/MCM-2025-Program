\documentclass[12pt,a4paper,figuresright]{book}

\usepackage{amsmath,amssymb}
\usepackage{tabularx,multirow,graphicx,url,wrapfig,xcolor,rotating,multicol,epsfig,colortbl,verbatim}

\setlength{\textheight}{25.2cm}
\setlength{\textwidth}{16.5cm} %\setlength{\textwidth}{18.2cm}
\setlength{\voffset}{-1.6cm}
\setlength{\hoffset}{-0.3cm} %\setlength{\hoffset}{-1.2cm}
\setlength{\evensidemargin}{-0.3cm}
\setlength{\oddsidemargin}{0.3cm}
\setlength{\parindent}{0cm}
\setlength{\parskip}{0.3cm}

\renewcommand{\topfraction}{1}
\renewcommand{\textfraction}{0}
\setlength{\floatsep}{12pt plus 2pt minus 2pt}

\newcommand{\organizer}[3]{%
	{\textit{#1}}\\\nopagebreak%
	#2\\\nopagebreak%
	\url{#3}\vspace{3mm}\\\nopagebreak%
	}

\newenvironment{session}[5] % [1] session title
							% [2] number of organizers
                            % [3] organizer 1 info
                            % [4] organizer 2 info
                            % [5] organizer 3 info
                            % [6] session id for later
 {%\needspace{6\baselineskip}
  \vskip 0pt\nopagebreak%
  %\label{#5}%
  \textbf{#1}\vspace{3mm}\\\nopagebreak%
  \ifthenelse{\equal{#2}{1}}{Organizer:}{Organizers:}%
  \vspace{2mm}\\\nopagebreak%
  #3
  \ifthenelse{\equal{#2}{2}}{#4}{}%
  \ifthenelse{\equal{#2}{3}}{#4#5}{}%
  \quad\\\nopagebreak%
  %Session Description:\vspace{3mm}\\\nopagebreak%
 }
 {\nopagebreak}%


\pagestyle{empty}

% ------------------------------------------------------------------------
% Document begins here
% ------------------------------------------------------------------------
\begin{document}

%Input the relevant information below
\begin{session}
  {Hardware or Software for (Quasi-)Monte Carlo Algorithms}% [1] session title
  {3} %[2]  number of organizers
  {\organizer{Sou-Cheng T.  Choi}% organizer one name
    {Illinois Institute of Technology}% orgnizer one affiliations
    {schoi32@iit.edu}}% organizer one email
  {\organizer{Pieterjan Robbe}% organizer two name, if needed
	{Sandia National Laboratories}% orgnizer two affiliations, if needed
	{pmrobbe@sandia.gov}}% organizer two email
  {\organizer{Mike Giles}% organizer one name
	{University of Oxford}% orgnizer one affiliations
	{mike.giles@maths.ox.ac.uk}}% organizer one email

Monte Carlo (MC) or quasi-Monte Carlo (QMC) algorithms are widely used in various fields such as finance, physics, and engineering for their ability to handle high-dimensional integration problems. The development and maintenance of software for (quasi-)Monte Carlo ((Q)MC) algorithms can significantly enhance the accessibility and usability of these techniques. This special session aims to bring together experts from academia and industry to discuss recent advances in (Q)MC software, share best practices, and explore future directions, fostering collaboration among researchers and practitioners.

Topics of interest for the session include:
\begin{itemize}
    \item Novel hardware or architectural designs for open-source (Q)MC libraries.
    \item Best collaborative practices for developing and maintaining efficient and reliable (Q)MC software.
    \item Challenges and opportunities in integrating (Q)MC methods with machine learning and AI techniques.
    \item High-performance computing solutions for (Q)MC software.
    \item Adaptation of (Q)MC software to application fields such as finance, computer graphics, sensitivity analysis, Bayesian optimization, and uncertainty quantification.
    \item Innovative approaches to enhancing and extending existing (Q)MC tools.
\end{itemize}

\end{session}
Committed Speakers and Topics:
\begin{itemize}
\item Part 1 of the Special Session:
\begin{itemize}
    \item Speaker 1: Pieterjan Robbe, Sandia National Laboratories, Multifidelity QMC development in Dakota (https://dakota.sandia.gov/), \texttt{pmrobbe@sandia.gov}
    \item Speaker 2: Irina-Beatrice Haas, University of Oxford,  MLMC for FPGAs, \newline \texttt{Irina-Beatrice.Haas@maths.ox.ac.uk} 
    \item Speaker 3: Mike Giles, University of Oxford, CUDA implementation of MLMC (\url{https://people.maths.ox.ac.uk/gilesm/mlmc/}), \texttt{mike.giles@maths.ox.ac.uk}
    \item Speaker 4: Chung Ming Loi,  Durham University, UM-Bridge (\url{https://github.com/um-bridge}), \texttt{chung.m.loi@durham.ac.uk} %PhD student of Anne Reinarz 
\end{itemize}
\item Part 2 of the Special Session:
\begin{itemize}
    \item Speaker 5:  Niklas Baumgarten, University of Heidelberg, Software for Multilevel Monte Carlo Methods, \texttt{niklas.baumgarten@uni-heidelberg.de}
    \item Speaker 6: Aleksei Sorokin,  Illinois Institute of Technology, QMCPy's Randomization Routines and Fast Kernel Interpolation, \texttt{asorokin@hawk.iit.edu}
    \item Speaker 7:  Johannes Krotz, University of Notre Dame, Methods and Software for Hybrid Q/MC Solvers for Radiation Transport, \texttt{jkrotz@nd.edu} %postdoc of Ryan McClarren
    \item Speaker 8: Joseph Farmer, University of Notre Dame, High Performance Calculations of Radiation Emission from High Temperature Fluid Flow, \texttt{jfarmer4@nd.edu} % PhD Student of Ryan McClarren

\end{itemize}
\end{itemize}


\medskip
\begin{comment}
If you would like to include references, please do so by creating a simple list numbered by [1], [2], [3], \ldots. See example below.
Please do not use the \texttt{bibliography} environment or \texttt{bibtex} files.
%APA reference style is recommended.
\begin{enumerate}
	\item[{[1]}] Niederreiter, Harald (1992). {\it Random number generation and quasi-Monte Carlo methods}. Society for Industrial and Applied Mathematics (SIAM).
	\item[{[2]}] Roberts, Gareth O, \& Rosenthal, Jeffrey S. (2002).  Optimal scaling for various Metropolis-Hastings algorithms, \textbf{16}(4), 351--367.
\end{enumerate}

Equations may be used if they are referenced. Please note that the equation numbers may be different (but will be cross-referenced correctly) in the final program book.
\end{comment}


\end{document}

