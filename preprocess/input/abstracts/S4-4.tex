\documentclass[12pt,a4paper,figuresright]{book}

\usepackage{amsmath,amssymb}
\usepackage{tabularx,graphicx,url,xcolor,rotating,multicol,epsfig,colortbl}

\setlength{\textheight}{25.2cm}
\setlength{\textwidth}{16.5cm} %\setlength{\textwidth}{18.2cm}
\setlength{\voffset}{-1.6cm}
\setlength{\hoffset}{-0.3cm} %\setlength{\hoffset}{-1.2cm}
\setlength{\evensidemargin}{-0.3cm} 
\setlength{\oddsidemargin}{0.3cm}
\setlength{\parindent}{0cm} 
\setlength{\parskip}{0.3cm}

% -- adding a talk
\newenvironment{talk}[6]% [1] talk title
                         % [2] speaker name, [3] affiliations, [4] email,
                         % [5] coauthors, [6] special session
                         % [7] time slot
                         % [8] talk id, [9] session id or photo
 {%\needspace{6\baselineskip}%
  \vskip 0pt\nopagebreak%
%   \colorbox{gray!20!white}{\makebox[0.99\textwidth][r]{}}\nopagebreak%
%   \ifthenelse{\equal{#9}{photo}}{%
%                     \\\\\colorbox{gray!20!white}{\makebox{\includegraphics[width=3cm]{#8}}}\nopagebreak}{}%
 \vskip 0pt\nopagebreak%
%  \label{#8}%
  \textbf{#1}\vspace{3mm}\\\nopagebreak%
  \textit{#2}\\\nopagebreak%
  #3\\\nopagebreak%
  \url{#4}\vspace{3mm}\\\nopagebreak%
  \ifthenelse{\equal{#5}{}}{}{Coauthor(s): #5\vspace{3mm}\\\nopagebreak}%
  \ifthenelse{\equal{#6}{}}{}{Special session: #6\quad \vspace{3mm}\\\nopagebreak}%
 }
 {\vspace{1cm}\nopagebreak}%

\pagestyle{empty}

% ------------------------------------------------------------------------
% Document begins here
% ------------------------------------------------------------------------
\begin{document}
	
\begin{talk}
  {Scalable and User-friendly QMC Sampling with UMBridge}% [1] talk title
  {Chung Ming Loi}% [2] speaker name
  {Durham University}% [3] affiliations
  {chung.m.loi@durham.ac.uk}% [4] email
  {Anne Reinarz}% [5] coauthors  Will and James?
  {Hardware and Software for Quasi-Monte Carlo Methods}% [6] special session. Leave this field empty for contributed talks. 
				% Insert the title of the special session if you were invited to give a talk in a special session.
			
Uncertainty quantification (UQ) plays a crucial role in geoscience: Bayesian inference determines model parameters, such as the permeability and porosity of the sub-surface, that are typically impossible to determine accurately from observations. In practice, it is crucial to study the uncertainty in the inferred parameters to correctly quantify risk and make decisions. Despite its scientific value, performing UQ for an application is often a lengthy process due to a need for interdisciplinary expertise in both UQ and advanced simulation codes. In this talk, we will look at improving the workflow and computational efficiency of quasi-Monte Carlo (i.e., sampling/ensemble based) approaches to UQ applications. We introduce UM-Bridge [2], a universal software interface that facilitates integration of complex simulation models with an entire range of leading UQ packages. By separating concerns between simulation and UQ, UM-Bridge allows rapid development of cutting-edge applications. The newly implemented load balancing framework in UM-Bridge further enables scaling workloads to High Performance Computing clusters.

\begin{enumerate}
    \item[{[1]}] Graham, I. G., Kuo, F. Y., Nuyens, D., Scheichl, R., and Sloan, I. H. (2011). Quasi-Monte Carlo methods for elliptic PDEs with random coefficients and applications. Journal of Computational Physics, 230(10), 3668-3694.
    \item[{[2]}]  L. Seelinger, V. Cheng-Seelinger, A. Davis, M. Parno, and A. Reinarz, “UM-Bridge: Uncertainty quantification and modeling bridge,” Journal of Open Source Software, vol. 8, no. 83, p. 4748, 2023. [Online]. Available: https://doi.org/10.21105/joss.04748
\end{enumerate}

\end{talk}

\end{document}

