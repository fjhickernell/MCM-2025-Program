\documentclass[12pt,a4paper,figuresright]{book}

\usepackage{amsmath,amssymb}
\usepackage{tabularx,graphicx,url,xcolor,rotating,multicol,epsfig,colortbl}

\setlength{\textheight}{25.2cm}
\setlength{\textwidth}{16.5cm} %\setlength{\textwidth}{18.2cm}
\setlength{\voffset}{-1.6cm}
\setlength{\hoffset}{-0.3cm} %\setlength{\hoffset}{-1.2cm}
\setlength{\evensidemargin}{-0.3cm} 
\setlength{\oddsidemargin}{0.3cm}
\setlength{\parindent}{0cm} 
\setlength{\parskip}{0.3cm}

% -- adding a talk
\newenvironment{talk}[6]% [1] talk title
                         % [2] speaker name, [3] affiliations, [4] email,
                         % [5] coauthors, [6] special session
                         % [7] time slot
                         % [8] talk id, [9] session id or photo
 {%\needspace{6\baselineskip}%
  \vskip 0pt\nopagebreak%
%   \colorbox{gray!20!white}{\makebox[0.99\textwidth][r]{}}\nopagebreak%
%   \ifthenelse{\equal{#9}{photo}}{%
%                     \\\\\colorbox{gray!20!white}{\makebox{\includegraphics[width=3cm]{#8}}}\nopagebreak}{}%
 \vskip 0pt\nopagebreak%
%  \label{#8}%
  \textbf{#1}\vspace{3mm}\\\nopagebreak%
  \textit{#2}\\\nopagebreak%
  #3\\\nopagebreak%
  \url{#4}\vspace{3mm}\\\nopagebreak%
  \ifthenelse{\equal{#5}{}}{}{Coauthor(s): #5\vspace{3mm}\\\nopagebreak}%
  \ifthenelse{\equal{#6}{}}{}{Special session: #6\quad \vspace{3mm}\\\nopagebreak}%
 }
 {\vspace{1cm}\nopagebreak}%

\pagestyle{empty}

% ------------------------------------------------------------------------
% Document begins here
% ------------------------------------------------------------------------
\begin{document}
	
\begin{talk}
  {Fast Approximate Matrix Inversion via MCMC for Linear System Solvers}% [1] talk title
  {Soumyadip Ghosh}% [2] speaker name
  {IBM Research}% [3] affiliations
  {ghoshs@us.ibm.com}% [4] email
  {Vassil Alexandrov, Lior Horesh, Vasilieos Kalantzis, Anton Lebedev,  WonKyung Lee, Yingdong Lu, Tomasz Nowicki, Shashanka Ubaru, Olha Yaman}% [5] coauthors
  {}% [6] special session. Leave this field empty for contributed talks. 
				% Insert the title of the special session if you were invited to give a talk in a special session.
			
%Your abstract goes here. Please do not use your own commands or macros.
A key prerequisite of modern iterative solvers of linear algebraic equations $Ax=b$ is the fast computation of a pre-conditioner matrix $P$ that gives a good approximation to the (generalized) inverse of $A$ such that the set of equations obtained by pre-multiplying with $P$, $PAx=Pb$, is solved quickly. 
We study the classical Ulan-von Neumann MCMC algorithm that was designed based on the Neumann infinite series representation of the inverse of a non-singular matrix. The parameters of the MCMC algorithm determine the overall time to solve $Ax=b$, which is a metric that 
reflects both the time to compute the MCMC preconditioner $P$ and its quality as a preconditioner to solve the linear equations. Our main focus is on how the MCMC parameters should be tuned to speed up computations in applications that require repeated calls to the solver with varying matrices $A$, a common scenario for instance in numerical approximations of physical phenomena. 
We present a model that relates key features of matrices $A$ with good choices of MCMC algorithm parameters that lead to a fast overall time to find a solution to $Ax=b$. A computationally efficient approach based on Bayesian experimental design is described to learn and update this model while minimizing the number of runs of the expensive solver in application settings that solve of linear system over well defined sets of $A$ matrices. We present numerical experiments to illustrate the efficacy of this approach.
In another contribution, we present a new MCMC algorithm which we term as \emph{regenerative Ulam-von Neumann} algorithm. It exploits a regenerative structure present in the Neumann series that underlies the original algorithm and improves on it by producing an unbiased estimator of the matrix inverse. A rigorous analysis of performance of the algorithm is provided. This includes the variance of the estimator, which allows one to estimate the time taken to obtain solutions of a desired quality. Finally, numerical experiments verify the qualitative effectiveness of the proposed scheme. 

\end{talk}

\end{document}

