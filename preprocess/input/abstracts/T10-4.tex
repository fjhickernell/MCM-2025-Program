\documentclass[12pt,a4paper,figuresright]{book}

\usepackage{amsmath,amssymb}
\usepackage{tabularx,graphicx,url,xcolor,rotating,multicol,epsfig,colortbl}
\usepackage{hyperref}

\setlength{\textheight}{25.2cm}
\setlength{\textwidth}{16.5cm} %\setlength{\textwidth}{18.2cm}
\setlength{\voffset}{-1.6cm}
\setlength{\hoffset}{-0.3cm} %\setlength{\hoffset}{-1.2cm}
\setlength{\evensidemargin}{-0.3cm} 
\setlength{\oddsidemargin}{0.3cm}
\setlength{\parindent}{0cm} 
\setlength{\parskip}{0.3cm}

% -- adding a talk
\newenvironment{talk}[6]% [1] talk title
                         % [2] speaker name, [3] affiliations, [4] email,
                         % [5] coauthors, [6] special session
                         % [7] time slot
                         % [8] talk id, [9] session id or photo
 {%\needspace{6\baselineskip}%
  \vskip 0pt\nopagebreak%
%   \colorbox{gray!20!white}{\makebox[0.99\textwidth][r]{}}\nopagebreak%
%   \ifthenelse{\equal{#9}{photo}}{%
%                     \\\\\colorbox{gray!20!white}{\makebox{\includegraphics[width=3cm]{#8}}}\nopagebreak}{}%
 \vskip 0pt\nopagebreak%
%  \label{#8}%
  \textbf{#1}\vspace{3mm}\\\nopagebreak%
  \textit{#2}\\\nopagebreak%
  #3\\\nopagebreak%
  \url{#4}\vspace{3mm}\\\nopagebreak%
  \ifthenelse{\equal{#5}{}}{}{Coauthor(s): #5\vspace{3mm}\\\nopagebreak}%
  \ifthenelse{\equal{#6}{}}{}{Special session: #6\quad \vspace{3mm}\\\nopagebreak}%
 }
 {\vspace{1cm}\nopagebreak}%

\pagestyle{empty}

% ------------------------------------------------------------------------
% Document begins here
% ------------------------------------------------------------------------
\begin{document}
	
\begin{talk}
  {Taming the Interacting Particle Langevin Algorithm – The Superlinear Case}% [1] talk title
  {Nikolaos Makras}% [2] speaker name
  {University of Edinburgh, School of Mathematics}% [3] affiliations
  {N.Makras@sms.ed.ac.uk}% [4] email
  {Tim Johnston, Sotirios Sabanis}% [5] coauthors
  {}% [6] special session. Leave this field empty for contributed talks. 
				% Insert the title of the special session if you were invited to give a talk in a special session.
			
Recent advances in stochastic optimization have yielded the interacting particle Langevin algorithm (IPLA), which leverages the notion of interacting particle systems (IPS) to efficiently sample from approximate posterior densities. This becomes particularly crucial in relation to the framework of Expectation-Maximization (EM), where the E-step is computationally challenging or even intractable. Although prior research has focused on scenarios involving convex cases with gradients of log densities that grow at most linearly, our work extends this framework to include polynomial growth. Taming techniques are employed to produce an explicit discretization scheme that yields a new class of stable, under such non-linearities, algorithms which are called tamed interacting particle Langevin algorithms (tIPLA). We obtain non-asymptotic convergence error estimates in Wasserstein-2 distance for the new class under the best known rate. 

\medskip

\begin{enumerate}
	\item[{[1]}] Tim Johnston, Nikolaos Makras and Sotirios Sabanis, {\it Taming the Interacting Particle Langevin Algorithm – The Superlinear Case} (2024). Preprint: \href{https://arxiv.org/abs/2403.19587}{arxiv:2403.19587} 
\end{enumerate}

\end{talk}

\end{document}

