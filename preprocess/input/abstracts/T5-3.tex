\documentclass[12pt,a4paper,figuresright]{book}

\usepackage{amsmath,amssymb}
\usepackage{tabularx,graphicx,url,xcolor,rotating,multicol,epsfig,colortbl}

\setlength{\textheight}{25.2cm}
\setlength{\textwidth}{16.5cm} %\setlength{\textwidth}{18.2cm}
\setlength{\voffset}{-1.6cm}
\setlength{\hoffset}{-0.3cm} %\setlength{\hoffset}{-1.2cm}
\setlength{\evensidemargin}{-0.3cm} 
\setlength{\oddsidemargin}{0.3cm}
\setlength{\parindent}{0cm} 
\setlength{\parskip}{0.3cm}

% -- adding a talk
\newenvironment{talk}[6]% [1] talk title
                         % [2] speaker name, [3] affiliations, [4] email,
                         % [5] coauthors, [6] special session
                         % [7] time slot
                         % [8] talk id, [9] session id or photo
 {%\needspace{6\baselineskip}%
  \vskip 0pt\nopagebreak%
%   \colorbox{gray!20!white}{\makebox[0.99\textwidth][r]{}}\nopagebreak%
%   \ifthenelse{\equal{#9}{photo}}{%
%                     \\\\\colorbox{gray!20!white}{\makebox{\includegraphics[width=3cm]{#8}}}\nopagebreak}{}%
 \vskip 0pt\nopagebreak%
%  \label{#8}%
  \textbf{#1}\vspace{3mm}\\\nopagebreak%
  \textit{#2}\\\nopagebreak%
  #3\\\nopagebreak%
  \url{#4}\vspace{3mm}\\\nopagebreak%
  \ifthenelse{\equal{#5}{}}{}{Coauthor(s): #5\vspace{3mm}\\\nopagebreak}%
  \ifthenelse{\equal{#6}{}}{}{Special session: #6\quad \vspace{3mm}\\\nopagebreak}%
 }
 {\vspace{1cm}\nopagebreak}%

\pagestyle{empty}

% ------------------------------------------------------------------------
% Document begins here
% ------------------------------------------------------------------------
\begin{document}
	
\begin{talk}
  {Use of rank-1 lattices in the Fourier neural operator}% [1] talk title
  {Jakob Dilen}% [2] speaker name
  {Department of Computer Science, KU Leuven}% [3] affiliations
  {jakob.dilen@student.kuleuven.be}% [4] email
  {Frances Y. Kuo, Dirk Nuyens}% [5] coauthors
  {}% [6] special session. Leave this field empty for contributed talks. 
				% Insert the title of the special session if you were invited to give a talk in a special session.

The ``Fourier neural operator'' [2] is a variant of the ``neural operator''.
Its defining characteristic, compared to the regular neural operator, is that it transforms the input to the Fourier domain at the start of each layer. This transformation uses the $d$-dimensional FFT on a regular grid in $d$ dimensions. We describe how to do this more efficiently using rank-$1$ lattice points, which allow for a one-dimensional FFT algorithm, see, e.g., [1]. 

\medskip
\begin{enumerate}
        \item[{[1]}] F. Y. Kuo, G. Migliorati, F. Nobile, and D. Nuyens. \textit{Function integration,
reconstruction and approximation using rank-1 lattices}. Math. Comp., 90, April
2021.   
        \item[{[2]}] Z. Li, N. B. Kovachki, K. Azizzadenesheli, B. liu, K. Bhattacharya, A. Stuart,
and A. Anandkumar. \textit{Fourier neural operator for parametric partial differential
equations}. International Conference on Learning Representations, 2021.
\end{enumerate}
\end{talk}

\end{document}

