\documentclass[12pt,a4paper,figuresright]{book}

\usepackage{amsmath,amssymb}
\usepackage{tabularx,graphicx,url,xcolor,rotating,multicol,epsfig,colortbl}

\setlength{\textheight}{25.2cm}
\setlength{\textwidth}{16.5cm} %\setlength{\textwidth}{18.2cm}
\setlength{\voffset}{-1.6cm}
\setlength{\hoffset}{-0.3cm} %\setlength{\hoffset}{-1.2cm}
\setlength{\evensidemargin}{-0.3cm} 
\setlength{\oddsidemargin}{0.3cm}
\setlength{\parindent}{0cm} 
\setlength{\parskip}{0.3cm}

% -- adding a talk
\newenvironment{talk}[6]% [1] talk title
                         % [2] speaker name, [3] affiliations, [4] email,
                         % [5] coauthors, [6] special session
                         % [7] time slot
                         % [8] talk id, [9] session id or photo
 {%\needspace{6\baselineskip}%
  \vskip 0pt\nopagebreak%
%   \colorbox{gray!20!white}{\makebox[0.99\textwidth][r]{}}\nopagebreak%
%   \ifthenelse{\equal{#9}{photo}}{%
%                     \\\\\colorbox{gray!20!white}{\makebox{\includegraphics[width=3cm]{#8}}}\nopagebreak}{}%
 \vskip 0pt\nopagebreak%
%  \label{#8}%
  \textbf{#1}\vspace{3mm}\\\nopagebreak%
  \textit{#2}\\\nopagebreak%
  #3\\\nopagebreak%
  \url{#4}\vspace{3mm}\\\nopagebreak%
  \ifthenelse{\equal{#5}{}}{}{Coauthor(s): #5\vspace{3mm}\\\nopagebreak}%
  \ifthenelse{\equal{#6}{}}{}{Special session: #6\quad \vspace{3mm}\\\nopagebreak}%
 }
 {\vspace{1cm}\nopagebreak}%

\pagestyle{empty}

% ------------------------------------------------------------------------
% Document begins here
% ------------------------------------------------------------------------
\begin{document}
	
\begin{talk}
  {Hybrid least squares for learning functions from highly noisy data}% [1] talk title
  {Yiming Xu}% [2] speaker name
  {University of Kentucky}% [3] affiliations
  {yiming.xu@uky.edu}% [4] email
  {Ben Adcock, Bernhard Hientzsch, Akil Narayan}% [5] coauthors
  {}% [6] special session. Leave this field empty for contributed talks. 
				% Insert the title of the special session if you were invited to give a talk in a special session.
			
Motivated by the request for efficient estimation of conditional expectations, we consider a least-squares function approximation problem with heavily polluted data. In such scenarios, existing methods based on the small noise assumption become suboptimal. We propose a hybrid approach that combines Christoffel sampling with optimal experimental design to address this issue. The proposed algorithm adheres to appropriate optimality criteria for both sample points generation and function evaluation, leading to improved computational efficiency and sample complexity. We also extend the algorithm to convex-constrained settings with similar theoretical guarantees. Moreover, when the target function is defined as the expectation of a random field, we introduce adaptive random subspaces to approximate the target function and establish results concerning its approximation capacity. Our findings are corroborated through numerical studies on synthetic data and a more challenging stochastic simulation problem in computational finance.
\medskip

\end{talk}

\end{document}

