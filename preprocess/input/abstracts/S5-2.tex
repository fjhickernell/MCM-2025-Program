\documentclass[12pt,a4paper,figuresright]{book}

\usepackage{amsmath,amssymb}
\usepackage{tabularx,graphicx,url,xcolor,rotating,multicol,epsfig,colortbl}

\setlength{\textheight}{25.2cm}
\setlength{\textwidth}{16.5cm} %\setlength{\textwidth}{18.2cm}
\setlength{\voffset}{-1.6cm}
\setlength{\hoffset}{-0.3cm} %\setlength{\hoffset}{-1.2cm}
\setlength{\evensidemargin}{-0.3cm} 
\setlength{\oddsidemargin}{0.3cm}
\setlength{\parindent}{0cm} 
\setlength{\parskip}{0.3cm}

% -- adding a talk
\newenvironment{talk}[6]% [1] talk title
                         % [2] speaker name, [3] affiliations, [4] email,
                         % [5] coauthors, [6] special session
                         % [7] time slot
                         % [8] talk id, [9] session id or photo
 {%\needspace{6\baselineskip}%
  \vskip 0pt\nopagebreak%
%   \colorbox{gray!20!white}{\makebox[0.99\textwidth][r]{}}\nopagebreak%
%   \ifthenelse{\equal{#9}{photo}}{%
%                     \\\\\colorbox{gray!20!white}{\makebox{\includegraphics[width=3cm]{#8}}}\nopagebreak}{}%
 \vskip 0pt\nopagebreak%
%  \label{#8}%
  \textbf{#1}\vspace{3mm}\\\nopagebreak%
  \textit{#2}\\\nopagebreak%
  #3\\\nopagebreak%
  \url{#4}\vspace{3mm}\\\nopagebreak%
  \ifthenelse{\equal{#5}{}}{}{Coauthor(s): #5\vspace{3mm}\\\nopagebreak}%
  \ifthenelse{\equal{#6}{}}{}{Special session: #6\quad \vspace{3mm}\\\nopagebreak}%
 }
 {\vspace{1cm}\nopagebreak}%

\pagestyle{empty}

% ------------------------------------------------------------------------
% Document begins here
% ------------------------------------------------------------------------
\begin{document}
	
\begin{talk}
  {Optimality of deterministic and randomized QMC-cubatures on several scales of function spaces}% [1] talk title
  {Michael Gnewuch}% [2] speaker name
  {University of Osnabrueck}% [3] affiliations
  {mgnewuch@uos.de}% [4] email
  {Josef Dick, Lev Markhasin, Winfried Sickel, Yannick Meiners}% [5] coauthors
  {Stochastic Computation and Complexity}% [6] special session. Leave this field empty for contributed talks. 
				% Insert the title of the special session if you were invited to give a talk in a special session.
			
			
We study the integration problem over the s-dimensional unit cube on four scales of Banach spaces of integrands. First we consider Haar wavelet spaces $H_{p, q, \alpha}$, $1\le p, q \le \infty$, $\alpha > 1/p$, consisting of functions whose Haar wavelet coefficients exhibit a certain decay behavior measured by the parameters $p,q$, and, most importantly, $\alpha$.  We study the worst case error of a deterministic cubature rule over the norm unit ball 
%(i.e., the operator norm of the difference of the integration functional and the cubature rule)
and provide upper bounds for quasi-Monte Carlo (QMC) cubature rules based on arbitrary $(t,m,s)$-nets as well as matching lower error bounds for arbitrary cubature rules. These results show that using arbitrary $(t,m,s)$-nets as integration nodes yields the best possible rate of convergence. In the Hilbert space setting $p=2 = q$ it was earlier shown by Heinrich, Hickernell and Yue [2]  that scrambled (t,m,s)-nets yield optimal convergence rates in the randomized setting, where the randomized worst case error is considered. 

We establish several suitable function space embeddings that allow to transfer the deterministic and randomized upper error bounds on Haar wavelet spaces
to certain spaces of fractional smoothness $1/p < \alpha  \le 1$ and to Sobolev and Besov spaces of dominating mixed smoothness $1/p < a \le 1$.
Known lower bounds for Sobolev and Besov spaces of dominating mixed smoothness show that (deterministic or suitably randomized) $(t,m,s)$-nets yield optimal convergence rates also on the corresponding scales of spaces.			

The talk is based on the preprint [1] and the master thesis of my student Yannick Meiners.
\medskip

\begin{enumerate}
	\item[{[1]}] M. Gnewuch, J. Dick, L. Markhasin, W. Sickel, QMC integration based on arbitrary $(t,m,s)$-nets yields optimal convergence rates on several scales of function spaces, preprint 2024, arXiv:2409.12879. 
        \item[{[2]}] S. Heinrich, F. J. Hickernell and R. X. Yue, Optimal quadrature for Haar wavelet spaces, Math. Comput., 73 (2004), 259–277.
	\end{enumerate}

\end{talk}

\end{document}

