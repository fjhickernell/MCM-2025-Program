\documentclass[12pt,a4paper,figuresright]{book}

\usepackage{amsmath,amssymb}
\usepackage{tabularx,graphicx,url,xcolor,rotating,multicol,epsfig,colortbl}

\setlength{\textheight}{25.2cm}
\setlength{\textwidth}{16.5cm} %\setlength{\textwidth}{18.2cm}
\setlength{\voffset}{-1.6cm}
\setlength{\hoffset}{-0.3cm} %\setlength{\hoffset}{-1.2cm}
\setlength{\evensidemargin}{-0.3cm} 
\setlength{\oddsidemargin}{0.3cm}
\setlength{\parindent}{0cm} 
\setlength{\parskip}{0.3cm}

% -- adding a talk
\newenvironment{talk}[6]% [1] talk title
                         % [2] speaker name, [3] affiliations, [4] email,
                         % [5] coauthors, [6] special session
                         % [7] time slot
                         % [8] talk id, [9] session id or photo
 {%\needspace{6\baselineskip}%
  \vskip 0pt\nopagebreak%
%   \colorbox{gray!20!white}{\makebox[0.99\textwidth][r]{}}\nopagebreak%
%   \ifthenelse{\equal{#9}{photo}}{%
%                     \\\\\colorbox{gray!20!white}{\makebox{\includegraphics[width=3cm]{#8}}}\nopagebreak}{}%
 \vskip 0pt\nopagebreak%
%  \label{#8}%
  \textbf{#1}\vspace{3mm}\\\nopagebreak%
  \textit{#2}\\\nopagebreak%
  #3\\\nopagebreak%
  \url{#4}\vspace{3mm}\\\nopagebreak%
  \ifthenelse{\equal{#5}{}}{}{Coauthor(s): #5\vspace{3mm}\\\nopagebreak}%
  \ifthenelse{\equal{#6}{}}{}{Special session: #6\quad \vspace{3mm}\\\nopagebreak}%
 }
 {\vspace{1cm}\nopagebreak}%

\pagestyle{empty}

% ------------------------------------------------------------------------
% Document begins here
% ------------------------------------------------------------------------
\begin{document}
	
\begin{talk}
  {Comparing Probabilistic Load Forecasters: Stochastic Differential Equations and Deep Learning}% [1] talk title
  {Riccardo Saporiti}% [2] speaker name
  {EPFL, Lausanne, Switzerland}% [3] affiliations
  {riccardo.saporiti@epfl.ch}% [4] email
  {Fabio Nobile, Celia García-Pareja}% [5] coauthors
  {}% [6] special session. Leave this field empty for contributed talks. 
				% Insert the title of the special session if you were invited to give a talk in a special session.

Generating probabilistic predictions for the electricity-load profile is the foundation of efficient use of renewable energy and diminishing carbon footprint.

In this talk, we consider the problem of creating probabilistic forecasts of the day-ahead electricity consumption profile of an agglomerate of buildings in the city of Lausanne (Switzerland) in the absence of an externally provided prediction function. 
 
We propose a nonparametric, data-driven, approach based on Itô' Stochastic Differential Equations (SDEs) [1]. Our work is novel in that the mean function of the SDE is expanded on a Fourier periodic basis, capturing intra-day and intra-week periodic features. 
Using a derivative tracking term, we impose the trajectories of the process to revert toward the mean. To model high-volatility levels associated with more uncertain electricity consumption regimes, we employ a square-root type diffusion coefficients. 

Maximum-Likelihood estimation is used to infer the parameters of the model coherently with the available observations of the time history. We show that the maximization problem is well posed and that it admits at least one solution over the feasible domain. 

We compare the probabilistic predictions generated by the SDE with Deep Learning based probabilistic forecaster. 
On the one hand, we introduce a Deep Learning forecaster based on Long short-term memory (LSTM) recurrent neural networks trained by minimizing the quantile loss function. This approach allows the generation of confidence intervals by sampling from the one-step-ahead univariate cumulative density function (CDF) associated with the electricity consumption of the future time instant. 
On the other hand, inspired by [2], we consider Multivariate Quantile Function Forecasters that, based on Normalizing Flows, learn the multivariate cumulative density function of the day-ahead electricity consumption.

Metrics such as Continuous ranked probability score and Prediction interval coverage percentage are used to assess the quality of the forecasts. 

We show that SDEs generate reliable and interpretable predictions while presenting the most parsimonious and computationally efficient structure among the three models.


\medskip

 

\begin{enumerate}
    \item[{[1]}] Riccardo Saporiti, Fabio Nobile, Celia García-Pareja. {\it Probabilistic Forecast of the Day-Ahead electricity consumption profile with Stochastic Differential Equations: a comparison with Deep Learning models}. In preparation.

	\item[{[2]}] Kelvin Kan, Francois-Xavier Aubet, Tim Januschowski, Youngsuk Park, Konstantinos Benidis, Lars Ruthotto, and Jan Gasthaus (2022). {\it Multivariate Quantile Function Forecaster}. Proceedings of The 25th International Conference on Artificial Intelligence and Statistics, PMLR 151:10603-10621, 2022.
 
\end{enumerate}
 
\end{talk}

\end{document}

