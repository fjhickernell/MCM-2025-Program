\documentclass[12pt,a4paper,figuresright]{book}

\usepackage{amsmath,amssymb}
\usepackage{tabularx,graphicx,url,xcolor,rotating,multicol,epsfig,colortbl}

\setlength{\textheight}{25.2cm}
\setlength{\textwidth}{16.5cm} %\setlength{\textwidth}{18.2cm}
\setlength{\voffset}{-1.6cm}
\setlength{\hoffset}{-0.3cm} %\setlength{\hoffset}{-1.2cm}
\setlength{\evensidemargin}{-0.3cm} 
\setlength{\oddsidemargin}{0.3cm}
\setlength{\parindent}{0cm} 
\setlength{\parskip}{0.3cm}

% -- adding a talk
\newenvironment{talk}[6]% [1] talk title
                         % [2] speaker name, [3] affiliations, [4] email,
                         % [5] coauthors, [6] special session
                         % [7] time slot
                         % [8] talk id, [9] session id or photo
 {%\needspace{6\baselineskip}%
  \vskip 0pt\nopagebreak%
%   \colorbox{gray!20!white}{\makebox[0.99\textwidth][r]{}}\nopagebreak%
%   \ifthenelse{\equal{#9}{photo}}{%
%                     \\\\\colorbox{gray!20!white}{\makebox{\includegraphics[width=3cm]{#8}}}\nopagebreak}{}%
 \vskip 0pt\nopagebreak%
%  \label{#8}%
  \textbf{#1}\vspace{3mm}\\\nopagebreak%
  \textit{#2}\\\nopagebreak%
  #3\\\nopagebreak%
  \url{#4}\vspace{3mm}\\\nopagebreak%
  \ifthenelse{\equal{#5}{}}{}{Coauthor(s): #5\vspace{3mm}\\\nopagebreak}%
  \ifthenelse{\equal{#6}{}}{}{Special session: #6\quad \vspace{3mm}\\\nopagebreak}%
 }
 {\vspace{1cm}\nopagebreak}%

\pagestyle{empty}

% ------------------------------------------------------------------------
% Document begins here
% ------------------------------------------------------------------------
\begin{document}
	
\begin{talk}
  {Quasi-uniform quasi-Monte Carlo lattice point sets}% [1] talk title
  {Kosuke Suzuki}% [2] speaker name
  {Yamagata University}% [3] affiliations
  {kosuke-suzuki@sci.kj.yamagata-u.ac.jp}% [4] email
  {Josef Dick, Takashi Goda, Gerhard Larcher, Friedrich Pillichshammer}% [5] coauthors
  {QMC and Applications}% [6] special session. Leave this field empty for contributed talks. 
				% Insert the title of the special session if you were invited to give a talk in a special session.
			

The discrepancy of a point set quantifies how well the points are distributed, with low-discrepancy point sets demonstrating exceptional uniform distribution properties. Such sets are integral to quasi-Monte Carlo methods, which approximate integrals over the unit cube for integrands of bounded variation. In contrast, quasi-uniform point sets are characterized by optimal separation and covering radii, making them well-suited for applications such as radial basis function approximation. This paper explores the quasi-uniformity properties of quasi-Monte Carlo point sets constructed from lattices. Specifically, we analyze rank-1 lattice point sets, Fibonacci lattice point sets, Frolov point sets, and $(n \boldsymbol{\alpha})$-sequences, providing insights into their potential for use in applications that require both low-discrepancy and quasi-uniform distribution. As an example, we show that the $(n \boldsymbol{\alpha})$-sequence with $\alpha_j = 2^{j/(d+1)}$ for $j \in \{1, 2, \ldots, d\}$ is quasi-uniform and has low-discrepancy.


\end{talk}

\end{document}

