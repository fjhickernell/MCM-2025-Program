\documentclass[12pt,a4paper,figuresright]{book}

\usepackage{amsmath,amssymb}
\usepackage{tabularx,graphicx,url,xcolor,rotating,multicol,epsfig,colortbl}

\setlength{\textheight}{25.2cm}
\setlength{\textwidth}{16.5cm} %\setlength{\textwidth}{18.2cm}
\setlength{\voffset}{-1.6cm}
\setlength{\hoffset}{-0.3cm} %\setlength{\hoffset}{-1.2cm}
\setlength{\evensidemargin}{-0.3cm} 
\setlength{\oddsidemargin}{0.3cm}
\setlength{\parindent}{0cm} 
\setlength{\parskip}{0.3cm}

% -- adding a talk
\newenvironment{talk}[6]% [1] talk title
                         % [2] speaker name, [3] affiliations, [4] email,
                         % [5] coauthors, [6] special session
                         % [7] time slot
                         % [8] talk id, [9] session id or photo
 {%\needspace{6\baselineskip}%
  \vskip 0pt\nopagebreak%
%   \colorbox{gray!20!white}{\makebox[0.99\textwidth][r]{}}\nopagebreak%
%   \ifthenelse{\equal{#9}{photo}}{%
%                     \\\\\colorbox{gray!20!white}{\makebox{\includegraphics[width=3cm]{#8}}}\nopagebreak}{}%
 \vskip 0pt\nopagebreak%
%  \label{#8}%
  \textbf{#1}\vspace{3mm}\\\nopagebreak%
  \textit{#2}\\\nopagebreak%
  #3\\\nopagebreak%
  \url{#4}\vspace{3mm}\\\nopagebreak%
  \ifthenelse{\equal{#5}{}}{}{Coauthor(s): #5\vspace{3mm}\\\nopagebreak}%
  \ifthenelse{\equal{#6}{}}{}{Special session: #6\quad \vspace{3mm}\\\nopagebreak}%
 }
 {\vspace{1cm}\nopagebreak}%

\pagestyle{empty}

% ------------------------------------------------------------------------
% Document begins here
% ------------------------------------------------------------------------
\begin{document}
	
\begin{talk}
  {QMC for Bayesian optimal experimental design with application to inverse problems governed by PDEs}% [1] talk title
  {Vesa Kaarnioja}% [2] speaker name
  {Free University of Berlin}% [3] affiliations
  {vesa.kaarnioja@fu-berlin.de}% [4] email
  {Claudia Schillings}% [5] coauthors
  {Nested expectations: models and estimators, Part I}% [6] special session. Leave this field empty for contributed talks. 
				% Insert the title of the special session if you were invited to give a talk in a special session.
			
The goal in Bayesian optimal experimental design (OED) is to maximize the expected information gain for the reconstruction of unknown quantities in an experiment by optimizing the placement of measurements. The objective function in the resulting optimization problem involves a multivariate double integral over the high-dimensional parameter and data domains. For the efficient approximation of these integrals, we consider a sparse tensor product combination of quasi-Monte Carlo (QMC) cubature rules over the parameter and data domains. For the parameterization of the unknown quantitites, we consider a model recently studied by Chernov and L\^{e} [1,2] as well as Harbrecht, Schmidlin, and Schwab [3] in which the input random field is assumed to belong to a Gevrey class. The Gevrey class contains functions that are infinitely many times continuously differentiable with a growth condition on the higher-order partial derivatives, but which are not analytic in general. Using the techniques developed in [4], we investigate efficient Bayesian OED for inverse problems governed by partial differential equations (PDEs).
\begin{enumerate}
	\item[{[1]}] Chernov, Alexey, \& L\^{e}, T\`{u}ng (2024). Analytic and Gevrey class regularity for parametric elliptic eigenvalue problems and applications. \emph{SIAM Journal on Numerical Analysis}, \textbf{62}(4), 1874--1900.
	\item[{[2]}] Chernov, Alexey, \& L\^{e}, T\`{u}ng (2024). Analytic and Gevrey class regularity for parametric semilinear reaction-diffusion problems and applications in uncertainty quantification. \emph{Computers \& Mathematics with Applications}, \textbf{164}, 116--130.
	\item[{[3]}] Harbrecht, Helmut, Schmidlin, Marc, \& Schwab, Christoph (2024). The Gevrey class implicit mapping theorem with applications to UQ of semilinear elliptic PDEs. \emph{Mathematical Models and Methods in Applied Sciences}, \textbf{34}(5), 881--917.
	\item[{[4]}] Kaarnioja, Vesa, \& Schillings, Claudia (2024). Quasi-Monte Carlo for Bayesian design of experiment problems governed by parametric PDEs. Preprint, \emph{arXiv:2405.03529 [math.NA]}.
\end{enumerate}

\end{talk}

\end{document}

