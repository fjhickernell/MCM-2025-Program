\documentclass[12pt,a4paper,figuresright]{book}

\usepackage{amsmath,amssymb}
\usepackage{tabularx,graphicx,url,xcolor,rotating,multicol,epsfig,colortbl}

\setlength{\textheight}{25.2cm}
\setlength{\textwidth}{16.5cm} %\setlength{\textwidth}{18.2cm}
\setlength{\voffset}{-1.6cm}
\setlength{\hoffset}{-0.3cm} %\setlength{\hoffset}{-1.2cm}
\setlength{\evensidemargin}{-0.3cm} 
\setlength{\oddsidemargin}{0.3cm}
\setlength{\parindent}{0cm} 
\setlength{\parskip}{0.3cm}

% -- adding a talk
\newenvironment{talk}[6]% [1] talk title
                         % [2] speaker name, [3] affiliations, [4] email,
                         % [5] coauthors, [6] special session
                         % [7] time slot
                         % [8] talk id, [9] session id or photo
 {%\needspace{6\baselineskip}%
  \vskip 0pt\nopagebreak%
%   \colorbox{gray!20!white}{\makebox[0.99\textwidth][r]{}}\nopagebreak%
%   \ifthenelse{\equal{#9}{photo}}{%
%                     \\\\\colorbox{gray!20!white}{\makebox{\includegraphics[width=3cm]{#8}}}\nopagebreak}{}%
 \vskip 0pt\nopagebreak%
%  \label{#8}%
  \textbf{#1}\vspace{3mm}\\\nopagebreak%
  \textit{#2}\\\nopagebreak%
  #3\\\nopagebreak%
  \url{#4}\vspace{3mm}\\\nopagebreak%
  \ifthenelse{\equal{#5}{}}{}{Coauthor(s): #5\vspace{3mm}\\\nopagebreak}%
  \ifthenelse{\equal{#6}{}}{}{Special session: #6\quad \vspace{3mm}\\\nopagebreak}%
 }
 {\vspace{1cm}\nopagebreak}%

\pagestyle{empty}

% ------------------------------------------------------------------------
% Document begins here
% ------------------------------------------------------------------------
\begin{document}
	
\begin{talk}
  {ATLAS: Adapting Trajectory Lengths and Step-Size for Hamiltonian Monte Carlo}% [1] talk title
  {Chirag Modi}% [2] speaker name
  {New York University}% [3] affiliations
  {modichirag@nyu.edu}% [4] email
  {}% [5] coauthors
  {}% [6] special session. Leave this field empty for contributed talks. 
				% Insert the title of the special session if you were invited to give a talk in a special session.

Hamiltonian Monte-Carlo (HMC) and its auto-tuned variant, the No U-Turn Sampler (NUTS)

can struggle to accurately sample distributions with complex geometries, e.g., varying cur-
vature, due to their constant step size for leapfrog integration and fixed mass matrix. In this

talk, I will present a strategy to locally adapt the step size parameter of HMC at every iter-
ation by evaluating a low-rank approximation of the local Hessian and estimating its largest

eigenvalue. I will then combine it with a strategy to similarly adapt the trajectory length
by monitoring the no U-turn condition, resulting in an adaptive sampler, ATLAS: adapting
trajectory length and step-size. I will further use a delayed rejection framework for making
multiple proposals that improves the computational efficiency of ATLAS, and develop an

approach for automatically tuning its hyperparameters during warmup. Finally, I will com-
pare ATLAS with NUTS on a suite of synthetic and real world examples, and show that

i) unlike NUTS, ATLAS is able to accurately sample difficult distributions with complex
geometries, ii) it is computationally competitive to NUTS for simpler distributions, and iii)
it is more robust to the tuning of hyperparamters.
\medskip

\end{talk}

\end{document}

