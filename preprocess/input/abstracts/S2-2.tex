\documentclass[12pt,a4paper,figuresright]{book}

\usepackage{amsmath,amssymb}
\usepackage{tabularx,graphicx,url,xcolor,rotating,multicol,epsfig,colortbl}

\setlength{\textheight}{25.2cm}
\setlength{\textwidth}{16.5cm} %\setlength{\textwidth}{18.2cm}
\setlength{\voffset}{-1.6cm}
\setlength{\hoffset}{-0.3cm} %\setlength{\hoffset}{-1.2cm}
\setlength{\evensidemargin}{-0.3cm} 
\setlength{\oddsidemargin}{0.3cm}
\setlength{\parindent}{0cm} 
\setlength{\parskip}{0.3cm}

% -- adding a talk
\newenvironment{talk}[6]% [1] talk title
                         % [2] speaker name, [3] affiliations, [4] email,
                         % [5] coauthors, [6] special session
                         % [7] time slot
                         % [8] talk id, [9] session id or photo
 {%\needspace{6\baselineskip}%
  \vskip 0pt\nopagebreak%
%   \colorbox{gray!20!white}{\makebox[0.99\textwidth][r]{}}\nopagebreak%
%   \ifthenelse{\equal{#9}{photo}}{%
%                     \\\\\colorbox{gray!20!white}{\makebox{\includegraphics[width=3cm]{#8}}}\nopagebreak}{}%
 \vskip 0pt\nopagebreak%
%  \label{#8}%
  \textbf{#1}\vspace{3mm}\\\nopagebreak%
  \textit{#2}\\\nopagebreak%
  #3\\\nopagebreak%
  \url{#4}\vspace{3mm}\\\nopagebreak%
  \ifthenelse{\equal{#5}{}}{}{Coauthor(s): #5\vspace{3mm}\\\nopagebreak}%
  \ifthenelse{\equal{#6}{}}{}{Special session: #6\quad \vspace{3mm}\\\nopagebreak}%
 }
 {\vspace{1cm}\nopagebreak}%

\pagestyle{empty}

% ------------------------------------------------------------------------
% Document begins here
% ------------------------------------------------------------------------
\begin{document}
	
\begin{talk}
  {Domain Uncertainty Quantification for Electromagnetic Wave Scattering via First-Order Sparse Boundary Element Approximation}% [1] talk title
  {Carlos Jerez-Hanckes}% [2] speaker name
  {INRIA Chile}% [3] affiliations
  {carlos.jerez@inria.cl}% [4] email
  {Paul Escapil-Inchausp\'e}% [5] coauthors
  {Domain uncertainty quantification}% [6] special session. Leave this field empty for contributed talks. 
				% Insert the title of the special session if you were invited to give a talk in a special session.
			
Quantifying the effects on electromagnetic waves scattered by objects of uncertain shape is key for robust design, particularly in high-precision applications. Assuming small random perturbations departing from a nominal domain, the first-order sparse boundary (FOSB) element method has been proven to directly compute statistical moments with poly-logarithmic complexity [1,2] for a prescribed accuracy, without resorting to computationally intense Monte Carlo (MC) simulations. However, implementing FOSB is not straightforward as the lack of compelling computational results for EM scattering attests [3]. In this work, we present a first full 3D implementation of FOSB for shape-related uncertainty quantification (UQ) in EM scattering [4]. In doing so, we address several implementation issues such as ill-conditioning and large computational and memory requirements and present a comprehensive, state-of-the-art, easy-to-use, open-source computational framework to directly apply this technique when dealing with complex objects. Exhaustive numerical experiments confirm our claims and demonstrate the technique's applicability and provide pathways for further improvement.

\medskip

%If you would like to include references, please do so by creating a simple list numbered by [1], [2], [3], \ldots. See example below.
%Please do not use the \texttt{bibliography} environment or \texttt{bibtex} files.
%APA reference style is recommended.
\begin{enumerate}
	\item[{[1]}] Jerez-Hanckes, Schwab (2017). {\it Electromagnetic Wave Scattering by Random Surfaces: Uncertainty Quantification via Sparse Tensor Boundary Elements}, IMA Journal of Numerical Analysis {\bf 37}(3), 1175--1210.
	\item[{[2]}] Hiptmair, Jerez-Hanckes, Schwab (2013). {\it Sparse Tensor Edge Elements}, BIT Numerical Mathematics {\bf 53}, 925--943.
	\item[{[3]}] Escapil-Inchausp\'e, Jerez-Hanckes (2020). {\it Helmholtz Scattering by Random Domains: First-Order Sparse Boundary Elements Approximation}, SIAM Journal of Scientific Computing {\bf 42}(5), A2561--A2592.
	\item[{[4]}] Escapil-Inchausp\'e, Jerez-Hanckes (2024). {\it Shape Uncertainty Quantification for Electromagnetic Wave Scattering via First-Order Sparse Boundary Element Approximation}, IEEE Transactions in Antennas \& Propagation {\bf 72}(8):6627--6637.
\end{enumerate}

%Equations may be used if they are referenced. Please note that the equation numbers may be different (but will be cross-referenced correctly) in the final program book.
\end{talk}

\end{document}

