\documentclass[12pt,a4paper,figuresright]{book}

\usepackage{amsmath,amssymb}
\usepackage{tabularx,graphicx,url,xcolor,rotating,multicol,epsfig,colortbl}

\setlength{\textheight}{25.2cm}
\setlength{\textwidth}{16.5cm} %\setlength{\textwidth}{18.2cm}
\setlength{\voffset}{-1.6cm}
\setlength{\hoffset}{-0.3cm} %\setlength{\hoffset}{-1.2cm}
\setlength{\evensidemargin}{-0.3cm} 
\setlength{\oddsidemargin}{0.3cm}
\setlength{\parindent}{0cm} 
\setlength{\parskip}{0.3cm}

% -- adding a talk
\newenvironment{talk}[6]% [1] talk title
                         % [2] speaker name, [3] affiliations, [4] email,
                         % [5] coauthors, [6] special session
                         % [7] time slot
                         % [8] talk id, [9] session id or photo
 {%\needspace{6\baselineskip}%
  \vskip 0pt\nopagebreak%
%   \colorbox{gray!20!white}{\makebox[0.99\textwidth][r]{}}\nopagebreak%
%   \ifthenelse{\equal{#9}{photo}}{%
%                     \\\\\colorbox{gray!20!white}{\makebox{\includegraphics[width=3cm]{#8}}}\nopagebreak}{}%
 \vskip 0pt\nopagebreak%
%  \label{#8}%
  \textbf{#1}\vspace{3mm}\\\nopagebreak%
  \textit{#2}\\\nopagebreak%
  #3\\\nopagebreak%
  \url{#4}\vspace{3mm}\\\nopagebreak%
  \ifthenelse{\equal{#5}{}}{}{Coauthor(s): #5\vspace{3mm}\\\nopagebreak}%
  \ifthenelse{\equal{#6}{}}{}{Special session: #6\quad \vspace{3mm}\\\nopagebreak}%
 }
 {\vspace{1cm}\nopagebreak}%

\pagestyle{empty}

% ------------------------------------------------------------------------
% Document begins here
% ------------------------------------------------------------------------
\begin{document}
	
\begin{talk}
  {Cheap permutation testing}% [1] talk title
  {Carles Domingo-Enrich}% [2] speaker name
  {Microsoft Research New England}% [3] affiliations
  {carlesd@microsoft.com}% [4] email
  {Raaz Dwivedi, Lester Mackey}% [5] coauthors
  {}% [6] special session. Leave this field empty for contributed talks. 
				% Insert the title of the special session if you were invited to give a talk in a special session.
			
Permutation tests are a popular choice for distinguishing distributions and testing independence, due to their exact, finite-sample control of false positives and their minimax optimality when paired with U-statistics. However, standard permutation tests are also expensive, requiring a test statistic to be computed hundreds or thousands of times to detect a separation between distributions. In this work, we offer a simple approach to accelerate testing: group your datapoints into bins and permute only those bins. For U and V-statistics, we prove that these cheap permutation tests have two remarkable properties. First, by storing appropriate sufficient statistics, a cheap test can be run in time comparable to evaluating a single test statistic. Second, cheap permutation power closely approximates standard permutation power. As a result, cheap tests inherit the exact false positive control and minimax optimality of standard permutation tests while running in a fraction of the time. We complement these findings with improved power guarantees for standard permutation testing and experiments demonstrating the benefits of cheap permutations over standard maximum mean discrepancy (MMD), Hilbert-Schmidt independence criterion (HSIC), random Fourier feature, Wilcoxon-Mann-Whitney, cross-MMD, and cross-HSIC tests.

% \medskip

% If you would like to include references, please do so by creating a simple list numbered by [1], [2], [3], \ldots. See example below.
% Please do not use the \texttt{bibliography} environment or \texttt{bibtex} files.
% APA reference style is recommended.
% \begin{enumerate}
% 	\item[{[1]}] Niederreiter, Harald (1992). {\it Random number generation and quasi-Monte Carlo methods}. Society for Industrial and Applied Mathematics (SIAM).
% 	\item[{[2]}] L’Ecuyer, Pierre, \& Christiane Lemieux. (2002). Recent advances in randomized quasi-Monte Carlo methods. Modeling uncertainty: An examination of stochastic theory, methods, and applications, 419-474.
% \end{enumerate}

% Equations may be used if they are referenced. Please note that the equation numbers may be different (but will be cross-referenced correctly) in the final program book.
\end{talk}

\end{document}
