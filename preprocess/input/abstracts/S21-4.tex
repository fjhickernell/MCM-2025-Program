\documentclass[12pt,a4paper,figuresright]{book}

\usepackage{amsmath,amssymb}
\usepackage{tabularx,graphicx,url,xcolor,rotating,multicol,epsfig,colortbl}

\setlength{\textheight}{25.2cm}
\setlength{\textwidth}{16.5cm} %\setlength{\textwidth}{18.2cm}
\setlength{\voffset}{-1.6cm}
\setlength{\hoffset}{-0.3cm} %\setlength{\hoffset}{-1.2cm}
\setlength{\evensidemargin}{-0.3cm} 
\setlength{\oddsidemargin}{0.3cm}
\setlength{\parindent}{0cm} 
\setlength{\parskip}{0.3cm}

% -- adding a talk
\newenvironment{talk}[6]% [1] talk title
                         % [2] speaker name, [3] affiliations, [4] email,
                         % [5] coauthors, [6] special session
                         % [7] time slot
                         % [8] talk id, [9] session id or photo
 {%\needspace{6\baselineskip}%
  \vskip 0pt\nopagebreak%
%   \colorbox{gray!20!white}{\makebox[0.99\textwidth][r]{}}\nopagebreak%
%   \ifthenelse{\equal{#9}{photo}}{%
%                     \\\\\colorbox{gray!20!white}{\makebox{\includegraphics[width=3cm]{#8}}}\nopagebreak}{}%
 \vskip 0pt\nopagebreak%
%  \label{#8}%
  \textbf{#1}\vspace{3mm}\\\nopagebreak%
  \textit{#2}\\\nopagebreak%
  #3\\\nopagebreak%
  \url{#4}\vspace{3mm}\\\nopagebreak%
  \ifthenelse{\equal{#5}{}}{}{Coauthor(s): #5\vspace{3mm}\\\nopagebreak}%
  \ifthenelse{\equal{#6}{}}{}{Special session: #6\quad \vspace{3mm}\\\nopagebreak}%
 }
 {\vspace{1cm}\nopagebreak}%

\pagestyle{empty}

% ------------------------------------------------------------------------
% Document begins here
% ------------------------------------------------------------------------
\begin{document}
	
\begin{talk}
  {WoS-NN: Collaborating Walk-on-Spheres with Machine Learning to Solve Elliptic PDEs}% [1] talk title
  {Silei Song}% [2] speaker name
  {Department of Computer Science, Florida State University}% [3] affiliations
  {ss19cu@fsu.edu}% [4] email
  {Michael Mascagni, Arash Fahim}% [5] coauthors
  {Monte Carlo Applications in High-performance Computing, Computer Graphics, and Computational Science}% [6] special session. Leave this field empty for contributed talks. 
				% Insert the title of the special session if you were invited to give a talk in a special session.
			
Solving elliptic partial differential equations (PDEs) is a fundamental step in various scientific and engineering studies. As a classic stochastic solver, the Walk on Spheres (WoS) method is a well-established and efficient algorithm that provides accurate local estimates for PDEs. However, limited by the curse of dimensionality, WoS may not offer sufficiently precise global estimations, which becomes more serious in high-dimensional scenarios. Recent developments in machine learning offer promising strategies to address this limitation. By integrating machine learning techniques with WoS and space discretization approaches, we developed a novel stochastic solver, WoS-NN. This new method solves elliptic problems with Dirichlet boundary conditions, facilitating precise and rapid global solutions and gradient approximations. A typical experimental result demonstrated that the proposed WoS-NN method provides accurate field estimations, reducing $76.32\%$ errors while using only $8\%$ of path samples compared to the conventional WoS method, which saves abundant computational time and resource consumption. WoS-NN can also be utilized as a fast and effective gradient estimator based on established implementations of the original WoS method. This new method reduced the impacts of the curse of dimensionality and can be widely applied to areas like geometry processing, bio-molecular modeling, financial mathematics, etc.

\medskip

\end{talk}

\end{document}

