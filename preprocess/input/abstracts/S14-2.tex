\documentclass[12pt,a4paper,figuresright]{book}

\usepackage{amsmath,amssymb}
\usepackage{tabularx,graphicx,url,xcolor,rotating,multicol,epsfig,colortbl}

\setlength{\textheight}{25.2cm}
\setlength{\textwidth}{16.5cm} %\setlength{\textwidth}{18.2cm}
\setlength{\voffset}{-1.6cm}
\setlength{\hoffset}{-0.3cm} %\setlength{\hoffset}{-1.2cm}
\setlength{\evensidemargin}{-0.3cm} 
\setlength{\oddsidemargin}{0.3cm}
\setlength{\parindent}{0cm} 
\setlength{\parskip}{0.3cm}

% -- adding a talk
\newenvironment{talk}[6]% [1] talk title
                         % [2] speaker name, [3] affiliations, [4] email,
                         % [5] coauthors, [6] special session
                         % [7] time slot
                         % [8] talk id, [9] session id or photo
 {%\needspace{6\baselineskip}%
  \vskip 0pt\nopagebreak%
%   \colorbox{gray!20!white}{\makebox[0.99\textwidth][r]{}}\nopagebreak%
%   \ifthenelse{\equal{#9}{photo}}{%
%                     \\\\\colorbox{gray!20!white}{\makebox{\includegraphics[width=3cm]{#8}}}\nopagebreak}{}%
 \vskip 0pt\nopagebreak%
%  \label{#8}%
  \textbf{#1}\vspace{3mm}\\\nopagebreak%
  \textit{#2}\\\nopagebreak%
  #3\\\nopagebreak%
  \url{#4}\vspace{3mm}\\\nopagebreak%
  \ifthenelse{\equal{#5}{}}{}{Coauthor(s): #5\vspace{3mm}\\\nopagebreak}%
  \ifthenelse{\equal{#6}{}}{}{Special session: #6\quad \vspace{3mm}\\\nopagebreak}%
 }
 {\vspace{1cm}\nopagebreak}%

\pagestyle{empty}

% ------------------------------------------------------------------------
% Document begins here
% ------------------------------------------------------------------------
\begin{document}
	
\begin{talk}
  {Estimating rare event probabilities associated with McKean--Vlasov SDEs}% [1] talk title
  {Shyam Mohan Subbiah Pillai}% [2] speaker name
  {RWTH Aachen University, Germany}% [3] affiliations
  {subbiah@uq.rwth-aachen.de}% [4] email
  {Nadhir Ben Rached, Abdul-Lateef Haji-Ali, Raùl Tempone}% [5] coauthors
  {Advances in Rare Events Simulation}% [6] special session. Leave this field empty for contributed talks. 
				% Insert the title of the special session if you were invited to give a talk in a special session.
			
McKean–Vlasov stochastic differential equations (MV-SDEs) arise as the mean-field limits of stochastic interacting particle systems, with applications in pedestrian dynamics, collective animal behavior, and financial market modelling. This work develops an efficient method for estimating rare event probabilities associated with MV-SDEs by combining multilevel Monte Carlo (MC) with importance sampling (IS). To apply a measure change for IS, we first reformulate the MV-SDE as a standard SDE by conditioning on its law, leading to the decoupled MV-SDE. We then formulate the problem of finding the optimal IS measure change as a stochastic optimal control problem that minimizes the variance of the MC estimator. The resulting partial differential equation is solved numerically to obtain the optimal IS measure change. Building on this IS scheme and the decoupling approach, we introduce a double loop Monte Carlo (DLMC) estimator. To further improve computational efficiency, we extend DLMC to a multilevel setting, reducing its computational complexity. To enhance variance convergence in the level differences for the discontinuous indicator function, we propose two key techniques: (1) numerical smoothing via one-dimensional integration over a carefully chosen variable and (2) an antithetic sampler to increase correlation between fine and coarse SDE paths. By integrating IS with efficient multilevel sampling, we develop the multilevel double loop Monte Carlo (MLDLMC) estimator. We demonstrate its effectiveness on the Kuramoto model from statistical physics, showing a reduction in computational complexity from $\mathcal{O}(\mathrm{TOL}_\mathrm{r}^{-4})$ using DLMC to $\mathcal{O}(\mathrm{TOL}_\mathrm{r}^{-3})$ using MLDLMC with IS, for estimating rare event probabilities up to a prescribed relative error tolerance $\mathrm{TOL}_\mathrm{r}$. This talk is primarily based on [1,2].

\medskip

%If you would like to include references, please do so by creating a simple list numbered by [1], [2], [3], \ldots. See example below.
%Please do not use the \texttt{bibliography} environment or \texttt{bibtex} files.
%APA reference style is recommended.
\begin{enumerate}
	\item[{[1]}] Ben Rached, N., Haji-Ali, A. L., Subbiah Pillai, S. M., \& Tempone, R. (2024). Double-loop importance sampling for McKean–Vlasov stochastic differential equation. Statistics and Computing, 34(6), 197.
	\item[{[2]}] Ben Rached, N., Haji-Ali, A. L., Subbiah Pillai, S. M., \& Tempone, R. (2025). Multilevel importance sampling for rare events associated with the McKean–Vlasov equation. Statistics and Computing, 35(1), 1.
\end{enumerate}

%Equations may be used if they are referenced. Please note that the equation numbers may be different (but will be cross-referenced correctly) in the final program book.
\end{talk}

\end{document}

