\documentclass[12pt,a4paper,figuresright]{book}

\usepackage{amsmath,amssymb}
\usepackage{tabularx,graphicx,url,xcolor,rotating,multicol,epsfig,colortbl}

\setlength{\textheight}{25.2cm}
\setlength{\textwidth}{16.5cm} %\setlength{\textwidth}{18.2cm}
\setlength{\voffset}{-1.6cm}
\setlength{\hoffset}{-0.3cm} %\setlength{\hoffset}{-1.2cm}
\setlength{\evensidemargin}{-0.3cm} 
\setlength{\oddsidemargin}{0.3cm}
\setlength{\parindent}{0cm} 
\setlength{\parskip}{0.3cm}

% -- adding a talk
\newenvironment{talk}[6]% [1] talk title
                         % [2] speaker name, [3] affiliations, [4] email,
                         % [5] coauthors, [6] special session
                         % [7] time slot
                         % [8] talk id, [9] session id or photo
 {%\needspace{6\baselineskip}%
  \vskip 0pt\nopagebreak%
%   \colorbox{gray!20!white}{\makebox[0.99\textwidth][r]{}}\nopagebreak%
%   \ifthenelse{\equal{#9}{photo}}{%
%                     \\\\\colorbox{gray!20!white}{\makebox{\includegraphics[width=3cm]{#8}}}\nopagebreak}{}%
 \vskip 0pt\nopagebreak%
%  \label{#8}%
  \textbf{#1}\vspace{3mm}\\\nopagebreak%
  \textit{#2}\\\nopagebreak%
  #3\\\nopagebreak%
  \url{#4}\vspace{3mm}\\\nopagebreak%
  \ifthenelse{\equal{#5}{}}{}{Coauthor(s): #5\vspace{3mm}\\\nopagebreak}%
  \ifthenelse{\equal{#6}{}}{}{Special session: #6\quad \vspace{3mm}\\\nopagebreak}%
 }
 {\vspace{1cm}\nopagebreak}%

\pagestyle{empty}

% ------------------------------------------------------------------------
% Document begins here
% ------------------------------------------------------------------------
\begin{document}
	
\begin{talk}
  {Halton Sequences, Scrambling and the Inverse Star-Discrepancy}% [1] talk title
  {Christian Wei\ss{}}% [2] speaker name
  {Ruhr West University of Applied Sciences}% [3] affiliations
  {christian.weiss@hs-ruhrwest.de}% [4] email
  {}% [5] coauthors
  {}% [6] special session. Leave this field empty for contributed talks. 
				% Insert the title of the special session if you were invited to give a talk in a special session.
			
Halton sequences are classical examples of multi-dimensional low-discrepancy sequences. Braaten and Weller discovered that scrambling strongly reduces their empirical star-discre-pancy. A similar approach may be applied to certain multi-parameter subsequences of Halton sequences. Indeed, results from p-adic analysis guarantee that these subsequences still have the theoretical low-discrepancy property while scrambling has strong effects on the empirical star-discrepancy. By optimizing the parameters of these subsequences known empiric bounds for the inverse star-discrepancy can be improved.

\medskip


\begin{enumerate}
	\item[{[1]}] E. Braaten, and G. Weller. (1979). An improved low-discrepancy sequence for multidimensional quasi-Monte Carlo integration. Journal of Computational Physics, 33(2): 249–258.
	\item[{[2]}] C. Wei\ss{}. (2024). Scrambled Halton Subsequences and Inverse Star-Discrepancy, arXiv: 2411.10363. 
\end{enumerate}

\end{talk}

\end{document}

