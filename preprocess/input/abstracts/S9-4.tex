\documentclass[12pt,a4paper,figuresright]{book}

\usepackage{amsmath,amssymb}
\usepackage{tabularx,graphicx,url,xcolor,rotating,multicol,epsfig,colortbl}

\setlength{\textheight}{25.2cm}
\setlength{\textwidth}{16.5cm} %\setlength{\textwidth}{18.2cm}
\setlength{\voffset}{-1.6cm}
\setlength{\hoffset}{-0.3cm} %\setlength{\hoffset}{-1.2cm}
\setlength{\evensidemargin}{-0.3cm} 
\setlength{\oddsidemargin}{0.3cm}
\setlength{\parindent}{0cm} 
\setlength{\parskip}{0.3cm}

% -- adding a talk
\newenvironment{talk}[6]% [1] talk title
                         % [2] speaker name, [3] affiliations, [4] email,
                         % [5] coauthors, [6] special session
                         % [7] time slot
                         % [8] talk id, [9] session id or photo
 {%\needspace{6\baselineskip}%
  \vskip 0pt\nopagebreak%
%   \colorbox{gray!20!white}{\makebox[0.99\textwidth][r]{}}\nopagebreak%
%   \ifthenelse{\equal{#9}{photo}}{%
%                     \\\\\colorbox{gray!20!white}{\makebox{\includegraphics[width=3cm]{#8}}}\nopagebreak}{}%
 \vskip 0pt\nopagebreak%
%  \label{#8}%
  \textbf{#1}\vspace{3mm}\\\nopagebreak%
  \textit{#2}\\\nopagebreak%
  #3\\\nopagebreak%
  \url{#4}\vspace{3mm}\\\nopagebreak%
  \ifthenelse{\equal{#5}{}}{}{Coauthor(s): #5\vspace{3mm}\\\nopagebreak}%
  \ifthenelse{\equal{#6}{}}{}{Special session: #6\quad \vspace{3mm}\\\nopagebreak}%
 }
 {\vspace{1cm}\nopagebreak}%

\pagestyle{empty}

% ------------------------------------------------------------------------
% Document begins here
% ------------------------------------------------------------------------
\begin{document}
	
\begin{talk}
  {On energy, discrepancy, group  invariant measures, alignment of neural data and Whitney extensions}% [1] talk title
  {Steven Damelin}% [2] speaker name
  {zbMATH Open, European Mathematical Society}% [3] affiliations
  {steve.damelin@gmail.com}% [4] email
  {D. Ragozin, F. Hickernell, X. Zeng, J. Levesley, X. Sun, C. Fefferman}% [5] coauthors
  {}% [6] special session. Leave this field empty for contributed talks. 
				% Insert the title of the special session if you were invited to give a talk in a special session.
			
Given $X$, some measurable subset of Euclidean space, one sometimes wants to construct a design, $P\subset X$ with a small energy or discrepancy. It is known that these two measures of design quality are equivalent when they are defined via positive definite kernels $K:X^2\to \mathbb R$. Our talk studies discrepancy and energy in the case when $X$ is the orbit of a compact, possibly non-Abelian group, $G$, acting as measurable transformations of $X$ and the kernel $K$ is invariant under the group action. For example, our results apply to the sphere, flat torus and Riesz kernel. It is shown that the minimizer measure for the energy is the normalized measure on $X$ induced by Haar measure on $G$. This allows us to calculate upper bounds on discrepancy. We will explain how this problem relates to interactions of neural data in the subject of Neural Science, Whitney extensions and manifold learning alignment . 


\medskip

%If you would like to include references, please do so by creating a simple list numbered by [1], [2], [3], \ldots. See example below.
%Please do not use the \texttt{bibliography} environment or \texttt{bibtex} files.
%APA reference style is recommended.
\begin{enumerate}
	\item[{[1]}] Damelin, Steven (2024). {\it Near extensions and Alignment of Data in $R^n$: Whitney extensions of smooth near isometries, shortest paths, equidistribution, clustering and non-rigid alignment of data in Euclidean space}. John Wiley \& Sons.
        \item[{[2]}] Damelin, Steven, \& Ragozin, David, \& Levesley, Jeremy, \& Sun, Xinping (2009). {\it Energies, Group Invariant Kernels and Numerical Integration on Compact Manifolds}. Journal of Complexity. \textbf{25}, 152-162.
        \item[{[3]}] Damelin, Steven, \& Ragozin, David, \& Hickernell, Fred, \& Zeng, X (2010). {\it  On energy, discrepancy and G invariant measures on measurable subsets of Euclidean space}. Journal of Fourier Analysis and its Applications. \textbf{16}, 813-839.
\end{enumerate}

%Equations may be used if they are referenced. Please note that the equation numbers may be different (but will be cross-referenced correctly) in the final %program book.
\end{talk}

\end{document}

