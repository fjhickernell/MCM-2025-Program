\documentclass[12pt,a4paper,figuresright]{book}

\usepackage{amsmath,amssymb}
\usepackage{tabularx,graphicx,url,xcolor,rotating,multicol,epsfig,colortbl}

\setlength{\textheight}{25.2cm}
\setlength{\textwidth}{16.5cm} %\setlength{\textwidth}{18.2cm}
\setlength{\voffset}{-1.6cm}
\setlength{\hoffset}{-0.3cm} %\setlength{\hoffset}{-1.2cm}
\setlength{\evensidemargin}{-0.3cm} 
\setlength{\oddsidemargin}{0.3cm}
\setlength{\parindent}{0cm} 
\setlength{\parskip}{0.3cm}

% -- adding a talk
\newenvironment{talk}[6]% [1] talk title
                         % [2] speaker name, [3] affiliations, [4] email,
                         % [5] coauthors, [6] special session
                         % [7] time slot
                         % [8] talk id, [9] session id or photo
 {%\needspace{6\baselineskip}%
  \vskip 0pt\nopagebreak%
%   \colorbox{gray!20!white}{\makebox[0.99\textwidth][r]{}}\nopagebreak%
%   \ifthenelse{\equal{#9}{photo}}{%
%                     \\\\\colorbox{gray!20!white}{\makebox{\includegraphics[width=3cm]{#8}}}\nopagebreak}{}%
 \vskip 0pt\nopagebreak%
%  \label{#8}%
  \textbf{#1}\vspace{3mm}\\\nopagebreak%
  \textit{#2}\\\nopagebreak%
  #3\\\nopagebreak%
  \url{#4}\vspace{3mm}\\\nopagebreak%
  \ifthenelse{\equal{#5}{}}{}{Coauthor(s): #5\vspace{3mm}\\\nopagebreak}%
  \ifthenelse{\equal{#6}{}}{}{Special session: #6\quad \vspace{3mm}\\\nopagebreak}%
 }
 {\vspace{1cm}\nopagebreak}%

\pagestyle{empty}

% ------------------------------------------------------------------------
% Document begins here
% ------------------------------------------------------------------------
\begin{document}
	
\begin{talk}
  {Robust Bayesian Optimal Experimental Design under Model Misspecification}% [1] talk title
  {Tommie A. Catanach}% [2] speaker name
  {Sandia National Laboratories}% [3] affiliations
  {tacatan@sandia.gov}% [4] email
  {}% [5] coauthors
  {Next-generation optimal experimental design: theory, scalability, and real world impact: Part II}% [6] special session. Leave this field empty for contributed talks. 
				% Insert the title of the special session if you were invited to give a talk in a special session.
			
Bayesian Optimal Experimental Design (BOED) has become a powerful tool for improving uncertainty quantification by strategically guiding data collection. However, the reliability of BOED depends critically on the validity of its underlying assumptions and the possibility of model discrepancy. In practice, the chosen data acquisition strategy may inadvertently reinforce prior assumptions—overlooking data that could challenge them—or rely on low-fidelity models whose error is not well characterized, leading to biased inferences. These biases can be particularly severe because BOED often targets extreme parameter regions as the most “informative,” potentially magnifying the impact of model error.

In this talk, we present a new information criterion, Expected Generalized Information Gain (EGIG)[1], that explicitly accounts for model discrepancy in BOED. EGIG augments standard Expected Information Gain by balancing the trade-off between experiment performance (i.e., how much information is gained) and robustness (i.e., how susceptible the design is to model misspecification). Concretely, EGIG measures how poorly inference under an incorrect model might perform, compared to a more appropriate model for the experiment. We will discuss the theoretical underpinnings of EGIG, as well as nested Monte Carlo algorithms for incorporating it into BOED for nonlinear inference problems. These methods handle both quantifiable discrepancies (e.g., low-fidelity vs. high-fidelity models) and unknown discrepancies represented by a distribution of potential errors, thereby enhancing the robustness and reliability of BOED in real-world settings.

SNL is managed and operated by NTESS under DOE NNSA contract DE-NA0003525. 
\medskip

\begin{enumerate}
	\item[{[1]}] Catanach, T. A., \& Das, N. (2023). {\it Metrics for bayesian optimal experiment design under model misspecification}. In 2023 62nd IEEE Conference on Decision and Control (CDC) (pp. 7707-7714). IEEE.
\end{enumerate}
\end{talk}

\end{document}

