\documentclass[12pt,a4paper,figuresright]{book}

\usepackage{amsmath,amssymb}
\usepackage{tabularx,graphicx,url,xcolor,rotating,multicol,epsfig,colortbl}
\usepackage{graphics}

\setlength{\textheight}{25.2cm}
\setlength{\textwidth}{16.5cm} %\setlength{\textwidth}{18.2cm}
\setlength{\voffset}{-1.6cm}
\setlength{\hoffset}{-0.3cm} %\setlength{\hoffset}{-1.2cm}
\setlength{\evensidemargin}{-0.3cm} 
\setlength{\oddsidemargin}{0.3cm}
\setlength{\parindent}{0cm} 
\setlength{\parskip}{0.3cm}

% -- adding a talk
\newenvironment{talk}[6]% [1] talk title
                         % [2] speaker name, [3] affiliations, [4] email,
                         % [5] coauthors, [6] special session
                         % [7] time slot
                         % [8] talk id, [9] session id or photo
 {%\needspace{6\baselineskip}%
  \vskip 0pt\nopagebreak%
%   \colorbox{gray!20!white}{\makebox[0.99\textwidth][r]{}}\nopagebreak%
%   \ifthenelse{\equal{#9}{photo}}{%
%                     \\\\\colorbox{gray!20!white}{\makebox{\includegraphics[width=3cm]{#8}}}\nopagebreak}{}%
 \vskip 0pt\nopagebreak%
%  \label{#8}%
  \textbf{#1}\vspace{3mm}\\\nopagebreak%
  \textit{#2}\\\nopagebreak%
  #3\\\nopagebreak%
  \url{#4}\vspace{3mm}\\\nopagebreak%
  \ifthenelse{\equal{#5}{}}{}{Coauthor(s): #5\vspace{3mm}\\\nopagebreak}%
  \ifthenelse{\equal{#6}{}}{}{Special session: #6\quad \vspace{3mm}\\\nopagebreak}%
 }
 {\vspace{1cm}\nopagebreak}%

\pagestyle{empty}

% ------------------------------------------------------------------------
% Document begins here
% ------------------------------------------------------------------------
\begin{document}
	
\begin{talk}
  {First-passage-based Last-passage Algorithm 
  	for Charge Density on a Conducting Surface}% [1] talk title
  {Chi-Ok Hwang}% [2] speaker name
  {Gwangju Institute of Science and Technology, Gwangju 61005, Republic of Korea }% [3] affiliations
  {chwang@gist.ac.kr}% [4] email
  {Jinseong Son, Maximiliano Islas Solis, Tsoggerel Tsogbadrakh}% [5] coauthors
  {}% [6] special session. Leave this field empty for contributed talks. 
				% Insert the title of the special session if you were invited to give a talk in a special session.
			
According to probabilistic potential theory, first- and last-passage algorithms have been devel-
oped. Usually the first-passage algorithms with an enclosing sphere are used for overall charge
distribution on a closed conducting object and last-passage algorithms for charge density at a
specific point on the conducting object. The first- and last-passage algorithms are inherently con-
nected. In this paper, we combine the first- and last-passage algorithms. We develop an algorithm
for computing charge density at a specific point on the conducting object via the overall charge
density distribution on a conducting object which is the simulation result of the first-passage al-
gorithm with an enclosing sphere. We demonstrate the algorithm for charge density on a sphere
and on the unit cube held at unit potential. The results show good agreements with theoretical or
other simulation ones.


\medskip

Acknowledgments: This work was supported by the GIST Research Institute (GRI) in 2024.
\begin{enumerate}
	\item[{[1]}] J. Son, J. Im, and C.-O. Hwang, Appl. Math. Comput. submitted (2021).
	\item[{[2]}] H. Jang, U. Yu, Y. Chung, and C.-O. Hwang, Adv. Theory Simul. 3(8) (2020).
	\item[{[3]}] H. Jang, J. Given, U. Yu, and C.-O. Hwang, Adv. Theory Simul.
	https://onlinelibrary.wiley.com/doi/full/10.1002/adts.202000268 (2021).
	\item[{[4]}] C.-O. Hwang and T. Won, J. Korean Phys. Soc. 47, S464 (2005).
	\item[{[5]}] J. A. Given, C.-O. Hwang, and M. Mascagni, Phys. Rev. E 66, 056704 (2002).
\end{enumerate}


\end{talk}

\end{document}

