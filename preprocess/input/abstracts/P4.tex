\documentclass[12pt,a4paper,figuresright]{book}

\usepackage{amsmath,amssymb}
\usepackage{tabularx,graphicx,url,xcolor,rotating,multicol,epsfig,colortbl}

\setlength{\textheight}{25.2cm}
\setlength{\textwidth}{16.5cm} %\setlength{\textwidth}{18.2cm}
\setlength{\voffset}{-1.6cm}
\setlength{\hoffset}{-0.3cm} %\setlength{\hoffset}{-1.2cm}
\setlength{\evensidemargin}{-0.3cm} 
\setlength{\oddsidemargin}{0.3cm}
\setlength{\parindent}{0cm} 
\setlength{\parskip}{0.3cm}

% -- adding a talk
\newenvironment{talk}[6]% [1] talk title
                         % [2] speaker name, [3] affiliations, [4] email,
                         % [5] coauthors, [6] special session
                         % [7] time slot
                         % [8] talk id, [9] session id or photo
 {%\needspace{6\baselineskip}%
  \vskip 0pt\nopagebreak%
%   \colorbox{gray!20!white}{\makebox[0.99\textwidth][r]{}}\nopagebreak%
%   \ifthenelse{\equal{#9}{photo}}{%
%                     \\\\\colorbox{gray!20!white}{\makebox{\includegraphics[width=3cm]{#8}}}\nopagebreak}{}%
 \vskip 0pt\nopagebreak%
%  \label{#8}%
  \textbf{#1}\vspace{3mm}\\\nopagebreak%
  \textit{#2}\\\nopagebreak%
  #3\\\nopagebreak%
  \url{#4}\vspace{3mm}\\\nopagebreak%
  \ifthenelse{\equal{#5}{}}{}{Coauthor(s): #5\vspace{3mm}\\\nopagebreak}%
  \ifthenelse{\equal{#6}{}}{}{Special session: #6\quad \vspace{3mm}\\\nopagebreak}%
 }
 {\vspace{1cm}\nopagebreak}%

\pagestyle{empty}

% ------------------------------------------------------------------------
% Document begins here
% ------------------------------------------------------------------------
\begin{document}
	
\begin{talk}
  {Sensitivity and Screening: From Monte Carlo to Experimental Design}% [1] talk title
  {Roshan Joseph}% [2] speaker name
  {Georgia Institute of Technology, Atlanta}% [3] affiliations
  {roshan@gatech.edu}% [4] email
  %{Names of coauthors go here, no affiliations of coauthors please, all affiliations will be included in an appendix of  the program book}% [5] coauthors
  {}% [6] special session. Leave this field empty for contributed talks. 
				% Insert the title of the special session if you were invited to give a talk in a special session.
			
Identifying the most important factors affecting the output of a system from a set of potentially important factors is an important problem in scientific investigations. If a computational model is available to predict the output, we can use global sensitivity analysis to quantify the importance of each factor. There are many Monte Carlo-based methods available to estimate global sensitivity indices. However, their computation can become costly if the model is computationally expensive. In such cases, carefully designed experiments can be used for screening the factors. In this talk, I will explain some of these techniques and the latest developments, including their applications in active learning. I will also briefly explain how to estimate the sensitivity indices from noisy data when we do not know or have access to the model that generated the data.
\medskip

\begin{enumerate}
	\item[{[1]}] Xiao, Q., Joseph, V. R., and Ray, D. M. (2023). {\it Maximum One-Factor-At-A-Time  Designs for Screening in Computer Experiments}. Technometrics, 65, 220-230.
    \item[{[2]}] Song, D. and Joseph, V. R. (2025). {\it Efficient Active Learning Strategies for Computer Experiments}. https://arxiv.org/abs/2501.13841.
	\item[{[3]}] Huang, C. and Joseph, V. R. (2025). {\it Factor Importance Ranking and Selection using Total Indices}. Technometrics, https://doi.org/10.1080/00401706.2025.2483531.
\end{enumerate}

%Equations may be used if they are referenced. Please note that the equation numbers may be different (but will be cross-referenced correctly) in the final program book.
\end{talk}

\end{document}

