\documentclass[12pt,a4paper,figuresright]{book}

\usepackage{amsmath,amssymb}
\usepackage{tabularx,graphicx,url,xcolor,rotating,multicol,epsfig,colortbl}

\setlength{\textheight}{25.2cm}
\setlength{\textwidth}{16.5cm} %\setlength{\textwidth}{18.2cm}
\setlength{\voffset}{-1.6cm}
\setlength{\hoffset}{-0.3cm} %\setlength{\hoffset}{-1.2cm}
\setlength{\evensidemargin}{-0.3cm} 
\setlength{\oddsidemargin}{0.3cm}
\setlength{\parindent}{0cm} 
\setlength{\parskip}{0.3cm}

% -- adding a talk
\newenvironment{talk}[6]% [1] talk title
                         % [2] speaker name, [3] affiliations, [4] email,
                         % [5] coauthors, [6] special session
                         % [7] time slot
                         % [8] talk id, [9] session id or photo
 {%\needspace{6\baselineskip}%
  \vskip 0pt\nopagebreak%
%   \colorbox{gray!20!white}{\makebox[0.99\textwidth][r]{}}\nopagebreak%
%   \ifthenelse{\equal{#9}{photo}}{%
%                     \\\\\colorbox{gray!20!white}{\makebox{\includegraphics[width=3cm]{#8}}}\nopagebreak}{}%
 \vskip 0pt\nopagebreak%
%  \label{#8}%
  \textbf{#1}\vspace{3mm}\\\nopagebreak%
  \textit{#2}\\\nopagebreak%
  #3\\\nopagebreak%
  \url{#4}\vspace{3mm}\\\nopagebreak%
  \ifthenelse{\equal{#5}{}}{}{Coauthor(s): #5\vspace{3mm}\\\nopagebreak}%
  \ifthenelse{\equal{#6}{}}{}{Special session: #6\quad \vspace{3mm}\\\nopagebreak}%
 }
 {\vspace{1cm}\nopagebreak}%

\pagestyle{empty}

% ------------------------------------------------------------------------
% Document begins here
% ------------------------------------------------------------------------
\begin{document}
	
\begin{talk}
  {Gradient-based MCMC in high dimensions}% [1] talk title
  {Reuben Cohn-Gordon}% [2] speaker name
  {University of California, Berkeley}% [3] affiliations
  {reubenharry@gmail.com}% [4] email
  {Jakob Robnik, Uroš Seljak}% [5] coauthors
  {}% [6] special session. Leave this field empty for contributed talks. 
				% Insert the title of the special session if you were invited to give a talk in a special session.
			
Sampling from distributions over $\mathbb{R}^d$ for $d$ larger than $10^4$ arises as a computational challenge in many of the physical sciences, including particle physics [1], condensed matter physics [2], cosmology [3] and chemistry [4], as well as in Bayesian statistics and machine learning [5]. Commonly used gradient-based variants of Markov Chain Monte Carlo such as Hamiltonian Monte Carlo (HMC) [6] and in particular the No U-Turn Sampler [7], are designed for differentiable multivariate densities, but struggle in very high dimensions. 
% The absence of a standard general purpose solution means that different fields use domain-specific approaches, despite the underlying similarity of the inference problem at hand. 
We propose a general purpose approach to the gradient-based high dimensional regime, based on two insights. First, in high dimensional cases where limited asymptotic bias is acceptable, Markov Chain algorithms without Metropolis-Hastings (MH) adjustment are more statistically efficient; we provide theoretical and numerical evidence for this claim and show how to choose a step size to limit the incurred bias to an acceptable level. Second, in the case that MH adjustment is required, we show that a particular 4th order integrator [8] drastically improves the statistical efficiency of HMC and related algorithms in high dimensions.

% (3) that Langevin noise on momenta in the dynamics can be chosen as a function of autocorrelation length and

\medskip

% If you would like to include references, please do so by creating a simple list numbered by [1], [2], [3], \ldots. See example below.
% Please do not use the \texttt{bibliography} environment or \texttt{bibtex} files.
% APA reference style is recommended.
\begin{enumerate}
\item[{[1]}] Duane, S., Kennedy, A. D., Pendleton, B. J., & Roweth, D. (1987). Hybrid monte carlo. Physics letters B, 195(2), 216-222.
\item[{[2]}] Lunts, P., Albergo, M. S., & Lindsey, M. (2023). Non-Hertz-Millis scaling of the antiferromagnetic quantum critical metal via scalable Hybrid Monte Carlo. Nature communications, 14(1), 2547.
\item[{[3]}] Lewis, A., & Bridle, S. (2002). Cosmological parameters from CMB and other data: A Monte Carlo approach. Physical Review D, 66(10), 103511.
\item[{[4]}] Tuckerman, M. E. (2023). Statistical mechanics: theory and molecular simulation. Oxford university press.
% \item[{[5]}] Betancourt, M., & Girolami, M. (2015). Hamiltonian Monte Carlo for hierarchical models. Current trends in Bayesian methodology with applications, 79(30), 2-4.
\item[{[5]}] Cobb, A. D., & Jalaian, B. (2021, December). Scaling Hamiltonian Monte Carlo inference for Bayesian neural networks with symmetric splitting. In Uncertainty in Artificial Intelligence (pp. 675-685). PMLR.
\item[{[6]}] Betancourt, M. (2017). A conceptual introduction to Hamiltonian Monte Carlo. arXiv preprint arXiv:1701.02434.
\item[{[7]}] Hoffman, M. D., & Gelman, A. (2014). The No-U-Turn sampler: adaptively setting path lengths in Hamiltonian Monte Carlo. J. Mach. Learn. Res., 15(1), 1593-1623.
% \item[{[6]}] Minary, P., Martyna, G. J., & Tuckerman, M. E. (2003). Algorithms and novel applications based on the isokinetic ensemble. I. Biophysical and path integral molecular dynamics. The Journal of chemical physics, 118(6), 2510-2526.
% \item[{[7]}] Robnik, J., De Luca, G. B., Silverstein, E., & Seljak, U. (2023). Microcanonical hamiltonian monte carlo. Journal of Machine Learning Research, 24(311), 1-34.
\item[{[8]}] Takaishi, T., & De Forcrand, P. (2006). Testing and tuning symplectic integrators for the hybrid Monte Carlo algorithm in lattice QCD. Physical Review E—Statistical, Nonlinear, and Soft Matter Physics, 73(3), 036706.
\end{enumerate}
% 	\item[{[1]}] Niederreiter, Harald (1992). {\it Random number generation and quasi-Monte Carlo methods}. Society for Industrial and Applied Mathematics (SIAM).
% 	\item[{[2]}] Roberts, Gareth O, \& Rosenthal, Jeffrey S. (2002).  Optimal scaling for various Metropolis-Hastings algorithms, \textbf{16}(4), 351--367.
% \end{enumerate}

% Equations may be used if they are referenced. Please note that the equation numbers may be different (but will be cross-referenced correctly) in the final program book.
\end{talk}

\end{document}
