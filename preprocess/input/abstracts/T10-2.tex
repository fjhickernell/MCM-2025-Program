\documentclass[12pt,a4paper,figuresright]{book}

\usepackage{amsmath,amssymb}
\usepackage{tabularx,graphicx,url,xcolor,rotating,multicol,epsfig,colortbl}

\setlength{\textheight}{25.2cm}
\setlength{\textwidth}{16.5cm} %\setlength{\textwidth}{18.2cm}
\setlength{\voffset}{-1.6cm}
\setlength{\hoffset}{-0.3cm} %\setlength{\hoffset}{-1.2cm}
\setlength{\evensidemargin}{-0.3cm} 
\setlength{\oddsidemargin}{0.3cm}
\setlength{\parindent}{0cm} 
\setlength{\parskip}{0.3cm}

% -- adding a talk
\newenvironment{talk}[6]% [1] talk title
                         % [2] speaker name, [3] affiliations, [4] email,
                         % [5] coauthors, [6] special session
                         % [7] time slot
                         % [8] talk id, [9] session id or photo
 {%\needspace{6\baselineskip}%
  \vskip 0pt\nopagebreak%
%   \colorbox{gray!20!white}{\makebox[0.99\textwidth][r]{}}\nopagebreak%
%   \ifthenelse{\equal{#9}{photo}}{%
%                     \\\\\colorbox{gray!20!white}{\makebox{\includegraphics[width=3cm]{#8}}}\nopagebreak}{}%
 \vskip 0pt\nopagebreak%
%  \label{#8}%
  \textbf{#1}\vspace{3mm}\\\nopagebreak%
  \textit{#2}\\\nopagebreak%
  #3\\\nopagebreak%
  \url{#4}\vspace{3mm}\\\nopagebreak%
  \ifthenelse{\equal{#5}{}}{}{Coauthor(s): #5\vspace{3mm}\\\nopagebreak}%
  \ifthenelse{\equal{#6}{}}{}{Special session: #6\quad \vspace{3mm}\\\nopagebreak}%
 }
 {\vspace{1cm}\nopagebreak}%

\pagestyle{empty}

% ------------------------------------------------------------------------
% Document begins here
% ------------------------------------------------------------------------
\begin{document}
	
\begin{talk}
  {Langevin-based strategies for nested particle filters}% [1] talk title
  {Sara Pérez-Vieites}% [2] speaker name
  {Aalto University}% [3] affiliations
  {sara.perezvieites@aalto.fi}% [4] email
  {Nicola Branchini, Víctor Elvira and Joaquín Míguez}% [5] coauthors
  {}% [6] special session. Leave this field empty for contributed talks. 
				% Insert the title of the special session if you were invited to give a talk in a special session.
			
Many problems in some of the most active fields of science require to estimate parameters and predict the evolution of complex dynamical systems using sequentially collected data. The nested particle filter (NPF) framework stands out since it is the only fully recursive probabilistic method for Bayesian inference. That is, it computes the joint posterior distribution of the parameters and states while maintaining a computational complexity of $\mathcal{O}(T)$, which makes it particularly suitable for long observation sequences. 

A key strategy to keep particle diversity in the parameter space, given the static nature of the parameters, is jittering. The parameter space is explored by perturbing a subset of particles with arbitrary variance or applying a controlled variance to all particles. As the perturbations are controlled, it ensures convergence to the true posterior distribution while keeping the full framework recursive. However, this is not an efficient exploration strategy, particularly for problems with a higher dimension in the parameter space.

To address this limitation, we propose a Langevin-based methodology within the NPF framework. A challenge is that the required score function is intractable. We propose to approximate the score with an accurate method that is provably stable over time, and to explore strategies to reduce its computational cost while retaining accuracy.
This approach significantly improves the scalability of NPF in the parameter dimension, while still ensuring asymptotic convergence to the true posterior, as well as maintaining computational feasibility.

\medskip


%If you would like to include references, please do so by creating a simple list numbered by [1], [2], [3], \ldots. See example below.
%Please do not use the \texttt{bibliography} environment or \texttt{bibtex} files.
%APA reference style is recommended.
%\begin{enumerate}
%	\item[{[1]}] Niederreiter, Harald (1992). {\it Random number generation and quasi-Monte Carlo methods}. Society for Industrial and Applied Mathematics (SIAM).
%	\item[{[2]}] Roberts, Gareth O, \& Rosenthal, Jeffrey S. (2002).  Optimal scaling for various Metropolis-Hastings algorithms, \textbf{16}(4), 351--367.
%\end{enumerate}

%Equations may be used if they are referenced. Please note that the equation numbers may be different (but will be cross-referenced correctly) in the final program book.
\end{talk}

\end{document}

