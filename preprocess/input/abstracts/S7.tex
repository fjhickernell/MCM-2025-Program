\documentclass[12pt,a4paper,figuresright]{book}

\usepackage{amsmath,amssymb}
\usepackage{tabularx,multirow,graphicx,url,wrapfig,xcolor,rotating,multicol,epsfig,colortbl}

\setlength{\textheight}{25.2cm}
\setlength{\textwidth}{16.5cm} %\setlength{\textwidth}{18.2cm}
\setlength{\voffset}{-1.6cm}
\setlength{\hoffset}{-0.3cm} %\setlength{\hoffset}{-1.2cm}
\setlength{\evensidemargin}{-0.3cm} 
\setlength{\oddsidemargin}{0.3cm}
\setlength{\parindent}{0cm} 
\setlength{\parskip}{0.3cm}

\renewcommand{\topfraction}{1}
\renewcommand{\textfraction}{0}
\setlength{\floatsep}{12pt plus 2pt minus 2pt}

\newcommand{\organizer}[3]{%
	{\textit{#1}}\\\nopagebreak%
	#2\\\nopagebreak%
	\url{#3}\vspace{3mm}\\\nopagebreak%
	}

\newenvironment{session}[5] % [1] session title
							% [2] number of organizers
                            % [3] organizer 1 info
                            % [4] organizer 2 info
                            % [5] organizer 3 info
                            % [6] session id for later
 {%\needspace{6\baselineskip}
  \vskip 0pt\nopagebreak%
  %\label{#5}%
  \textbf{#1}\vspace{3mm}\\\nopagebreak%
  \ifthenelse{\equal{#2}{1}}{Organizer:}{Organizers:}%
  \vspace{2mm}\\\nopagebreak%
  #3
  \ifthenelse{\equal{#2}{2}}{#4}{}%
  \ifthenelse{\equal{#2}{3}}{#4#5}{}%
  \quad\\\nopagebreak%
  %Session Description:\vspace{3mm}\\\nopagebreak%
 }
 {\nopagebreak}%

\pagestyle{empty}

% ------------------------------------------------------------------------
% Document begins here
% ------------------------------------------------------------------------
\begin{document}
	
%Input the relevant information below
\begin{session}
  {Computational Methods for Low-discrepancy Sampling and Applications}% [1] session title
  {2} %[2]  number of organizers
  {\organizer{Nathan Kirk}% organizer one name
    {Illinois Institute of Technology}% orgnizer one affiliations
    {nkirk@iit.edu}}% organizer one email
  {\organizer{François Clément}% organizer two name, if needed
	{University of Washington}% orgnizer two affiliations, if needed
	{fclement@uw.edu}}% organizer two email
  {\organizer{Name three}% organizer one name
	{Affiliation(s) three}% orgnizer one affiliations
	{organizer-three-email-goes@here}}% organizer one email

This session aims to showcase recent advancements in the optimization of sample point distributions [1, 2] and their applications. Some of the methods on display will range from deep learning methods to permutation optimization and greedy approaches [3], showcasing the usefulness of the $L_2$-discrepancies in optimizing the $L_{\infty}$ discrepancies. As a consequence of some of these improved low-discrepancy sets, an application will be shown in improved path planning in robotics [4]. Several other applications will be explored in the context of using the median over the mean of $r$ RQMC estimates as proposed in several recent papers including [5].

\medskip


\begin{enumerate}
	\item[{[1]}] T. K. Rusch, N. Kirk, M. Bronstein, C. Lemieux and D. Rus, \textit{Message-Passing Monte Carlo: Generating low-discrepancy point sets via graph neural networks}, PNAS \textbf{121} (40) e2409913121 (2024)

	\item[{[2]}] F. Clément, C. Doerr, K. Klamroth, L. Paquete, \textit{Transforming the Challenge of Constructing Low-Discrepancy Point Sets into a Permutation Selection Problem}, \url{https://arxiv.org/abs/2407.11533}.
    \item[{[3]}] F. Cl\'ement, \textit{Outperforming the Best {1D} Low-Discrepancy Constructions with a Greedy Algorithm}, \url{https://arxiv.org/abs/2406.18132}.
        \item[{[4]}] M. Chahine, T. K. Rusch, Z. J. Patterson and D. Rus, \textit{Improving Efficiency of Sampling-based Motion Planning via Message-Passing Monte Carlo}, \url{https://arxiv.org/abs/2410.03909}
        \item[{[5]}] P. L'Ecuyer, M. K. Nayakama, A. B. Owen and B. Tuffin, \textit{Confidence Intervals for Randomized Quasi-Monte Carlo Estimators}, Proceedings of the 2023 Winter Simulation Conference (2023)
\end{enumerate}


\end{session}

\end{document}
