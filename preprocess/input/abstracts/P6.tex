\documentclass[12pt,a4paper,figuresright]{book}

\usepackage{amsmath,amssymb}
\usepackage{tabularx,graphicx,url,xcolor,rotating,multicol,epsfig,colortbl}

\setlength{\textheight}{25.2cm}
\setlength{\textwidth}{16.5cm} %\setlength{\textwidth}{18.2cm}
\setlength{\voffset}{-1.6cm}
\setlength{\hoffset}{-0.3cm} %\setlength{\hoffset}{-1.2cm}
\setlength{\evensidemargin}{-0.3cm} 
\setlength{\oddsidemargin}{0.3cm}
\setlength{\parindent}{0cm} 
\setlength{\parskip}{0.3cm}

% -- adding a talk
\newenvironment{talk}[6]% [1] talk title
                         % [2] speaker name, [3] affiliations, [4] email,
                         % [5] coauthors, [6] special session
                         % [7] time slot
                         % [8] talk id, [9] session id or photo
 {%\needspace{6\baselineskip}%
  \vskip 0pt\nopagebreak%
%   \colorbox{gray!20!white}{\makebox[0.99\textwidth][r]{}}\nopagebreak%
%   \ifthenelse{\equal{#9}{photo}}{%
%                     \\\\\colorbox{gray!20!white}{\makebox{\includegraphics[width=3cm]{#8}}}\nopagebreak}{}%
 \vskip 0pt\nopagebreak%
%  \label{#8}%
  \textbf{#1}\vspace{3mm}\\\nopagebreak%
  \textit{#2}\\\nopagebreak%
  #3\\\nopagebreak%
  \url{#4}\vspace{3mm}\\\nopagebreak%
  \ifthenelse{\equal{#5}{}}{}{Coauthor(s): #5\vspace{3mm}\\\nopagebreak}%
  \ifthenelse{\equal{#6}{}}{}{Special session: #6\quad \vspace{3mm}\\\nopagebreak}%
 }
 {\vspace{1cm}\nopagebreak}%

\pagestyle{empty}

% ------------------------------------------------------------------------
% Document begins here
% ------------------------------------------------------------------------
\begin{document}
	
\begin{talk}
  {Gradient-Based MCMC Sampling: Methods and Optimization Strategies}% [1] talk title
  {Uro\v s Seljak}% [2] speaker name
  {UC Berkeley and Lawrence Berkeley National Laboratory}% [3] affiliations
  {useljak@berkeley.edu}% [4] email
  {Reuben Cohn-Gordon, Jakob Robnik}% [5] coauthors
  {}% [6] special session. Leave this field empty for contributed talks. 
				% Insert the title of the special session if you were invited to give a talk in a special session.
			
\vspace{-5ex}Gradient-based Markov Chain Monte Carlo (MCMC) methods significantly outperform gradient-free alternatives in sampling efficiency, particularly in high-dimensional spaces where they have become the standard approach. These methods leverage gradient information to guide the sampling process more intelligently than random-walk approaches.
Two fundamental approaches that dominate this field are 1)
Hamiltonian Monte Carlo (HMC), which  employs principles from classical mechanics, treating the sampling problem as simulating Hamiltonian dynamics on an extended phase space. This approach naturally incorporates momentum variables that help the sampler traverse the parameter space more efficiently than simple random walks. 2)
Langevin Monte Carlo (LMC), which utilizes stochastic differential equations that incorporate both gradient information and controlled noise injection. 
Recent theoretical developments have produced microcanonical versions of both Hamiltonian and Langevin samplers (MCHMC and MCLMC). These variants demonstrate measurably superior sampling efficiency compared to their canonical predecessors.

In addition to the choice of 
the method, 
practitioners face numerous algorithmic choices that can significantly impact performance:
1) Metropolis Adjustment: The decision whether to include Metropolis-Hastings correction steps involves trading exact preservation of the target distribution against computational speed.
2) Preconditioning: Incorporating problem-specific geometric information through preconditioning matrices can dramatically improve convergence rates, particularly for ill-conditioned target distributions.
3) Hyperparameter Tuning: Critical parameters include step sizes, trajectory lengths for HMC, and damping coefficients for Langevin methods. Recently, well tuned black-box methods have been developed that approach optimal performance. 
4) Parallelization Strategy: parallel sampling on a GPU or CPU cluster enables dramatically reduced wall clock time to reach the required target accuracy. 
5) Numerical Integration: Higher-order integrators can improve accuracy at the cost of additional gradient evaluations per step.

This goal of this talk is to provide
guidance to 
the optimal choice among these methods, which depends on specific application requirements including computational budget, accuracy demands, and problem dimensionality. Understanding the theoretical trade-offs enables practitioners to select and configure samplers that best match their particular constraints and objectives.

\vspace{-10ex}
\end{talk}

\end{document}


