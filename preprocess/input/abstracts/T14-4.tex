\documentclass[12pt,a4paper,figuresright]{book}

\usepackage{amsmath,amssymb}
\usepackage{tabularx,graphicx,url,xcolor,rotating,multicol,epsfig,colortbl}

\setlength{\textheight}{25.2cm}
\setlength{\textwidth}{16.5cm} %\setlength{\textwidth}{18.2cm}
\setlength{\voffset}{-1.6cm}
\setlength{\hoffset}{-0.3cm} %\setlength{\hoffset}{-1.2cm}
\setlength{\evensidemargin}{-0.3cm} 
\setlength{\oddsidemargin}{0.3cm}
\setlength{\parindent}{0cm} 
\setlength{\parskip}{0.3cm}

% -- adding a talk
\newenvironment{talk}[6]% [1] talk title
                         % [2] speaker name, [3] affiliations, [4] email,
                         % [5] coauthors, [6] special session
                         % [7] time slot
                         % [8] talk id, [9] session id or photo
 {%\needspace{6\baselineskip}%
  \vskip 0pt\nopagebreak%
%   \colorbox{gray!20!white}{\makebox[0.99\textwidth][r]{}}\nopagebreak%
%   \ifthenelse{\equal{#9}{photo}}{%
%                     \\\\\colorbox{gray!20!white}{\makebox{\includegraphics[width=3cm]{#8}}}\nopagebreak}{}%
 \vskip 0pt\nopagebreak%
%  \label{#8}%
  \textbf{#1}\vspace{3mm}\\\nopagebreak%
  \textit{#2}\\\nopagebreak%
  #3\\\nopagebreak%
  \url{#4}\vspace{3mm}\\\nopagebreak%
  \ifthenelse{\equal{#5}{}}{}{Coauthor(s): #5\vspace{3mm}\\\nopagebreak}%
  \ifthenelse{\equal{#6}{}}{}{Special session: #6\quad \vspace{3mm}\\\nopagebreak}%
 }
 {\vspace{1cm}\nopagebreak}%

\pagestyle{empty}

% ------------------------------------------------------------------------
% Document begins here
% ------------------------------------------------------------------------
\begin{document}
\begin{talk}
  {Randomness in the quantum age: A Comparative Study of Classical and Quantum Random Number Generators}% [1] talk title
  {Hongmei Chi}% [2] speaker name
  {Florida A\&M University}% [3] affiliations
  {Hongmei.chi@famu.edu}% [4] email
 % {Names of coauthors go here, no affiliations of coauthors please, all affiliations will be included in an appendix of the program book}% [5] coauthors
 {}% [6] special session. Leave this field empty for contributed talks. 
				% Insert the title of the special session if you were invited to give a talk in a special session. 
The advent of quantum computing marks a pivotal transformation in the domain of computational sciences, particularly in the generation and utilization of randomness. This paper provides a critical examination of Random Number Generators (RNGs) within the context of the quantum computing era, with a specific focus on the emerging class of Quantum Random Number Generators (QRNGs), which leverage the inherent unpredictability of quantum mechanical processes to produce true random numbers, offering significant advantages over classical pseudorandom number generators. 
In this paper, we present a comprehensive analysis of QRNG principles, architectures, and implementations, focusing on key quantum phenomena such as photon path selection, quantum vacuum fluctuations, and quantum phase noise. We examine the theoretical foundations that guarantee randomness, explore entropy extraction techniques, and evaluate the performance metrics critical for cryptographic and high-security applications [1]. A comparative assessment of existing commercial and laboratory-scale QRNG systems is provided, highlighting trade-offs between speed, complexity, and certifiability of randomness [2].  We address the challenges in device-independent randomness certification and integration into modern communication and computing infrastructures. Our findings underscore QRNGs as essential components in advancing secure quantum technologies, with potential for wide-scale deployment in future quantum networks and cryptographic protocols [3].

 %We investigate the theoretical underpinnings of QRNGs, grounded in fundamental quantum phenomena such as superposition and entanglement, which guarantee true randomness as prescribed by quantum indeterminacy. This work contrasts classical and quantum approaches, evaluates current QRNG implementations, and addresses the practical challenges in achieving scalable, secure, and certifiable quantum randomness. 
 Finally, we explore the implications of QRNGs for post-quantum cryptographic protocols and assess the broader impact of quantum randomness on algorithmic design and security frameworks. This study underscores the necessity of integrating quantum-enhanced randomness into next-generation computational systems to ensure robustness in the face of quantum adversaries [4].

\medskip

\begin{enumerate}
\item[{[1]}] Cirauqui, D., García-March, M. Á., Guigó Corominas, G., Graß, T., Grzybowski, P. R., Muñoz-Gil, G., ... \& Lewenstein, M. (2024). Comparing pseudo-and quantum-random number generators with Monte Carlo simulations. APL Quantum,  \textbf{1}(3).

\item[{[2]}]  Lin, H., Lu, H., Wong, M. S., Almogbel, A., Alyamani, A., Ng, T. K., ... \& Ooi, B. S. (2025). Micro-LED-based quantum random number generators. Optics Express, \textbf{33}(11), 22154-22164.

\item[{[3]}]  Yang, Y., Lin, Y., Xiao, J., \& Zhong, Z. (2025). Feasibility discussion of quantum cryptography for Internet of Things security: a literature review. Optical and Quantum Electronics, \textbf{57}(5), 264.

\item [{[4]}]  Jacak, M. M., Jóźwiak, P., Niemczuk, J., \& Jacak, J. E. (2021). Quantum generators of random numbers. Scientific Reports, \textbf{11}(1), 16108.
\end{enumerate}

\end{talk}

\end{document}

