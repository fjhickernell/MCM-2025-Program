
\documentclass[12pt,a4paper,figuresright]{book}

\usepackage{amsmath,amssymb}
\usepackage{tabularx,graphicx,url,xcolor,rotating,multicol,epsfig,colortbl}

\setlength{\textheight}{25.2cm}
\setlength{\textwidth}{16.5cm} %\setlength{\textwidth}{18.2cm}
\setlength{\voffset}{-1.6cm}
\setlength{\hoffset}{-0.3cm} %\setlength{\hoffset}{-1.2cm}
\setlength{\evensidemargin}{-0.3cm} 
\setlength{\oddsidemargin}{0.3cm}
\setlength{\parindent}{0cm} 
\setlength{\parskip}{0.3cm}

% -- adding a talk
\newenvironment{talk}[6]% [1] talk title
                         % [2] speaker name, [3] affiliations, [4] email,
                         % [5] coauthors, [6] special session
                         % [7] time slot
                         % [8] talk id, [9] session id or photo
 {%\needspace{6\baselineskip}%
  \vskip 0pt\nopagebreak%
%   \colorbox{gray!20!white}{\makebox[0.99\textwidth][r]{}}\nopagebreak%
%   \ifthenelse{\equal{#9}{photo}}{%
%                     \\\\\colorbox{gray!20!white}{\makebox{\includegraphics[width=3cm]{#8}}}\nopagebreak}{}%
 \vskip 0pt\nopagebreak%
%  \label{#8}%
  \textbf{#1}\vspace{3mm}\\\nopagebreak%
  \textit{#2}\\\nopagebreak%
  #3\\\nopagebreak%
  \url{#4}\vspace{3mm}\\\nopagebreak%
  \ifthenelse{\equal{#5}{}}{}{Coauthor(s): #5\vspace{3mm}\\\nopagebreak}%
  \ifthenelse{\equal{#6}{}}{}{Special session: #6\quad \vspace{3mm}\\\nopagebreak}%
 }
 {\vspace{1cm}\nopagebreak}%

\pagestyle{empty}

% ------------------------------------------------------------------------
% Document begins here
% ------------------------------------------------------------------------
\begin{document}
	
\begin{talk}
  {Examining the Fault Tolerance of High-Performance Monte Carlo Applications through Simulation}% [1] talk title
  {Sharanya Jayaraman}% [2] speaker name
  {Department of Computer Science, Florida State University}% [3] affiliations
  {sjayaraman@fsu.edu}% [4] email
  {}% [5] coauthors
  {Monte Carlo Applications in High-performance Computing, Computer Graphics, and Computational Science}% [6] special session. 
			
Monte Carlo methods are naturally fault-tolerant due to their stochastic nature and are highly scalable, making them well-suited for high-performance computing environments$^{[1]}$. In this work, we use a custom-built, cloud-based exascale simulator to investigate the resilience of Monte Carlo and multi-level Monte Carlo methods under various hardware and software fault scenarios. We examine the behavior of the calculations and the propagation of inaccuracies in the presence and absence of common mitigation strategies, such as checkpointing$^{[2]}$. Our simulations offer insights into how faults propagate through computations, enabling us to evaluate techniques for constraining and mitigating the impact of unpredictable failures. The results may be used as guidance for deploying Monte Carlo-based applications on exascale systems, thereby enhancing their reliability.

\medskip

References
\begin{enumerate}
	\item[{[1]}] Pauli, Stefan, Arbenz, Peter, \&  Schwab, Christoph, (2015). {\it Intrinsic fault tolerance of multilevel Monte Carlo methods}, Journal of Parallel and Distributed Computing \textbf{84}, 24-36.
	\item[{[2]}] Chang, C., Deringer, V.L., Katti, K.S. et al. (2023) {\it Simulations in the era of exascale computing}, Nat Rev Mater \textbf{8}, 309–313 
\end{enumerate}

\end{talk}

\end{document}

