\documentclass[12pt,a4paper,figuresright]{book}

\usepackage{amsmath,amssymb,mathtools,amssymb, bm,amsthm}
\usepackage{tabularx,graphicx,url,xcolor,rotating,multicol,epsfig,colortbl}

\setlength{\textheight}{25.2cm}
\setlength{\textwidth}{16.5cm} %\setlength{\textwidth}{18.2cm}
\setlength{\voffset}{-1.6cm}
\setlength{\hoffset}{-0.3cm} %\setlength{\hoffset}{-1.2cm}
\setlength{\evensidemargin}{-0.3cm} 
\setlength{\oddsidemargin}{0.3cm}
\setlength{\parindent}{0cm} 
\setlength{\parskip}{0.3cm}

% -- adding a talk
\newenvironment{talk}[6]% [1] talk title
                         % [2] speaker name, [3] affiliations, [4] email,
                         % [5] coauthors, [6] special session
                         % [7] time slot
                         % [8] talk id, [9] session id or photo
 {%\needspace{6\baselineskip}%
  \vskip 0pt\nopagebreak%
%   \colorbox{gray!20!white}{\makebox[0.99\textwidth][r]{}}\nopagebreak%
%   \ifthenelse{\equal{#9}{photo}}{%
%                     \\\\\colorbox{gray!20!white}{\makebox{\includegraphics[width=3cm]{#8}}}\nopagebreak}{}%
 \vskip 0pt\nopagebreak%
%  \label{#8}%
  \textbf{#1}\vspace{3mm}\\\nopagebreak%
  \textit{#2}\\\nopagebreak%
  #3\\\nopagebreak%
  \url{#4}\vspace{3mm}\\\nopagebreak%
  \ifthenelse{\equal{#5}{}}{}{Coauthor(s): #5\vspace{3mm}\\\nopagebreak}%
  \ifthenelse{\equal{#6}{}}{}{Special session: #6\quad \vspace{3mm}\\\nopagebreak}%
 }
 {\vspace{1cm}\nopagebreak}%

\pagestyle{empty}

% ------------------------------------------------------------------------
% Document begins here
% ------------------------------------------------------------------------
\begin{document}
	
\begin{talk}
  {The Quality of Lattice Sequences}% [1] talk title
  {Larysa Matiukha}% [2] speaker name
  {Illinois Institute of Technology}% [3] affiliations
  {lmatiukha@hawk.iit.edu}% [4] email
  {Yuhan Ding, Fred J. Hickernell}% [5] coauthors
  {}% [6] special session. Leave this field empty for contributed talks. 
				% Insert the title of the special session if you were invited to give a talk in a special session.

Lattices are a popular choice of nodes for approximating multidimensional integrals by a sample mean, typically using sample sizes of the form $n = b^m$ for some prime base $b$ and non-negative integer $m$. However, a computational time budget or hardware failure may prevent us from choosing the preferred number of samples. In this talk, we present an upper bound on the figure of merit $P_\alpha$ for extensible lattice sequences with arbitrary $n$, derived in Banach space setting. We show that while the error decays relatively slowly for general $n$, it improves significantly when $n$ is a small integer multiple of a power of the base, reaching the optimal decay of $\mathcal{O}(n^{-\alpha})$. % maybe not "optimal" 
Additionally, we investigated a related figure of merit $P_{\alpha,2}$ in a Hilbert space, allowing easier computation for arbitrary $n$. Numerical results for $P_{1,2}$ using both equal and optimally chosen sample weights show a decay of $\mathcal{O}(n^{-1})$ in both cases, while optimal weights lead to a generally smaller and non-increasing error. 


% Motivated by a scenario where, due to computational time constraint or computer failure, we cannot choose the preferred sample size $n = b^m$, we derive an upper bound on the figure of merit $P_\alpha$ for lattice sequences with an arbitrary number of points $n$. 
% Our derivation demonstrates that the error decays relatively slowly for  general $n$, but when $n$ is a small multiple of a power of the base $b$, the decay is close to that of the preferred values of $n$.
%Your abstract goes here. Please do not use your own commands or macros.

% \medskip

% If you would like to include references, please do so by creating a simple list numbered by [1], [2], [3], \ldots. See example below.
% Please do not use the \texttt{bibliography} environment or \texttt{bibtex} files.
% APA reference style is recommended.
% \begin{enumerate}
% 	\item[{[1]}] Niederreiter, Harald (1992). {\it Random number generation and quasi-Monte Carlo methods}. Society for Industrial and Applied Mathematics (SIAM).
% 	\item[{[2]}] Roberts, Gareth O, \& Rosenthal, Jeffrey S. (2002).  Optimal scaling for various Metropolis-Hastings algorithms, \textbf{16}(4), 351--367.
% \end{enumerate}

% Equations may be used if they are referenced. Please note that the equation numbers may be different (but will be cross-referenced correctly) in the final program book.
\end{talk}

\end{document}

