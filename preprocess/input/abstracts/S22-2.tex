\documentclass[12pt,a4paper,figuresright]{book}

\usepackage{amsmath,amssymb}
\usepackage{tabularx,graphicx,url,xcolor,rotating,multicol,epsfig,colortbl}

\setlength{\textheight}{25.2cm}
\setlength{\textwidth}{16.5cm} %\setlength{\textwidth}{18.2cm}
\setlength{\voffset}{-1.6cm}
\setlength{\hoffset}{-0.3cm} %\setlength{\hoffset}{-1.2cm}
\setlength{\evensidemargin}{-0.3cm} 
\setlength{\oddsidemargin}{0.3cm}
\setlength{\parindent}{0cm} 
\setlength{\parskip}{0.3cm}

% -- adding a talk
\newenvironment{talk}[6]% [1] talk title
                         % [2] speaker name, [3] affiliations, [4] email,
                         % [5] coauthors, [6] special session
                         % [7] time slot
                         % [8] talk id, [9] session id or photo
 {%\needspace{6\baselineskip}%
  \vskip 0pt\nopagebreak%
%   \colorbox{gray!20!white}{\makebox[0.99\textwidth][r]{}}\nopagebreak%
%   \ifthenelse{\equal{#9}{photo}}{%
%                     \\\\\colorbox{gray!20!white}{\makebox{\includegraphics[width=3cm]{#8}}}\nopagebreak}{}%
 \vskip 0pt\nopagebreak%
%  \label{#8}%
  \textbf{#1}\vspace{3mm}\\\nopagebreak%
  \textit{#2}\\\nopagebreak%
  #3\\\nopagebreak%
  \url{#4}\vspace{3mm}\\\nopagebreak%
  \ifthenelse{\equal{#5}{}}{}{Coauthor(s): #5\vspace{3mm}\\\nopagebreak}%
  \ifthenelse{\equal{#6}{}}{}{Special session: #6\quad \vspace{3mm}\\\nopagebreak}%
 }
 {\vspace{1cm}\nopagebreak}%

\pagestyle{empty}

% ------------------------------------------------------------------------
% Document begins here
% ------------------------------------------------------------------------
\begin{document}
	
\begin{talk}
  {High-dimensional density estimation on  unbounded domain}% [1] talk title
  {Zhijian He}% [2] speaker name
  {South China University of Technology}% [3] affiliations
  {hezhijian@scut.edu.cn}% [4] email
  {Ziyang Ye, Haoyuan Tan, Xiaoqun Wang}% [5] coauthors
  {QMC and Applications Parts I and II}% [6] special session. Leave this field empty for contributed talks. 
				% Insert the title of the special session if you were invited to give a talk in a special session.
			
This talk will present a kernel-based method to approximate probability density functions of unbounded random variables taking values in high-dimensional spaces. Building upon the framework of Kazashi and Nobile [1], our estimator is a linear combination of kernel functions whose coefficients are determined a linear equation. We first transform the unbounded sample domain into a hyper cube and then use rank-1 lattice points as the interpolation nodes.
We establish a rigorous error analysis for the mean integrated squared error (MISE) under an exponential decay conditon.  Under a suitable smoothness assumption, our method attains an MISE rate  approaching $O(N^{-1})$ for $N$ independent identically distributed observations. Numerical experiments validate our theoretical findings and demonstrate the superior performance of the proposed estimator compared to state-of-the-art alternatives.
\medskip


\begin{enumerate}
	\item[{[1]}] Kazashi, Yoshihito, \& Nobile, Fabio. (2023). Density estimation in {RKHS} with application to Korobov spaces in high dimensions, SIAM Journal on Numerical Analysis, \textbf{61}(2), 1080-1102.
\end{enumerate}

\end{talk}

\end{document}

