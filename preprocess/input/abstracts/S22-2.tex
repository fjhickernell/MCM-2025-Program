\documentclass[12pt,a4paper,figuresright]{book}

\usepackage{amsmath,amssymb}
\usepackage{tabularx,graphicx,url,xcolor,rotating,multicol,epsfig,colortbl}

\setlength{\textheight}{25.2cm}
\setlength{\textwidth}{16.5cm} %\setlength{\textwidth}{18.2cm}
\setlength{\voffset}{-1.6cm}
\setlength{\hoffset}{-0.3cm} %\setlength{\hoffset}{-1.2cm}
\setlength{\evensidemargin}{-0.3cm} 
\setlength{\oddsidemargin}{0.3cm}
\setlength{\parindent}{0cm} 
\setlength{\parskip}{0.3cm}

% -- adding a talk
\newenvironment{talk}[6]% [1] talk title
                         % [2] speaker name, [3] affiliations, [4] email,
                         % [5] coauthors, [6] special session
                         % [7] time slot
                         % [8] talk id, [9] session id or photo
 {%\needspace{6\baselineskip}%
  \vskip 0pt\nopagebreak%
%   \colorbox{gray!20!white}{\makebox[0.99\textwidth][r]{}}\nopagebreak%
%   \ifthenelse{\equal{#9}{photo}}{%
%                     \\\\\colorbox{gray!20!white}{\makebox{\includegraphics[width=3cm]{#8}}}\nopagebreak}{}%
 \vskip 0pt\nopagebreak%
%  \label{#8}%
  \textbf{#1}\vspace{3mm}\\\nopagebreak%
  \textit{#2}\\\nopagebreak%
  #3\\\nopagebreak%
  \url{#4}\vspace{3mm}\\\nopagebreak%
  \ifthenelse{\equal{#5}{}}{}{Coauthor(s): #5\vspace{3mm}\\\nopagebreak}%
  \ifthenelse{\equal{#6}{}}{}{Special session: #6\quad \vspace{3mm}\\\nopagebreak}%
 }
 {\vspace{1cm}\nopagebreak}%

\pagestyle{empty}

% ------------------------------------------------------------------------
% Document begins here
% ------------------------------------------------------------------------
\begin{document}
	
\begin{talk}
  {$L_2$-approximation: using randomized lattice algorithms and QMC hyperinterpolation}% [1] talk title
  {Mou Cai}% [2] speaker name
  {Graduate School of Engineering, The University of Tokyo}% [3] affiliations
  {caimoumou@g.ecc.u-tokyo.ac.jp}% [4] email
  {Congpei An, Takashi Goda, Yoshihito Kazashi}% [5] coauthors
  {}% [6] special session. Leave this field empty for contributed talks. 
				% Insert the title of the special session if you were invited to give a talk in a special session.
Abstract

			
We propose a randomized lattice algorithm for approximating multivariate periodic functions over the $d$-dimensional unit cube from the weighted Korobov space with mixed smoothness $\alpha > 1/2$ and product weights $\gamma_1,\gamma_2,\ldots\in [0,1]$. This randomization involves drawing the number of points for function evaluations randomly, and selecting a good generating vector for rank-1 lattice points using the randomized component-by-component algorithm. We prove that our randomized algorithm achieves a worst-case root mean squared $L_2$-approximation error of order $M^{-\alpha/2 - 1/8 + \varepsilon}$ for an arbitrarily small $\varepsilon > 0$, where $M$ denotes the maximum number of function evaluations, and that the error bound is independent of the dimension $d$ if the weights satisfy $\sum_{j=1}^\infty \gamma_j^{1/\alpha} < \infty$. Our upper bound converges faster than a lower bound on the worst-case $L_2$-approximation error for deterministic rank-1 lattice-based approximation proved by Byrenheid, K\"{a}mmerer, Ullrich, and Volkmer (2017). We also show a lower error bound of order $M^{-\alpha/2-1/2}$ for our randomized algorithm. Finally, we present a generalization of hyperinterpolation over the unit cube named Quasi-Monte Carlo (QMC) hyperinterpolation. This new approximation scheme can be integrated with the Lasso technique to enhance sparsity and denoising capabilities.

\medskip


\end{talk}

\end{document}

