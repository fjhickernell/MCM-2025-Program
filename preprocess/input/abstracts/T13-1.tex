\documentclass[12pt,a4paper,figuresright]{book}

\usepackage{amsmath,amssymb}
\usepackage{tabularx,graphicx,url,xcolor,rotating,multicol,epsfig,colortbl}

\setlength{\textheight}{25.2cm}
\setlength{\textwidth}{16.5cm} %\setlength{\textwidth}{18.2cm}
\setlength{\voffset}{-1.6cm}
\setlength{\hoffset}{-0.3cm} %\setlength{\hoffset}{-1.2cm}
\setlength{\evensidemargin}{-0.3cm} 
\setlength{\oddsidemargin}{0.3cm}
\setlength{\parindent}{0cm} 
\setlength{\parskip}{0.3cm}

% -- adding a talk
\newenvironment{talk}[6]% [1] talk title
                         % [2] speaker name, [3] affiliations, [4] email,
                         % [5] coauthors, [6] special session
                         % [7] time slot
                         % [8] talk id, [9] session id or photo
 {%\needspace{6\baselineskip}%
  \vskip 0pt\nopagebreak%
%   \colorbox{gray!20!white}{\makebox[0.99\textwidth][r]{}}\nopagebreak%
%   \ifthenelse{\equal{#9}{photo}}{%
%                     \\\\\colorbox{gray!20!white}{\makebox{\includegraphics[width=3cm]{#8}}}\nopagebreak}{}%
 \vskip 0pt\nopagebreak%
%  \label{#8}%
  \textbf{#1}\vspace{3mm}\\\nopagebreak%
  \textit{#2}\\\nopagebreak%
  #3\\\nopagebreak%
  \url{#4}\vspace{3mm}\\\nopagebreak%
  \ifthenelse{\equal{#5}{}}{}{Coauthor(s): #5\vspace{3mm}\\\nopagebreak}%
  \ifthenelse{\equal{#6}{}}{}{Special session: #6\quad \vspace{3mm}\\\nopagebreak}%
 }
 {\vspace{1cm}\nopagebreak}%

\pagestyle{empty}

% ------------------------------------------------------------------------
% Document begins here
% ------------------------------------------------------------------------
\begin{document}
	
\begin{talk}
  {Learning cooling strategies in simulated annealing through binary interactions}% [1] talk title
  {Frédéric Blondeel}% [2] speaker name
  {KULeuven \& UniFe}% [3] affiliations
  {frederic.blondeel@kuleuven.be}% [4] email
  {Lorenzo Pareschi, Giovanni Samaey}% [5] coauthors
  {}% [6] special session. Leave this field empty for contributed talks. 
				% Insert the title of the special session if you were invited to give a talk in a special session.
			
Global optimization is amongst the hardest to solve problems. This is because finding the global minimum can usually only be guaranteed to be found in infinite time. Therefore one usually relies on meta-heuristic algorithms to guide the search and improve chances of successfully identifying the minimum. One particular family of algorithms is simulated annealing (SA). This family of algorithms is inspired by real-world metallurgy and is based on the Metropolis-Hastings algorithm. It works by randomly sampling $N$ particles on the search space, each with a given temperature $T$. The movement of these particles is analogous to a Brownian random walk with step size proportional to temperature. The temperature is gradually cooled down over the course of the simulation according to some predefined schedule. Due to the Metropolis-Hastings-like acceptance-rejection rule, these particles can jump out of local minima and are expected to move towards the lowest energy state, i.e., the global minimum. It is the design of these cooling schedules for SA that we wish to improve. This is because they directly impact the efficiency of the optimization tool. Typically cooling schedules are inverse logarithmic or geometric decays in time. Here, we consider a collective SA dynamic where particles \textit{interact} to learn the optimal temperature cooling strategy. This is inspired by the well-known particle-swapping technique known as parallel tempering (PT). To this aim we introduce a Boltzmann-type description where particles (partially) exchange their temperatures, therefore slowly cooling down the overall mean temperature. In order to simulate the dynamic we use a direct simulation Monte Carlo (DSMC) algorithm known as Nanbu-Babovsky. We show on various test functions (Ackley, Rastrigin, etc.) that this novel approach outperforms the standard SA with logarithmic and geometric annealing schedules.

\end{talk}

\end{document}

