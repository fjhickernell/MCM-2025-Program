\documentclass[12pt,a4paper,figuresright]{book}

\usepackage{amsmath,amssymb}
\usepackage{tabularx,graphicx,url,xcolor,rotating,multicol,epsfig,colortbl}

\setlength{\textheight}{25.2cm}
\setlength{\textwidth}{16.5cm} %\setlength{\textwidth}{18.2cm}
\setlength{\voffset}{-1.6cm}
\setlength{\hoffset}{-0.3cm} %\setlength{\hoffset}{-1.2cm}
\setlength{\evensidemargin}{-0.3cm} 
\setlength{\oddsidemargin}{0.3cm}
\setlength{\parindent}{0cm} 
\setlength{\parskip}{0.3cm}

% -- adding a talk
\newenvironment{talk}[6]% [1] talk title
                         % [2] speaker name, [3] affiliations, [4] email,
                         % [5] coauthors, [6] special session
                         % [7] time slot
                         % [8] talk id, [9] session id or photo
 {%\needspace{6\baselineskip}%
  \vskip 0pt\nopagebreak%
%   \colorbox{gray!20!white}{\makebox[0.99\textwidth][r]{}}\nopagebreak%
%   \ifthenelse{\equal{#9}{photo}}{%
%                     \\\\\colorbox{gray!20!white}{\makebox{\includegraphics[width=3cm]{#8}}}\nopagebreak}{}%
 \vskip 0pt\nopagebreak%
%  \label{#8}%
  \textbf{#1}\vspace{3mm}\\\nopagebreak%
  \textit{#2}\\\nopagebreak%
  #3\\\nopagebreak%
  \url{#4}\vspace{3mm}\\\nopagebreak%
  \ifthenelse{\equal{#5}{}}{}{Coauthor(s): #5\vspace{3mm}\\\nopagebreak}%
  \ifthenelse{\equal{#6}{}}{}{Special session: #6\quad \vspace{3mm}\\\nopagebreak}%
 }
 {\vspace{1cm}\nopagebreak}%

\pagestyle{empty}

% ------------------------------------------------------------------------
% Document begins here
% ------------------------------------------------------------------------
\begin{document}
	
\begin{talk}
  {Algorithmic Discrepancy Theory: An Overview}% [1] talk title
  {Haotian Jiang}% [2] speaker name
  {University of Chicago}% [3] affiliations
  {jhtdavid@uchicago.edu}% [4] email
  {}
  {Recent Progress on Algorithmic Discrepancy Theory and Applications}
  %{Names of coauthors go here, no affiliations of coauthors please, all affiliations will be included in an appendix of the program book}% [5] coauthors{}% [6] special session. Leave this field empty for contributed talks. 
				% Insert the title of the special session if you were invited to give a talk in a special session.
			
Combinatorial discrepancy theory studies the following question: given a universe of elements $U=\{1,\ldots, n\}$ and a collection $\mathcal{S} = \{S_1, \ldots, S_m\}$ of subsets of $U$, how well can we partition $U$ into two pieces, so that all sets in $\mathcal{S}$ are split as evenly as possible.
%Discrepancy theory studies the irregularities of distributions. Typical questions studied in discrepancy theory include: ``What is the most uniform way of distributing $n$ points in the unit square, and how big must the irregularity be?", ``What is the best way to divide a set of $n$ objects into two groups that are as 'similar' as possible?" 
The study of this question has found extensive applications to many areas of mathematics, computer science, statistics, finance, etc. 

The past decade has seen tremendous progress in designing efficient algorithms for this problem. These developments have led to many surprising applications in areas such as differential privacy, graph sparsification, approximation algorithms and rounding, kernel density estimation, randomized controlled trials, and quasi-Monte Carlo methods.

In this talk, I will briefly survey some recent algorithmic developments in this area. 

%\medskip

%If you would like to include references, please do so by creating a simple list numbered by [1], [2], [3], \ldots. See example below.
%Please do not use the \texttt{bibliography} environment or \texttt{bibtex} files.
% APA reference style is recommended.
% \begin{enumerate}
% 	\item[{[1]}] Niederreiter, Harald (1992). {\it Random number generation and quasi-Monte Carlo methods}. Society for Industrial and Applied Mathematics (SIAM).
% 	\item[{[2]}] L’Ecuyer, Pierre, \& Christiane Lemieux. (2002). Recent advances in randomized quasi-Monte Carlo methods. Modeling uncertainty: An examination of stochastic theory, methods, and applications, 419-474.
% \end{enumerate}

% Equations may be used if they are referenced. Please note that the equation numbers may be different (but will be cross-referenced correctly) in the final program book.
\end{talk}

\end{document}

