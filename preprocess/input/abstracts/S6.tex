\documentclass[12pt,a4paper,figuresright]{book}

\usepackage{amsmath,amssymb}
\usepackage{tabularx,multirow,graphicx,url,wrapfig,xcolor,rotating,multicol,epsfig,colortbl,verbatim}

\setlength{\textheight}{25.2cm}
\setlength{\textwidth}{16.5cm} %\setlength{\textwidth}{18.2cm}
\setlength{\voffset}{-1.6cm}
\setlength{\hoffset}{-0.3cm} %\setlength{\hoffset}{-1.2cm}
\setlength{\evensidemargin}{-0.3cm}
\setlength{\oddsidemargin}{0.3cm}
\setlength{\parindent}{0cm}
\setlength{\parskip}{0.3cm}

\renewcommand{\topfraction}{1}
\renewcommand{\textfraction}{0}
\setlength{\floatsep}{12pt plus 2pt minus 2pt}

\newcommand{\organizer}[3]{%
	{\textit{#1}}\\\nopagebreak%
	#2\\\nopagebreak%
	\url{#3}\vspace{3mm}\\\nopagebreak%
	}

\newenvironment{session}[5] % [1] session title
							% [2] number of organizers
                            % [3] organizer 1 info
                            % [4] organizer 2 info
                            % [5] organizer 3 info
                            % [6] session id for later
 {%\needspace{6\baselineskip}
  \vskip 0pt\nopagebreak%
  %\label{#5}%
  \textbf{#1}\vspace{3mm}\\\nopagebreak%
  \ifthenelse{\equal{#2}{1}}{Organizer:}{Organizers:}%
  \vspace{2mm}\\\nopagebreak%
  #3
  \ifthenelse{\equal{#2}{2}}{#4}{}%
  \ifthenelse{\equal{#2}{3}}{#4#5}{}%
  \quad\\\nopagebreak%
  %Session Description:\vspace{3mm}\\\nopagebreak%
 }
 {\nopagebreak}%


\pagestyle{empty}

% ------------------------------------------------------------------------
% Document begins here
% ------------------------------------------------------------------------
\begin{document}

%Input the relevant information below
\begin{session}
  {Recent advances in optimization under uncertainty}% [1] session title
  {3} %[2]  number of organizers
  {\organizer{Philipp A. Guth}% organizer one name
    {RICAM, Austrian Academy of Sciences}% orgnizer one affiliations
    {philipp.guth@ricam.oeaw.ac.at}}% organizer one email
  {\organizer{Vesa Kaarnioja}}% organizer two name, if needed
	{Free University of Berlin}% orgnizer two affiliations, if needed
	{vesa.kaarnioja@fu-berlin.de}}% organizer two email
  {}% organizer three name
	{}% orgnizer three affiliations
	{}% organizer three email

The quantification of uncertainties associated with large-scale optimization problems based on partial differential equation models typically involves a number of uncertainties. Some uncertain parameters may include, e.g., the material parameters, domain shape or sensor locations used to collect measurement data. The associated challenging high-dimensional integration problems can be solved efficiently using, e.g., multilevel Monte Carlo or quasi-Monte Carlo methods. This session aims to cover some recent developments in the computational and theoretical treatment of this actively developing field of research.



\end{session}

\end{document}

