\documentclass[12pt,a4paper,figuresright]{book}

\usepackage{amsmath,amssymb}
\usepackage{tabularx,graphicx,url,xcolor,rotating,multicol,epsfig,colortbl}

\setlength{\textheight}{25.2cm}
\setlength{\textwidth}{16.5cm} %\setlength{\textwidth}{18.2cm}
\setlength{\voffset}{-1.6cm}
\setlength{\hoffset}{-0.3cm} %\setlength{\hoffset}{-1.2cm}
\setlength{\evensidemargin}{-0.3cm} 
\setlength{\oddsidemargin}{0.3cm}
\setlength{\parindent}{0cm} 
\setlength{\parskip}{0.3cm}

% -- adding a talk
\newenvironment{talk}[6]% [1] talk title
                         % [2] speaker name, [3] affiliations, [4] email,
                         % [5] coauthors, [6] special session
                         % [7] time slot
                         % [8] talk id, [9] session id or photo
 {%\needspace{6\baselineskip}%
  \vskip 0pt\nopagebreak%
%   \colorbox{gray!20!white}{\makebox[0.99\textwidth][r]{}}\nopagebreak%
%   \ifthenelse{\equal{#9}{photo}}{%
%                     \\\\\colorbox{gray!20!white}{\makebox{\includegraphics[width=3cm]{#8}}}\nopagebreak}{}%
 \vskip 0pt\nopagebreak%
%  \label{#8}%
  \textbf{#1}\vspace{3mm}\\\nopagebreak%
  \textit{#2}\\\nopagebreak%
  #3\\\nopagebreak%
  \url{#4}\vspace{3mm}\\\nopagebreak%
  \ifthenelse{\equal{#5}{}}{}{Coauthor(s): #5\vspace{3mm}\\\nopagebreak}%
  \ifthenelse{\equal{#6}{}}{}{Special session: #6\quad \vspace{3mm}\\\nopagebreak}%
 }
 {\vspace{1cm}\nopagebreak}%

\pagestyle{empty}

% ------------------------------------------------------------------------
% Document begins here
% ------------------------------------------------------------------------
\begin{document}
	
\begin{talk}
  {Quasi-Monte Carlo Generators, Randomization Routines, and Fast Kernel Methods}% [1] talk title
  {Aleksei G Sorokin}% [2] speaker name
  {Illinois Institute of Technology, Department of Applied Mathematics. \\ Sandia National Laboratories.}% [3] affiliations
  {asorokin@hawk.iit.edu}% [4] email
  {}%Fred J Hickernell, Sou-Cheng T Choi, Aadit Jain, Pieterjan Robbe}% [5] coauthors
  {}% [6] special session. Leave this field empty for contributed talks. 
				% Insert the title of the special session if you were invited to give a talk in a special session.
			
        This talk will highlight recent improvements to the QMCPy Python package: a unified library for Quasi-Monte Carlo methods and computations related to low discrepancy sequences. We will describe routines for low discrepancy point set generation, randomization, and application to fast kernel methods. Specifically, we will discuss generators for lattices, digital nets, and Halton point sets with randomizations including random permutations / shifts, linear matrix scrambling, and nested uniform scrambling. Routines for working with higher-order digital nets and scramblings will also be detailed. For kernel methods, we provide implementations of special shift-invariant and digitally-shift invariant kernels along with fast Gram matrix operations facilitated by the bit-reversed Fast Fourier Transform (FFT), the bit-reversed inverse FFT (IFFT), and the Fast Walsh Hadamard Transform (FWHT). We will also describe methods to quickly update the matrix-vector product or linear system solution after doubling the number of points in a lattice or digital net in natural order. Generalizations to fast Gaussian process regression with derivative information will be discussed if time permits. 
\medskip

\begin{enumerate}
	\item[{[1]}] Sorokin, A. (2025). A Unified Implementation of Quasi-Monte Carlo Generators, Randomization Routines, and Fast Kernel Methods. arXiv preprint arXiv:2502.14256.
\end{enumerate}

\end{talk}

\end{document}