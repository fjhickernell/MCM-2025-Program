\documentclass[12pt,a4paper,figuresright]{book}

\usepackage{amsmath,amssymb}
\usepackage{tabularx,graphicx,url,xcolor,rotating,multicol,epsfig,colortbl}

\setlength{\textheight}{25.2cm}
\setlength{\textwidth}{16.5cm} %\setlength{\textwidth}{18.2cm}
\setlength{\voffset}{-1.6cm}
\setlength{\hoffset}{-0.3cm} %\setlength{\hoffset}{-1.2cm}
\setlength{\evensidemargin}{-0.3cm} 
\setlength{\oddsidemargin}{0.3cm}
\setlength{\parindent}{0cm} 
\setlength{\parskip}{0.3cm}

% -- adding a talk
\newenvironment{talk}[6]% [1] talk title
                         % [2] speaker name, [3] affiliations, [4] email,
                         % [5] coauthors, [6] special session
                         % [7] time slot
                         % [8] talk id, [9] session id or photo
 {%\needspace{6\baselineskip}%
  \vskip 0pt\nopagebreak%
%   \colorbox{gray!20!white}{\makebox[0.99\textwidth][r]{}}\nopagebreak%
%   \ifthenelse{\equal{#9}{photo}}{%
%                     \\\\\colorbox{gray!20!white}{\makebox{\includegraphics[width=3cm]{#8}}}\nopagebreak}{}%
 \vskip 0pt\nopagebreak%
%  \label{#8}%
  \textbf{#1}\vspace{3mm}\\\nopagebreak%
  \textit{#2}\\\nopagebreak%
  #3\\\nopagebreak%
  \url{#4}\vspace{3mm}\\\nopagebreak%
  \ifthenelse{\equal{#5}{}}{}{Coauthor(s): #5\vspace{3mm}\\\nopagebreak}%
  \ifthenelse{\equal{#6}{}}{}{Special session: #6\quad \vspace{3mm}\\\nopagebreak}%
 }
 {\vspace{1cm}\nopagebreak}%

\pagestyle{empty}

% ------------------------------------------------------------------------
% Document begins here
% ------------------------------------------------------------------------
\begin{document}
	
\begin{talk}
  {ARCANE Reweighting: A technique to tackle the sign problem in the simulation of collider events in high energy physics}% [1] talk title
  {Prasanth Shyamsundar}% [2] speaker name
  {Fermi National Accelerator Laboratory}% [3] affiliations
  {prasanth@fnal.gov}% [4] email
  % {Names of coauthors go here, no affiliations of coauthors please, all affiliations will be included in an appendix of 
  % the program book}% [5] coauthors
  {}%[5] coauthors
  {}% [6] special session. Leave this field empty for contributed talks. 
				% Insert the title of the special session if you were invited to give a talk in a special session.
			
% Your abstract goes here. Please do not use your own commands or macros.

Negatively weighted events, which appear in the Monte Carlo (MC) simulation of particle collisions, significantly increase the computational resource requirements of current and future collider experiments in high energy physics. This work introduces an MC technique called ARCANE reweighting for reducing or eliminating negatively weighted events. The technique works by redistributing (via an additive reweighting) the contributions of different pathways within the simulator that lead to the same final event. The technique is exact and does not introduce any biases in the distributions of physical observables. ARCANE reweighting can be thought of as a variant of the parametrized control variates technique, with the added nuance that redistribution is performed using a deferred additive reweighting. The technique is demonstrated for the simulation of a specific collision process, namely $e^+ e^- \longrightarrow q \bar{q} + 1\,jet$. The technique can be extended to several other collision processes of interest as well. This talk is based on the Refs~[1] and [2].

\medskip

% If you would like to include references, please do so by creating a simple list numbered by [1], [2], [3], \ldots. See example below.
% Please do not use the \texttt{bibliography} environment or \texttt{bibtex} files.
% APA reference style is recommended.
\begin{enumerate}
	% \item[{[1]}] Niederreiter, Harald (1992). {\it Random number generation and quasi-Monte Carlo methods}. Society for Industrial and Applied Mathematics (SIAM).
	% \item[{[2]}] L’Ecuyer, Pierre, \& Christiane Lemieux. (2002). Recent advances in randomized quasi-Monte Carlo methods. Modeling uncertainty: An examination of stochastic theory, methods, and applications, 419-474.
    \item[{[1]}] Shyamsundar, Prasanth (2025). {\it ARCANE Reweighting: A Monte Carlo Technique to Tackle the Negative Weights Problem in Collider Event Generation}. arXiv:2502.08052 [hep-ph].
    \item[{[2]}] Shyamsundar, Prasanth (2025). {\it A Demonstration of ARCANE Reweighting: Reducing the Sign Problem in the MC@NLO Generation of $e^+ e^- \longrightarrow q\bar{q} + 1\,jet$ Events}. arXiv:2502.08052 [hep-ph].
\end{enumerate}

% Equations may be used if they are referenced. Please note that the equation numbers may be different (but will be cross-referenced correctly) in the final program book.
\end{talk}

\end{document}

