\documentclass[12pt,a4paper,figuresright]{book}

\usepackage{amsmath,amssymb}
\usepackage{tabularx,graphicx,url,xcolor,rotating,multicol,epsfig,colortbl}

\setlength{\textheight}{25.2cm}
\setlength{\textwidth}{16.5cm} %\setlength{\textwidth}{18.2cm}
\setlength{\voffset}{-1.6cm}
\setlength{\hoffset}{-0.3cm} %\setlength{\hoffset}{-1.2cm}
\setlength{\evensidemargin}{-0.3cm} 
\setlength{\oddsidemargin}{0.3cm}
\setlength{\parindent}{0cm} 
\setlength{\parskip}{0.3cm}

% -- adding a talk
\newenvironment{talk}[6]% [1] talk title
                         % [2] speaker name, [3] affiliations, [4] email,
                         % [5] coauthors, [6] special session
                         % [7] time slot
                         % [8] talk id, [9] session id or photo
 {%\needspace{6\baselineskip}%
  \vskip 0pt\nopagebreak%
%   \colorbox{gray!20!white}{\makebox[0.99\textwidth][r]{}}\nopagebreak%
%   \ifthenelse{\equal{#9}{photo}}{%
%                     \\\\\colorbox{gray!20!white}{\makebox{\includegraphics[width=3cm]{#8}}}\nopagebreak}{}%
 \vskip 0pt\nopagebreak%
%  \label{#8}%
  \textbf{#1}\vspace{3mm}\\\nopagebreak%
  \textit{#2}\\\nopagebreak%
  #3\\\nopagebreak%
  \url{#4}\vspace{3mm}\\\nopagebreak%
  \ifthenelse{\equal{#5}{}}{}{Coauthor(s): #5\vspace{3mm}\\\nopagebreak}%
  \ifthenelse{\equal{#6}{}}{}{Special session: #6\quad \vspace{3mm}\\\nopagebreak}%
 }
 {\vspace{1cm}\nopagebreak}%

\pagestyle{empty}

% ------------------------------------------------------------------------
% Document begins here
% ------------------------------------------------------------------------
\begin{document}
	
\begin{talk}
  {Bayesian Analysis of Latent Underdispersion Using Discrete Order Statistics}% [1] talk title
  {Jimmy Lederman}% [2] speaker name
  {Department of Statistics, University of Chicago}% [3] affiliations
  {jlederman@uchicago.edu}% [4] email
  {Aaron Schein}% [5] coauthors
  {}% [6] special session. Leave this field empty for contributed talks. 
				% Insert the title of the special session if you were invited to give a talk in a special session.


Researchers routinely analyze count data using models based on a Poisson likelihood, for which there exist many analytically convenient and computationally efficient strategies for posterior inference. A limitation of such models however is the equidsipersion constraint of the Poisson distribution. This restriction prevents the model's likelihood, and by extension its posterior predictive distribution, from concentrating around its mode. As a result, these models are parametrically bound to produce probabilistic predictions with high uncertainty, even in cases where low uncertainty is supported by the data. While count data often exhibits overdispersion \textit{marginally}, such data may nevertheless be consistent with a likelihood that is underdispersed \textit{conditionally}, given parameters and latent variables. Detecting conditional underdispersion, however, requires one to fit the ``right" model and thus the ability to build, fit, and critique a variety of different models with underdispersed likelihoods. Towards this end, we introduce a novel family of models for conditionally underdispersed count data whose likelihoods are based on order statistics of Poisson random variables. More specifically, we assume that each observed count coincides with the $j^{\textrm{th}}$ order-statistic of $D$ latent i.i.d.~Poisson random variables, where $j$ and $D$ are user-defined hyperparameters. To perform efficient MCMC-based posterior inference in this family of models, we derive a data-augmentation strategy which samples the other $D{-}1$ latent variables from their exact conditional, given the observed $(j,D)$-order statistic. By relying on the explicit construction of a Poisson order statistic, this data augmentation strategy can be modularly combined with the many existing inference strategies for Poisson-based models. We generalize this approach beyond the Poisson to any non-negative discrete parent distribution and, in particular, show that models based on negative binomial order statistics can flexibly capture both conditional under and overdispersion.  To illustrate our approach empirically, we build and fit models to three real count data sets of flight times, COVID-19 cases counts, and RNA-sequence data, and we demonstrate how models with underdispersed likelihoods can leverage latent structure to make more precise probabilistic predictions. Although the possibility of conditional underdispersion is often overlooked in practice, we argue that this is at least in part due to the lack of tools for modeling underdispersion in settings where complex latent structure is present.

	

% \medskip

% If you would like to include references, please do so by creating a simple list numbered by [1], [2], [3], \ldots. See example below.
% Please do not use the \texttt{bibliography} environment or \texttt{bibtex} files.
% APA reference style is recommended.
% \begin{enumerate}
% 	\item[{[1]}] Niederreiter, Harald (1992). {\it Random number generation and quasi-Monte Carlo methods}. Society for Industrial and Applied Mathematics (SIAM).
% 	\item[{[2]}] L’Ecuyer, Pierre, \& Christiane Lemieux. (2002). Recent advances in randomized quasi-Monte Carlo methods. Modeling uncertainty: An examination of stochastic theory, methods, and applications, 419-474.
% \end{enumerate}

% Equations may be used if they are referenced. Please note that the equation numbers may be different (but will be cross-referenced correctly) in the final program book.
\end{talk}

\end{document}

