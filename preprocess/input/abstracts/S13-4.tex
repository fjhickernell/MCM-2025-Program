\documentclass[12pt,a4paper,figuresright]{book}

\usepackage{amsmath,amssymb}
\usepackage{tabularx,graphicx,url,xcolor,rotating,multicol,epsfig,colortbl}

\setlength{\textheight}{25.2cm}
\setlength{\textwidth}{16.5cm} %\setlength{\textwidth}{18.2cm}
\setlength{\voffset}{-1.6cm}
\setlength{\hoffset}{-0.3cm} %\setlength{\hoffset}{-1.2cm}
\setlength{\evensidemargin}{-0.3cm} 
\setlength{\oddsidemargin}{0.3cm}
\setlength{\parindent}{0cm} 
\setlength{\parskip}{0.3cm}

% -- adding a talk
\newenvironment{talk}[6]% [1] talk title
                         % [2] speaker name, [3] affiliations, [4] email,
                         % [5] coauthors, [6] special session
                         % [7] time slot
                         % [8] talk id, [9] session id or photo
 {%\needspace{6\baselineskip}%
  \vskip 0pt\nopagebreak%
%   \colorbox{gray!20!white}{\makebox[0.99\textwidth][r]{}}\nopagebreak%
%   \ifthenelse{\equal{#9}{photo}}{%
%                     \\\\\colorbox{gray!20!white}{\makebox{\includegraphics[width=3cm]{#8}}}\nopagebreak}{}%
 \vskip 0pt\nopagebreak%
%  \label{#8}%
  \textbf{#1}\vspace{3mm}\\\nopagebreak%
  \textit{#2}\\\nopagebreak%
  #3\\\nopagebreak%
  \url{#4}\vspace{3mm}\\\nopagebreak%
  \ifthenelse{\equal{#5}{}}{}{Coauthor(s): #5\vspace{3mm}\\\nopagebreak}%
  \ifthenelse{\equal{#6}{}}{}{Special session: #6\quad \vspace{3mm}\\\nopagebreak}%
 }
 {\vspace{1cm}\nopagebreak}%

\pagestyle{empty}

% ------------------------------------------------------------------------
% Document begins here
% ------------------------------------------------------------------------
\begin{document}
	
\begin{talk}
  {Optimal Pilot Sampling for Multi-fidelity Monte Carlo Methods}% [1] talk title
  {Xun Huan}% [2] speaker name
  {University of Michigan, Department of Mechanical Engineering}% [3] affiliations
  {xhuan@umich.edu}% [4] email
  {Thomas Coons, Aniket Jivani}% [5] coauthors
  {}% [6] special session. Leave this field empty for contributed talks. 
				% Insert the title of the special session if you were invited to give a talk in a special session.

Bayesian optimal experimental design (OED) aims to maximize an expected utility, often chosen to be the expected information gain (EIG), over a given design space. Estimating EIG typically relies on Monte Carlo methods, which requires repeated evaluations of a computational model simulating the experimental process. 
However, when the model is expensive to evaluate, standard Monte Carlo becomes impractical.
%When performing forward uncertainty quantification (UQ) of a computationally intensive mathematical model, traditional sampling methods such as Monte Carlo can be prohibitively expensive. 

Multi-fidelity variants of Monte Carlo, such as Approximate Control Variate (ACV) estimators, can significantly expedite such estimations by leveraging an ensemble of low-fidelity models that approximate the high-fidelity model with varying degrees of accuracy and cost. 
To apply these techniques in an error-optimal manner, the covariance matrix across model outputs must be estimated from independent pilot model evaluations. This step incurs a significant but often overlooked computational cost. 
Furthermore, the optimal allocation of computational resources between 
%the model evaluations needed for 
covariance estimation and 
%the model evaluations needed for 
ACV estimation remains an open problem.  Existing approaches fail to accommodate optimal estimators and may not be accurate with small pilot sample sizes.

In this work, we introduce a novel framework for dynamically allocating resources between these two tasks.
%prescribing the budget allocation between covariance estimation and ACV evaluation that 
Our method employs Bayesian inference
to quantify uncertainty in the covariance matrix and derives an adaptive expected loss metric
%that is adaptively estimated and minimized as pilot samples are drawn, informing the user when 
to determine when to terminate pilot sampling. 
We demonstrate and analyze our framework through a benchmark nonlinear OED problem. 

\medskip

%If you would like to include references, please do so by creating a simple list numbered by [1], [2], [3], \ldots. See example below.
%Please do not use the \texttt{bibliography} environment or \texttt{bibtex} files. APA reference style is recommended.
%\begin{enumerate}
%	\item[{[1]}] Niederreiter, Harald (1992). {\it Random number generation and quasi-Monte Carlo methods}. Society for Industrial and Applied Mathematics (SIAM).
%	\item[{[2]}] L’Ecuyer, Pierre, \& Christiane Lemieux. (2002). Recent advances in randomized quasi-Monte Carlo methods. Modeling uncertainty: An examination of stochastic theory, methods, and applications, 419-474.
%\end{enumerate}

\end{talk}

\end{document}

