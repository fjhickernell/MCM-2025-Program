\documentclass[12pt,a4paper,figuresright]{book}

\usepackage{amsmath,amssymb}
\usepackage{tabularx,graphicx,url,xcolor,rotating,multicol,epsfig,colortbl}

\setlength{\textheight}{25.2cm}
\setlength{\textwidth}{16.5cm} %\setlength{\textwidth}{18.2cm}
\setlength{\voffset}{-1.6cm}
\setlength{\hoffset}{-0.3cm} %\setlength{\hoffset}{-1.2cm}
\setlength{\evensidemargin}{-0.3cm} 
\setlength{\oddsidemargin}{0.3cm}
\setlength{\parindent}{0cm} 
\setlength{\parskip}{0.3cm}

% -- adding a talk
\newenvironment{talk}[6]% [1] talk title
                         % [2] speaker name, [3] affiliations, [4] email,
                         % [5] coauthors, [6] special session
                         % [7] time slot
                         % [8] talk id, [9] session id or photo
 {%\needspace{6\baselineskip}%
  \vskip 0pt\nopagebreak%
%   \colorbox{gray!20!white}{\makebox[0.99\textwidth][r]{}}\nopagebreak%
%   \ifthenelse{\equal{#9}{photo}}{%
%                     \\\\\colorbox{gray!20!white}{\makebox{\includegraphics[width=3cm]{#8}}}\nopagebreak}{}%
 \vskip 0pt\nopagebreak%
%  \label{#8}%
  \textbf{#1}\vspace{3mm}\\\nopagebreak%
  \textit{#2}\\\nopagebreak%
  #3\\\nopagebreak%
  \url{#4}\vspace{3mm}\\\nopagebreak%
  \ifthenelse{\equal{#5}{}}{}{Coauthor(s): #5\vspace{3mm}\\\nopagebreak}%
  \ifthenelse{\equal{#6}{}}{}{Special session: #6\quad \vspace{3mm}\\\nopagebreak}%
 }
 {\vspace{1cm}\nopagebreak}%

\pagestyle{empty}

% ------------------------------------------------------------------------
% Document begins here
% ------------------------------------------------------------------------
\begin{document}
	
\begin{talk}
  {Golden ratio nets and sequences}% [1] talk title
  {Christiane Lemieux}% [2] speaker name
  {University of Waterloo}% [3] affiliations
  {clemieux@uwaterloo.ca}% [4] email
  {Nathan Kirk and Jaspar Wiart}% [5] coauthors
  {}% [6] special session. Leave this field empty for contributed talks. 
				% Insert the title of the special session if you were invited to give a talk in a special session.
			

In this talk, we discuss nets and sequences constructed in an irrational base, focusing on the case of a base given by the golden ratio $\varphi$. We provide a complete framework to study equidistribution properties of nets in base $\varphi$, which among other things requires the introduction of a new concept of prime elementary intervals which differ from the standard definition used for integer bases. We define the one-dimensional van der Corput sequence in base $\varphi$ and two-dimensional Hammersley point sets in base $\varphi$ and we prove some properties for $(0,1)-$sequences and $(0,m,2)-$nets in base $\varphi$, respectively. This part of the talk is based on [1].


Building on this new framework, we propose 
an {\em interlaced Halton sequence} that makes use of integer \textit{and} irrational-based van der Corput sequences and show empirically improved performance compared to the traditional Halton sequence [2]. In addition, we propose a scrambling algorithm for irrational-based digital sequences, which leverages dependence properties of scrambled digital nets [3].

\medskip

%If you would like to include references, please do so by creating a simple list numbered by [1], [2], [3], \ldots. See example below.
%Please do not use the \texttt{bibliography} environment or \texttt{bibtex} files.
%APA reference style is recommended.
\begin{enumerate}
	\item[{[1]}] N. Kirk, C. Lemieux and J. Wiart. Golden ratio nets and sequences. To appear in {\em Functiones and Approximatio}, 2025.
	\item[{[2]}] N. Kirk, C. Lemieux. An improved Halton sequence for implementation in quasi-Monte Carlo methods. {\em Proceedings of the 2024 Winter Simulation Conference}, 431--442, IEEE Press, Piscataway, NJ, 2024. 
    \item[{[3]}] C. Lemieux and J. Wiart. On the distribution of scrambled $(0, m, s)$-nets over unanchored
boxes. In: {\em Monte Carlo and Quasi-Monte Carlo Methods 2020}, A. Keller (ed), 
Springer, 187-230, 2022.
\end{enumerate}

%Equations may be used if they are referenced. Please note that the equation numbers may be different (but will be cross-referenced correctly) in the final program book.
\end{talk}

\end{document}
