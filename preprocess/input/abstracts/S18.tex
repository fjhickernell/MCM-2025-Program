\documentclass[12pt,a4paper,figuresright]{book}

\usepackage{amsmath,amssymb}
\usepackage{tabularx,multirow,graphicx,url,wrapfig,xcolor,rotating,multicol,epsfig,colortbl}

\setlength{\textheight}{25.2cm}
\setlength{\textwidth}{16.5cm} %\setlength{\textwidth}{18.2cm}
\setlength{\voffset}{-1.6cm}
\setlength{\hoffset}{-0.3cm} %\setlength{\hoffset}{-1.2cm}
\setlength{\evensidemargin}{-0.3cm}
\setlength{\oddsidemargin}{0.3cm}
\setlength{\parindent}{0cm}
\setlength{\parskip}{0.3cm}

\renewcommand{\topfraction}{1}
\renewcommand{\textfraction}{0}
\setlength{\floatsep}{12pt plus 2pt minus 2pt}

\newcommand{\organizer}[3]{%
	{\textit{#1}}\\\nopagebreak%
	#2\\\nopagebreak%
	\url{#3}\vspace{3mm}\\\nopagebreak%
	}

\newenvironment{session}[5] % [1] session title 
							% [2] number of organizers 
                            % [3] organizer 1 info 
                            % [4] organizer 2 info 
                            % [5] organizer 3 info 
                            % [6] session id for later 
 {%\needspace{6\baselineskip}
  \vskip 0pt\nopagebreak%
  %\label{#5}%
  \textbf{#1}\vspace{3mm}\\\nopagebreak%
  \ifthenelse{\equal{#2}{1}}{Organizer:}{Organizers:}%
  \vspace{2mm}\\\nopagebreak%
  #3 
  \ifthenelse{\equal{#2}{2}}{#4}{}%
  \ifthenelse{\equal{#2}{3}}{#4#5}{}%
  \quad\\\nopagebreak%
  %Session Description:\vspace{3mm}\\\nopagebreak%
 }
 {\nopagebreak}%


\pagestyle{empty}

% ------------------------------------------------------------------------
% Document begins here 
% ------------------------------------------------------------------------
\begin{document}

%Input the relevant information below
\begin{session}
  {Advances in Adaptive Hamiltonian Monte Carlo}% [1] session title
  {1} %[2]  number of organizers
  {\organizer{Art B. Owen}% organizer one name
    {Stanford University}% organizer one affiliations
    {owen@stanford.edu}}% organizer one email
  {\organizer{}% organizer two name, if needed
	{}% organizer two affiliations, if needed
	{}}% organizer two email
  {\organizer{}% organizer three name
	{}% organizer three affiliations
	{}}% organizer three email
	
Hamiltonian Monte Carlo (HMC) is one of the most effective tools for high-dimensional Markov chain Monte Carlo. It is the default algorithm used in probabilistic programming languages for Bayesian computation, including \texttt{Stan} [2], \texttt{PyMC} and \texttt{NumPyro} (Python), and \texttt{Turing.jl} (Julia).  While HMC handles some of the most difficult MCMC problems, it does so through the use of several tuning parameters, which can be challenging to set. 

Significant progress came from the no-U-turn sampler (NUTS) [3] and apogee-to-apogee path sampler [7], both of which dynamically adapt path lengths. More recent progress includes delayed rejection HMC [5], which locally adapts step sizes, and 
Gibbs self tuning (GIST) [1], which treats tuning parameters as random variables to maintain detailed balance.  Chirag Modi's ATLAS [5] leverages local Hessians, delayed rejection, and GIST.  AutoStep MCMC [4] adapts step sizes locally to match the variable geometry of target distributions.

The speakers, in tentative order, will be: \\[-5ex]
\begin{itemize}
    \item Bob Carpenter, Center for Computational Mathematics, Flatiron Institute. \\[-4ex]
    \item Nawaf Bou-Rabee, Department of Mathematical Sciences, Rutgers University. \\[-4ex]
    \item Chirag Modi, Center for Cosmology and Particle Physics, Department of Physics, New York University. \\[-4ex]
    \item Trevor Campbell, University of British Columbia.\\[-4ex]
\end{itemize}

Bob Carpenter will provide an introduction to HMC, NUTS, and GIST for non-specialists. Nawaf Bou-Rabee will elaborate on GIST. Chirag Modi will discuss delayed rejection and ATLAS. Trevor Campbell will present AutoStep and related methods.


\medskip
{\bf\large References}\\[-5ex]
\begin{description}
\item{[1]} Bou-Rabee, B. Carpenter, and M. Marsden. GIST: Gibbs self-tuning for locally adaptive HMC. arXiv:2404.15253, 2024.

\item{[2]} Carpenter, B. et al. 2017. Stan: A probabilistic programming language. \textit{J. Stat. Soft.},~76

\item{[3]} Hoffman, M.~D.\ and Gelman, A. 2014. The no-U-turn sampler: Adaptively setting path lengths in HMC. \textit{J. Mach. Learn. Res.}, 15(1).
  
\item{[4]} Liu, T., Campbell, T., et al. 2024. AutoStep: Locally adaptive involutive MCMC. arXiv:2410.18929, 2024.

\item{[5]} Modi, C. 2024. ATLAS: Adapting trajectory lengths and step-size for HMC. arXiv:2410.21587

\item{[6]} Modi, C., Barnett, A. and Carpenter, B. 2024. Delayed rejection Hamiltonian Monte Carlo for sampling multiscale distributions. \textit{Bayesian Analysis}, 19(3), 2024.

\item{[7]} Sherlock, C., Urbas, S. and Ludkin, M. 2023. The apogee to apogee path sampler. \textit{JCGS}, 32(4).
\end{description}

\end{document}
