\documentclass[12pt,a4paper,figuresright]{book}

\usepackage{amsmath,amssymb}
\usepackage{tabularx,graphicx,url,xcolor,rotating,multicol,epsfig,colortbl}

\setlength{\textheight}{25.2cm}
\setlength{\textwidth}{16.5cm} %\setlength{\textwidth}{18.2cm}
\setlength{\voffset}{-1.6cm}
\setlength{\hoffset}{-0.3cm} %\setlength{\hoffset}{-1.2cm}
\setlength{\evensidemargin}{-0.3cm} 
\setlength{\oddsidemargin}{0.3cm}
\setlength{\parindent}{0cm} 
\setlength{\parskip}{0.3cm}

% -- adding a talk
\newenvironment{talk}[6]% [1] talk title
                         % [2] speaker name, [3] affiliations, [4] email,
                         % [5] coauthors, [6] special session
                         % [7] time slot
                         % [8] talk id, [9] session id or photo
 {%\needspace{6\baselineskip}%
  \vskip 0pt\nopagebreak%
%   \colorbox{gray!20!white}{\makebox[0.99\textwidth][r]{}}\nopagebreak%
%   \ifthenelse{\equal{#9}{photo}}{%
%                     \\\\\colorbox{gray!20!white}{\makebox{\includegraphics[width=3cm]{#8}}}\nopagebreak}{}%
 \vskip 0pt\nopagebreak%
%  \label{#8}%
  \textbf{#1}\vspace{3mm}\\\nopagebreak%
  \textit{#2}\\\nopagebreak%
  #3\\\nopagebreak%
  \url{#4}\vspace{3mm}\\\nopagebreak%
  \ifthenelse{\equal{#5}{}}{}{Coauthor(s): #5\vspace{3mm}\\\nopagebreak}%
  \ifthenelse{\equal{#6}{}}{}{Special session: #6\quad \vspace{3mm}\\\nopagebreak}%
 }
 {\vspace{1cm}\nopagebreak}%

\pagestyle{empty}

% ------------------------------------------------------------------------
% Document begins here
% ------------------------------------------------------------------------
\begin{document}
	
\begin{talk}
  {Searching Permutations for Constructing Low-Discrepancy Point Sets and Inverstigating the Kritzinger Sequence}% [1] talk title
  {Fran\c{c}ois Cl\'{e}ment}% [2] speaker name
  {Department of Mathematics, University of Washington}% [3] affiliations
  {fclement@uw.edu}% [4] email
  {Carola Doerr, Kathrin Klamroth, Luís Paquete}% [5] coauthors
  {}% [6] special session. Leave this field empty for contributed talks. 
				% Insert the title of the special session if you were invited to give a talk in a special session.
			
This talk focuses on two different approaches for the construction of low-discrepancy sets that are quite different from traditional approaches, yet yield excellent empirical results in two dimensions. The first of these, which will be the main focus of my talk, is based on selecting the relative position of the different points we wish to place, before using non-linear programming methods to obtain a point set with extremely low star discrepancy. In [1], we showed that this method consistently outperformed all other existing techniques. It is however subject to computational limits: finding good permutation choices, with or without optimization, is the next key step in improving our understanding of low-discrepancy structures.

In the second part of the talk, I will quickly highlight some extended numerical experiments on the sequence introduced by Kritzinger in [2], showing that despite the lack of theoretical results proving that it is a low-discrepancy sequence, it performs at least as well as known sequences in one dimension, despite being constructed greedily. 

\medskip

\begin{enumerate}
	\item[{[1]}] F. Clément, C. Doerr, K. Klamroth, L.Paquete (2024). {\it Transforming the Challenge of Constructing Low-Discrepancy Point Sets into a Permutation Selection Problem}., to appear, arxiv: https://arxiv.org/abs/2407.11533.
	\item[{[2]}] R. Kritzinger (2022).  {\it Uniformly Distributed Sequence generated by a greedy minimization of the $L_2$ discrepancy}., Moscow Journal of Combinatorics and Number Theory, \textbf{11}(2), 215--236.
\end{enumerate}

\end{talk}

\end{document}

