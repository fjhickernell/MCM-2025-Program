\documentclass[12pt,a4paper,figuresright]{book}

\usepackage{amsmath,amssymb}
\usepackage{tabularx,graphicx,url,xcolor,rotating,multicol,epsfig,colortbl}

\setlength{\textheight}{25.2cm}
\setlength{\textwidth}{16.5cm} %\setlength{\textwidth}{18.2cm}
\setlength{\voffset}{-1.6cm}
\setlength{\hoffset}{-0.3cm} %\setlength{\hoffset}{-1.2cm}
\setlength{\evensidemargin}{-0.3cm} 
\setlength{\oddsidemargin}{0.3cm}
\setlength{\parindent}{0cm} 
\setlength{\parskip}{0.3cm}

% -- adding a talk
\newenvironment{talk}[6]% [1] talk title
                         % [2] speaker name, [3] affiliations, [4] email,
                         % [5] coauthors, [6] special session
                         % [7] time slot
                         % [8] talk id, [9] session id or photo
 {%\needspace{6\baselineskip}%
  \vskip 0pt\nopagebreak%
%   \colorbox{gray!20!white}{\makebox[0.99\textwidth][r]{}}\nopagebreak%
%   \ifthenelse{\equal{#9}{photo}}{%
%                     \\\\\colorbox{gray!20!white}{\makebox{\includegraphics[width=3cm]{#8}}}\nopagebreak}{}%
 \vskip 0pt\nopagebreak%
%  \label{#8}%
  \textbf{#1}\vspace{3mm}\\\nopagebreak%
  \textit{#2}\\\nopagebreak%
  #3\\\nopagebreak%
  \url{#4}\vspace{3mm}\\\nopagebreak%
  \ifthenelse{\equal{#5}{}}{}{Coauthor(s): #5\vspace{3mm}\\\nopagebreak}%
  \ifthenelse{\equal{#6}{}}{}{Special session: #6\quad \vspace{3mm}\\\nopagebreak}%
 }
 {\vspace{1cm}\nopagebreak}%

\pagestyle{empty}

% ------------------------------------------------------------------------
% Document begins here
% ------------------------------------------------------------------------
\begin{document}
	
\begin{talk}
  {An Empirical Evaluation of Robust Estimators for RQMC}%
% {Cost-Free improvements to the median estimator for Random-parameter QMC}% [1] talk title
  {Gregory Seljak}% [2] speaker name
  {Universit\'e de Montr\'eal}% [3] affiliations
  {gregory.de.salaberry.seljak@umontreal.ca}% [4] email
  {Pierre L'Ecuyer, Christiane Lemieux}% [5] coauthors
  {Computational Methods for Low-discrepancy Sampling and Applications}% [6] special session. 

\medskip

Randomized quasi-Monte Carlo (RQMC) traditionally takes a low-discrepancy (QMC) point set,
makes $r$ independent randomizations of it to obtain $r$ replicates of an unbiased RQMC estimator, 
then computes the average and variance of these $r$ estimates to obtain a final estimate 
and perhaps a confidence interval [4]. 
Some methods construct the points by optimizing refined figures-of-merit adapted to the considered integrand
and apply minimal randomization such as a random (digital) shift.  Other methods randomize
the parameters of the QMC point sets more extensively (e.g., the generating vectors or matrices).
The second kind of method is easier to apply because it requires much less knowledge of the integrand,
but bad parameter values may be drawn once in a while, leading to (rare) RQMC replicates having 
a large conditional variance that produce bad outliers.
To reduce the impact of such outliers, one approach studied recently is to use the median 
of the $r$ replicates instead of the mean as a final estimator [2, 3, 5, 6].
Other types of robust estimators could also be used in place of the median [1].
In this talk, we report extensive experiments that compare the mean square errors 
and convergence of various estimators (the mean, the median, and other robust estimators) 
defined in terms of $r$ RQMC replicates.
We also discuss the computation of confidence intervals for the mean when using such estimators.

{\list{[\arabic{enumi}]}{\settowidth\labelwidth{[5]}\leftmargin\labelwidth
  \advance\leftmargin\labelsep\usecounter{enumi}}
%
\item
E.~Gobet, M.~Lerasle, and D.~M{\'e}tivier.
Accelerated convergence of error quantiles using robust randomized
  quasi {Monte Carlo} methods.
\url{https://hal.science/hal-03631879}, 2024.

\item
T.~Goda and P.~L'Ecuyer.
Construction-free median quasi-{Monte Carlo} rules for function
  spaces with unspecified smoothness and general weights.
{\em {SIAM} Journal on Scientific Computing}, 44(4):A2765--A2788, 2022.

\item
T.~Goda, K.~Suzuki, and M.~Matsumoto.
A universal median quasi-{Monte Carlo} integration.
{\em {SIAM} Journal on Numerical Analysis}, 62(1):533--566, 2024.

\item
P.~L'Ecuyer, M.~Nakayama, A.~B. Owen, and B.~Tuffin.
Confidence intervals for randomized quasi-{Monte Carlo} estimators.
In {\em Proceedings of the 2023 Winter Simulation Conference}, pages
  445--456. IEEE Press, 2023.

\item
Z.~Pan and A.~B. Owen.
Super-polynomial accuracy of one dimensional randomized nets using
  the median of means.
{\em Mathematics of Computation}, 92(340):805--837, 2023.

\item
Z.~Pan and A.~B. Owen.
Super-polynomial accuracy of multidimensional randomized nets using
  the median of means.
{\em Mathematics of Computation}, 93(349):2265--2289, 2024.

}
\end{talk}

\end{document}
