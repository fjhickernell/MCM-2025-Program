\documentclass[12pt,a4paper,figuresright]{book}

\usepackage{amsmath,amssymb}
\usepackage{tabularx,graphicx,url,xcolor,rotating,multicol,epsfig,colortbl}

\setlength{\textheight}{25.2cm}
\setlength{\textwidth}{16.5cm} %\setlength{\textwidth}{18.2cm}
\setlength{\voffset}{-1.6cm}
\setlength{\hoffset}{-0.3cm} %\setlength{\hoffset}{-1.2cm}
\setlength{\evensidemargin}{-0.3cm} 
\setlength{\oddsidemargin}{0.3cm}
\setlength{\parindent}{0cm} 
\setlength{\parskip}{0.3cm}

% -- adding a talk
\newenvironment{talk}[6]% [1] talk title
                         % [2] speaker name, [3] affiliations, [4] email,
                         % [5] coauthors, [6] special session
                         % [7] time slot
                         % [8] talk id, [9] session id or photo
 {%\needspace{6\baselineskip}%
  \vskip 0pt\nopagebreak%
%   \colorbox{gray!20!white}{\makebox[0.99\textwidth][r]{}}\nopagebreak%
%   \ifthenelse{\equal{#9}{photo}}{%
%                     \\\\\colorbox{gray!20!white}{\makebox{\includegraphics[width=3cm]{#8}}}\nopagebreak}{}%
 \vskip 0pt\nopagebreak%
%  \label{#8}%
  \textbf{#1}\vspace{3mm}\\\nopagebreak%
  \textit{#2}\\\nopagebreak%
  #3\\\nopagebreak%
  \url{#4}\vspace{3mm}\\\nopagebreak%
  \ifthenelse{\equal{#5}{}}{}{Coauthor(s): #5\vspace{3mm}\\\nopagebreak}%
  \ifthenelse{\equal{#6}{}}{}{Special session: #6\quad \vspace{3mm}\\\nopagebreak}%
 }
 {\vspace{1cm}\nopagebreak}%

\pagestyle{empty}

% ------------------------------------------------------------------------
% Document begins here
% ------------------------------------------------------------------------
\begin{document}
	
\begin{talk}
  {A nested Multilevel Monte Carlo framework for efficient simulations on FPGAs}% [1] talk title
  {Irina-Beatrice Haas}% [2] speaker name
  {Mathematical Institute, University of Oxford}% [3] affiliations
  {irina-beatrice.haas@maths.ox.ac.uk}% [4] email
  {Michael B. Giles}% [5] coauthors
  {Hardware or Software for (Quasi-)Monte Carlo Algorithm}% [6] special session. Leave this field empty for contributed talks. 
				% Insert the title of the special session if you were invited to give a talk in a special session.
			

Multilevel Monte Carlo (MLMC) is a computational method that reduces the cost of Monte Carlo simulations by combining SDE approximations with multiple resolutions. A further avenue for significantly reducing cost and improving power efficiency of MLMC, notably for financial option pricing, is the use of low precision calculations on configurable hardware devices such as Field-Programmable Gate Arrays (FPGAs). With this goal in mind, in this talk we propose a new MLMC framework that exploits approximate random variables and fixed-point operations with optimised precision to compute most SDE paths with a lower cost.

For the generation of random Normal increments, we discuss several methods based on the approximation of the inverse normal CDF (see e.g. [3]), and we argue that these methods could be implemented on FPGAs using small Look-Up-Tables to generate random numbers more efficiently than on CPUs or GPUs. 

To set the bit-width of variables in the path generation we propose a rounding error model and optimise the precision of all variables on each Monte Carlo level. This optimisation stage is independent of the desired overall accuracy, and can therefore be performed off-line. 

With these two key improvements, our proposed framework [2] offers higher computational
savings than the existing mixed-precision MLMC framework [1].


\medskip

%If you would like to include references, please do so by creating a simple list numbered by [1], [2], [3], \ldots. See example below.
%Please do not use the \texttt{bibliography} environment or \texttt{bibtex} files.
%APA reference style is recommended.

\begin{enumerate}

    \item[{[1]}] Brugger, C., de Schryver, C., Wehn, N., Omland, S., Hefter, M., Ritter, K., Kostiuk, A., \& Korn, R. (2014). Mixed precision multilevel Monte Carlo on hybrid computing systems. In Proceedings of the IEEE Conference on Computational Intelligence for Financial Engineering \& Economics (CIFEr) (pp. 215–222). IEEE. \url{https://doi.org/10.1109/CIFEr.2014.6924076}
    
    \item[{[2]}] Haas, I. B., \& Giles, M. B. (2025). A nested MLMC framework for efficient simulations on FPGAs. Accepted to appear in Monte Carlo Methods and Applications. arXiv preprint arXiv:2502.07123. \url{https://arxiv.org/abs/2502.07123}

     \item[{[3]}] Giles, M.B., \& Sheridan-Methven, O. (2023). Approximating Inverse Cumulative Distribution Functions to Produce Approximate Random Variables. ACM Transactions on Mathematical Software, 49(3), no 26. \url{https://doi.org/10.1145/3604935}
\end{enumerate}

%Equations may be used if they are referenced. Please note that the equation numbers may be different (but will be cross-referenced correctly) in the final program book.
\end{talk}

\end{document}

