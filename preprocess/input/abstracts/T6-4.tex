\documentclass[12pt,a4paper,figuresright]{book}

\usepackage{amsmath,amssymb}
\usepackage{tabularx,graphicx,url,xcolor,rotating,multicol,epsfig,colortbl}

\setlength{\textheight}{25.2cm}
\setlength{\textwidth}{16.5cm} %\setlength{\textwidth}{18.2cm}
\setlength{\voffset}{-1.6cm}
\setlength{\hoffset}{-0.3cm} %\setlength{\hoffset}{-1.2cm}
\setlength{\evensidemargin}{-0.3cm} 
\setlength{\oddsidemargin}{0.3cm}
\setlength{\parindent}{0cm} 
\setlength{\parskip}{0.3cm}

% -- adding a talk
\newenvironment{talk}[6]% [1] talk title
                         % [2] speaker name, [3] affiliations, [4] email,
                         % [5] coauthors, [6] special session
                         % [7] time slot
                         % [8] talk id, [9] session id or photo
 {%\needspace{6\baselineskip}%
  \vskip 0pt\nopagebreak%
%   \colorbox{gray!20!white}{\makebox[0.99\textwidth][r]{}}\nopagebreak%
%   \ifthenelse{\equal{#9}{photo}}{%
%                     \\\\\colorbox{gray!20!white}{\makebox{\includegraphics[width=3cm]{#8}}}\nopagebreak}{}%
 \vskip 0pt\nopagebreak%
%  \label{#8}%
  \textbf{#1}\vspace{3mm}\\\nopagebreak%
  \textit{#2}\\\nopagebreak%
  #3\\\nopagebreak%
  \url{#4}\vspace{3mm}\\\nopagebreak%
  \ifthenelse{\equal{#5}{}}{}{Coauthor(s): #5\vspace{3mm}\\\nopagebreak}%
  \ifthenelse{\equal{#6}{}}{}{Special session: #6\quad \vspace{3mm}\\\nopagebreak}%
 }
 {\vspace{1cm}\nopagebreak}%

\pagestyle{empty}

% ------------------------------------------------------------------------
% Document begins here
% ------------------------------------------------------------------------
\begin{document}
	
\begin{talk}
  {Importance Sampling for Hawkes Processes}% [1] talk title
  {Alex Shkolnik}% [2] speaker name
  {University of California, Santa Barbara}% [3] affiliations
  {shkolnik@ucsb.edu}% [4] email
  {Baeho Kim}% [5] coauthors
  {}% [6] special session. Leave this field empty for contributed talks. 
				% Insert the title of the special session if you were invited to give a talk in a special session.
			
In 1971, Alan Hawkes [1] introduced a highly influential point
process $N$ for which, given a constant $\mu > 0$ and a
$[0,\infty)$-valued function $g$, the intensity process $X$
takes the form,
\begin{align} \label{HawkesIntensity} 
 \quad X_t = \mu + \int_0^t g(t-s) \, \mathrm{d} N_s \, ,
 \quad (t \ge 0) \, .
\end{align}
By now, Hawkes processes have found a wide array of application
in the sciences, engineering, statistics, operations research,
mathematical finance and machine learning. We develop importance
sampling estimators for rare-event probabilities of the form
$\mathbb{P}(N_t \ge c \, t)$ and general functions $g$ in
$(\ref{HawkesIntensity})$. This problem has received little 
attention to
date, as for most $g$, the process $(N,X)$ is
non-Markovian and lends to little mathematical tractability.
Our approach is based on a Girsanov change of intensity coupled
with a conditioning on the rare-event.  We prove asymptotic
optimality of the resulting importance sampling estimators in
the limit $t \to \infty$. Related large deviations results and
an extension to fully nonlinear models of $N$ with intensity
$\phi(X)$ are presented.  Numerical simulations illustrate the
performance of our importance sampling estimators relative to
Monte Carlo for various functions $g$ as well as to exponential
tilting in the case of an exponential $g$ (the sole tractable
model).


%Please do not use your own commands or macros.
\medskip

%If you would like to include references, please do so by creating a simple list numbered by [1], [2], [3], \ldots. See example below.

%Please do not use the \texttt{bibliography} environment or \texttt{bibtex} files. APA reference style is recommended.
\begin{enumerate}
	\item[{[1]}] Hawkes, A. G. (1971). {\it Spectra of some 
self-exciting and mutually exciting point processes}. 
Biometrika 58(1), 83–90.
\end{enumerate}


%Equations may be used if they are referenced. Please note that the equation numbers may be different (but will be cross-referenced correctly) in the final program book.
\end{talk}

\end{document}

