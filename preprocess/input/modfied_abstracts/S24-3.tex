\documentclass[12pt,a4paper,figuresright]{book}

\usepackage{amsmath,amssymb}
\usepackage{tabularx,graphicx,url,xcolor,rotating,multicol,epsfig,colortbl}

\setlength{\textheight}{25.2cm}
\setlength{\textwidth}{16.5cm} %\setlength{\textwidth}{18.2cm}
\setlength{\voffset}{-1.6cm}
\setlength{\hoffset}{-0.3cm} %\setlength{\hoffset}{-1.2cm}
\setlength{\evensidemargin}{-0.3cm} 
\setlength{\oddsidemargin}{0.3cm}
\setlength{\parindent}{0cm} 
\setlength{\parskip}{0.3cm}

% -- adding a talk
\newenvironment{talk}[6]% [1] talk title
                         % [2] speaker name, [3] affiliations, [4] email,
                         % [5] coauthors, [6] special session
                         % [7] time slot
                         % [8] talk id, [9] session id or photo
 {%\needspace{6\baselineskip}%
  \vskip 0pt\nopagebreak%
%   \colorbox{gray!20!white}{\makebox[0.99\textwidth][r]{}}\nopagebreak%
%   \ifthenelse{\equal{#9}{photo}}{%
%                     \\\\\colorbox{gray!20!white}{\makebox{\includegraphics[width=3cm]{#8}}}\nopagebreak}{}%
 \vskip 0pt\nopagebreak%
%  \label{#8}%
  \textbf{#1}\vspace{3mm}\\\nopagebreak%
  \textit{#2}\\\nopagebreak%
  #3\\\nopagebreak%
  \url{#4}\vspace{3mm}\\\nopagebreak%
  \ifthenelse{\equal{#5}{}}{}{Coauthor(s): #5\vspace{3mm}\\\nopagebreak}%
  \ifthenelse{\equal{#6}{}}{}{Special session: #6\quad \vspace{3mm}\\\nopagebreak}%
 }
 {\vspace{1cm}\nopagebreak}%

\pagestyle{empty}

% ------------------------------------------------------------------------
% Document begins here
% ------------------------------------------------------------------------
\begin{document}

\begin{talk}
  {Stochastic gradient with least-squares control variates}% [1] talk title
  {Matteo Raviola}% [2] speaker name
  {École polytechnique fédérale de Lausanne}% [3] affiliations
  {matteo.raviola@epfl.ch}% [4] email
  {Fabio Nobile, Nathan Schaeffer}% [5] coauthors
  {}% [6] special session. Leave this field empty for contributed talks. 
  % Insert the title of the special session if you were invited to give a talk in a special session.
  
  The stochastic gradient (SG) method is a widely used approach for solving stochastic optimization problems, but its convergence is typically slow.
  Existing variance reduction techniques, such as SAGA [1], improve convergence by leveraging stored gradient information; however, they are restricted to settings where the objective functional is a finite sum, and their performance degrades when the number of terms in the sum is large.
  In this work, we propose a novel approach which also works when the objective is given by an expectation over random variables with a continuous probability distribution.
  Our method constructs a control variate by fitting a linear model to past gradient evaluations using weighted discrete least-squares, effectively reducing variance while preserving computational efficiency.
  We establish theoretical sublinear convergence guarantees and demonstrate the method's effectiveness through numerical experiments on random PDE-constrained optimization.
  
  \medskip
  
  \begin{enumerate}
    \item[{[1]}] Defazio, A., Bach, F., \& Lacoste-Julien, S. (2014). {\it SAGA: A fast incremental gradient method with support for non-strongly convex composite objectives.} Advances in neural information processing systems, 27.
  \end{enumerate}
  
\end{talk}

\end{document}

