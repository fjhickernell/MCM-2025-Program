\documentclass[12pt,a4paper,figuresright]{book}

\usepackage{amsmath,amssymb}
\usepackage{tabularx,graphicx,url,xcolor,rotating,multicol,epsfig,colortbl}

\setlength{\textheight}{25.2cm}
\setlength{\textwidth}{16.5cm} %\setlength{\textwidth}{18.2cm}
\setlength{\voffset}{-1.6cm}
\setlength{\hoffset}{-0.3cm} %\setlength{\hoffset}{-1.2cm}
\setlength{\evensidemargin}{-0.3cm} 
\setlength{\oddsidemargin}{0.3cm}
\setlength{\parindent}{0cm} 
\setlength{\parskip}{0.3cm}

% -- adding a talk
\newenvironment{talk}[6]% [1] talk title
                         % [2] speaker name, [3] affiliations, [4] email,
                         % [5] coauthors, [6] special session
                         % [7] time slot
                         % [8] talk id, [9] session id or photo
 {%\needspace{6\baselineskip}%
  \vskip 0pt\nopagebreak%
%   \colorbox{gray!20!white}{\makebox[0.99\textwidth][r]{}}\nopagebreak%
%   \ifthenelse{\equal{#9}{photo}}{%
%                     \\\\\colorbox{gray!20!white}{\makebox{\includegraphics[width=3cm]{#8}}}\nopagebreak}{}%
 \vskip 0pt\nopagebreak%
%  \label{#8}%
  \textbf{#1}\vspace{3mm}\\\nopagebreak%
  \textit{#2}\\\nopagebreak%
  #3\\\nopagebreak%
  \url{#4}\vspace{3mm}\\\nopagebreak%
  \ifthenelse{\equal{#5}{}}{}{Coauthor(s): #5\vspace{3mm}\\\nopagebreak}%
  \ifthenelse{\equal{#6}{}}{}{Special session: #6\quad \vspace{3mm}\\\nopagebreak}%
 }
 {\vspace{1cm}\nopagebreak}%

\pagestyle{empty}

% ------------------------------------------------------------------------
% Document begins here
% ------------------------------------------------------------------------
\begin{document}
	
\begin{talk}
  {Sharp Characterization and Control of Global Dynamics of SGDs with Heavy Tails}% [1] talk title
  {Xingyu Wang}% [2] speaker name
  {University of Amsterdam}% [3] affiliations
  {x.wang4@uva.nl}% [4] email
  {Chang-Han Rhee, Sewoong Oh}% [5] coauthors
  {Heavy tailed sampling }% [6] special session. Leave this field empty for contributed talks. 
				% Insert the title of the special session if you were invited to give a talk in a special session.
			
The empirical success of deep learning is often attributed to the mysterious ability of stochastic gradient descents (SGDs) to avoid sharp local minima in the loss landscape, as sharp minima are believed to lead to poor generalization. To unravel this mystery and potentially further enhance such capability of SGDs, it is imperative to go beyond the traditional local convergence analysis and obtain a comprehensive understanding of SGDs' global dynamics within complex non-convex loss landscapes. In this talk, we characterize the global dynamics of SGDs through the heavy-tailed large deviations and local stability framework. This framework systematically characterizes the rare events in heavy-tailed dynamical systems; building on this, we characterize intricate phase transitions in the first exit times, which leads to the heavy-tailed counterparts of the classical Freidlin-Wentzell and Eyring-Kramers theories. Moreover, applying this framework to SGD, we reveal a fascinating phenomenon in deep learning: by injecting and then truncating heavy-tailed noises during the training phase, SGD can almost completely avoid sharp minima and hence achieve better generalization performance for the test data.

% \medskip

% If you would like to include references, please do so by creating a simple list numbered by [1], [2], [3], \ldots. See example below.
% Please do not use the \texttt{bibliography} environment or \texttt{bibtex} files.
% APA reference style is recommended.
% \begin{enumerate}
% 	\item[{[1]}] Niederreiter, Harald (1992). {\it Random number generation and quasi-Monte Carlo methods}. Society for Industrial and Applied Mathematics (SIAM).
% 	\item[{[2]}] L’Ecuyer, Pierre, \& Christiane Lemieux. (2002). Recent advances in randomized quasi-Monte Carlo methods. Modeling uncertainty: An examination of stochastic theory, methods, and applications, 419-474.
% \end{enumerate}

% Equations may be used if they are referenced. Please note that the equation numbers may be different (but will be cross-referenced correctly) in the final program book.
\end{talk}

\end{document}

