\documentclass[12pt,a4paper,figuresright]{book}

\usepackage{amsmath,amssymb}
\usepackage{tabularx,graphicx,url,xcolor,rotating,multicol,epsfig,colortbl}

\setlength{\textheight}{25.2cm}
\setlength{\textwidth}{16.5cm} %\setlength{\textwidth}{18.2cm}
\setlength{\voffset}{-1.6cm}
\setlength{\hoffset}{-0.3cm} %\setlength{\hoffset}{-1.2cm}
\setlength{\evensidemargin}{-0.3cm} 
\setlength{\oddsidemargin}{0.3cm}
\setlength{\parindent}{0cm} 
\setlength{\parskip}{0.3cm}

% -- adding a talk
\newenvironment{talk}[6]% [1] talk title
                         % [2] speaker name, [3] affiliations, [4] email,
                         % [5] coauthors, [6] special session
                         % [7] time slot
                         % [8] talk id, [9] session id or photo
 {%\needspace{6\baselineskip}%
  \vskip 0pt\nopagebreak%
%   \colorbox{gray!20!white}{\makebox[0.99\textwidth][r]{}}\nopagebreak%
%   \ifthenelse{\equal{#9}{photo}}{%
%                     \\\\\colorbox{gray!20!white}{\makebox{\includegraphics[width=3cm]{#8}}}\nopagebreak}{}%
 \vskip 0pt\nopagebreak%
%  \label{#8}%
  \textbf{#1}\vspace{3mm}\\\nopagebreak%
  \textit{#2}\\\nopagebreak%
  #3\\\nopagebreak%
  \url{#4}\vspace{3mm}\\\nopagebreak%
  \ifthenelse{\equal{#5}{}}{}{Coauthor(s): #5\vspace{3mm}\\\nopagebreak}%
  \ifthenelse{\equal{#6}{}}{}{Special session: #6\quad \vspace{3mm}\\\nopagebreak}%
 }
 {\vspace{1cm}\nopagebreak}%

\pagestyle{empty}

% ------------------------------------------------------------------------
% Document begins here
% ------------------------------------------------------------------------
\begin{document}
	
\begin{talk}
  {State and parameter inference in stochastic reaction networks}% [1] talk title
  {Muruhan Rathinam}% [2] speaker name
  {University of Maryland Baltimore County}% [3] affiliations
  {muruhan@umbc.edu}% [4] email
  {Mingkai Yu}% [5] coauthors
  {}% [6] special session. Leave this field empty for contributed talks. 
				% Insert the title of the special session if you were invited to give a talk in a special session.
Continuous time Markov chain models are widely used to model intracellular chemical reactions networks that arise in systems and synthetic biology. In this talk, we address the problem of inference of state and parameters of such systems from partial observations. We present details of recent particle filtering methods that are applicable to two different scenarios: one in which the observations are made continuously in time and the other in which the observations are made in discrete snapshots of time.            We provide the theoretical justification as well as numerical results to illustrate these methods.       
			
\medskip

%If you would like to include references, please do so by creating a simple list numbered by [1], [2], [3], \ldots. See example below.
%Please do not use the \texttt{bibliography} environment or \texttt{bibtex} files.
%APA reference style is recommended.
\begin{enumerate}
	\item[{[1]}] Rathinam, Muruhan \& Yu, Mingkai (2021). {\it State and parameter estimation from exact partial state observation in stochastic reaction networks}. The Journal of Chemical Physics. 
 154(3).
	\item[{[2]}] Rathinam, Muruhan \& Yu, Mingkai (2023). Stochastic Filtering of Reaction Networks Partially Observed in Time Snapshots. Journal of Computational Physics. 
Volume 515, 15 October 2024, 113265. 

\end{enumerate}

%Equations may be used if they are referenced. Please note that the equation numbers may be different (but will be cross-referenced correctly) in the final program book.
\end{talk}

\end{document}

