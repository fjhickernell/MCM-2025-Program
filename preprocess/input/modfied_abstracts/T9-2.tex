\documentclass[12pt,a4paper,figuresright]{book}

\usepackage{amsmath,amssymb}
\usepackage{tabularx,graphicx,url,xcolor,rotating,multicol,epsfig,colortbl}

\setlength{\textheight}{25.2cm}
\setlength{\textwidth}{16.5cm} %\setlength{\textwidth}{18.2cm}
\setlength{\voffset}{-1.6cm}
\setlength{\hoffset}{-0.3cm} %\setlength{\hoffset}{-1.2cm}
\setlength{\evensidemargin}{-0.3cm} 
\setlength{\oddsidemargin}{0.3cm}
\setlength{\parindent}{0cm} 
\setlength{\parskip}{0.3cm}

% -- adding a talk
\newenvironment{talk}[6]% [1] talk title
                         % [2] speaker name, [3] affiliations, [4] email,
                         % [5] coauthors, [6] special session
                         % [7] time slot
                         % [8] talk id, [9] session id or photo
 {%\needspace{6\baselineskip}%
  \vskip 0pt\nopagebreak%
%   \colorbox{gray!20!white}{\makebox[0.99\textwidth][r]{}}\nopagebreak%
%   \ifthenelse{\equal{#9}{photo}}{%
%                     \\\\\colorbox{gray!20!white}{\makebox{\includegraphics[width=3cm]{#8}}}\nopagebreak}{}%
 \vskip 0pt\nopagebreak%
%  \label{#8}%
  \textbf{#1}\vspace{3mm}\\\nopagebreak%
  \textit{#2}\\\nopagebreak%
  #3\\\nopagebreak%
  \url{#4}\vspace{3mm}\\\nopagebreak%
  \ifthenelse{\equal{#5}{}}{}{Coauthor(s): #5\vspace{3mm}\\\nopagebreak}%
  \ifthenelse{\equal{#6}{}}{}{Special session: #6\quad \vspace{3mm}\\\nopagebreak}%
 }
 {\vspace{1cm}\nopagebreak}%

\pagestyle{empty}

% ------------------------------------------------------------------------
% Document begins here
% ------------------------------------------------------------------------
\begin{document}
	
\begin{talk}
  {Quantitative results on sampling from quasi-stationary distributions}% [1] talk title
  {Daniel Yukimura}% [2] speaker name
  {IMPA, Rio de Janeiro, Brazil}% [3] affiliations
  {yukimura@impa.br}% [4] email
  {Roberto I. Oliveira}% [5] coauthors
  {}% [6] special session. Leave this field empty for contributed talks. 
				% Insert the title of the special session if you were invited to give a talk in a special session.
			
    We study the rate of convergence of Sequential Monte Carlo (SMC) methods for approximating the quasi-stationary ditribution (QSD) of Markov processes. 
    For processes with killing or absorption, the QSD appears as a stable behavior observed before extinction, or as the limiting distribution of the process conditioned on not being absorbed. 
    We give quantitative lower and upper bounds for the particle filter approximation of these distributions.
    For the lower bound, we show that fast mixing is not enough to guarantee that simulation methods can converge in few steps.
    In the upper bound, we show that SMC with adaptive resampling has a rate depending on the number of steps, the mixing time, and in how fast the processed gets killed.
    Our seems to be the first result to have a quantitative dependency of this form that is valid for discrete time Markov chains in general state spaces.
    Our techniques and concentration results for bounding the approximations of SMC with adaptive resampling are also novel, and we believe might be applicable in other scenarios that can benefit from the lower variance obtained due to an adaptive approach.
    % Finally, as an example, we apply these results to get quantitative estimates for a large class of killed processes in $\mathbb{Z}^d$, where we obtain the first bounds with the ``natural" time scale.
    
    

\medskip


\end{talk}

\end{document}

