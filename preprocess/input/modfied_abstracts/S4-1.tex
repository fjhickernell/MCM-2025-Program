\documentclass[12pt,a4paper,figuresright]{book}

\usepackage{amsmath,amssymb}
\usepackage{tabularx,graphicx,url,xcolor,rotating,multicol,epsfig,colortbl}

\setlength{\textheight}{25.2cm}
\setlength{\textwidth}{16.5cm} %\setlength{\textwidth}{18.2cm}
\setlength{\voffset}{-1.6cm}
\setlength{\hoffset}{-0.3cm} %\setlength{\hoffset}{-1.2cm}
\setlength{\evensidemargin}{-0.3cm} 
\setlength{\oddsidemargin}{0.3cm}
\setlength{\parindent}{0cm} 
\setlength{\parskip}{0.3cm}

% -- adding a talk
\newenvironment{talk}[6]% [1] talk title
                         % [2] speaker name, [3] affiliations, [4] email,
                         % [5] coauthors, [6] special session
                         % [7] time slot
                         % [8] talk id, [9] session id or photo
 {%\needspace{6\baselineskip}%
  \vskip 0pt\nopagebreak%
%   \colorbox{gray!20!white}{\makebox[0.99\textwidth][r]{}}\nopagebreak%
%   \ifthenelse{\equal{#9}{photo}}{%
%                     \\\\\colorbox{gray!20!white}{\makebox{\includegraphics[width=3cm]{#8}}}\nopagebreak}{}%
 \vskip 0pt\nopagebreak%
%  \label{#8}%
  \textbf{#1}\vspace{3mm}\\\nopagebreak%
  \textit{#2}\\\nopagebreak%
  #3\\\nopagebreak%
  \url{#4}\vspace{3mm}\\\nopagebreak%
  \ifthenelse{\equal{#5}{}}{}{Coauthor(s): #5\vspace{3mm}\\\nopagebreak}%
  \ifthenelse{\equal{#6}{}}{}{Special session: #6\quad \vspace{3mm}\\\nopagebreak}%
 }
 {\vspace{1cm}\nopagebreak}%

\pagestyle{empty}

% ------------------------------------------------------------------------
% Document begins here
% ------------------------------------------------------------------------
\begin{document}
	
\begin{talk}
  {Multilevel quasi-Monte Carlo without replications}% [1] talk title
  {Pieterjan Robbe}% [2] speaker name
  {Sandia National Laboratories}% [3] affiliations
  {pmrobbe@sandia.gov}% [4] email
  {Aleksei Sorokin, Gianluca Geraci, Fred J. Hickernell, Mike Eldred}% [5] coauthors
  {Hardware or Software for (Quasi-)Monte Carlo Algorithms}% [6] special session. Leave this field empty for contributed talks. 
				% Insert the title of the special session if you were invited to give a talk in a special session.
			
In this talk, we explore a novel approach to multilevel quasi-Monte Carlo (MLQMC) sampling that eliminates the need for stochastic replications. Our approach for estimating the level-wise variances is based on the Bayesian cubature framework introduced in [1]. Empirical results from a series of numerical experiments illustrate the effectiveness of our method in various applications. We discuss the integration of our new method in Dakota, Sandia's flagship UQ software package.

\medskip

\begin{enumerate}
  \item[{[1]}] Jagadeeswaran, R., \& Hickernell, F.\,J. (2019). Fast automatic Bayesian cubature using lattice sampling. Statistics and Computing, \textbf{29}(6), 1215--1229.
\end{enumerate}

\end{talk}

\end{document}

