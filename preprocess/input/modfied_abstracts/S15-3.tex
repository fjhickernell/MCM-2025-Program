\documentclass[12pt,a4paper,figuresright]{book}

\usepackage{amsmath,amssymb}
\usepackage{tabularx,graphicx,url,xcolor,rotating,multicol,epsfig,colortbl}

\setlength{\textheight}{25.2cm}
\setlength{\textwidth}{16.5cm} %\setlength{\textwidth}{18.2cm}
\setlength{\voffset}{-1.6cm}
\setlength{\hoffset}{-0.3cm} %\setlength{\hoffset}{-1.2cm}
\setlength{\evensidemargin}{-0.3cm} 
\setlength{\oddsidemargin}{0.3cm}
\setlength{\parindent}{0cm} 
\setlength{\parskip}{0.3cm}

% -- adding a talk
\newenvironment{talk}[6]% [1] talk title
                         % [2] speaker name, [3] affiliations, [4] email,
                         % [5] coauthors, [6] special session
                         % [7] time slot
                         % [8] talk id, [9] session id or photo
 {%\needspace{6\baselineskip}%
  \vskip 0pt\nopagebreak%
%   \colorbox{gray!20!white}{\makebox[0.99\textwidth][r]{}}\nopagebreak%
%   \ifthenelse{\equal{#9}{photo}}{%
%                     \\\\\colorbox{gray!20!white}{\makebox{\includegraphics[width=3cm]{#8}}}\nopagebreak}{}%
 \vskip 0pt\nopagebreak%
%  \label{#8}%
  \textbf{#1}\vspace{3mm}\\\nopagebreak%
  \textit{#2}\\\nopagebreak%
  #3\\\nopagebreak%
  \url{#4}\vspace{3mm}\\\nopagebreak%
  \ifthenelse{\equal{#5}{}}{}{Coauthor(s): #5\vspace{3mm}\\\nopagebreak}%
  \ifthenelse{\equal{#6}{}}{}{Special session: #6\quad \vspace{3mm}\\\nopagebreak}%
 }
 {\vspace{1cm}\nopagebreak}%

\pagestyle{empty}

% ------------------------------------------------------------------------
% Document begins here
% ------------------------------------------------------------------------
\begin{document}
	
\begin{talk}
  {Boosting the inference for generative models by (Quasi-)Monte Carlo resampling}% [1] talk title
  {Ziang Niu}% [2] speaker name
  {University of Pennsylvania}% [3] affiliations
  {ziangniu@wharton.upenn.edu}% [4] email
  {Bhaswar B. Bhattacharya, François-Xavier Briol, Anirban Chatterjee, Johanna Meier.}% [5] coauthors
  {Frontiers in (Quasi-)Monte Carlo and Markov Chain Monte Carlo Methods}% [6] special session. Leave this field empty for contributed talks. 
				% Insert the title of the special session if you were invited to give a talk in a special session.
			
In the era of generative models, statistical inference based on classical likelihood for such models has faced a challenge. This is due to highly nontrivial model structures and thus computing the likelihood functions is almost impossible. Often time, information on these generative models can only be obtained by sampling from the models but the necessity of sampling can further pose a tradeoff between computational burden and statistical accuracy. In this talk, we propose a framework for statistical inference for generative models leveraging the techniques from (Quasi-)Monte Carlo. Despite the unavoidable balance of statistical accuracy and computation, both computational and statistical performances can be boosted by employing (Quasi-)Monte Carlo techniques. The presentation will be based on two papers.

\begin{enumerate}
	\item[{[1]}] A. Chatterjee, Z. Niu \& B. Bhattacharya (2024). {\it A kernel-based conditional two-sample test using nearest neighbors (with applications to calibration, regression curves, and simulation-based inference)}. Preprint.
	\item[{[2]}] Z. Niu, J. Meier \& F-X. Briol (2023). \it{Discrepancy-based inference for
  intractable generative models using
  Quasi-Monte Carlo.} Electronic Journal of Statistics.
\end{enumerate}

\medskip

\end{talk}



\end{document}

