\documentclass[12pt,a4paper,figuresright]{book}

\usepackage{amsmath,amssymb}
\usepackage{tabularx,graphicx,url,xcolor,rotating,multicol,epsfig,colortbl}

\setlength{\textheight}{25.2cm}
\setlength{\textwidth}{16.5cm} %\setlength{\textwidth}{18.2cm}
\setlength{\voffset}{-1.6cm}
\setlength{\hoffset}{-0.3cm} %\setlength{\hoffset}{-1.2cm}
\setlength{\evensidemargin}{-0.3cm} 
\setlength{\oddsidemargin}{0.3cm}
\setlength{\parindent}{0cm} 
\setlength{\parskip}{0.3cm}

% -- adding a talk
\newenvironment{talk}[6]% [1] talk title
                         % [2] speaker name, [3] affiliations, [4] email,
                         % [5] coauthors, [6] special session
                         % [7] time slot
                         % [8] talk id, [9] session id or photo
 {%\needspace{6\baselineskip}%
  \vskip 0pt\nopagebreak%
%   \colorbox{gray!20!white}{\makebox[0.99\textwidth][r]{}}\nopagebreak%
%   \ifthenelse{\equal{#9}{photo}}{%
%                     \\\\\colorbox{gray!20!white}{\makebox{\includegraphics[width=3cm]{#8}}}\nopagebreak}{}%
 \vskip 0pt\nopagebreak%
%  \label{#8}%
  \textbf{#1}\vspace{3mm}\\\nopagebreak%
  \textit{#2}\\\nopagebreak%
  #3\\\nopagebreak%
  \url{#4}\vspace{3mm}\\\nopagebreak%
  \ifthenelse{\equal{#5}{}}{}{Coauthor(s): #5\vspace{3mm}\\\nopagebreak}%
  \ifthenelse{\equal{#6}{}}{}{Special session: #6\quad \vspace{3mm}\\\nopagebreak}%
 }
 {\vspace{1cm}\nopagebreak}%

\pagestyle{empty}

% ------------------------------------------------------------------------
% Document begins here
% ------------------------------------------------------------------------
\begin{document}
	
\begin{talk}
  {High-order adaptive methods for exit times of diffusion processes and reflected diffusions}% [1] talk title
  {H{\aa}kon Hoel}% [2] speaker name
  {University of Oslo}% [3] affiliations
  {haakonah@math.uio.no}% [4] email
  {}% [5] coauthors
  {}% [6] special session. Leave this field empty for contributed talks. 
				% Insert the title of the special session if you were invited to give a talk in a special session.
			
\medskip
The Feynman--Kac formula connects domain-exit and boundary-reflection properties of stochastic differential equations (SDEs) and parabolic partial differential equations. The SDE viewpoint is particularly interesting for numerical methods, as it can be used with Monte Carlo methods to overcome the curse of dimensionality when solving high-dimensional PDE. This however hinges on having an efficient numerical method for simulating the exit times of SDEs. Since exit times of diffusion processes are very sensitive to perturbations in initial conditions, it is challenging to construct such numerical methods.

This talk presents a high-order method with adaptive time-stepping for strong approximations of exit times. The method employs a high-order Itô--Taylor scheme for simulating SDE paths and carefully decreases the step size in the numerical integration as the diffusion process approaches the domain's boundary. These techniques complement each other well: adaptive time-stepping improves the accuracy of the exit time by reducing the overshoot out of the domain, and high-order schemes improve the state approximation of the diffusion process, which is useful feedback to control the step size. We will also consider an ongoing extension of the numerical method to reflected diffusions.

\end{talk}

\end{document}

