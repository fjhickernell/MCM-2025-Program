\documentclass[12pt,a4paper,figuresright]{book}

\usepackage{amsmath,amssymb}
\usepackage{tabularx,graphicx,url,xcolor,rotating,multicol,epsfig,colortbl}

\setlength{\textheight}{25.2cm}
\setlength{\textwidth}{16.5cm} %\setlength{\textwidth}{18.2cm}
\setlength{\voffset}{-1.6cm}
\setlength{\hoffset}{-0.3cm} %\setlength{\hoffset}{-1.2cm}
\setlength{\evensidemargin}{-0.3cm} 
\setlength{\oddsidemargin}{0.3cm}
\setlength{\parindent}{0cm} 
\setlength{\parskip}{0.3cm}

% -- adding a talk
\newenvironment{talk}[6]% [1] talk title
                         % [2] speaker name, [3] affiliations, [4] email,
                         % [5] coauthors, [6] special session
                         % [7] time slot
                         % [8] talk id, [9] session id or photo
 {%\needspace{6\baselineskip}%
  \vskip 0pt\nopagebreak%
  \vskip 0pt\nopagebreak%
  \textbf{#1}\vspace{3mm}\\\nopagebreak%
  \textit{#2}\\\nopagebreak%
  #3\\\nopagebreak%
  \url{#4}\vspace{3mm}\\\nopagebreak%
  \ifthenelse{\equal{#5}{}}{}{Coauthor(s): #5\vspace{3mm}\\\nopagebreak}%
  \ifthenelse{\equal{#6}{}}{}{Special session: #6\quad \vspace{3mm}\\\nopagebreak}%
 }
 {\vspace{1cm}\nopagebreak}%

\pagestyle{empty}

% ------------------------------------------------------------------------
% Document begins here
% ------------------------------------------------------------------------
\begin{document}
	
\begin{talk}
  {Domain UQ for stationary and time-dependent PDEs using
QMC}% [1] talk title
  {André-Alexander Zepernick}% [2] speaker name
  {Free University of Berlin}% [3] affiliations
  {a.zepernick@fu-berlin.de}% [4] email
  {Ana Djurdjevac, Vesa Kaarnioja, Claudia Schillings}% [5] coauthors
  {Domain Uncertainty Quantification}% [6] special session

The problem of modelling processes with partial differential equations posed on random domains arises in various applications like biology or engineering. We study uncertainty quantification for partial differential equations subject to domain uncertainty. For the random domain parameterization, we adopt an approach, which was also examined by Chernov and L\^{e} [1,2] as well as Harbrecht, Schmidlin, and Schwab [3], where one assumes the input random field to be Gevrey regular. This approach has the advantage of being substantially more general than models which assume a particular parametric representation of the input random field such as a Karhunen--Lo\`eve series expansion. As model problems we consider both the Poisson equation as well as the heat equation and design randomly shifted lattice quasi-Monte Carlo (QMC) cubature rules for the computation of response statistics subject to domain uncertainty. The QMC rules obtained in [4] exhibit dimension-independent, faster-than-Monte Carlo cubature convergence rates. Our theoretical results are illustrated by numerical examples.
\begin{enumerate}
    \item[{[1]}] Chernov, Alexey, \& L\^{e}, Tùng (2024). Analytic and Gevrey class regularity for parametric elliptic eigenvalue problems and applications. \emph{SIAM Journal on Numerical Analysis}, \textbf{62}(4), 1874--1900.
    \item[{[2]}] Chernov, A., \& L\^{e}, Tùng (2024). Analytic and Gevrey class regularity for parametric semilinear reaction-diffusion problems and applications in uncertainty quantification. \emph{Computers \& Mathematics with Applications}, \textbf{164}, 116--130.
    \item[{[3]}] Harbrecht, Helmut, Schmidlin, Marc, \& Schwab, Christoph (2024). The Gevrey class implicit mapping theorem with application to UQ of semilinear elliptic PDEs. \emph{Mathematical Models and Methods in Applied Sciences.}, \textbf{34}(5), 881--917.
    \item[{[4]}] Djurdjevac, Ana, Kaarnioja, Vesa, Schillings, Claudia \& Zepernick, André-Alexander (2025). Uncertainty quantification for stationary and time-dependent PDEs subject to Gevrey regular random domain deformations. Preprint, \emph{arXiv:2502.12345 [math.NA]}.
\end{enumerate}

\medskip

\end{talk}

\end{document}

