\documentclass[12pt,a4paper,figuresright]{book}

\usepackage{amsmath,amssymb}
\usepackage{tabularx,graphicx,url,xcolor,rotating,multicol,epsfig,colortbl}

\setlength{\textheight}{25.2cm}
\setlength{\textwidth}{16.5cm} %\setlength{\textwidth}{18.2cm}
\setlength{\voffset}{-1.6cm}
\setlength{\hoffset}{-0.3cm} %\setlength{\hoffset}{-1.2cm}
\setlength{\evensidemargin}{-0.3cm} 
\setlength{\oddsidemargin}{0.3cm}
\setlength{\parindent}{0cm} 
\setlength{\parskip}{0.3cm}

% -- adding a talk
\newenvironment{talk}[6]% [1] talk title
                         % [2] speaker name, [3] affiliations, [4] email,
                         % [5] coauthors, [6] special session
                         % [7] time slot
                         % [8] talk id, [9] session id or photo
 {%\needspace{6\baselineskip}%
  \vskip 0pt\nopagebreak%
%   \colorbox{gray!20!white}{\makebox[0.99\textwidth][r]{}}\nopagebreak%
%   \ifthenelse{\equal{#9}{photo}}{%
%                     \\\\\colorbox{gray!20!white}{\makebox{\includegraphics[width=3cm]{#8}}}\nopagebreak}{}%
 \vskip 0pt\nopagebreak%
%  \label{#8}%
  \textbf{#1}\vspace{3mm}\\\nopagebreak%
  \textit{#2}\\\nopagebreak%
  #3\\\nopagebreak%
  \url{#4}\vspace{3mm}\\\nopagebreak%
  \ifthenelse{\equal{#5}{}}{}{Coauthor(s): #5\vspace{3mm}\\\nopagebreak}%
  \ifthenelse{\equal{#6}{}}{}{Special session: #6\quad \vspace{3mm}\\\nopagebreak}%
 }
 {\vspace{1cm}\nopagebreak}%

\pagestyle{empty}

% ------------------------------------------------------------------------
% Document begins here
% ------------------------------------------------------------------------
\begin{document}
	
\begin{talk}
  {Revisiting self-normalized importance sampling: new methods and diagnostics}% [1] talk title
  {Nicola Branchini}% [2] speaker name
  {University of Edinburgh}% [3] affiliations
  {n.branchini@sms.ed.ac.uk}% [4] email
  {Víctor Elvira}% [5] coauthors
  {}% [6] special session. Leave this field empty for contributed talks. 
				% Insert the title of the special session if you were invited to give a talk in a special session.
			
Importance sampling (IS) can often be implemented only with normalized weights, yielding the popular self-normalized IS (SNIS) estimator. However, proposal distributions are often learned and evaluated using criteria designed for the unnormalized IS (UIS) estimator.

In this talk, we aim to present a unified perspective on recent methodological advances in understanding and improving SNIS.
We propose and compare two new frameworks for adaptive importance sampling (AIS) methods tailored to SNIS. Our first framework exploits the view of SNIS as a ratio of two UIS estimators, coupling two separate AIS samplers in a joint distribution selected to minimize asymptotic variance. Our second framework instead proposes the first MCMC-driven AIS sampler directly targeting the (often overlooked) optimal SNIS proposal.

We also establish a close connection between the optimal SNIS proposal and so-called subtractive mixture models (SMMs), where negative coefficients are possible - motivating the study of the properties of the first IS estimators using SMMs.

Finally, we propose new Monte Carlo diagnostics specifically for SNIS. They extend existing diagnostics for numerator and denominator by incorporating their statistical dependence, drawing on different notions of tail dependence from multivariate extreme value theory.

\medskip

\begin{enumerate}
	\item[{[1]}] Branchini, N., \& Elvira, V. (2024). Generalizing self-normalized importance sampling with couplings. arXiv preprint arXiv:2406.19974.
	\item[{[2]}] Branchini, N., \& Elvira, V. (2025). Towards adaptive self-normalized importance samplers. In submission at Statistical Signal Processing Workshop (SSP), 2025.
    \item[{[3]}] \textbf{Zellinger, L. \& Branchini, N. (equal contribution)}, Elvira, V., \& Vergari, A. Scalable expectation estimation with subtractive mixture models. In submission at Frontiers in Probabilistic Inference: Learning meets Sampling (workshop at ICLR 2025).
    \item[{[4]}] Branchini, N., \& Elvira, V. The role of tail dependence in estimating posterior expectations. In NeurIPS 2024 Workshop on Bayesian Decision-making and Uncertainty.


\end{enumerate}

\end{talk}

\end{document}

