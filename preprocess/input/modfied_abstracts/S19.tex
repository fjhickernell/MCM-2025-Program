\documentclass[12pt,a4paper,figuresright]{book}

\usepackage{amsmath,amssymb}
\usepackage{tabularx,multirow,graphicx,url,wrapfig,xcolor,rotating,multicol,epsfig,colortbl}

\setlength{\textheight}{25.2cm}
\setlength{\textwidth}{16.5cm} %\setlength{\textwidth}{18.2cm}
\setlength{\voffset}{-1.6cm}
\setlength{\hoffset}{-0.3cm} %\setlength{\hoffset}{-1.2cm}
\setlength{\evensidemargin}{-0.3cm}
\setlength{\oddsidemargin}{0.3cm}
\setlength{\parindent}{0cm}
\setlength{\parskip}{0.3cm}

\renewcommand{\topfraction}{1}
\renewcommand{\textfraction}{0}
\setlength{\floatsep}{12pt plus 2pt minus 2pt}

\newcommand{\organizer}[3]{%
	{\textit{#1}}\\\nopagebreak%
	#2\\\nopagebreak%
	\url{#3}\vspace{3mm}\\\nopagebreak%
	}

\newenvironment{session}[5] % [1] session title
							% [2] number of organizers
                            % [3] organizer 1 info
                            % [4] organizer 2 info
                            % [5] organizer 3 info
                            % [6] session id for later
 {%\needspace{6\baselineskip}
  \vskip 0pt\nopagebreak%
  %\label{#5}%
  \textbf{#1}\vspace{3mm}\\\nopagebreak%
  \ifthenelse{\equal{#2}{1}}{Organizer:}{Organizers:}%
  \vspace{2mm}\\\nopagebreak%
  #3
  \ifthenelse{\equal{#2}{2}}{#4}{}%
  \ifthenelse{\equal{#2}{3}}{#4#5}{}%
  \quad\\\nopagebreak%
  %Session Description:\vspace{3mm}\\\nopagebreak%
 }
 {\nopagebreak}%


\pagestyle{empty}

% ------------------------------------------------------------------------
% Document begins here
% ------------------------------------------------------------------------
\begin{document}

%Input the relevant information below
\begin{session}
  {Stochastic Optimization}% [1] session title
  {1} %[2]  number of organizers
  {\organizer{Shane G. Henderson}% organizer one name
    {Cornell University}% orgnizer one affiliations
    {sgh9@cornell.edu}}% organizer one email
%  {\organizer{Name two}% organizer two name, if needed
%	{Affiliation(s) two}% orgnizer two affiliations, if needed
%	{organizer-two-email-goes@here}}% organizer two email
%  {\organizer{Name three}% organizer one name
%	{Affiliation(s) three}% orgnizer one affiliations
%	{organizer-three-email-goes@here}}% organizer one email

~In many applications, one wishes to solve an optimization problem
$\min_{x \in X} f(x)$, where $f(\cdot)$ and/or its derivatives can
only be evaluated through noisy estimates obtained using Monte Carlo
simulation. Such problems are ubiquitous in machine learning, and also
arise in stochastic simulation applications. This session will consist
of 3 talks in the area.

Proposed Speakers
\begin{enumerate}
\item Raghu Bollapragada, Graduate Program in Operations Research and Industrial Engineering, University of Texas
  at Austin.
\item Raghu Pasupathy, Department of Statistics, Purdue University.
\item Shane G. Henderson, School of Operations Research and Information
    Engineering, Cornell University.
\end{enumerate}

\medskip

%If you would like to include references, please do so by creating a simple list numbered by [1], [2], [3], \ldots. See example below.
%Please do not use the \texttt{bibliography} environment or \texttt{bibtex} files.
%APA reference style is recommended.
%\begin{enumerate}
%	\item[{[1]}] Niederreiter, Harald (1992). {\it Random number generation and quasi-Monte Carlo methods}. Society for Industrial and Applied Mathematics (SIAM).
%	\item[{[2]}] Roberts, Gareth O, \& Rosenthal, Jeffrey S. (2002).  Optimal scaling for various Metropolis-Hastings algorithms, \textbf{16}(4), 351--367.
%\end{enumerate}

%Equations may be used if they are referenced. Please note that the equation numbers may be different (but will be cross-referenced correctly) in the final program book.

\end{session}

\end{document}

