\documentclass[12pt,a4paper,figuresright]{book}

\usepackage{amsmath,amssymb}
\usepackage{tabularx,graphicx,url,xcolor,rotating,multicol,epsfig,colortbl}

\setlength{\textheight}{25.2cm}
\setlength{\textwidth}{16.5cm} %\setlength{\textwidth}{18.2cm}
\setlength{\voffset}{-1.6cm}
\setlength{\hoffset}{-0.3cm} %\setlength{\hoffset}{-1.2cm}
\setlength{\evensidemargin}{-0.3cm} 
\setlength{\oddsidemargin}{0.3cm}
\setlength{\parindent}{0cm} 
\setlength{\parskip}{0.3cm}

% -- adding a talk
\newenvironment{talk}[6]% [1] talk title
                         % [2] speaker name, [3] affiliations, [4] email,
                         % [5] coauthors, [6] special session
                         % [7] time slot
                         % [8] talk id, [9] session id or photo
 {%\needspace{6\baselineskip}%
  \vskip 0pt\nopagebreak%
%   \colorbox{gray!20!white}{\makebox[0.99\textwidth][r]{}}\nopagebreak%
%   \ifthenelse{\equal{#9}{photo}}{%
%                     \\\\\colorbox{gray!20!white}{\makebox{\includegraphics[width=3cm]{#8}}}\nopagebreak}{}%
 \vskip 0pt\nopagebreak%
%  \label{#8}%
  \textbf{#1}\vspace{3mm}\\\nopagebreak%
  \textit{#2}\\\nopagebreak%
  #3\\\nopagebreak%
  \url{#4}\vspace{3mm}\\\nopagebreak%
  \ifthenelse{\equal{#5}{}}{}{Coauthor(s): #5\vspace{3mm}\\\nopagebreak}%
  \ifthenelse{\equal{#6}{}}{}{Special session: #6\quad \vspace{3mm}\\\nopagebreak}%
 }
 {\vspace{1cm}\nopagebreak}%

\pagestyle{empty}

% ------------------------------------------------------------------------
% Document begins here
% ------------------------------------------------------------------------
\begin{document}
	
\begin{talk}
  {Respecting the boundaries: Space-filling designs for surrogate modeling with boundary information}% [1] talk title
  {Simon Mak}% [2] speaker name
  {Duke University}% [3] affiliations
  {sm769@duke.edu}% [4] email
  {Yen-Chun Liu}% [5] coauthors
  {}% [6] special session. Leave this field empty for contributed talks. 
				% Insert the title of the special session if you were invited to give a talk in a special session.
			
Gaussian process (GP) surrogate models are widely used for emulating expensive computer simulators, and have led to important advances in science and engineering. One challenge with fitting such surrogates is the costly generation of training data, which can require thousands of CPU hours per run. Recent promising work has investigated the integration of known boundary information for surrogate modeling, which can greatly reduce its required training sample size and thus computational cost. There is, however, little work exploring the important question of how such experiments should be designed given boundary information. We propose here a new class of space-filling designs, called boundary maximin designs, for effective GP surrogates with boundary information. Our designs rely on a new space-filling criterion derived from the asymptotic D-optimal designs of the boundary GPs of Vernon et al. (2019) and Ding et al. (2019), which can incorporate a broad class of known boundaries, including axis-parallel and/or perpendicular boundaries. To account for effect sparsity given many input parameters, we further propose a new boundary maximum projection design that jointly factors in boundary information and ensures good projective properties. Numerical experiments and an application in particle physics demonstrate improved surrogate performance with the proposed boundary maximin designs over the state-of-the-art.
\medskip

% If you would like to include references, please do so by creating a simple list numbered by [1], [2], [3], \ldots. See example below.
% Please do not use the \texttt{bibliography} environment or \texttt{bibtex} files.
% APA reference style is recommended.
% \begin{enumerate}
% 	\item[{[1]}] Niederreiter, Harald (1992). {\it Random number generation and quasi-Monte Carlo methods}. Society for Industrial and Applied Mathematics (SIAM).
% 	\item[{[2]}] Roberts, Gareth O, \& Rosenthal, Jeffrey S. (2002).  Optimal scaling for various Metropolis-Hastings algorithms, \textbf{16}(4), 351--367.
% \end{enumerate}

% Equations may be used if they are referenced. Please note that the equation numbers may be different (but will be cross-referenced correctly) in the final program book.
\end{talk}

\end{document}

