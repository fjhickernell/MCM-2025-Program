\documentclass[12pt,a4paper,figuresright]{book}

\usepackage{amsmath,amssymb}
\usepackage{tabularx,graphicx,url,xcolor,rotating,multicol,epsfig,colortbl}

\setlength{\textheight}{25.2cm}
\setlength{\textwidth}{16.5cm} %\setlength{\textwidth}{18.2cm}
\setlength{\voffset}{-1.6cm}
\setlength{\hoffset}{-0.3cm} %\setlength{\hoffset}{-1.2cm}
\setlength{\evensidemargin}{-0.3cm} 
\setlength{\oddsidemargin}{0.3cm}
\setlength{\parindent}{0cm} 
\setlength{\parskip}{0.3cm}

% -- adding a talk
\newenvironment{talk}[6]% [1] talk title
                         % [2] speaker name, [3] affiliations, [4] email,
                         % [5] coauthors, [6] special session
                         % [7] time slot
                         % [8] talk id, [9] session id or photo
 {%\needspace{6\baselineskip}%
  \vskip 0pt\nopagebreak%
%   \colorbox{gray!20!white}{\makebox[0.99\textwidth][r]{}}\nopagebreak%
%   \ifthenelse{\equal{#9}{photo}}{%
%                     \\\\\colorbox{gray!20!white}{\makebox{\includegraphics[width=3cm]{#8}}}\nopagebreak}{}%
 \vskip 0pt\nopagebreak%
%  \label{#8}%
  \textbf{#1}\vspace{3mm}\\\nopagebreak%
  \textit{#2}\\\nopagebreak%
  #3\\\nopagebreak%
  \url{#4}\vspace{3mm}\\\nopagebreak%
  \ifthenelse{\equal{#5}{}}{}{Coauthor(s): #5\vspace{3mm}\\\nopagebreak}%
  \ifthenelse{\equal{#6}{}}{}{Special session: #6\quad \vspace{3mm}\\\nopagebreak}%
 }
 {\vspace{1cm}\nopagebreak}%

\pagestyle{empty}

% ------------------------------------------------------------------------
% Document begins here
% ------------------------------------------------------------------------
\begin{document}
	
\begin{talk}
  {Multiple Importance Sampling for Rare Event Simulation in Communication Systems}% [1] talk title
  {V\'ictor Elvira}% [2] speaker name
  {University of Edinburgh}% [3] affiliations
  {victor.elvira@ed.ac.uk}% [4] email
  {Igancio Santamar\'ia}% [5] coauthors
  {}% [6] special session. Leave this field empty for contributed talks. 
				% Insert the title of the special session if you were invited to give a talk in a special session.
			
Estimating the symbol error rate (SER) in multiple-input multiple-output (MIMO) detectors involves computing an integral in a high-dimensional space, typically outside the Voronoi region of the transmitted symbol. Since a closed-form solution does not exist, the standard approach relies on crude Monte Carlo (MC) simulations, which become highly inefficient in rare-event regimes.  In this talk, we propose a multiple importance sampling (MIS) method that provides unbiased SER estimates with significantly reduced variance compared to naive MC. The method constructs an adaptive mixture proposal distribution, where the number of components, their parameters, and weights are selected automatically. This flexibility ensures robustness across a range of scenarios, making the method efficient, easy to implement, and theoretically sound. Our simulations demonstrate that SERs as low as $10^{-8}$ can be accurately estimated with just $10^4$ samples, achieving multiple orders of magnitude improvement over standard MC.
\medskip

%If you would like to include references, please do so by creating a simple list numbered by [1], [2], [3], \ldots. See example below.
%Please do not use the \texttt{bibliography} environment or \texttt{bibtex} files.
%APA reference style is recommended.
%\begin{enumerate}
%	\item[{[1]}] Niederreiter, Harald (1992). {\it Random number generation and quasi-Monte Carlo methods}. Society for Industrial and Applied Mathematics (SIAM).
%	\item[{[2]}] Roberts, Gareth O, \& Rosenthal, Jeffrey S. (2002).  Optimal scaling for various Metropolis-Hastings algorithms, \textbf{16}(4), 351--367.
%\end{enumerate}

%Equations may be used if they are referenced. Please note that the equation numbers may be different (but will be cross-referenced correctly) in the final program book.
\end{talk}

\end{document}

