\documentclass[12pt,a4paper,figuresright]{book}

\usepackage{amsmath,amssymb}
\usepackage{tabularx,graphicx,url,xcolor,rotating,multicol,epsfig,colortbl}

\setlength{\textheight}{25.2cm}
\setlength{\textwidth}{16.5cm} %\setlength{\textwidth}{18.2cm}
\setlength{\voffset}{-1.6cm}
\setlength{\hoffset}{-0.3cm} %\setlength{\hoffset}{-1.2cm}
\setlength{\evensidemargin}{-0.3cm} 
\setlength{\oddsidemargin}{0.3cm}
\setlength{\parindent}{0cm} 
\setlength{\parskip}{0.3cm}

% -- adding a talk
\newenvironment{talk}[6]% [1] talk title
                         % [2] speaker name, [3] affiliations, [4] email,
                         % [5] coauthors, [6] special session
                         % [7] time slot
                         % [8] talk id, [9] session id or photo
 {%\needspace{6\baselineskip}%
  \vskip 0pt\nopagebreak%
%   \colorbox{gray!20!white}{\makebox[0.99\textwidth][r]{}}\nopagebreak%
%   \ifthenelse{\equal{#9}{photo}}{%
%                     \\\\\colorbox{gray!20!white}{\makebox{\includegraphics[width=3cm]{#8}}}\nopagebreak}{}%
 \vskip 0pt\nopagebreak%
%  \label{#8}%
  \textbf{#1}\vspace{3mm}\\\nopagebreak%
  \textit{#2}\\\nopagebreak%
  #3\\\nopagebreak%
  \url{#4}\vspace{3mm}\\\nopagebreak%
  \ifthenelse{\equal{#5}{}}{}{Coauthor(s): #5\vspace{3mm}\\\nopagebreak}%
  \ifthenelse{\equal{#6}{}}{}{Special session: #6\quad \vspace{3mm}\\\nopagebreak}%
 }
 {\vspace{1cm}\nopagebreak}%

\pagestyle{empty}

% ------------------------------------------------------------------------
% Document begins here
% ------------------------------------------------------------------------
\begin{document}
	
\begin{talk}
  {Empirical Statistical Comparative Analysis of SNP Heritability Estimators and Gradient Boosting Machines (GBM) Using Genetic Data from the UK Biobank}% [1] talk title
  {Kazeem Adeleke$^{*1}$} % [2] speaker name
  {University of the West of England, UK}% [3] affiliations
  {adedayo.adeleke@uwe.ac.uk} % [4] email
  { Peter Ogunyinka$^{2}$, Emmanuel Ologunleko$^{3}$ and Dawud Agunbiade$^{4}$} % [5] coauthors
  {}% [6] special session. Leave this field empty for contributed talks. 
				% Insert the title of the special session if you were invited to give a talk in a special session.
			
This study addresses the methodological challenges in estimating genetic heritability by comparing traditional statistical approaches with advanced machine learning techniques. We evaluated three distinct methods: sibling regression, LD-score regression, and Gradient Boosting Machines (GBMs), using both simulated datasets and real-world data from the UK Biobank. Our methodology involved generating simulated genotypes following Mendelian inheritance patterns and creating corresponding phenotypes incorporating family-specific genetic effect sizes. We conducted Genome-Wide Association Studies (GWAS) on firstborn children from each family and performed comprehensive heritability analyses using all three methods. Results demonstrated that while sibling regression effectively captured within-family genetic similarities and LD-score regression accounted for population-wide linkage disequilibrium patterns, GBMs showed superior capability in predicting phenotypes by capturing complex genetic interactions. The integration of GBMs with traditional methods revealed enhanced predictive power and provided new insights into the genetic architecture of complex traits. Our findings emphasize the value of combining conventional statistical approaches with machine learning techniques for more robust heritability estimation in large-scale UK Biobank studies.

\medskip

%If you would like to include references, please do so by creating a simple list numbered by [1], [2], [3], \ldots. See example below.
%Please do not use the \texttt{bibliography} environment or \texttt{bibtex} files.
%APA reference style is recommended.
%\begin{enumerate}
%	\item[{[1]}] Niederreiter, Harald (1992). {\it Random number generation and quasi-Monte Carlo methods}. Society for Industrial and Applied Mathematics (SIAM).
%	\item[{[2]}] Roberts, Gareth O, \& Rosenthal, Jeffrey S. (2002).  Optimal scaling for various Metropolis-Hastings algorithms, \textbf{16}(4), 351--367.
%\end{enumerate}

%Equations may be used if they are referenced. Please note that the equation numbers may be different (but will be cross-referenced correctly) in the final program book.
\end{talk}

\end{document}

