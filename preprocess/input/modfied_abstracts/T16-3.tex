\documentclass[12pt,a4paper,figuresright]{book}

\usepackage{amsmath,amssymb}
\usepackage{tabularx,graphicx,url,xcolor,rotating,multicol,epsfig,colortbl}

\setlength{\textheight}{25.2cm}
\setlength{\textwidth}{16.5cm} %\setlength{\textwidth}{18.2cm}
\setlength{\voffset}{-1.6cm}
\setlength{\hoffset}{-0.3cm} %\setlength{\hoffset}{-1.2cm}
\setlength{\evensidemargin}{-0.3cm} 
\setlength{\oddsidemargin}{0.3cm}
\setlength{\parindent}{0cm} 
\setlength{\parskip}{0.3cm}

% -- adding a talk
\newenvironment{talk}[6]% [1] talk title
                         % [2] speaker name, [3] affiliations, [4] email,
                         % [5] coauthors, [6] special session
                         % [7] time slot
                         % [8] talk id, [9] session id or photo
 {%\needspace{6\baselineskip}%
  \vskip 0pt\nopagebreak%
%   \colorbox{gray!20!white}{\makebox[0.99\textwidth][r]{}}\nopagebreak%
%   \ifthenelse{\equal{#9}{photo}}{%
%                     \\\\\colorbox{gray!20!white}{\makebox{\includegraphics[width=3cm]{#8}}}\nopagebreak}{}%
 \vskip 0pt\nopagebreak%
%  \label{#8}%
  \textbf{#1}\vspace{3mm}\\\nopagebreak%
  \textit{#2}\\\nopagebreak%
  #3\\\nopagebreak%
  \url{#4}\vspace{3mm}\\\nopagebreak%
  \ifthenelse{\equal{#5}{}}{}{Coauthor(s): #5\vspace{3mm}\\\nopagebreak}%
  \ifthenelse{\equal{#6}{}}{}{Special session: #6\quad \vspace{3mm}\\\nopagebreak}%
 }
 {\vspace{1cm}\nopagebreak}%

\pagestyle{empty}

% ------------------------------------------------------------------------
% Document begins here
% ------------------------------------------------------------------------
\begin{document}
	
\begin{talk}
  {Moving PCG beyond LCGs}% [1] talk title
  {Christopher Draper}% [2] speaker name
  {Florida State University}% [3] affiliations
  {chd16@fsu.edu}% [4] email
  { Michael Mascagni}% [5] coauthors
  {}% [6] special session. Leave this field empty for contributed talks. 
				% Insert the title of the special session if you were invited to give a talk in a special session.
			
PCG is a set of generators released by Melissa E. O’Neill in 2014 [1]. The original technical report outlined a number of lightweight
scrambling techniques. Each scrambling technique offered some improvement to the quality of the linear congruential generators
they were designed for. However the real strength of the scrambling techniques was that they could easily be combined in different
combinations to offer much stronger improvements. The PCG technical report concludes with the creation of the PCG library, a
popular PRNG library that implements a number of generators described in the technical report. Starting from the observation that the
PCG work was narrowly focused on implementing their scrambling techniques for specific linear congruential generators, we explore
the PCG scrambling techniques and their potential application for being applied to other PRNGs. We show the steps taken to generalize the PCG
scrambling techniques to work with any arbitrary amount of bits and parameter values. Then test the PCG scrambling techniques
across different linear congruential generators and then test the PCG scrambling techniques across a number of different PRNGs.
\medskip

\begin{enumerate}
	\item[{[1]}] Melissa E. O’Neill. 2014. PCG: A Family of Simple Fast Space-Efficient Statistically Good Algorithms for Random Number Generation. Technical Report HMC-CS-2014-0905. Harvey Mudd College, Claremont, CA.
\end{enumerate}

\end{talk}

\end{document}

