\documentclass[12pt,a4paper,figuresright]{book}

\usepackage{amsmath,amssymb}
\usepackage{tabularx,graphicx,url,xcolor,rotating,multicol,epsfig,colortbl}

\setlength{\textheight}{25.2cm}
\setlength{\textwidth}{16.5cm} %\setlength{\textwidth}{18.2cm}
\setlength{\voffset}{-1.6cm}
\setlength{\hoffset}{-0.3cm} %\setlength{\hoffset}{-1.2cm}
\setlength{\evensidemargin}{-0.3cm}
\setlength{\oddsidemargin}{0.3cm}
\setlength{\parindent}{0cm}
\setlength{\parskip}{0.3cm}

% -- adding a talk
\newenvironment{talk}[6]% [1] talk title
                         % [2] speaker name, [3] affiliations, [4] email,
                         % [5] coauthors, [6] special session
                         % [7] time slot
                         % [8] talk id, [9] session id or photo
 {%\needspace{6\baselineskip}%
  \vskip 0pt\nopagebreak%
%   \colorbox{gray!20!white}{\makebox[0.99\textwidth][r]{}}\nopagebreak%
%   \ifthenelse{\equal{#9}{photo}}{%
%                     \\\\\colorbox{gray!20!white}{\makebox{\includegraphics[width=3cm]{#8}}}\nopagebreak}{}%
 \vskip 0pt\nopagebreak%
%  \label{#8}%
  \textbf{#1}\vspace{3mm}\\\nopagebreak%
  \textit{#2}\\\nopagebreak%
  #3\\\nopagebreak%
  \url{#4}\vspace{3mm}\\\nopagebreak%
  \ifthenelse{\equal{#5}{}}{}{Coauthor(s): #5\vspace{3mm}\\\nopagebreak}%
  \ifthenelse{\equal{#6}{}}{}{Special session: #6\quad \vspace{3mm}\\\nopagebreak}%
 }
 {\vspace{1cm}\nopagebreak}%

\pagestyle{empty}

% ------------------------------------------------------------------------
% Document begins here
% ------------------------------------------------------------------------
\begin{document}

\begin{talk}
  {An Adaptive Sampling Algorithm for Level-set Approximation}% [1] talk title
  {Abdul-Latefe Haji-Ali}% [2] speaker name
  {Heriot-Watt University}% [3] affiliations
  {a.hajiali@hw.ac.uk}% [4] email
  {Matteo Croci and Ian CJ Powell}% [5] coauthors
  {}% [6] special session. Leave this field empty for contributed talks.
				% Insert the title of the special session if you were invited to give a talk in a special session.

  Let $D\subset\joinrel\subset\rm{I\!R}$$^d$ be a $d$-dimensional domain with compact
closure. We consider the approximation of the $d-1$ dimensional zero level-set
$\mathcal{L}_0 := \{x \in \overline{D} : f(x) = 0\}$ where the Lipschitz function $f$ is either
accessible directly or when $f(x) = \mathbb{E}$$\left[\tilde{f}(x)\right]$ for
all $x \in \overline{D}$. Given an approximation scheme with a priori error bounds
and $L^p$ bounds on the $\tilde{f}$-approximation error, we propose a
grid-based adaptive sampling scheme which produces an approximation to $\mathcal{L}_0$
with expected cost-complexity reduction of $\varepsilon^{\left(\frac{p+1}{\alpha p}\right)}$
compared to a non-adaptive scheme, where $\alpha$ is the known convergence rate of
an interpolation scheme. We provide the numerical analysis and experiments to
show that these rates hold in practice.

\medskip

% If you would like to include references, please do so by creating a simple list numbered by [1], [2], [3], \ldots. See example below.
% Please do not use the \texttt{bibliography} environment or \texttt{bibtex} files.
% APA reference style is recommended.
% \begin{enumerate}
% 	\item[{[1]}] Niederreiter, Harald (1992). {\it Random number generation and quasi-Monte Carlo methods}. Society for Industrial and Applied Mathematics (SIAM).
% 	\item[{[2]}] Roberts, Gareth O, \& Rosenthal, Jeffrey S. (2002).  Optimal scaling for various Metropolis-Hastings algorithms, \textbf{16}(4), 351--367.
% \end{enumerate}

% Equations may be used if they are referenced. Please note that the equation numbers may be different (but will be cross-referenced correctly) in the final program book.
\end{talk}

\end{document}
