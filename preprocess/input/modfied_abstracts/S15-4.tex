\documentclass[12pt,a4paper,figuresright]{book}

\usepackage{amsmath,amssymb}
\usepackage{tabularx,graphicx,url,xcolor,rotating,multicol,epsfig,colortbl}

\setlength{\textheight}{25.2cm}
\setlength{\textwidth}{16.5cm} %\setlength{\textwidth}{18.2cm}
\setlength{\voffset}{-1.6cm}
\setlength{\hoffset}{-0.3cm} %\setlength{\hoffset}{-1.2cm}
\setlength{\evensidemargin}{-0.3cm} 
\setlength{\oddsidemargin}{0.3cm}
\setlength{\parindent}{0cm} 
\setlength{\parskip}{0.3cm}

% -- adding a talk
\newenvironment{talk}[6]% [1] talk title
                         % [2] speaker name, [3] affiliations, [4] email,
                         % [5] coauthors, [6] special session
                         % [7] time slot
                         % [8] talk id, [9] session id or photo
 {%\needspace{6\baselineskip}%
  \vskip 0pt\nopagebreak%
%   \colorbox{gray!20!white}{\makebox[0.99\textwidth][r]{}}\nopagebreak%
%   \ifthenelse{\equal{#9}{photo}}{%
%                     \\\\\colorbox{gray!20!white}{\makebox{\includegraphics[width=3cm]{#8}}}\nopagebreak}{}%
 \vskip 0pt\nopagebreak%
%  \label{#8}%
  \textbf{#1}\vspace{3mm}\\\nopagebreak%
  \textit{#2}\\\nopagebreak%
  #3\\\nopagebreak%
  \url{#4}\vspace{3mm}\\\nopagebreak%
  \ifthenelse{\equal{#5}{}}{}{Coauthor(s): #5\vspace{3mm}\\\nopagebreak}%
  \ifthenelse{\equal{#6}{}}{}{Special session: #6\quad \vspace{3mm}\\\nopagebreak}%
 }
 {\vspace{1cm}\nopagebreak}%

\pagestyle{empty}

% ------------------------------------------------------------------------
% Document begins here
% ------------------------------------------------------------------------
\begin{document}
	
\begin{talk}
  {A hit and run approach for sampling and analyzing ranking models}% [1] talk title
  {Chenyang Zhong}% [2] speaker name
  {Department of Statistics, Columbia University}% [3] affiliations
  {cz2755@columbia.edu}% [4] email
  {}% [5] coauthors
  {Frontiers in (Quasi-)Monte Carlo and Markov Chain Monte Carlo Methods}% [6] special session. Leave this field empty for contributed talks. 
				% Insert the title of the special session if you were invited to give a talk in a special session.
			
%Your abstract goes here. Please do not use your own commands or macros.
The analysis of ranking data has gained much recent interest across various applications, including recommender systems, market research, and electoral studies. This talk focuses on the Mallows permutation model, a probabilistic model for ranking data introduced by C. L. Mallows. The Mallows model specifies a family of non-uniform probability distributions on permutations and is characterized by a distance metric on permutations. We focus on two popular choices: the $L^1$ (Spearman’s footrule) and $L^2$ (Spearman’s rank correlation) distances. Despite their widespread use in statistics and machine learning, Mallows models with these metrics present significant computational challenges due to the intractability of their normalizing constants.

Hit and run algorithms form a broad class of MCMC algorithms, including Swendsen-Wang and data augmentation. In this talk, I will first explain how to sample from Mallows models using hit and run algorithms. For both models, we establish $O(\log n)$ mixing time upper bounds, which provide the first theoretical guarantees for efficient sampling and enable computationally feasible Monte Carlo maximum likelihood estimation. Then, I will also discuss how the hit and run algorithms can be utilized to prove theorems about probabilistic properties of the Mallows models.

\medskip

%If you would like to include references, please do so by creating a simple list numbered by [1], [2], [3], \ldots. See example below.
%Please do not use the \texttt{bibliography} environment or \texttt{bibtex} files.
%APA reference style is recommended.
%\begin{enumerate}
%	\item[{[1]}] Niederreiter, Harald (1992). {\it Random number generation and quasi-Monte Carlo methods}. Society for Industrial and Applied Mathematics (SIAM).
%	\item[{[2]}] Roberts, Gareth O, \& Rosenthal, Jeffrey S. (2002).  Optimal scaling for various Metropolis-Hastings algorithms, \textbf{16}(4), 351--367.
%\end{enumerate}

%Equations may be used if they are referenced. Please note that the equation numbers may be different (but will be cross-referenced correctly) in the final program book.
\end{talk}

\end{document}

