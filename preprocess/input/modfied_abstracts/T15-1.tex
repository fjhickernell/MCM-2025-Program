\documentclass[12pt,a4paper,figuresright]{book}

\usepackage{amsmath,amssymb}
\usepackage{tabularx,graphicx,url,xcolor,rotating,multicol,epsfig,colortbl}

\setlength{\textheight}{25.2cm}
\setlength{\textwidth}{16.5cm} %\setlength{\textwidth}{18.2cm}
\setlength{\voffset}{-1.6cm}
\setlength{\hoffset}{-0.3cm} %\setlength{\hoffset}{-1.2cm}
\setlength{\evensidemargin}{-0.3cm} 
\setlength{\oddsidemargin}{0.3cm}
\setlength{\parindent}{0cm} 
\setlength{\parskip}{0.3cm}
\usepackage{bm}

% -- adding a talk
\newenvironment{talk}[6]% [1] talk title
                         % [2] speaker name, [3] affiliations, [4] email,
                         % [5] coauthors, [6] special session
                         % [7] time slot
                         % [8] talk id, [9] session id or photo
 {%\needspace{6\baselineskip}%
  \vskip 0pt\nopagebreak%
%   \colorbox{gray!20!white}{\makebox[0.99\textwidth][r]{}}\nopagebreak%
%   \ifthenelse{\equal{#9}{photo}}{%
%                     \\\\\colorbox{gray!20!white}{\makebox{\includegraphics[width=3cm]{#8}}}\nopagebreak}{}%
 \vskip 0pt\nopagebreak%
%  \label{#8}%
  \textbf{#1}\vspace{3mm}\\\nopagebreak%
  \textit{#2}\\\nopagebreak%
  #3\\\nopagebreak%
  \url{#4}\vspace{3mm}\\\nopagebreak%
  \ifthenelse{\equal{#5}{}}{}{Coauthor(s): #5\vspace{3mm}\\\nopagebreak}%
  \ifthenelse{\equal{#6}{}}{}{Special session: #6\quad \vspace{3mm}\\\nopagebreak}%
 }
 {\vspace{1cm}\nopagebreak}%

\pagestyle{empty}

% ------------------------------------------------------------------------
% Document begins here
% ------------------------------------------------------------------------
\begin{document}
	
\begin{talk}
  {Combining quasi-Monte Carlo with Stochastic Optimal Control for Trajectory Optimization of Autonomous Vehicles in Mine Counter Measure Simulations}% [1] talk title
  {Philippe Blondeel}% [2] speaker name
  {Belgian Royal Military Academy}% [3] affiliations
  {Philippe.blondeel@mil.be}% [4] email
  {Filip Van Utterbeeck, Ben Lauwens}% [5] coauthors
  {}% [6] special session. Leave this field empty for contributed talks. 
				% Insert the title of the special session if you were invited to give a talk in a special session.
			
Modelling and simulating mine countermeasures (MCM) search missions performed by autonomous vehicles is a challenging endeavour. The goal of these simulations typically consists of calculating trajectories for autonomous vehicles in a designated zone such that the coverage (residual risk) of the zone is below a certain user defined threshold. We have chosen to model and implement the MCM problem as a stochastic optimal control problem, see [1]. Mathematically, the MCM problem is defined as minimizing the total mission time needed to survey a designated zone $\Omega$ for a given residual risk of not detecting sea mines in the  user-chosen square  domain, i.e., 
\begin{equation}
\text{min}\, T_f,
\label{eq:min}
\end{equation}
subjected to
\begin{equation}
 \mathbb{E}[q\left(T_F\right)] :=  \int_\Omega \text{e}^{-\int_0^{T_F} \gamma\left(\bm{x}\left(\tau\right),\bm{\omega}\right)\, d\,\tau}\phi\left(\bm{\omega}\right) d\,\bm{\omega} \leq \text{Residual Risk}.
\label{eq:exp}
\end{equation}
The output of our stochastic optimal control implementation consists of an optimal trajectory in the square domain for the autonomous vehicle.  As shown in Eq.\,\eqref{eq:exp}, the residual risk is mathematically represented as an expected value integral. In [2], we presented a novel relaxation strategy for the computation of the residual MCM risk, used in our stochastic optimal control formulation. This novel relaxation strategy ensures that the  residual risk obtained at the end of the optimisation run is below the maximally allowed user requested residual risk. This was however not the case  with our initial `naive' implementation of the MCM problem. Our proposed relaxation strategy ensures that the user requested risk is satisfied by sequentially solving the stochastic optimal control problem with an ever increasing size of the domain. We combine this strategy with  a quasi-Monte Carlo  sampling scheme based on a Rank-1 Lattice rule for the computation of the expected value integral. We observe a speedup up to a factor two in terms of total computational cost in favour of quasi-Monte Carlo when compared to standard Monte Carlo.


%In order to compute a solution for this expected value integral, we use on the one hand   , and on the other hand we use  the traditional Monte Carlo (MC) sampling scheme. However,  In order to remedy to this issue, we developed a multi-domain relaxation strategy. The main idea of our multi-domain relaxation strategy consists of sequentially solving the stochastic optimal control problem with an ever increasing size of the domain until the requested risk is satisfied. The qMC or MC points we use are generated once at the start of the simulation on the unit cube domain, after which they are mapped to the desired domains.   
\medskip
\begin{enumerate}
	\item[{[1]}] Blondeel, P., Van Utterbeeck, F., Lauwens, B. (2024). Modeling sand ripples in mine countermeasure simulations by means of stochastic optimal control. In: The 19th European Congress on Computational Methods in Applied Sciences and Engineering, ECCOMAS, Lisbon, Portugal (2024).
	\item[{[2]}] Blondeel, P., Van Utterbeeck, F., Lauwens, B. (2025).  Application of quasi-Monte Carlo in Mine Countermeasure Simulations with a Stochastic Optimal Control Framework, \textit{arXiv preprint}
\end{enumerate}

\end{talk}

\end{document}

