\documentclass[12pt,a4paper,figuresright]{book}

\usepackage{amsmath,amssymb}
\usepackage{tabularx,multirow,graphicx,url,wrapfig,xcolor,rotating,multicol,epsfig,colortbl}
\usepackage{hyperref}
\usepackage{color}

%------ for easier commenting; remove later -----
\usepackage[textsize=tiny]{todonotes}
\setlength{\marginparwidth}{1.7cm}

\definecolor{ff}{rgb}{0,0,1}
\newcommand{\ff}[1]{{\color{ff}{[FF: #1]}}}
\newcommand{\rrtd}[2][]{\todo[color=ff!40,#1]{FF: #2}}

\definecolor{xh}{rgb}{1,0.53,0.0}
\newcommand{\xh}[1]{{\color{xh}{[XH: #1]}}}
\newcommand{\xhtd}[2][]{\todo[color=xh!40,#1]{XH: #2}}

\definecolor{ymm}{rgb}{0.1,0.45,0.1}
\newcommand{\ym}[1]{{\color{ymm}{[YMM: #1]}}}
\newcommand{\ymtd}[2][]{\todo[color=ymm!40,#1]{YMM: #2}}

%---------------------------------------------

\setlength{\textheight}{25.2cm}
\setlength{\textwidth}{16.5cm} %\setlength{\textwidth}{18.2cm}
\setlength{\voffset}{-1.6cm}
\setlength{\hoffset}{-0.3cm} %\setlength{\hoffset}{-1.2cm}
\setlength{\evensidemargin}{-0.3cm} 
\setlength{\oddsidemargin}{0.3cm}
\setlength{\parindent}{0cm} 
\setlength{\parskip}{0.3cm}

\renewcommand{\topfraction}{1}
\renewcommand{\textfraction}{0}
\setlength{\floatsep}{12pt plus 2pt minus 2pt}

\newcommand{\organizer}[3]{%
	{\textit{#1}}\\\nopagebreak%
	#2\\\nopagebreak%
	\url{#3}\vspace{3mm}\\\nopagebreak%
	}

\newenvironment{session}[5] % [1] session title
							% [2] number of organizers
                            % [3] organizer 1 info
                            % [4] organizer 2 info
                            % [5] organizer 3 info
                            % [6] session id for later
 {%\needspace{6\baselineskip}
  \vskip 0pt\nopagebreak%
  %\label{#5}%
  \textbf{#1}\vspace{3mm}\\\nopagebreak%
  \ifthenelse{\equal{#2}{1}}{Organizer:}{Organizers:}%
  \vspace{2mm}\\\nopagebreak%
  #3
  \ifthenelse{\equal{#2}{2}}{#4}{}%
  \ifthenelse{\equal{#2}{3}}{#4#5}{}%
  \quad\\\nopagebreak%
  %Session Description:\vspace{3mm}\\\nopagebreak%
 }
 {\nopagebreak}%

\pagestyle{empty}

% ------------------------------------------------------------------------
% Document begins here
% ------------------------------------------------------------------------
\begin{document}
	
%Input the relevant information below
\begin{session}
  {Next-generation optimal experimental design: theory, scalability, and real world impact: Part II}% [1] session title
  {3} %[2]  number of organizers
  {\organizer{Florence Forbes}% organizer one name
    {Inria, France}% orgnizer one affiliations
    {florence.forbes@inria.fr}}% organizer one email
  {\organizer{Xun Huan}% organizer two name, if needed
	{University of Michigan, USA}% orgnizer two affiliations, if needed
	{xhuan@umich.edu}}% organizer two email
  {\organizer{Youssef Marzouk}% organizer one name
	{Massachusetts Institute of Technology, USA}% orgnizer one affiliations
	{ymarz@mit.edu}}% organizer one email


Optimal experimental design (OED) provides a mathematical framework for identifying candidate data or experimental configurations that are most useful for inference, prediction, or some other downstream goal. Though OED is hardly a new topic,\footnote{At the first session of the Indian Statistical Conference in 1938, R.\ Fisher supposedly said, ``To call in the statistician after the experiment is done may be no more than asking him to perform a postmortem examination: he may be able to say what the experiment died of.''} the need for advances in OED has never been greater than it is today. Myriad application areas have witnessed, on the one hand, an explosion in the volume of data that can be acquired, and on the other, the use of increasingly complex and computationally intensive models to interpret these data. Yet large volumes of data do not by default carry a commensurate amount of information. Moreover, we inevitably face constraints: on the costs of experimentation or data acquisition, on data storage and communication, and on the computational effort of statistical inference in complex models. OED directly addresses the associated trade-offs---e.g., between experimentation, measurement, and/or processing \textit{costs} and 
the \textit{quality} of subsequent and decision making. See {\it e.g.} [1] for a recent review of OED topics, which provides numerous other references.

This session will highlight computational developments at the OED research frontier. Methods for stochastic simulation, high-dimensional approximation or integration, and stochastic optimization are central to modern OED and to the scaling of OED to large parameter spaces and complex statistical models. Modern machine learning methodologies---from neural network surrogates, to deep reinforcement learning in sequential OED, to modern generative models and transport methods for simulation-based inference---also play a catalyzing role in such OED approaches. Talks in this session will illuminate these emerging interactions and their role in realizing Bayesian, decision-theoretic, and information-theoretic formulations of OED for truly complex problems. Session speakers will also discuss ongoing work to develop theoretical guarantees for these new OED methodologies, and showcase applications to real-world problems ranging from sensor steering to seismology.

\medskip

\begin{enumerate}
	\item[{[1]}] X.\ Huan, J.\ Jagalur, Y.\ Marzouk (2024). Optimal experimental design: Formulations and computations, \textit{Acta Numerica} \textbf{33}, 715--840.
	%\item[{[2]}] T. Rainforth, A. Foster, D.R. Ivanova, F. Bickford Smith. Modern Bayesian experimental design, Statistical Science 39 (1), 100-114, 2024.
\end{enumerate}


\textbf{Speakers and their affiliations}

\begin{itemize}
\item Alen Alexanderian,  aalexan3@ncsu.edu, North Carolina State University, USA

\item Jacopo Iollo, jacopo.iollo@inria.fr, INRIA, Universit\'e Grenoble Alpes, France

\item Tommie Catanach, tacatan@sandia.gov , Sandia National Laboratories, USA

\end{itemize}

\end{session}

\end{document}