\documentclass[12pt,a4paper,figuresright]{book}

\usepackage{amsmath,amssymb}
\usepackage{tabularx,multirow,graphicx,url,wrapfig,xcolor,rotating,multicol,epsfig,colortbl,verbatim}

\setlength{\textheight}{25.2cm}
\setlength{\textwidth}{16.5cm} %\setlength{\textwidth}{18.2cm}
\setlength{\voffset}{-1.6cm}
\setlength{\hoffset}{-0.3cm} %\setlength{\hoffset}{-1.2cm}
\setlength{\evensidemargin}{-0.3cm}
\setlength{\oddsidemargin}{0.3cm}
\setlength{\parindent}{0cm}
\setlength{\parskip}{0.3cm}

\renewcommand{\topfraction}{1}
\renewcommand{\textfraction}{0}
\setlength{\floatsep}{12pt plus 2pt minus 2pt}

\newcommand{\organizer}[3]{%
	{\textit{#1}}\\\nopagebreak%
	#2\\\nopagebreak%
	\url{#3}\vspace{3mm}\\\nopagebreak%
	}

\newenvironment{session}[5] % [1] session title
							% [2] number of organizers
                            % [3] organizer 1 info
                            % [4] organizer 2 info
                            % [5] organizer 3 info
                            % [6] session id for later
 {%\needspace{6\baselineskip}
  \vskip 0pt\nopagebreak%
  %\label{#5}%
  \textbf{#1}\vspace{3mm}\\\nopagebreak%
  \ifthenelse{\equal{#2}{1}}{Organizer:}{Organizers:}%
  \vspace{2mm}\\\nopagebreak%
  #3
  \ifthenelse{\equal{#2}{2}}{#4}{}%
  \ifthenelse{\equal{#2}{3}}{#4#5}{}%
  \quad\\\nopagebreak%
  %Session Description:\vspace{3mm}\\\nopagebreak%
 }
 {\nopagebreak}%


\pagestyle{empty}

% ------------------------------------------------------------------------
% Document begins here
% ------------------------------------------------------------------------
\begin{document}
	
%Input the relevant information below
\begin{session}
  {Analysis of Langevin and Related Sampling Algorithms, Part II}% [1] session title
  {3} %[2]  number of organizers
  {\organizer{Yifan Chen}% organizer one name
    {Courant Institute of Mathematical Sciences, New York University}% orgnizer one affiliations
    {yifan.chen@nyu.edu}}% organizer one email
  {\organizer{Xiaoou Cheng}% organizer two name, if needed
	{Courant Institute of Mathematical Sciences, New York University}% orgnizer two affiliations, if needed
	{chengxo@nyu.edu}}% organizer two email
  {\organizer{Jonathan Weare}% organizer three name
  {Courant Institute of Mathematical Sciences, New York University}% organizer three affiliations
  {weare@nyu.edu}}% organizer three email
 

% Your special session abstract goes here, including the list of speakers and their affiliations. Please do not use your own commands or macros.

% Many Markov Chain Monte Carlo (MCMC) samplers follow certain stochastic dynamics. Unadjust Langevin algorithm provides a conceptual starting point for a giant family of extensions, which include the kinetic/underdamped Langevin, Hamiltonian Monte Carlo, and No-U-Turn Sampler (NUTS).

Many Markov Chain Monte Carlo (MCMC) samplers are based on stochastic dynamics. Langevin dynamics serves as a fundamental basis for a vast family of extensions, such as unadjusted Langevin algorithms, kinetic/underdamped Langevin algorithms, Hamiltonian Monte Carlo, and the No-U-Turn Sampler (NUTS). The gradient flow structure of Langevin dynamics also motivates the development of a large class of novel algorithms such as stein variational gradient descent, birth-death process approaches, and those based on Fisher-Rao gradient flows. These methods have become ubiquitous across various fields, including molecular dynamics, Bayesian statistics, and machine learning. Recent years have seen significant theoretical advances in analyzing such methods, particularly in high-dimensional settings and non-convex cases. This special session aims to bring together researchers from different communities (probability, statistics, scientific computing, theoretical computer science, machine learning, etc.)\ working on analysis of sampling dynamics of Langevin and beyond to present recent progress, discuss challenges, and share ideas. 

\end{session}

%[\textbf{To conference organizers}: One of our speakers in our two-part sessions has other commitment until July 30. Thus, we prefer both of our sessions to be scheduled during July 31 -- Aug 1. July 30 is also acceptable if the later dates are impossible. We really appreciate your help to make our sessions possible.]



% If you would like to include references, please do so by creating a simple list numbered by [1], [2], [3], \ldots. See example below.
% Please do not use the \texttt{bibliography} environment or \texttt{bibtex} files.

% \begin{enumerate}
% 	\item[{[1]}] Niederreiter, Harald (1992). {\it Random number generation and quasi-Monte Carlo methods}. Society for Industrial and Applied Mathematics (SIAM).
% 	\item[{[2]}] L’Ecuyer, Pierre, \& Christiane Lemieux. (2002). Recent advances in randomized quasi-Monte Carlo methods. Modeling uncertainty: An examination of stochastic theory, methods, and applications, 419-474.
% \end{enumerate}

% Equations may be used if they are referenced. Please note that the equation numbers may be different (but will be cross-referenced correctly) in the final program book.
  


\end{document}

