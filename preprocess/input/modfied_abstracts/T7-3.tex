\documentclass[12pt,a4paper,figuresright]{book}

\usepackage{amsmath,amssymb}
\usepackage{tabularx,graphicx,url,xcolor,rotating,multicol,epsfig,colortbl}

\setlength{\textheight}{25.2cm}
\setlength{\textwidth}{16.5cm} %\setlength{\textwidth}{18.2cm}
\setlength{\voffset}{-1.6cm}
\setlength{\hoffset}{-0.3cm} %\setlength{\hoffset}{-1.2cm}
\setlength{\evensidemargin}{-0.3cm} 
\setlength{\oddsidemargin}{0.3cm}
\setlength{\parindent}{0cm} 
\setlength{\parskip}{0.3cm}

% -- adding a talk
\newenvironment{talk}[6]%
 {
  \vskip 0pt\nopagebreak%
 \vskip 0pt\nopagebreak%
  \textbf{#1}\vspace{3mm}\\\nopagebreak%
  \textit{#2}\\\nopagebreak%
  #3\\\nopagebreak%
  \url{#4}\vspace{3mm}\\\nopagebreak%
  \ifthenelse{\equal{#5}{}}{}{Coauthor(s): #5\vspace{3mm}\\\nopagebreak}%
  \ifthenelse{\equal{#6}{}}{}{Special session: #6\quad \vspace{3mm}\\\nopagebreak}%
 }
 {\vspace{1cm}\nopagebreak}%

\pagestyle{empty}

% ------------------------------------------------------------------------
% Document begins here
% ------------------------------------------------------------------------
\begin{document}
	
\begin{talk}
  {Polynomial approximation for efficient transport-based sampling}% [1] talk title
  {Josephine Westermann}% [2] speaker name
  {Heidelberg University}% [3] affiliations
  {josephine.westermann@uni-heidelberg.de}% [4] email
  {}% [5] coauthors
  {}

Sampling from non-trivial probability distributions is a fundamental challenge in uncertainty quantification and inverse problems, particularly when dealing with high-dimensional domains, costly-to-evaluate or unnormalized density functions, and non-trivial support structures. Measure transport via polynomial density surrogates [1] provides a systematic and constructive solution to this problem by reformulating it as a convex optimization task with deterministic error bounds. This approach is particularly efficient for smooth problems but can become computationally demanding when dealing with highly concentrated posterior distributions, as constructing the density surrogate requires many evaluations in regions where the density is nearly zero—leading to an inefficient allocation of computational resources. In this talk, we explore how the computational cost in such cases can be further reduced by first approximating the potential function with a polynomial. This additional approximation step shifts the focus of expensive evaluations toward capturing the underlying system more effectively. The surrogate-based posterior can then be evaluated cheaply and approximated with high accuracy, enabling efficient transport-based sampling. We discuss the implications of this strategy and examine its potential to enhance performance in demanding inference problems.

\medskip

\begin{enumerate}
	\item[{[1]}] Westermann, J., \& Zech, J. (2025). \textit{Measure transport via polynomial density surrogates}. \textit{Foundations of Data Science}. \url{https://doi.org/10.3934/fods.2025001}
\end{enumerate}

\end{talk}

\end{document}

