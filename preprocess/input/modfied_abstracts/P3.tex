\documentclass[12pt,a4paper,figuresright]{book}

\usepackage{amsmath,amssymb}
\usepackage{tabularx,graphicx,url,xcolor,rotating,multicol,epsfig,colortbl}

\setlength{\textheight}{25.2cm}
\setlength{\textwidth}{16.5cm} %\setlength{\textwidth}{18.2cm}
\setlength{\voffset}{-1.6cm}
\setlength{\hoffset}{-0.3cm} %\setlength{\hoffset}{-1.2cm}
\setlength{\evensidemargin}{-0.3cm} 
\setlength{\oddsidemargin}{0.3cm}
\setlength{\parindent}{0cm} 
\setlength{\parskip}{0.3cm}

% -- adding a talk
\newenvironment{talk}[6]% [1] talk title
                         % [2] speaker name, [3] affiliations, [4] email,
                         % [5] coauthors, [6] special session
                         % [7] time slot
                         % [8] talk id, [9] session id or photo
 {%\needspace{6\baselineskip}%
  \vskip 0pt\nopagebreak%
%   \colorbox{gray!20!white}{\makebox[0.99\textwidth][r]{}}\nopagebreak%
%   \ifthenelse{\equal{#9}{photo}}{%
%                     \\\\\colorbox{gray!20!white}{\makebox{\includegraphics[width=3cm]{#8}}}\nopagebreak}{}%
 \vskip 0pt\nopagebreak%
%  \label{#8}%
  \textbf{#1}\vspace{3mm}\\\nopagebreak%
  \textit{#2}\\\nopagebreak%
  #3\\\nopagebreak%
  \url{#4}\vspace{3mm}\\\nopagebreak%
  \ifthenelse{\equal{#5}{}}{}{Coauthor(s): #5\vspace{3mm}\\\nopagebreak}%
  \ifthenelse{\equal{#6}{}}{}{Special session: #6\quad \vspace{3mm}\\\nopagebreak}%
 }
 {\vspace{1cm}\nopagebreak}%

\pagestyle{empty}

% ------------------------------------------------------------------------
% Document begins here
% ------------------------------------------------------------------------
\begin{document}
	
\begin{talk}
  {Combining Simulation and Linear Algebra: COSIMLA}% [1] talk title
  {Peter W. Glynn}% [2] speaker name
  {Stanford University}% [3] affiliations
  {glynn@stanford.edu}% [4] email
  {Zeyu Zheng}% [5] coauthors
  {}% [6] special session. Leave this field empty for contributed talks. 
				% Insert the title of the special session if you were invited to give a talk in a special session.
			
In numerical computation for Markov chains and jump processes, matrix-based linear algebraic methods leverage fully the special structure of such models, allowing one to efficiently compute highly accurate solutions quickly. When the number of states is large or infinite, Monte Carlo simulation is an appealing alternative, typically allowing low accuracy solutions to be computed efficiently. In this talk, we describe COSIMLA, COmbined SIMulations and Linear Algebra. This new class of algorithms combines the best of the two numerical approaches, using matrix methods to compute expectations and probabilities in the truncated core of the state space, while one uses Monte Carlo to simulate path excursions outside the truncation. As a result, one can now compute high accuracy solutions for models with a very large state space. We show how the method applies to computing equilibrium quantities and various transient characteristics of Markov chains. These algorithms can typically be viewed as an application of conditional Monte Carlo. We also discuss how stratification can be conveniently applied in this setting to provide further variance reductions.  

\end{talk}

\end{document}

