\documentclass[12pt,a4paper,figuresright]{book}

\usepackage{amsmath,amssymb}
\usepackage{tabularx,graphicx,url,xcolor,rotating,multicol,epsfig,colortbl}

\setlength{\textheight}{25.2cm}
\setlength{\textwidth}{16.5cm} %\setlength{\textwidth}{18.2cm}
\setlength{\voffset}{-1.6cm}
\setlength{\hoffset}{-0.3cm} %\setlength{\hoffset}{-1.2cm}
\setlength{\evensidemargin}{-0.3cm} 
\setlength{\oddsidemargin}{0.3cm}
\setlength{\parindent}{0cm} 
\setlength{\parskip}{0.3cm}

% -- adding a talk
\newenvironment{talk}[6]% [1] talk title
                         % [2] speaker name, [3] affiliations, [4] email,
                         % [5] coauthors, [6] special session
                         % [7] time slot
                         % [8] talk id, [9] session id or photo
 {%\needspace{6\baselineskip}%
  \vskip 0pt\nopagebreak%
%   \colorbox{gray!20!white}{\makebox[0.99\textwidth][r]{}}\nopagebreak%
%   \ifthenelse{\equal{#9}{photo}}{%
%                     \\\\\colorbox{gray!20!white}{\makebox{\includegraphics[width=3cm]{#8}}}\nopagebreak}{}%
 \vskip 0pt\nopagebreak%
%  \label{#8}%
  \textbf{#1}\vspace{3mm}\\\nopagebreak%
  \textit{#2}\\\nopagebreak%
  #3\\\nopagebreak%
  \url{#4}\vspace{3mm}\\\nopagebreak%
  \ifthenelse{\equal{#5}{}}{}{Coauthor(s): #5\vspace{3mm}\\\nopagebreak}%
  \ifthenelse{\equal{#6}{}}{}{Special session: #6\quad \vspace{3mm}\\\nopagebreak}%
 }
 {\vspace{1cm}\nopagebreak}%

\pagestyle{empty}

% ------------------------------------------------------------------------
% Document begins here
% ------------------------------------------------------------------------
\begin{document}
	
\begin{talk}
  {Stacking designs: designing multifidelity computer experiments with target predictive accuracy}% [1] talk title
  {Chih-Li Sung}% [2] speaker name
  {Michigan State University}% [3] affiliations
  {sungchih@msu.edu}% [4] email
  {Yi Ji, Simon Mak, Wenjia Wang, Tao Tang}% [5] coauthors
  {}% [6] special session. Leave this field empty for contributed talks. 
				% Insert the title of the special session if you were invited to give a talk in a special session.
			
In an era where scientific experiments can be very costly, multi-fidelity emulators provide a useful tool for cost-efficient predictive scientific computing. For scientific applications, the experimenter is often limited by a tight computational budget, and thus wishes to (i) maximize predictive power of the multi-fidelity emulator via a careful design of experiments, and (ii) ensure this model achieves a desired error tolerance with some notion of confidence. Existing design methods, however, do not jointly tackle objectives (i) and (ii). We propose a novel stacking design approach that addresses both goals. A  multi-level reproducing kernel Hilbert space (RKHS) interpolator is first introduced to build the emulator, under which our stacking design provides a sequential approach for designing multi-fidelity runs such that a desired prediction error of $\epsilon > 0$ is met under regularity assumptions. We then prove a novel cost complexity theorem that, under this multi-level interpolator, establishes a bound on the computation cost (for training data simulation) needed to achieve a prediction bound of $\epsilon$. This result provides novel insights on conditions under which the proposed multi-fidelity approach improves upon a conventional RKHS interpolator which relies on a single fidelity level. Finally, we demonstrate the effectiveness of stacking designs in a suite of simulation experiments and an application to finite element analysis.

\end{talk}

\end{document}

