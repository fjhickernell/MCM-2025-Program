\documentclass[12pt,a4paper,figuresright]{book}

\usepackage{amsmath,amssymb}
\usepackage{tabularx,graphicx,url,xcolor,rotating,multicol,epsfig,colortbl}

\setlength{\textheight}{25.2cm}
\setlength{\textwidth}{16.5cm} %\setlength{\textwidth}{18.2cm}
\setlength{\voffset}{-1.6cm}
\setlength{\hoffset}{-0.3cm} %\setlength{\hoffset}{-1.2cm}
\setlength{\evensidemargin}{-0.3cm} 
\setlength{\oddsidemargin}{0.3cm}
\setlength{\parindent}{0cm} 
\setlength{\parskip}{0.3cm}

% -- adding a talk
\newenvironment{talk}[6]% [1] talk title
                         % [2] speaker name, [3] affiliations, [4] email,
                         % [5] coauthors, [6] special session
                         % [7] time slot
                         % [8] talk id, [9] session id or photo
 {%\needspace{6\baselineskip}%
  \vskip 0pt\nopagebreak%
%   \colorbox{gray!20!white}{\makebox[0.99\textwidth][r]{}}\nopagebreak%
%   \ifthenelse{\equal{#9}{photo}}{%
%                     \\\\\colorbox{gray!20!white}{\makebox{\includegraphics[width=3cm]{#8}}}\nopagebreak}{}%
 \vskip 0pt\nopagebreak%
%  \label{#8}%
  \textbf{#1}\vspace{3mm}\\\nopagebreak%
  \textit{#2}\\\nopagebreak%
  #3\\\nopagebreak%
  \url{#4}\vspace{3mm}\\\nopagebreak%
  \ifthenelse{\equal{#5}{}}{}{Coauthor(s): #5\vspace{3mm}\\\nopagebreak}%
  \ifthenelse{\equal{#6}{}}{}{Special session: #6\quad \vspace{3mm}\\\nopagebreak}%
 }
 {\vspace{1cm}\nopagebreak}%

\pagestyle{empty}

% ------------------------------------------------------------------------
% Document begins here
% ------------------------------------------------------------------------
\begin{document}
	
\begin{talk}
  {Efficient expected information gain estimators based on the randomized quasi-Monte Carlo method}% [1] talk title
  {Arved Bartuska}% [2] speaker name
  {King Abdullah University of Science and Technology/RWTH Aachen University}% [3] affiliations
  {arved.bartuska@kaust.edu.sa}% [4] email
  {Andr\'{e} Gustavo Carlon, Luis Espath, Sebastian Krumscheid, Ra\'{u}l Tempone}% [5] coauthors
  {}% [6] special session. Leave this field empty for contributed talks. 
				% Insert the title of the special session if you were invited to give a talk in a special session.
			
Efficient estimation of the expected information gain (EIG) of an experiment allows for design optimization in a Bayesian setting. This task faces computational challenges, particularly when the experiment model requires numerical discretization schemes. We demonstrate various methods to make such estimations feasible, combining quasi-Monte Carlo (QMC), randomized QMC (rQMC), and multilevel methods.

Analytical error bounds are made possible by Owen's [1] and He et al.'s [2] work on singular integrands combined with a truncation scheme of the observation noise present in experiment models. Applications from Bayesian experimental design demonstrate the improved convergence behavior of the proposed methods compared to traditional Monte Carlo-based estimators.

\medskip

\begin{enumerate}
	\item[{[1]}] Owen, Art B. (2006). {\it Halton sequences avoid the origin}. SIAM Review, 48:487–503.
	\item[{[2]}] He, Zhijian, \& Zheng, Zhan, \& Wang, Xiaoqun. (2023).{\it On the error rate of importance sampling with randomized quasi-Monte Carlo}. SIAM Journal
on Numerical Analysis,  61(2):10.1137/22M1510121.
\end{enumerate}

\end{talk}

\end{document}

