\documentclass[12pt,a4paper,figuresright]{book}

\usepackage{amsmath,amssymb}
\usepackage{tabularx,graphicx,url,xcolor,rotating,multicol,epsfig,colortbl}

\setlength{\textheight}{25.2cm}
\setlength{\textwidth}{16.5cm} %\setlength{\textwidth}{18.2cm}
\setlength{\voffset}{-1.6cm}
\setlength{\hoffset}{-0.3cm} %\setlength{\hoffset}{-1.2cm}
\setlength{\evensidemargin}{-0.3cm} 
\setlength{\oddsidemargin}{0.3cm}
\setlength{\parindent}{0cm} 
\setlength{\parskip}{0.3cm}

% -- adding a talk
\newenvironment{talk}[6]% [1] talk title
                         % [2] speaker name, [3] affiliations, [4] email,
                         % [5] coauthors, [6] special session
                         % [7] time slot
                         % [8] talk id, [9] session id or photo
 {%\needspace{6\baselineskip}%
  \vskip 0pt\nopagebreak%
%   \colorbox{gray!20!white}{\makebox[0.99\textwidth][r]{}}\nopagebreak%
%   \ifthenelse{\equal{#9}{photo}}{%
%                     \\\\\colorbox{gray!20!white}{\makebox{\includegraphics[width=3cm]{#8}}}\nopagebreak}{}%
 \vskip 0pt\nopagebreak%
%  \label{#8}%
  \textbf{#1}\vspace{3mm}\\\nopagebreak%
  \textit{#2}\\\nopagebreak%
  #3\\\nopagebreak%
  \url{#4}\vspace{3mm}\\\nopagebreak%
  \ifthenelse{\equal{#5}{}}{}{Coauthor(s): #5\vspace{3mm}\\\nopagebreak}%
  \ifthenelse{\equal{#6}{}}{}{Special session: #6\quad \vspace{3mm}\\\nopagebreak}%
 }
 {\vspace{1cm}\nopagebreak}%

\pagestyle{empty}

% ------------------------------------------------------------------------
% Document begins here
% ------------------------------------------------------------------------
\begin{document}
	
\begin{talk}
  {Efficient Pricing for Variable Annuity via Simulation}% [1] talk title
  {Hao Quan}% [2] speaker name
  {University of Waterloo}% [3] affiliations
  {h5quan@uwaterloo.ca}% [4] email
  {Ben (Mingbin) Feng}% [5] coauthors
  {}% [6] special session. Leave this field empty for contributed talks. 
				% Insert the title of the special session if you were invited to give a talk in a special session.

Variable Annuities (VAs) are insurance products that offer policyholders exposure to financial market upside potential while safeguarding against downside risk through optional riders, such as Guaranteed Minimum Death Benefits (GMDBs), Guaranteed Minimum Accumulation Benefits (GMABs), and Guaranteed Minimum Withdrawal Benefits (GMWBs). These riders, tailored to policyholders’ needs, introduce a complex risk profile combining mortality and financial uncertainties, rendering VA pricing and fee determination computationally challenging. Due to this complexity, Monte Carlo simulation is often the only practical approach for valuing these contracts.

In this study, we address the problem of setting fair management fees for VA rider combinations using the equivalence principle, which balances the expected present value of premiums and benefits. We formulate fee determination as a stochastic root-finding problem, expressed as
\[
E[V(\varphi)] - P = 0,
\]
where \( E[V(\varphi)] \) denotes the expected present value of VA benefits under fee structure \( \varphi \), and \( P \) represents the premium. The VA benefit \( V(\varphi) \) reflects the evolution of the contract’s shadow account value and various benefit guarantees over the contract’s lifetime. As a result, estimating its expected value is computationally challenging. Moreover, solving the root-finding problem requires estimating the gradient of this expectation. To solve this, we employ stochastic gradient estimation techniques, such as finite differences and infinitesimal perturbation analysis (IPA). We analyze the theoretical properties and computational performance of the proposed root-finding algorithms, offering insights into their efficacy for VA pricing. Our results show that gradient estimation techniques have a significant impact on the efficiency and accuracy of estimating fair fees for various rider combinations.
\end{talk}

\end{document}

