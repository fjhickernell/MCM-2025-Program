\documentclass[12pt,a4paper,figuresright]{book}

\usepackage{amsmath,amssymb}
\usepackage{tabularx,graphicx,url,xcolor,rotating,multicol,epsfig,colortbl}

\setlength{\textheight}{25.2cm}
\setlength{\textwidth}{16.5cm} %\setlength{\textwidth}{18.2cm}
\setlength{\voffset}{-1.6cm}
\setlength{\hoffset}{-0.3cm} %\setlength{\hoffset}{-1.2cm}
\setlength{\evensidemargin}{-0.3cm} 
\setlength{\oddsidemargin}{0.3cm}
\setlength{\parindent}{0cm} 
\setlength{\parskip}{0.3cm}

% -- adding a talk
\newenvironment{talk}[6]% [1] talk title
                         % [2] speaker name, [3] affiliations, [4] email,
                         % [5] coauthors, [6] special session
                         % [7] time slot
                         % [8] talk id, [9] session id or photo
 {%\needspace{6\baselineskip}%
  \vskip 0pt\nopagebreak%
%   \colorbox{gray!20!white}{\makebox[0.99\textwidth][r]{}}\nopagebreak%
%   \ifthenelse{\equal{#9}{photo}}{%
%                     \\\\\colorbox{gray!20!white}{\makebox{\includegraphics[width=3cm]{#8}}}\nopagebreak}{}%
 \vskip 0pt\nopagebreak%
%  \label{#8}%
  \textbf{#1}\vspace{3mm}\\\nopagebreak%
  \textit{#2}\\\nopagebreak%
  #3\\\nopagebreak%
  \url{#4}\vspace{3mm}\\\nopagebreak%
  \ifthenelse{\equal{#5}{}}{}{Coauthor(s): #5\vspace{3mm}\\\nopagebreak}%
  \ifthenelse{\equal{#6}{}}{}{Special session: #6\quad \vspace{3mm}\\\nopagebreak}%
 }
 {\vspace{1cm}\nopagebreak}%

\pagestyle{empty}

% ------------------------------------------------------------------------
% Document begins here
% ------------------------------------------------------------------------
\begin{document}
	
\begin{talk}
  {Dynamical Low-Rank Approximation for SDEs: an interacting particle-system ROM}% [1] talk title
  {Fabio Zoccolan}% [2] speaker name
  {École Polytechnique Fédérale de Lausanne}% [3] affiliations
  {fabio.zoccolan@epfl.ch}% [4] email
  {Dr. Yoshihito Kazashi, Prof. Fabio Nobile}% [5] coauthors
  {}% [6] special session. Leave this field empty for contributed talks. 
The Dynamical Low-Rank Approximation (DLRA) technique is a time-dependent reduced-order model (ROM) known for its significant advantages in terms of computational time and accuracy. Its appeal in uncertainty quantification is due to the fact that its solution is composed of time-dependent deterministic and stochastic bases, allowing the approximation to better track the dynamics of the studied system. In the context of stochastic differential equations (SDEs) a rigorous mathematical setting was presented in [1], using the so-called Dynamically Orthogonal (DO) framework. The well-posedness of this setting is nontrivial due to the coupled nature of the DO system: for instance, the deterministic basis depends on all stochastic basis paths, and the equations involve the inversion of a Gramian matrix. When coming to stochastic discretization through a Monte-Carlo procedure, these features imply to deal with a interacting noisy particle dynamics. We proposed two fully discretized schemes based on the Monte-Carlo method, investigating their errors and analyzing possible issues arisen by the discretization of the Gramian inverse [2]. Theoretical results will be supported by numerical simulations.

\medskip
\begin{enumerate}
	\item[{[1]}] Yoshihito Kazashi, Fabio Nobile, and Fabio Zoccolan. {\it Dynamical low-rank approximation for stochastic differential equations}. Mathematics of Computation (2024).
	\item[{[2]}] Yoshihito Kazashi, Fabio Nobile, and Fabio Zoccolan. {\it Numerical Methods for Dynamical low-rank approximation of stochastic differential equations, Part I \& II}, in preparation (2025).
\end{enumerate}

\end{talk}

\end{document}

