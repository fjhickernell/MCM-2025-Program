\documentclass[12pt,a4paper,figuresright]{book}

\usepackage{amsmath,amssymb}
\usepackage{tabularx,graphicx,url,xcolor,rotating,multicol,epsfig,colortbl}

\setlength{\textheight}{25.2cm}
\setlength{\textwidth}{16.5cm} %\setlength{\textwidth}{18.2cm}
\setlength{\voffset}{-1.6cm}
\setlength{\hoffset}{-0.3cm} %\setlength{\hoffset}{-1.2cm}
\setlength{\evensidemargin}{-0.3cm} 
\setlength{\oddsidemargin}{0.3cm}
\setlength{\parindent}{0cm} 
\setlength{\parskip}{0.3cm}

% -- adding a talk
\newenvironment{talk}[6]% [1] talk title
                         % [2] speaker name, [3] affiliations, [4] email,
                         % [5] coauthors, [6] special session
                         % [7] time slot
                         % [8] talk id, [9] session id or photo
 {%\needspace{6\baselineskip}%
  \vskip 0pt\nopagebreak%
%   \colorbox{gray!20!white}{\makebox[0.99\textwidth][r]{}}\nopagebreak%
%   \ifthenelse{\equal{#9}{photo}}{%
%                     \\\\\colorbox{gray!20!white}{\makebox{\includegraphics[width=3cm]{#8}}}\nopagebreak}{}%
 \vskip 0pt\nopagebreak%
%  \label{#8}%
  \textbf{#1}\vspace{3mm}\\\nopagebreak%
  \textit{#2}\\\nopagebreak%
  #3\\\nopagebreak%
  \url{#4}\vspace{3mm}\\\nopagebreak%
  \ifthenelse{\equal{#5}{}}{}{Coauthor(s): #5\vspace{3mm}\\\nopagebreak}%
  \ifthenelse{\equal{#6}{}}{}{Special session: #6\quad \vspace{3mm}\\\nopagebreak}%
 }
 {\vspace{1cm}\nopagebreak}%

\pagestyle{empty}

% ------------------------------------------------------------------------
% Document begins here
% ------------------------------------------------------------------------
\begin{document}
	
\begin{talk}
  {Exact discretization, tight frames and recovery via $D$-optimal designs}% [1] talk title
  {Felix Bartel}% [2] speaker name
  {University of New South Wales}% [3] affiliations
  {f.bartel@unsw.edu.au}% [4] email
  {Lutz Kämmerer, Kateryna Pozharska, Martin Schäfer, and Tino Ullrich}% [5] coauthors
  {QMC and Applications Part I or II}% [6] special session. Leave this field empty for contributed talks. 
				% Insert the title of the special session if you were invited to give a talk in a special session.
			
    $D$-optimal designs originate in statistics literature as an approach for optimal experimental designs. In numerical analysis points and weights resulting from maximal determinants turned out to be useful for quadrature and interpolation. Also recently, two of the present authors and coauthors investigated a connection to the discretization problem for the uniform norm. Here we use this approach of maximizing the determinant of a certain Gramian matrix with respect to points and weights for the construction of tight frames and exact Marcinkiewicz-Zygmund inequalities in $L_2$. We present a direct and constructive approach resulting in a discrete measure with at most $N\le n^2+1$ atoms, which discretely and accurately subsamples the $L_2$-norm of complex-valued functions contained in a given n-dimensional subspace. This approach can as well be used for the reconstruction of functions from general RKHS in $L_2$ where one only has access to the most important eigenfunctions. We verifiably and deterministically construct points and weights for a weighted least squares recovery procedure and pay in the rate of convergence compared to earlier optimal, however probabilistic approaches. The general results apply to the d-sphere or multivariate trigonometric polynomials on Td spectrally supported on arbitrary finite index sets $I\subset\mathbb Z^d$. They can be discretized using at most $|I|^2-|I|+1$ points and weights. Numerical experiments indicate the sharpness of this result. As a negative result we prove that, in general, it is not possible to control the number of points in a reconstructing lattice rule only in the cardinality $|I|$ without additional condition on the structure of $I$. We support our findings with numerical experiments.

\medskip

\end{talk}

\end{document}

