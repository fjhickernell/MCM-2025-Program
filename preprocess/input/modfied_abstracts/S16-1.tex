\documentclass[12pt,a4paper,figuresright]{book}

\usepackage{amsmath,amssymb}
\usepackage{tabularx,graphicx,url,xcolor,rotating,multicol,epsfig,colortbl}

\setlength{\textheight}{25.2cm}
\setlength{\textwidth}{16.5cm} %\setlength{\textwidth}{18.2cm}
\setlength{\voffset}{-1.6cm}
\setlength{\hoffset}{-0.3cm} %\setlength{\hoffset}{-1.2cm}
\setlength{\evensidemargin}{-0.3cm} 
\setlength{\oddsidemargin}{0.3cm}
\setlength{\parindent}{0cm} 
\setlength{\parskip}{0.3cm}

% -- adding a talk
\newenvironment{talk}[6]% [1] talk title
                         % [2] speaker name, [3] affiliations, [4] email,
                         % [5] coauthors, [6] special session
                         % [7] time slot
                         % [8] talk id, [9] session id or photo
 {%\needspace{6\baselineskip}%
  \vskip 0pt\nopagebreak%
%   \colorbox{gray!20!white}{\makebox[0.99\textwidth][r]{}}\nopagebreak%
%   \ifthenelse{\equal{#9}{photo}}{%
%                     \\\\\colorbox{gray!20!white}{\makebox{\includegraphics[width=3cm]{#8}}}\nopagebreak}{}%
 \vskip 0pt\nopagebreak%
%  \label{#8}%
  \textbf{#1}\vspace{3mm}\\\nopagebreak%
  \textit{#2}\\\nopagebreak%
  #3\\\nopagebreak%
  \url{#4}\vspace{3mm}\\\nopagebreak%
  \ifthenelse{\equal{#5}{}}{}{Coauthor(s): #5\vspace{3mm}\\\nopagebreak}%
  \ifthenelse{\equal{#6}{}}{}{Special session: #6\quad \vspace{3mm}\\\nopagebreak}%
 }
 {\vspace{1cm}\nopagebreak}%

\pagestyle{empty}

% ------------------------------------------------------------------------
% Document begins here
% ------------------------------------------------------------------------
\begin{document}
	
\begin{talk}
  {On the quantum complexity of parametric integration in Sobolev spaces}% [1] talk title
  {Stefan Heinrich}% [2] speaker name
  {RPTU Kaiserslautern-Landau, 
Germany}% [3] affiliations
  {heinrich@informatik.uni-kl.de}% [4] email
  {}% [5] coauthors
  {Stochastic computation and complexity}% [6] special session. Leave this field empty for contributed talks. 
				% Insert the title of the special session if you were invited to give a talk in a special session.
			
We consider the following problem of parametric integration in Sobolev spaces. We seek to approximate
$$
S:W_p^r(D)\to L_q(D_1), \quad (Sf)(s)=\int_{D_2}f(s,t)dt \quad (s\in D_1),
$$ 
where 
%
\begin{eqnarray*}
&&D=[0,1]^d=D_1\times D_2,\quad D_1=[0,1]^{d_1}, \quad D_2=[0,1]^{d_2}, 
\\
&&1\le p,q\le \infty, \quad d,d_1,d_2,r\in {\bf N},\quad d=d_1+d_2,\quad \frac{r}{d_1}>\left(\frac{1}{p}-\frac{1}{q}\right)_+\, .
\end{eqnarray*}
%
We study the complexity of this problem in the quantum setting of Information-Based Complexity [1]. Under the assumption that $W_p^r(D)$ is embedded into $C(D)$ (embedding condition) the case $p=q$ was solved by Wiegand [2]. Here we treat the case $p=q$ without embedding condition and the general case $p\ne q$ with or without the embedding condition. We also compare the rates with those in the (classical) randomized setting [3].


\medskip

\begin{enumerate}

\item[{[1]}] Heinrich, Stefan (2002).\  Quantum summation with an application to integration. 
Journal of Complexity 18, 1--50.
\item[{[2]}] Wiegand, Carsten (2006).\ {\it Optimal Monte Carlo and Quantum Algorithms for Parametric Integration}. Shaker Verlag.
\item[{[3]}] Heinrich, Stefan (2024).\  Randomized complexity of parametric integration and
the role of adaption  II. Sobolev spaces. Journal of Complexity 82, 101823.
\end{enumerate}

\end{talk}

\end{document}

