\documentclass[12pt,a4paper,figuresright]{book}

\usepackage{amsmath,amssymb}
\usepackage{tabularx,graphicx,url,xcolor,rotating,multicol,epsfig,colortbl}

\setlength{\textheight}{25.2cm}
\setlength{\textwidth}{16.5cm} %\setlength{\textwidth}{18.2cm}
\setlength{\voffset}{-1.6cm}
\setlength{\hoffset}{-0.3cm} %\setlength{\hoffset}{-1.2cm}
\setlength{\evensidemargin}{-0.3cm} 
\setlength{\oddsidemargin}{0.3cm}
\setlength{\parindent}{0cm} 
\setlength{\parskip}{0.3cm}

% -- adding a talk
\newenvironment{talk}[6]% [1] talk title
                         % [2] speaker name, [3] affiliations, [4] email,
                         % [5] coauthors, [6] special session
                         % [7] time slot
                         % [8] talk id, [9] session id or photo
 {%\needspace{6\baselineskip}%
  \vskip 0pt\nopagebreak%
%   \colorbox{gray!20!white}{\makebox[0.99\textwidth][r]{}}\nopagebreak%
%   \ifthenelse{\equal{#9}{photo}}{%
%                     \\\\\colorbox{gray!20!white}{\makebox{\includegraphics[width=3cm]{#8}}}\nopagebreak}{}%
 \vskip 0pt\nopagebreak%
%  \label{#8}%
  \textbf{#1}\vspace{3mm}\\\nopagebreak%
  \textit{#2}\\\nopagebreak%
  #3\\\nopagebreak%
  \url{#4}\vspace{3mm}\\\nopagebreak%
  \ifthenelse{\equal{#5}{}}{}{Coauthor(s): #5\vspace{3mm}\\\nopagebreak}%
  \ifthenelse{\equal{#6}{}}{}{Special session: #6\quad \vspace{3mm}\\\nopagebreak}%
 }
 {\vspace{1cm}\nopagebreak}%

\pagestyle{empty}

% ------------------------------------------------------------------------
% Document begins here
% ------------------------------------------------------------------------
\usepackage{ulem}  %Remove this before submitting
\begin{document}

\newcommand{\Fredcomment}[1]{{\color{blue}{#1}}}   %Remove this before submitting
	
\begin{talk}
  {Investigating the Optimum RQMC Batch Size for Betting and Empirical Bernstein Confidence Intervals}% [1] talk title
  {Aadit Jain}% [2] speaker name
  {Rancho Bernardo High School}% [3] affiliations
  {aaditdjain@gmail.com}% [4] email
  {Fred J.\ Hickernell, Art B.\ Owen, Aleksei G.\ Sorokin}% [5] coauthors
  {}% [6] special session. Leave this field empty for contributed talks. 
				% Insert the title of the special session if you were invited to give a talk in a special session.
			

The Betting [1] and Empirical Bernstein (EB) [1,2] confidence intervals (CIs) are finite sample (non-asymptotic) and require IID samples. Since both are non-asymptotic, they are much wider than confidence intervals based on the Central Limit Theorem (CLT) due to the stronger coverage property they provide. To apply these finite sample CIs to randomized quasi-Monte Carlo (RQMC), we take $R$ independent replications of $n$ RQMC points, averaging the $n$ function evaluations within each replication. Given a fixed budget $N = nR$, we investigate the optimal $n$ that minimizes the CI widths for both methods.  
%\medskip  

Using the code from [1], we ran simulations on various integrands (smooth, rough, one-dimensional, multi-dimensional) and ridge functions. Interestingly, the optimal $n$ was quite small compared to $N$, often just 1 (plain IID), 2, or 4 when $N = 2^{10}$. Moreover, the optimal $n$ appeared to grow quite slowly as $N$ increased. Notably, both CI methods applied to RQMC outperformed plain IID when the optimal $n$ is greater than $1$.  
%\medskip  

This experimental trend aligns with our analysis of Bennett’s inequality for EB [2], which suggests that the optimum $n$ is $O(N^{1/(2\theta + 1)})$ for $\theta > 1/2$. 
Specifically, for $\theta = 3/2$, which occurs for smoother integrands, we obtain $n = O(N^{1/4})$. For $\theta = 1$, which corresponds to a typical Koksma-Hlawka rate, we get $n = O(N^{1/3})$. The ratio of RQMC EB CI widths to plain IID EB CI widths is $\Theta( N^{(1-2\theta)/(4\theta+2)})$. For $\theta=1$,
we get a ratio of $\Theta(N^{-1/6})$,
while for $\theta=3/2$,
we get a more favorable width ratio of $\Theta(N^{-1/4})$. 
%\medskip  


On the other hand, CLT based CIs using RQMC  are only asymptotically valid. The value of $R$ could be any reasonable number that isn't too small, and remains constant as the total sample size, $N$, increases.  This means that $n = \Theta(N)$, which takes full advantage of the power of QMC.  It is also important to note that both Betting and EB require the random variables to be bounded between $0$ and $1$, unlike CLT based CIs. 


\begin{enumerate}
	\item[{[1]}] I. Waubdy-Smith and A. Ramdas. Estimating means of bounded random
variables by betting. J.\ Roy.\ Statist.\ Soc.\ B, 86:1–27, 2024.
	\item[{[2]}] A. Maurer and M. Pontil. Empirical Bernstein bounds and sample variance penalization. In Proceedings of the 22nd Annual Conference
on Learning Theory (COLT), pages 1–9, 2009.
\end{enumerate}


\end{talk}

\end{document}
While we have established a theoretical basis for EB, \Fredcomment{[Fred: Do you mean we have established a basis for the optimal size of the QMC sample?]}we haven't done so yet for Betting. For now, we extend Bennett’s analysis to Betting and are happy \Fredcomment{[Fred: Normally we are not so informal to use the word ``happy'']}to know that the Betting CIs will be at least 11\% narrower than Maurer and Pontil [2] (expecting Betting to show a similar asymptotic behavior as Maurer and Pontil [2]), but to strengthen our argument, we aim to analyze \textbf{Theorem 4.6} [3] in a manner similar to Bennett's inequality. Another promising avenue for future exploration is investigating the optimum $n$ for Betting and EB Confidence Sequences given a fixed budget $N$, which are sequences of CIs valid not just for a fixed sample size but for all sizes from one to infinity. 

     \item[{[3]}] S. Shekhar and A. Ramdas. On the near-optimality of betting confidence sets for bounded means. arXiv preprint, arXiv:2310.01547, 2023.

\Fredcomment{[Fred:  I think that you should say more about CLT based CIs and their relation to the finite sample CIS.  Do not put in the abstract what you plan to do.  You can add it to the talk, but you should not add it to the abstract.}