\documentclass[12pt,a4paper,figuresright]{book}

\usepackage{amsmath,amssymb}
\usepackage{tabularx,graphicx,url,xcolor,rotating,multicol,epsfig,colortbl}

\setlength{\textheight}{25.2cm}
\setlength{\textwidth}{16.5cm} %\setlength{\textwidth}{18.2cm}
\setlength{\voffset}{-1.6cm}
\setlength{\hoffset}{-0.3cm} %\setlength{\hoffset}{-1.2cm}
\setlength{\evensidemargin}{-0.3cm} 
\setlength{\oddsidemargin}{0.3cm}
\setlength{\parindent}{0cm} 
\setlength{\parskip}{0.3cm}

% -- adding a talk
\newenvironment{talk}[6]% [1] talk title
                         % [2] speaker name, [3] affiliations, [4] email,
                         % [5] coauthors, [6] special session
                         % [7] time slot
                         % [8] talk id, [9] session id or photo
 {%\needspace{6\baselineskip}%
  \vskip 0pt\nopagebreak%
%   \colorbox{gray!20!white}{\makebox[0.99\textwidth][r]{}}\nopagebreak%
%   \ifthenelse{\equal{#9}{photo}}{%
%                     \\\\\colorbox{gray!20!white}{\makebox{\includegraphics[width=3cm]{#8}}}\nopagebreak}{}%
 \vskip 0pt\nopagebreak%
%  \label{#8}%
  \textbf{#1}\vspace{3mm}\\\nopagebreak%
  \textit{#2}\\\nopagebreak%
  #3\\\nopagebreak%
  \url{#4}\vspace{3mm}\\\nopagebreak%
  \ifthenelse{\equal{#5}{}}{}{Coauthor(s): #5\vspace{3mm}\\\nopagebreak}%
  \ifthenelse{\equal{#6}{}}{}{Special session: #6\quad \vspace{3mm}\\\nopagebreak}%
 }
 {\vspace{1cm}\nopagebreak}%

\pagestyle{empty}

% ------------------------------------------------------------------------
% Document begins here
% ------------------------------------------------------------------------
\begin{document}
	
\begin{talk}
  {Randomized QMC with one categorical variable}% [1] talk title
  {Art B. Owen}% [2] speaker name
  {Stanford University}% [3] affiliations
  {owen@stanford.edu}% [4] email
  {Valerie Ho, Zexin Pan}% [5] coauthors
  {}% [6] special session. Leave this field empty for contributed talks. 
				% Insert the title of the special session if you were invited to give a talk in a special session.
			
  Randomized quasi-Monte Carlo (RQMC) methods benefit from smoothness
  in the integrand.  In some applications, one of the input variables takes
  only a modest finite number $L\geqslant2$
of values and can be considered a categorical
  variable. Such a variable introduces a discontinuity in the integrand.  It is
  an RQMC-friendly discontinuity because the discontinuity is axis parallel
  when each level of the categorical variable corresponds to a single
  sub-interval of $[0,1]$.  A naturally occuring use case has
  the categorical variable correspond to a component in a distribution
  that is a mixture of $L$ other distributions.  Within the mixture
  setting, mixture importance sampling is a prominent example.

 If category $\ell$  has probability $\alpha_\ell$ in the motivating
  model then it would naturally get an interval of width $\alpha_\ell$.
If the RQMC method uses scrambled Sobol' points
then we can make the problem even more RQMC-friendly by instead giving
  level $\ell$ an interval of width $\beta_\ell = 2^{-\kappa_\ell}$
  for some integers $\kappa_\ell \geqslant1$ and then
  incorporating sampling ratios $\alpha_\ell/\beta_\ell$.
  These $\beta_\ell$ are negative powers of $2$ that sum to $1$.
  The number of ways to choose $\beta_1,\dots,\beta_L$
  as a function of $L$ is given by sequence A002572 in the
  online encyclopedia of integer sequences.

  Under usual assumptions on the convergence rates for RQMC
  estimates, we find that the asymptotically optimal $\beta_\ell$
  are more nearly equal than the original $\alpha_\ell$ are.
  We find some rules for making that  allocation under uncertainty
  about the appropriate RQMC rate.

  The mixing problem was studied by [1] where they use instead
  $L$ different QMC samples instead of one sample in which
  one of the variables yields $\ell$. The first QMC variable
  was used by [2] to allocate points to different processors
  in parallel computing.

\medskip
\begin{enumerate}
	\item[{[1]}] Cui, T., J. Dick, and F. Pillichshammer (2023). Quasi-Monte Carlo methods
for mixture distributions and approximated distributions via piecewise linear
interpolation. Technical report, arXiv:2304.14786 
\item[{[2]}]
Keller, A. and L. Gr¨unschloß (2012). Parallel quasi-monte carlo integration by
partitioning low discrepancy sequences. In Monte Carlo and Quasi-Monte
Carlo Methods 2010, pp. 487–498. Springer.
\end{enumerate}

\end{talk}

\end{document}

