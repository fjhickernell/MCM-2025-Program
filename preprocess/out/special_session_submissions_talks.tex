\chapter{Special Sessions}

\begin{session}
 {Stochastic Computation and Complexity, Part I}% [1] session title
 {Stefan Heinrich}% [2] organizer one name
 {RPTU Kaiserslautern-Landau}% [3] organizer one affiliations
 {heinrich@informatik.uni-kl.de}% [4] organizer one email
 {Thomas M\"uller-Gronbach}% [5] organizer two name. Leave unchanged if there is no second organizer, otherwise fill in accordingly.
 {University of Passau}% [6] affiliations. Leave unchanged if there is no second organizer, otherwise fill in accordingly.
 {Thomas.Mueller-Gronbach@uni-passau.de}% [7] email. Leave unchanged if there is no second organizer, otherwise fill in accordingly.
 {S1}% [8] session id
 {\thirdorganizer{Larisa Yaroslavtseva}{University of Graz}{larisa.yaroslavtseva@uni-graz.at}}% [9] third organizer, if any

 The session is devoted to algorithms and complexity for
 quadrature and strong approximation of SDEs and SPDEs, in particular under nonstandard assumptions,
 high and infinite dimensional integration and approximation, and
 stochastic optimization and neural networks,
 including connections to functional analysis and stochastic analysis.
 \medskip
\end{session}

\sessionPart{}% [1] part
{\hfill\timeslot{Mon, July 28, 2024 -- Morning}
{10:30}{12:30} % Start and End time
{}} % Room 
\sessionTalk{Quantitative approximation of stochastic kinetic equations: from discrete to continuum}
{Chengcheng Ling}
{S1-1}
\sessionTalk{A strong order 1.5 boundary preserving discretization scheme for scalar SDEs defined in a domain}
{Andreas Neuenkirch}
{S1-2}
\sessionTalk{An adaptive Milstein-type method for strong approximation of systems of SDEs with a discontinuous drift coefficient}
{Christopher Rauhögger}
{S1-3}
\sessionTalk{Stong order 1 adaptive approximation of jump-diffusion SDEs with discontinuous drift}
{Verena Schwarz}
{S1-4}


\clearpage

\begin{session}
 {Nested expectations: models and estimators, Part I}% [1] session title
 {Arved Bartuska}% [2] organizer one name
 {King Abdullah University of Science and Technology and RWTH Aachen University}% [3] organizer one affiliations
 {sarved.bartuska@kaust.edu.sa}% [4] organizer one email
 {Abdul-Lateef Haji-Ali}% [5] organizer two name. Leave unchanged if there is no second organizer, otherwise fill in accordingly.
 {Heriot-Watt University}% [6] affiliations. Leave unchanged if there is no second organizer, otherwise fill in accordingly.
 {a.hajiali@hw.ac.uk}% [7] email. Leave unchanged if there is no second organizer, otherwise fill in accordingly.
 {S3}% [8] session id
{}

 Nested expectations
 arise in many applications, such as in engineering, mathematical finance, and medical decision-making. In addition to their nested structure, numerical estimations of such expectations are often complicated by singularities or discontinuities. Moreover, approximations when evaluating inner expectations using, for example, finite element or time-stepping schemes render traditional estimation methods such as double-loop Monte Carlo prohibitively expensive. This session will explore models and applications with this structure and methods for efficient estimation.
\end{session}

\sessionPart{}% [1] part
{\hfill\timeslot{Mon, Jul 28, 2025 -- Morning}
{10:30}{12:30} % Start and End time
{PH Auditorium}} % Room 
\sessionTalk{An Adaptive Sampling Algorithm for Level-set Approximation}
{Abdul Lateef Haji Ali}
{S3-1}
\sessionTalk{Posterior-Free A-Optimal Bayesian Design of Experiments via Conditional Expectation}
{Vinh Hoang}
{S3-2}
\sessionTalk{QMC for Bayesian optimal experimental design with application to inverse problems governed by PDEs}
{Vesa Kaarnioja}
{S3-3}


\clearpage

\begin{session}
 {Stochastic Computation and Complexity, Part II}% [1] session title
 {Stefan Heinrich}% [2] organizer one name
 {RPTU Kaiserslautern-Landau}% [3] organizer one affiliations
 {heinrich@informatik.uni-kl.de}% [4] organizer one email
 {Thomas M\"uller-Gronbach}% [5] organizer two name. Leave unchanged if there is no second organizer, otherwise fill in accordingly.
 {University of Passau}% [6] affiliations. Leave unchanged if there is no second organizer, otherwise fill in accordingly.
 {Thomas.Mueller-Gronbach@uni-passau.de}% [7] email. Leave unchanged if there is no second organizer, otherwise fill in accordingly.
 {S5}% [8] session id
 {\thirdorganizer{Larisa Yaroslavtseva}{University of Graz}{larisa.yaroslavtseva@uni-graz.at}}% [9] third organizer, if any

 The session is devoted to algorithms and complexity for
 quadrature and strong approximation of SDEs and SPDEs, in particular under nonstandard assumptions,
 high and infinite dimensional integration and approximation, and
 stochastic optimization and neural networks,
 including connections to functional analysis and stochastic analysis.
 \medskip
\end{session}

\sessionPart{}% [1] part
{\hfill\timeslot{Mon, Jul 28, 2025 -- Afternoon}
{15:30}{17:30} % Start and End time
{HH Auditorium}} % Room 
\sessionTalk{Optimality of deterministic and randomized QMC-cubatures on several scales of function spaces}
{Michael Gnewuch}
{S5-1}
\sessionTalk{Optimal designs for function discretization and construction of tight frames}
{Kateryna Pozharska}
{S5-2}
\sessionTalk{Complexity of approximating piecewise smooth functions in the presence of deterministic or random noise}
{Leszek Plaskota}
{S5-3}


\clearpage

\begin{session}
 {Stochastic Computation and Complexity, Part III}% [1] session title
 {Stefan Heinrich}% [2] organizer one name
 {RPTU Kaiserslautern-Landau}% [3] organizer one affiliations
 {heinrich@informatik.uni-kl.de}% [4] organizer one email
 {Thomas M\"uller-Gronbach}% [5] organizer two name. Leave unchanged if there is no second organizer, otherwise fill in accordingly.
 {University of Passau}% [6] affiliations. Leave unchanged if there is no second organizer, otherwise fill in accordingly.
 {Thomas.Mueller-Gronbach@uni-passau.de}% [7] email. Leave unchanged if there is no second organizer, otherwise fill in accordingly.
 {S8}% [8] session id
 {\thirdorganizer{Larisa Yaroslavtseva}{University of Graz}{larisa.yaroslavtseva@uni-graz.at}}% [9] third organizer, if any

 The session is devoted to algorithms and complexity for
 quadrature and strong approximation of SDEs and SPDEs, in particular under nonstandard assumptions,
 high and infinite dimensional integration and approximation, and
 stochastic optimization and neural networks,
 including connections to functional analysis and stochastic analysis.
 \medskip
\end{session}

\sessionPart{}% [1] part
{\hfill\timeslot{Tue, Jul 29, 2025 -- Morning}
{10:30}{12:30} % Start and End time
{HH Auditorium}} % Room 
\sessionTalk{Computing the stationary measure of McKean-Vlasov SDEs}
{Jean-François Chassagneux}
{S8-1}
\sessionTalk{On the convergence of the Euler-Maruyama scheme for McKean-Vlasov SDEs}
{Noufel Frikha}
{S8-3}
\sessionTalk{Wasserstein Convergence of Score-based Generative Models under Semiconvexity and Discontinuous Gradients}
{Sotirios Sabanis}
{S8-4}


\clearpage

\begin{session}
 {Stochastic Computation and Complexity, Part IV}% [1] session title
 {Stefan Heinrich}% [2] organizer one name
 {RPTU Kaiserslautern-Landau}% [3] organizer one affiliations
 {heinrich@informatik.uni-kl.de}% [4] organizer one email
 {Thomas M\"uller-Gronbach}% [5] organizer two name. Leave unchanged if there is no second organizer, otherwise fill in accordingly.
 {University of Passau}% [6] affiliations. Leave unchanged if there is no second organizer, otherwise fill in accordingly.
 {Thomas.Mueller-Gronbach@uni-passau.de}% [7] email. Leave unchanged if there is no second organizer, otherwise fill in accordingly.
 {S12}% [8] session id
 {\thirdorganizer{Larisa Yaroslavtseva}{University of Graz}{larisa.yaroslavtseva@uni-graz.at}}% [9] third organizer, if any

 The session is devoted to algorithms and complexity for
 quadrature and strong approximation of SDEs and SPDEs, in particular under nonstandard assumptions,
 high and infinite dimensional integration and approximation, and
 stochastic optimization and neural networks,
 including connections to functional analysis and stochastic analysis.
 \medskip
\end{session}

\sessionPart{}% [1] part
{\hfill\timeslot{Tue, Jul 29, 2025 -- Afternoon}
{15:30}{17:30} % Start and End time
{}} % Room 
\sessionTalk{Optimal strong approximation of SDEs with H\"older continuous drift coefficient}
{Larisa Yaroslavtseva}
{S12-1}
\sessionTalk{Malliavin Differentiation of Lipschitz SDEs and BSDEs and an Application to Quadratic Forward-Backward SDEs}
{Alexander Steinicke}
{S12-2}
\sessionTalk{Tractability of $L_2$-approximation and integration in weighted Hermite spaces of finite smoothness}
{Gunther Leobacher}
{S12-3}


\clearpage

\begin{session}
 {Stochastic Computation and Complexity, Part V}% [1] session title
 {Stefan Heinrich}% [2] organizer one name
 {RPTU Kaiserslautern-Landau}% [3] organizer one affiliations
 {heinrich@informatik.uni-kl.de}% [4] organizer one email
 {Thomas M\"uller-Gronbach}% [5] organizer two name. Leave unchanged if there is no second organizer, otherwise fill in accordingly.
 {University of Passau}% [6] affiliations. Leave unchanged if there is no second organizer, otherwise fill in accordingly.
 {Thomas.Mueller-Gronbach@uni-passau.de}% [7] email. Leave unchanged if there is no second organizer, otherwise fill in accordingly.
 {S16}% [8] session id
 {\thirdorganizer{Larisa Yaroslavtseva}{University of Graz}{larisa.yaroslavtseva@uni-graz.at}}% [9] third organizer, if any

 The session is devoted to algorithms and complexity for
 quadrature and strong approximation of SDEs and SPDEs, in particular under nonstandard assumptions,
 high and infinite dimensional integration and approximation, and
 stochastic optimization and neural networks,
 including connections to functional analysis and stochastic analysis.
 \medskip
\end{session}

\sessionPart{}% [1] part
{\hfill\timeslot{Wed, Jul 30, 2025 -- Morning}
{10:30}{12:30} % Start and End time
{}} % Room 
\sessionTalk{On the quantum complexity of parametric integration in Sobolev spaces}
{Stefan Heinrich}
{S16-1}
\sessionTalk{Quantum Integration in Tensor Product  Besov Spaces}
{Bernd Käßemodel}
{S16-2}


\clearpage

\begin{session}
 {Nested expectations: models and estimators, Part II}% [1] session title
 {}% [2] organizer one name
 {}% [3] organizer one affiliations
 {}% [4] organizer one email
 {}% [5] organizer two name. Leave unchanged if there is no second organizer, otherwise fill in accordingly.
 {}% [6] affiliations. Leave unchanged if there is no second organizer, otherwise fill in accordingly.
 {}% [7] email. Leave unchanged if there is no second organizer, otherwise fill in accordingly.
 {S24}% [8] session id
 {\thirdorganizer{}{}{}}% [9] third organizer, if any

 {Arved Bartuska}% [2] organizer name
 {King Abdullah University of Science and Technology/RWTH Aachen University}% [3] affiliations
 {arved.bartuska@kaust.edu.sa}% [4] email
 {Abdul-Lateef Haji-Ali }% [5] second organizer name. Leave unchanged if there is no second organizer, otherwise fill in accordingly.
 {Heriot-Watt University}% [6] second organizer affiliations. Leave unchanged if there is no second organizer, otherwise fill in accordingly.
 {a.hajiali@hw.ac.uk}% [7] second organizer email. Leave unchanged if there is no second organizer, otherwise fill in accordingly.
 Nested expectations arise in many applications, such as in engineering, mathematical finance, and medical decision-making. In addition to their nested structure, numerical estimations of such expectations are often complicated by singularities or discontinuities. Moreover, approximations when evaluating inner expectations using, for example, finite element or time-stepping schemes render traditional estimation methods such as double-loop Monte Carlo prohibitively expensive. This session will explore models and applications with this structure and methods for efficient estimation.
 List of speakers:
 Ra\'{u}l Tempone (King Abdullah University of Science and Technology/RWTH Aachen University)
 Andr\'{e} Gustavo Carlon (RWTH Aachen University)
 Zhijian He (South China University of Technology)
 Philipp Guth (Johann Radon Institute for Computational and Applied Mathematics)
\end{session}

\sessionPart{}% [1] part
{\hfill\timeslot{Thu, Jul 31, 2025 -- Morning}
{10:30}{12:30} % Start and End time
{}} % Room 
\sessionTalk{Multilevel randomized quasi-Monte Carlo estimator for nested expectations}
{RAUL TEMPONE}
{S24-1}
\sessionTalk{Stochastic gradient with least-squares control variates}
{Matteo Raviola}
{S24-2}
\sessionTalk{A one-shot method for Bayesian optimal experimental design}
{Philipp Guth}
{S24-3}


\clearpage

\begin{session}
 {Recent Advances in Stochastic Gradient Descent}% [1] session title
 {Jing Dong}% [2] organizer one name
 {Columbia University}% [3] organizer one affiliations
 {jing.dong@gsb.columbia.edu}% [4] organizer one email
{}{}{}
 {S27}% [8] session id
{}

 Stochastic Gradient Descent (SGD) is a cornerstone optimization method in machine learning,
 renowned for its efficiency in handling large-scale data. Its iterative approach enables
 the processing of extensive datasets by updating model parameters using randomly selected
 data subsets, thereby reducing computational costs. Despite its widespread adoption, traditional
 SGD faces challenges such as convergence to sharp minima, and sensitivity to data
 distribution shifts. Addressing these challenges is crucial for enhancing model generalization,
 robustness, and overall performance in diverse applications. This session aims to delve into
 recent developments that address these challenges in SGD, presenting innovative methodologies
 and theoretical insights to enhance its effectiveness in complex learning scenarios.
 The session will have three to four speakers. Currently, the confirmed speakers are Jose
 Blanchet (Stanford University), Chang-Han Rhee (Northwestern University), and Jing Dong
 (Columbia University). Each will present their recent works on stochastic gradient descent,
 ranging from SGD and heavy-tailed phenomenon to SGD with adaptively generated data.
 Collectively, these talks will shed light on cutting-edge advancements in SGD methodologies,
 providing both theoretical frameworks and practical strategies to enhance optimization in
 complex, real-world applications.
\end{session}

\sessionPart{}% [1] part
{\hfill\timeslot{Thu, Jul 31, 2025 -- Afternoon}
{15:30}{17:30} % Start and End time
{PH Auditorium}} % Room 
\sessionTalk{Inference for Stochastic Gradient Descent with Infinite Variance}
{Jose Blanchet}
{S27-1}
\sessionTalk{TBD}
{rhee}
{S27-2}
\sessionTalk{Stochastic Gradient Descent with Adaptive Data}
{Jing Dong}
{S27-3}
\sessionTalk{TBD}
{lovas}
{S27-4}


\clearpage

\begin{session}
 {Forward and Inverse Problems for Stochastic Reaction Networks}% [1] session title
 {Sophia Münker}% [2] organizer one name
 {RWTH Aachen University}% [3] organizer one affiliations
 {muenker@uq.rwth-aachen.de}% [4] organizer one email
 {Chiheb Ben Hammouda}% [5] organizer two name. Leave unchanged if there is no second organizer, otherwise fill in accordingly.
 {Utrecht University}% [6] affiliations. Leave unchanged if there is no second organizer, otherwise fill in accordingly.
 {c.benhammouda@uu.nl}% [7] email. Leave unchanged if there is no second organizer, otherwise fill in accordingly.
 {S28}% [8] session id
 {\thirdorganizer{Raúl Tempone}{RWTH Aachen University}{tempone@uq.rwth-aachen.de}}% [9] third organizer, if any

 This session aims to bring together experts working on stochastic reaction networks and pure jump processes for modeling stochastic biological and chemical systems. The session is about recent advances in Monte Carlo methods, variance and dimension reduction techniques that are relevant for tackling forward and inverse problems.
 \medskip
 The speakers are:
 \begin{itemize}
 \item Zhou Fang (Academy of Mathematics and Systems Science, Chinese Academy of Sciences)
 \item Sophia Münker (RWTH Aachen University)
 \item Maksim Chupin (King Abdullah University of Science and Technology (KAUST))
 \item Muruhan Rathinam (University of Maryland Baltimore County)
 \end{itemize}
\end{session}

\sessionPart{}% [1] part
{\hfill\timeslot{Fri, August 1, 2024 -- Morning}
{9:00}{10:30} % Start and End time
{}} % Room 
\sessionTalk{Filtered Markovian Projection: Dimensionality Reduction in Filtering for Stochastic Reaction Networks}
{Maksim Chupin}
{S28-1}
\sessionTalk{Fixed-budget simulation method for growing cell populations}
{Zhou Fang}
{S28-2}
\sessionTalk{State and parameter inference in stochastic reaction networks}
{Muruhan Rathinam}
{S28-3}
\sessionTalk{Dimensionality Reduction for Efficient Rare Event Estimation}
{Sophia Münker}
{S28-4}


