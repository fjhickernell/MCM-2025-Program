% ------------------------------------------------------------------------
% ------------------------------------------------------------------------
% ------------------------------------------------------------------------
\title{The Fifteenth International Conference on \\
	Monte Carlo and Applications \\
	(MCM 2025) }
%\chapter{The Fifteenth International Conference on Monte Carlo and Applications (MCM 2025) }
\date{\today}
\maketitle

\thispagestyle{empty} \tableofcontents

% ------------------------------------------------------------------------

\chapter{Welcome}

\section{Message from the Organizers}

%\section{Welcome}

We are delighted to welcome you to Chicago and Illinois Institute of Technology (Illinois Tech) for the \emph{15th International Conference on Monte Carlo and Applications (MCM)}. MCM was last held in the United States twenty years ago and last held in North America eight years ago.  Our Monte Carlo community is truly international, and although videoconferencing is easier than ever, gathering in person promotes greater understanding and insight.

MCM features over 150 presentations, including eight plenary talks,  dozens of special sessions, and many contributed talks.  Our speakers represent a variety of academic backgrounds, institutions, and career stages.  This interplay of perspectives will doubtless spur progress in Monte Carlo.  As a reminder to our speakers, your audience may vary quite a bit in their knowledge of your expertise; please keep this in mind as you speak.

Illinois Tech is a private, research university emphasizing architecture, business, computing, design, engineering, law, and science and letters.  Our roots date to the late 1800s.  From our earliest days we have striven to provide upward educational and economic mobility for our students.  We are proud to be the top university in Illinois and \#32 in the US for lifting or lifting students from families in the bottom 20\% of income to the top 20\%.

Chicago is known for its economic and cultural influence, and we like to think of ourselves as a more pleasant large city.  
While you are visiting here, we hope that you will enjoy our beautiful lakefront, diverse traditions, and cultural attractions. The visitor's guide at \href{https://www.choosechicago.com/articles/bucket-list/first-time-visitors-guide-to-chicago/}{\nolinkurl{www.choosechicago.com/articles/bucket-list/first-time-visitors-guide-to-chicago/}} may be of help.

We wish you a productive and interesting week at MCM 2025! If we can be of help, please approach any of us.

We wish you a productive and interesting week at MCM 2025!


\vspace{5ex}

The MCM 2025 Organizers 

\smallskip

Sou-Cheng Choi, \emph{Illinois Institute of Technology} \\
Yuhan Ding, \emph{Illinois Institute of Technology} \\
Fred J. Hickernell, \emph{Illinois Institute of Technology} \\
Tim Hobbs, \emph{Argonne National Laboratory} \\
Faith Kancauski, \emph{Illinois Institute of Technology} \\
Lulu Kang, \emph{University of Massachusetts Amherst} \\
Nathan Kirk, \emph{Illinois Institute of Technology} \\
Yiou Li, \emph{DePaul University} \\
David Minh, \emph{Illinois Institute of Technology} \\
Chang-Han Rhee, \emph{Northwestern University} \\
Daniel Sanz-Alonso, \emph{University of Chicago}


\vspace{0.5cm}
Conference website: \url{https://mcm2025chicago.org} \\
Conference email: \url{info@mcm2025chicago.org}

% ------------------------------------------------------------------------

%\section{About MCM 2025}

\section{Steering Committee and History}


The biennial International Conference on Monte Carlo Methods and Applications (MCM) is an international gathering of researchers devoted to the theory, methodology, and application of Monte Carlo methods and related subjects. It is held in odd numbered years and is guided by a steering committee: 

Ronald Cools, \emph{KU Leuven} \\
Mike Giles, \emph{Oxford University} \\
Emmanuel Gobet, \emph{Ecole Polytechnique, Palaiseau} \\
Frances Kuo, \emph{University of New South Wales} \\
Christiane Lemieux, \emph{University of Waterloo} \\
Gunter Leobacher, \emph{University of Graz} \\
Thomas Müller-Gronbach, \emph{Universität Passau} \\
Bruno Tuffin, \emph{Inria Rennes Bretagne-Atlantique}

This is the fifteenth MCM conference.  The previous  conferences were held in
\begin{enumerate}
\item Paris, France, July 2023
\item Mannheim, Germany, August 2021
\item Sydney, Australia, July 2019
\item Montreal, Canada, July 2017
\item Linz, Austria, July 2015
\item Annecy-le-Vieux, France, July 2013
\item Borovets, Bulgaria, August 2011
\item Brussels, Belgium, September 2009
\item Reading, UK, June 2007
\item Tallahassee, USA, May 2005
\item Berlin, Germany, September 2003
\item Salzburg, Austria, September 2001
\item Varna, Bulgaria, June 1999
\item Brussels, Belgium, April 1997 
\end{enumerate}

\section{Plenary Speakers}

We are delighted that eight Monte Carlo experts accepted our invitation to provide plenary talks that introduce the audience to recent results while providing context and motivation.

Nicolas Chopin, \emph{ENSAE, Institut Polytechnique de Paris} \\
Peter W Glynn, \emph{Stanford University} \\
Roshan Joseph, \emph{Georgia Institute of Technology} \\
Christiane Lemieux, \emph{University of Waterloo} \\
Veronika Rockova, \emph{University of Chicago} \\
Rohan Sawhney, \emph{NVIDIA} \\
Uros Seljak, \emph{University of California, Berkeley} \\
Michaela Szölgyenyi, \emph{University of Klagenfurt (AAU)}

\section{Conference Topics}

MCM 2025  include active topics of research in Monte Carlo methods—those
with a long history as well as those emerging topics. These include:

\setlength{\columnsep}{1cm}
\begin{multicols}{2}
	\raggedright
	• Markov chain Monte Carlo
	
	• Hamiltonian Monte Carlo
	
	• Sequential Monte Carlo, particle filters
	
	• Non-equilibrium candidate Monte Carlo
	
	• Bridge sampling
	
	• Rare event simulation
	
	• Multi-level Monte Carlo
	
	• (Randomized) quasi-Monte Carlo
	
	• Digital nets and lattice rules
	
	• Discrepancy theory
	
	• Complexity and tractability of multivariate problems
	
	• Variance reduction
	
	• Monte Carlo simulation on high-performance architectures
	
	• Uncertainty quantification
	
	• Experimental design
	
	• Generative models from artificial intelligence
	
	• Variational inference
	
	• Probabilistic numerics
	
	• Monte Carlo methods for quantum computers
	
	• Stochastic gradient and other stochastic optimization methods
	
	• Statistical learning and Monte Carlo sampling
	
	• Reinforcement learning and control
	
	• Bayesian inference
	
	• Computational statistical physics
	
	• Economic, engineering, industrial, and scientific applications
	
\end{multicols}






\section{Scientific Committee}

Academics engaged in Monte Carlo theory, methodoglogy, and practice served on the Scientific Committee.  These colleagues nominated plenary speakers and organized special sessions.


\setlength{\columnsep}{1cm}
\begin{multicols}{2}
\raggedright
Miguel Arratia (Department of Physics and Astronomy, U California, Riverside)

Ronald Cools (Department of Computer Science, KU Leuven)

Xinwei Deng (Department of Statistics, Virginia Polytechnic and State U)

Jing Dong (Graduate School of Business, Columbia)

Mike Giles (Mathematical Institute, Oxford U)

Emmanuel Gobet (Centre de Mathématiques Appliquées, École Polytechnique)

Shane Henderson (School of Operations Research and Information Engineering, Cornell U)

Xuhui Huang (Department of Chemistry, UW Madison)

Joshua Isaacson (Fermilab)

Peter Kritzer (Johann Radon Institute for Computational and Applied Mathematics, Austrian Academy of Sciences)

Frances Kuo (School of Mathematics and Statistics, U New South Wales)

Pierre L'Ecuyer (Département d'informatique et de recherche opérationnelle, U Montréal)

Christiane Lemieux (Department of Statistics and Actuarial Science, U Waterloo)

Gunther Leobacher (Institute of Mathematics and Scientific Computing, U Graz)

Chunfang Devon Lin (Department of Mathematics and Statistics, Queens U)

Simon Mak (Department of Statistical Science, Duke U)

Michael Mascagni (Department of Computer Science, Florida State U)

Thomas Müller-Gronbach (Faculty of Computer Science and Mathematics, U Passau)

Ben Nachman (Lawrence Berkeley National Lab)

Chris Oates (School of Mathematics, Statistics, and Physics, U Newcastle Upon Tyne)

Art Owen (Department of Statistics, Stanford U)

Raghu Pasupathy (Department of Statistics, Purdue U)

Natesh Pillai (Department of Statistics, Harvard U)

Pieterjan Robbe (Sandia National Labs)

Veronika Rockova (Chicago Booth School of Business, U Chicago)

Jeffrey Rosenthal (Department of Statistics, U Toronto)

Aretha Teckentrup (School of Mathematics, U Edinburgh)

Bruno Tuffin (INRIA Rennes Bretagne-Atlantique)

Jonathan Weare (Courant Institute of Mathematical Sciences, New York U)

\end{multicols}



\section{Student Assistants}
\update{TODO}

% \colorbox{gray!20!white}{\makebox{%
%   \includegraphics[width=3cm]{organizer-Sedgers}}}

\subsection{Sponsors}

We are grateful to our financial sponsors, who have made it possible to offer the conference program at a reduced cost to participants.

Institute for Mathematical and Statistical Innovation (IMSI), a US National Science Foundation research institute based at the University of Chicago\\
\url{https://www.imsi.institute/} \\
supported our plenary speakers, many of our early career participants, and some of logistical expenses

Illinois Institute of Technology\\
\url{https://www.iit.edu/} \\
made space available at reduced cost and supported the financial infrastructure

Committee on Computational and Applied Mathematics, University of Chicago\\
\url{https://cam.uchicago.edu/}

Argonne National Laboratory\\
\url{https://www.anl.gov/}

NYU Courant Institute\\
\url{https://cims.nyu.edu/}

BeeInventor: IoT for Smart Construction\\
\url{https://www.beeinventor.com/}

Xcelerator Business Summit\\
\url{https://www.xbsinfo.com/}

A few Illinois Tech alumni also supported this conference through financial gifts.

\vspace{-4ex}
% Sponsor logos
\begin{center}

	    \begin{minipage}{0.2\textwidth}
		\centering
		\includegraphics[height=3cm]{Photos/nsf_logo.png}
	\end{minipage}
	%\hspace{2em}
	\begin{minipage}{0.2\textwidth}
		\centering
		\includegraphics[height=3cm]{Photos/imsi_logo.png}
	\end{minipage} \\[1em]
	
\includegraphics[height=1.5cm]{Photos/illinois_tech_logo_full.png} \\[1em]
\includegraphics[height=1.5cm]{Photos/uchicago_cam_logo.png} \\[1em]
\includegraphics[height=2.5cm]{Photos/nyu_courant_logo.png} \\[1em]
\includegraphics[height=2.5cm]{Photos/argonne_logo.png} \\[1em]
\includegraphics[height=1.5cm]{Photos/beeinventor_logo.png} \\[1em]
\includegraphics[height=3cm]{Photos/xcelerator_logo.png}

\end{center}


% ------------------------------------------------------------------------
\subsection{Special Thanks}

The conference organizers would like to thank all sponsors for making this
event possible. We especially want to express our gratitude to the Institute for Mathematical and Statistical Innovation (IMSI), a US National Science Foundation research institute, for their generous support and funding for travel assistance to conference participants.

We also want to express our gratitude to Illinois Institute of Technology for providing us with the venue, resources, and institutional support to host this conference. Special thanks to the Department of Applied Mathematics and the entire Illinois Tech community for their assistance with the conference organization.

We are grateful to our partner institutions including the University of Chicago's Committee on Computational and Applied Mathematics, NYU Courant Institute, and Argonne National Laboratory for their support and collaboration.

We also thank our industry sponsors BeeInventor and Xcelerator Business Summit for their contributions to making this conference possible.

We wish to extend our thanks to the entire Steering Committee and
Scientific Program Committee, and past MCM conference organizers for their
contribution and support. We also thank our plenary speakers Nicolas Chopin, Peter W Glynn, Roshan Joseph, Christiane Lemieux, Veronika Rockova, Rohan Sawhney, Uros Seljak, and Michaela Szölgyenyi, as well as all special session organizers and session chairs for their help and support with the scientific organization of the conference.

Last but not least, we are extremely grateful to many friends in the MCM community 
who helped us in various ways to organize MCM 2025, in particular Mike Giles, 
Takashi Goda, Frances Y. Kuo, Christiane Lemieux, and Art B. Owen. \update{Names}

\clearpage

% ------------------------------------------------------------------------
% ------------------------------------------------------------------------
% ------------------------------------------------------------------------