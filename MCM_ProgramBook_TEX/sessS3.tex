\sessionPart{}% [1] part
{\hfill\timeslot{Mon, July 28, 2024 -- Morning}
{10:30}{12:30} % Start and End time
{}} % Room 
\sessionTalk{QMC for Bayesian optimal experimental design with application to inverse problems governed by PDEs}
{Vesa Kaarnioja}
{S3-1}
\sessionTalk{Double-loop randomized quasi-Monte Carlo estimator for nested integration}
{Sebastian Krumscheid}
{S3-2}
\sessionTalk{Posterior-Free A-Optimal Bayesian Design of Experiments via Conditional Expectation}
{Vinh Hoang}
{S3-3}
